
\chapter{Odpowiedzi skokowe dla DMC}
Wartość sygnałów w punkcie pracy wynosi $U_{pp}=Y_{pp}=Z_{pp}=0$. Zatem wykonując skok sygnału sterującego lub zakłócenia z wartości 0 do wartości 1,  mamy prawie gotową odpowiedź skokową na potrzeby algorytmu DMC. Wystarczy bowiem przysunąć wykres tak aby skok występował w chwili k=0.

\begin{figure}[H]
\centering
% This file was created by matlab2tikz.
%
%The latest updates can be retrieved from
%  http://www.mathworks.com/matlabcentral/fileexchange/22022-matlab2tikz-matlab2tikz
%where you can also make suggestions and rate matlab2tikz.
%
\definecolor{mycolor1}{rgb}{0.00000,0.44700,0.74100}%
%
\begin{tikzpicture}

\begin{axis}[%
width=4.521in,
height=3.566in,
at={(0.758in,0.481in)},
scale only axis,
xmin=0,
xmax=300,
xlabel style={font=\color{white!15!black}},
xlabel={k},
ymin=0,
ymax=2.5,
ylabel style={font=\color{white!15!black}},
ylabel={Y(k)},
axis background/.style={fill=white}
]
\addplot[const plot, color=mycolor1, forget plot] table[row sep=crcr] {%
1	0\\
2	0\\
3	0\\
4	0\\
5	0\\
6	0\\
7	0.14788\\
8	0.287003368\\
9	0.4178875887248\\
10	0.541019681620169\\
11	0.656857981328252\\
12	0.765833815583596\\
13	0.868353086271741\\
14	0.964797759090307\\
15	1.05552726696406\\
16	1.14087983209031\\
17	1.2211737112286\\
18	1.29670836859919\\
19	1.36776558051679\\
20	1.43461047566053\\
21	1.4974925146662\\
22	1.55664641252315\\
23	1.61229300706489\\
24	1.6646400766589\\
25	1.71388311002751\\
26	1.76020603096696\\
27	1.8037818805757\\
28	1.84477345945558\\
29	1.88333393220961\\
30	1.91960739642823\\
31	1.95372941823058\\
32	1.98582753630976\\
33	2.01602173631893\\
34	2.04442489733007\\
35	2.07114321199739\\
36	2.09627658196347\\
37	2.11991898995759\\
38	2.14215884995146\\
39	2.16307933665916\\
40	2.18275869559299\\
41	2.20127053481677\\
42	2.2186840994719\\
43	2.23506453008862\\
44	2.25047310563613\\
45	2.26496747220972\\
46	2.27860185820011\\
47	2.29142727674155\\
48	2.30349171618796\\
49	2.31484031932293\\
50	2.32551555196791\\
51	2.33555736161404\\
52	2.34500332666624\\
53	2.35388879685375\\
54	2.3622470253288\\
55	2.37010929294415\\
56	2.37750502517179\\
57	2.38446190209753\\
58	2.39100596190081\\
59	2.39716169820485\\
60	2.40295215165975\\
61	2.40839899609941\\
62	2.41352261959357\\
63	2.41834220069678\\
64	2.42287578017859\\
65	2.42714032850259\\
66	2.43115180930567\\
67	2.43492523911456\\
68	2.43847474352232\\
69	2.44181361003448\\
70	2.44495433778203\\
71	2.44790868428693\\
72	2.45068770945463\\
73	2.45330181695808\\
74	2.45576079316756\\
75	2.45807384377197\\
76	2.46024962822822\\
77	2.46229629216758\\
78	2.46422149788002\\
79	2.46603245299044\\
80	2.46773593743402\\
81	2.46933832883159\\
82	2.47084562635968\\
83	2.47226347320473\\
84	2.47359717768524\\
85	2.47485173312098\\
86	2.47603183652342\\
87	2.47714190617744\\
88	2.47818609817987\\
89	2.47916832199701\\
90	2.48009225509899\\
91	2.48096135672601\\
92	2.48177888083776\\
93	2.48254788829452\\
94	2.48327125831562\\
95	2.48395169925794\\
96	2.48459175875496\\
97	2.48519383325411\\
98	2.4857601769884\\
99	2.48629291041548\\
100	2.48679402815619\\
101	2.48726540646188\\
102	2.48770881023874\\
103	2.48812589965527\\
104	2.48851823635772\\
105	2.48888728931666\\
106	2.48923444032666\\
107	2.48956098917959\\
108	2.48986815853095\\
109	2.49015709847745\\
110	2.49042889086289\\
111	2.49068455332858\\
112	2.49092504312342\\
113	2.49115126068779\\
114	2.49136405302485\\
115	2.49156421687177\\
116	2.49175250168271\\
117	2.49192961243486\\
118	2.4920962122679\\
119	2.49225292496687\\
120	2.49240033729765\\
121	2.4925390012039\\
122	2.49266943587359\\
123	2.4927921296829\\
124	2.49290754202472\\
125	2.49301610502877\\
126	2.49311822517946\\
127	2.49321428483788\\
128	2.4933046436734\\
129	2.49338964001034\\
130	2.4934695920947\\
131	2.49354479928574\\
132	2.4936155431768\\
133	2.49368208864968\\
134	2.49374468486638\\
135	2.49380356620202\\
136	2.49385895312237\\
137	2.49391105300933\\
138	2.4939600609374\\
139	2.49400616040409\\
140	2.49404952401695\\
141	2.49409031413988\\
142	2.49412868350103\\
143	2.49416477576464\\
144	2.49419872606899\\
145	2.49423066153234\\
146	2.49426070172891\\
147	2.4942889591366\\
148	2.49431553955811\\
149	2.49434054251712\\
150	2.49436406163089\\
151	2.49438618496083\\
152	2.49440699534216\\
153	2.49442657069415\\
154	2.49444498431178\\
155	2.49446230514023\\
156	2.49447859803301\\
157	2.49449392399478\\
158	2.49450834040979\\
159	2.4945219012567\\
160	2.49453465731073\\
161	2.49454665633373\\
162	2.49455794325308\\
163	2.49456856032983\\
164	2.49457854731699\\
165	2.49458794160836\\
166	2.49459677837853\\
167	2.4946050907146\\
168	2.49461290974005\\
169	2.49462026473128\\
170	2.49462718322725\\
171	2.49463369113254\\
172	2.4946398128144\\
173	2.49464557119396\\
174	2.49465098783205\\
175	2.49465608300993\\
176	2.49466087580524\\
177	2.49466538416342\\
178	2.4946696249649\\
179	2.49467361408832\\
180	2.49467736646996\\
181	2.49468089615971\\
182	2.49468421637361\\
183	2.49468733954338\\
184	2.49469027736297\\
185	2.49469304083232\\
186	2.49469564029856\\
187	2.49469808549475\\
188	2.49470038557633\\
189	2.49470254915539\\
190	2.49470458433291\\
191	2.49470649872912\\
192	2.49470829951201\\
193	2.49470999342416\\
194	2.49471158680801\\
195	2.4947130856296\\
196	2.4947144955009\\
197	2.49471582170082\\
198	2.49471706919499\\
199	2.49471824265435\\
200	2.49471934647263\\
201	2.4947203847828\\
202	2.49472136147257\\
203	2.49472228019891\\
204	2.49472314440176\\
205	2.49472395731691\\
206	2.49472472198814\\
207	2.49472544127856\\
208	2.49472611788139\\
209	2.49472675432998\\
210	2.49472735300737\\
211	2.49472791615517\\
212	2.49472844588193\\
213	2.4947289441711\\
214	2.49472941288838\\
215	2.49472985378879\\
216	2.49473026852315\\
217	2.49473065864435\\
218	2.4947310256131\\
219	2.49473137080342\\
220	2.49473169550779\\
221	2.49473200094199\\
222	2.49473228824964\\
223	2.4947325585065\\
224	2.49473281272447\\
225	2.49473305185541\\
226	2.49473327679469\\
227	2.49473348838454\\
228	2.4947336874172\\
229	2.49473387463791\\
230	2.49473405074766\\
231	2.49473421640586\\
232	2.49473437223277\\
233	2.49473451881185\\
234	2.49473465669193\\
235	2.49473478638926\\
236	2.49473490838947\\
237	2.49473502314935\\
238	2.4947351310986\\
239	2.49473523264141\\
240	2.49473532815797\\
241	2.49473541800593\\
242	2.49473550252169\\
243	2.49473558202172\\
244	2.49473565680367\\
245	2.49473572714755\\
246	2.49473579331675\\
247	2.49473585555901\\
248	2.49473591410739\\
249	2.49473596918112\\
250	2.49473602098639\\
251	2.49473606971718\\
252	2.49473611555596\\
253	2.49473615867435\\
254	2.4947361992338\\
255	2.49473623738617\\
256	2.49473627327433\\
257	2.49473630703263\\
258	2.49473633878749\\
259	2.4947363686578\\
260	2.4947363967554\\
261	2.49473642318549\\
262	2.49473644804705\\
263	2.49473647143315\\
264	2.49473649343136\\
265	2.49473651412405\\
266	2.49473653358869\\
267	2.49473655189816\\
268	2.49473656912103\\
269	2.49473658532178\\
270	2.49473660056106\\
271	2.49473661489594\\
272	2.49473662838009\\
273	2.49473664106399\\
274	2.49473665299515\\
275	2.49473666421824\\
276	2.49473667477526\\
277	2.49473668470577\\
278	2.49473669404693\\
279	2.49473670283372\\
280	2.49473671109904\\
281	2.49473671887384\\
282	2.49473672618723\\
283	2.4947367330666\\
284	2.49473673953769\\
285	2.49473674562475\\
286	2.49473675135056\\
287	2.49473675673656\\
288	2.49473676180292\\
289	2.49473676656861\\
290	2.49473677105147\\
291	2.49473677105147\\
292	2.49473677105147\\
293	2.49473677105147\\
294	2.49473677105147\\
295	2.49473677105147\\
296	2.49473677105147\\
297	2.49473677105147\\
298	2.49473677105147\\
299	2.49473677105147\\
300	2.49473677105147\\
};
\end{axis}
\end{tikzpicture}%
\caption{Odpowiedź skokowa toru sterowanie-wyjście}
\end{figure}

\begin{figure}[H]
\centering
% This file was created by matlab2tikz.
%
%The latest updates can be retrieved from
%  http://www.mathworks.com/matlabcentral/fileexchange/22022-matlab2tikz-matlab2tikz
%where you can also make suggestions and rate matlab2tikz.
%
\definecolor{mycolor1}{rgb}{0.00000,0.44700,0.74100}%
%
\begin{tikzpicture}

\begin{axis}[%
width=4.521in,
height=3.566in,
at={(0.758in,0.481in)},
scale only axis,
xmin=0,
xmax=300,
xlabel style={font=\color{white!15!black}},
xlabel={k},
ymin=0,
ymax=1.8,
ylabel style={font=\color{white!15!black}},
ylabel={Y(k)},
axis background/.style={fill=white}
]
\addplot[const plot, color=mycolor1, forget plot] table[row sep=crcr] {%
1	0\\
2	0\\
3	0.20202\\
4	0.381363772\\
5	0.5405745884792\\
6	0.681910601930065\\
7	0.807376801662065\\
8	0.918753389257166\\
9	1.01762097400741\\
10	1.10538294469291\\
11	1.18328533412576\\
12	1.25243445742761\\
13	1.31381257352359\\
14	1.36829179137883\\
15	1.41664641767974\\
16	1.45956392062001\\
17	1.49765466487875\\
18	1.53146055549941\\
19	1.56146271294626\\
20	1.58808828791264\\
21	1.61171651228869\\
22	1.63268407189206\\
23	1.6512898769727\\
24	1.66779929798444\\
25	1.68244792655289\\
26	1.69544491485312\\
27	1.70697594064711\\
28	1.71720583993623\\
29	1.7262809444819\\
30	1.73433115727287\\
31	1.74147179531066\\
32	1.7478052257929\\
33	1.75342231885168\\
34	1.75840373740865\\
35	1.7628210824046\\
36	1.76673790961448\\
37	1.77021063244259\\
38	1.77328932347903\\
39	1.77601842616608\\
40	1.77843738665143\\
41	1.78058121477559\\
42	1.78248098213803\\
43	1.7841642642963\\
44	1.78565553336142\\
45	1.78697650655121\\
46	1.78814645563942\\
47	1.78918248168534\\
48	1.79009975893699\\
49	1.7909117513645\\
50	1.791630404893\\
51	1.79226631806014\\
52	1.7928288935179\\
53	1.7933264725271\\
54	1.79376645435222\\
55	1.79415540225022\\
56	1.79449913755705\\
57	1.79480282320733\\
58	1.7950710378724\\
59	1.79530784176955\\
60	1.79551683507687\\
61	1.79570120978355\\
62	1.79586379571225\\
63	1.79600710136771\\
64	1.79613335019233\\
65	1.79624451274432\\
66	1.79634233525612\\
67	1.79642836497965\\
68	1.79650397267911\\
69	1.79657037259177\\
70	1.79662864014108\\
71	1.7966797276547\\
72	1.79672447831156\\
73	1.79676363851692\\
74	1.79679786888231\\
75	1.79682775396694\\
76	1.79685381091999\\
77	1.79687649714742\\
78	1.79689621711284\\
79	1.79691332837016\\
80	1.79692814691417\\
81	1.79694095192608\\
82	1.79695198998197\\
83	1.79696147878477\\
84	1.79696961047339\\
85	1.79697655455666\\
86	1.79698246051441\\
87	1.79698746010321\\
88	1.79699166940006\\
89	1.79699519061374\\
90	1.79699811368989\\
91	1.79700051773321\\
92	1.79700247226746\\
93	1.79700403835162\\
94	1.79700526956837\\
95	1.79700621289954\\
96	1.79700690950111\\
97	1.79700739538939\\
98	1.79700770204824\\
99	1.79700785696642\\
100	1.797007884113\\
101	1.79700780435784\\
102	1.79700763584335\\
103	1.79700739431327\\
104	1.79700709340309\\
105	1.79700674489673\\
106	1.79700635895323\\
107	1.79700594430688\\
108	1.79700550844388\\
109	1.79700505775816\\
110	1.79700459768881\\
111	1.79700413284126\\
112	1.79700366709395\\
113	1.79700320369234\\
114	1.79700274533158\\
115	1.79700229422931\\
116	1.79700185218951\\
117	1.7970014206587\\
118	1.79700100077522\\
119	1.79700059341239\\
120	1.79700019921639\\
121	1.79699981863939\\
122	1.7969994519685\\
123	1.79699909935105\\
124	1.79699876081664\\
125	1.79699843629634\\
126	1.79699812563939\\
127	1.79699782862767\\
128	1.79699754498826\\
129	1.79699727440427\\
130	1.7969970165242\\
131	1.79699677096999\\
132	1.79699653734388\\
133	1.79699631523434\\
134	1.79699610422104\\
135	1.79699590387911\\
136	1.79699571378272\\
137	1.79699553350805\\
138	1.7969953626358\\
139	1.79699520075323\\
140	1.7969950474558\\
141	1.79699490234855\\
142	1.79699476504709\\
143	1.79699463517849\\
144	1.7969945123818\\
145	1.79699439630853\\
146	1.79699428662288\\
147	1.7969941830019\\
148	1.79699408513551\\
149	1.79699399272646\\
150	1.79699390549016\\
151	1.79699382315454\\
152	1.79699374545974\\
153	1.79699367215784\\
154	1.79699360301259\\
155	1.79699353779898\\
156	1.79699347630295\\
157	1.79699341832097\\
158	1.79699336365969\\
159	1.79699331213552\\
160	1.79699326357426\\
161	1.79699321781069\\
162	1.79699317468824\\
163	1.79699313405853\\
164	1.79699309578107\\
165	1.79699305972288\\
166	1.79699302575812\\
167	1.79699299376777\\
168	1.79699296363927\\
169	1.79699293526624\\
170	1.79699290854814\\
171	1.79699288339001\\
172	1.79699285970213\\
173	1.79699283739981\\
174	1.7969928164031\\
175	1.79699279663651\\
176	1.79699277802883\\
177	1.79699276051287\\
178	1.79699274402526\\
179	1.79699272850622\\
180	1.79699271389937\\
181	1.79699270015157\\
182	1.79699268721271\\
183	1.79699267503556\\
184	1.79699266357561\\
185	1.79699265279091\\
186	1.79699264264193\\
187	1.79699263309143\\
188	1.79699262410431\\
189	1.79699261564753\\
190	1.79699260768995\\
191	1.79699260020224\\
192	1.79699259315678\\
193	1.79699258652757\\
194	1.79699258029012\\
195	1.79699257442135\\
196	1.79699256889957\\
197	1.79699256370431\\
198	1.79699255881634\\
199	1.79699255421754\\
200	1.79699254989084\\
201	1.79699254582019\\
202	1.79699254199048\\
203	1.79699253838747\\
204	1.79699253499778\\
205	1.79699253180881\\
206	1.7969925288087\\
207	1.79699252598627\\
208	1.79699252333103\\
209	1.79699252083309\\
210	1.79699251848315\\
211	1.79699251627244\\
212	1.79699251419274\\
213	1.79699251223629\\
214	1.79699251039579\\
215	1.79699250866438\\
216	1.7969925070356\\
217	1.79699250550338\\
218	1.796992504062\\
219	1.79699250270607\\
220	1.79699250143053\\
221	1.79699250023063\\
222	1.79699249910187\\
223	1.79699249804005\\
224	1.7969924970412\\
225	1.79699249610159\\
226	1.7969924952177\\
227	1.79699249438624\\
228	1.79699249360409\\
229	1.79699249286833\\
230	1.79699249217621\\
231	1.79699249152515\\
232	1.79699249091271\\
233	1.7969924903366\\
234	1.79699248979466\\
235	1.79699248928488\\
236	1.79699248880533\\
237	1.79699248835424\\
238	1.79699248792991\\
239	1.79699248753075\\
240	1.79699248715528\\
241	1.79699248680208\\
242	1.79699248646984\\
243	1.79699248615731\\
244	1.79699248586332\\
245	1.79699248558678\\
246	1.79699248532664\\
247	1.79699248508194\\
248	1.79699248485176\\
249	1.79699248463524\\
250	1.79699248443157\\
251	1.79699248423998\\
252	1.79699248405976\\
253	1.79699248389023\\
254	1.79699248373076\\
255	1.79699248358076\\
256	1.79699248343965\\
257	1.79699248330692\\
258	1.79699248318207\\
259	1.79699248306463\\
260	1.79699248295415\\
261	1.79699248285023\\
262	1.79699248275248\\
263	1.79699248266053\\
264	1.79699248257403\\
265	1.79699248249267\\
266	1.79699248241613\\
267	1.79699248234414\\
268	1.79699248227642\\
269	1.79699248221272\\
270	1.7969924821528\\
271	1.79699248209644\\
272	1.79699248204342\\
273	1.79699248199354\\
274	1.79699248194663\\
275	1.7969924819025\\
276	1.79699248186099\\
277	1.79699248182194\\
278	1.79699248178521\\
279	1.79699248175066\\
280	1.79699248171816\\
281	1.79699248168759\\
282	1.79699248165883\\
283	1.79699248163178\\
284	1.79699248160634\\
285	1.7969924815824\\
286	1.79699248155989\\
287	1.79699248153871\\
288	1.79699248151878\\
289	1.79699248150005\\
290	1.79699248148242\\
291	1.79699248148242\\
292	1.79699248148242\\
293	1.79699248148242\\
294	1.79699248148242\\
295	1.79699248148242\\
296	1.79699248148242\\
297	1.79699248148242\\
298	1.79699248148242\\
299	1.79699248148242\\
300	1.79699248148242\\
};
\end{axis}
\end{tikzpicture}%
\caption{Odpowiedź skokowa toru zakłócenie-wyjście}
\end{figure}