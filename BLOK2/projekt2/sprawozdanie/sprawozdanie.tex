%! TEX encoding = utf8
\documentclass[a4paper,titlepage,11pt,twosides,floatssmall]{mwrep}
\usepackage[left=2.5cm,right=2.5cm,top=2.5cm,bottom=2.5cm]{geometry}
\usepackage[OT1]{fontenc}
\usepackage{polski}
\usepackage{amsmath}
\usepackage{amsfonts}
\usepackage{amssymb}
\usepackage{graphicx}
\usepackage{float}
\usepackage{url}
\usepackage{tikz}
\usetikzlibrary{arrows,calc,decorations.markings,math,arrows.meta}
\usepackage{rotating}
\usepackage[percent]{overpic}
\usepackage[utf8]{inputenc}
\usepackage{xcolor}
\usepackage{colortbl}
\usepackage{pgfplots}
\usetikzlibrary{pgfplots.groupplots}
\usepackage{listings}
\usepackage{matlab-prettifier}
\usepackage{enumitem,amssymb}
\definecolor{szary}{rgb}{0.95,0.95,0.95}
\usepackage{siunitx}
\usepackage{etoolbox}
\makeatletter
\patchcmd{\chapter}{\if@openright\cleardoublepage\else\clearpage\fi}{}{}{}
\makeatother
\sisetup{detect-weight,exponent-product=\cdot,output-decimal-marker={,},per-mode=symbol,binary-units=true,range-phrase={-},range-units=single}
\SendSettingsToPgf
%konfiguracje pakietu listings
\lstset{
	backgroundcolor=\color{szary},
	frame=single,
	breaklines=true,
}
\lstdefinestyle{customlatex}{
	basicstyle=\footnotesize\ttfamily,
	%basicstyle=\small\ttfamily,
}
\lstdefinestyle{customc}{
	breaklines=true,
	frame=tb,
	language=C,
	xleftmargin=0pt,
	showstringspaces=false,
	basicstyle=\small\ttfamily,
	keywordstyle=\bfseries\color{green!40!black},
	commentstyle=\itshape\color{purple!40!black},
	identifierstyle=\color{blue},
	stringstyle=\color{orange},
}
\lstdefinestyle{custommatlab}{
	captionpos=t,
	breaklines=true,
	frame=tb,
	xleftmargin=0pt,
	language=matlab,
	showstringspaces=false,
	basicstyle=\footnotesize\ttfamily,
	%basicstyle=\scriptsize\ttfamily,
	keywordstyle=\bfseries\color{green!40!black},
	commentstyle=\itshape\color{purple!40!black},
	identifierstyle=\color{blue},
	stringstyle=\color{orange},
}

%wymiar tekstu (bez �ywej paginy)
\textwidth 160mm \textheight 247mm

%ustawienia pakietu pgfplots
\pgfplotsset{
tick label style={font=\scriptsize},
label style={font=\small},
legend style={font=\small},
title style={font=\small}
}

\def\figurename{Rys.}
\def\tablename{Tab.}

%konfiguracja liczby p�ywaj�cych element�w
\setcounter{topnumber}{0}%2
\setcounter{bottomnumber}{3}%1
\setcounter{totalnumber}{5}%3
\renewcommand{\textfraction}{0.01}%0.2
\renewcommand{\topfraction}{0.95}%0.7
\renewcommand{\bottomfraction}{0.95}%0.3
\renewcommand{\floatpagefraction}{0.35}%0.5

\begin{document}
\frenchspacing
\pagestyle{uheadings}

%strona tytu�owa
\title{\bf Sprawozdanie z projektu i ćwiczenia laboratoryjnego nr 1, zadanie nr 1\vskip 0.1cm}
\author{Stanislau Stankevich, Rafał Bednarz, Ostrysz Jakub}
\date{2021}

\makeatletter
\renewcommand{\maketitle}{\begin{titlepage}
\begin{center}{\LARGE {\bf
Wydział Elektroniki i Technik Informacyjnych}}\\
\vspace{0.4cm}
{\LARGE {\bf Politechnika Warszawska}}\\
\vspace{0.3cm}
\end{center}
\vspace{5cm}
\begin{center}
{\bf \LARGE Projektowanie układów sterowania\\ (projekt grupowy) \vskip 0.1cm}
\end{center}
\vspace{1cm}
\begin{center}
{\bf \LARGE \@title}
\end{center}
\vspace{2cm}
\begin{center}
{\bf \Large \@author \par}
\end{center}
\vspace*{\stretch{6}}
\begin{center}
\bf{\large{Warszawa, \@date\vskip 0.1cm}}
\end{center}
\end{titlepage}
}
\makeatother

\maketitle

\tableofcontents
%! TEX encoding = utf8
\chapter{Sprawdzenie poprawności podanych wartości}
Żeby sprawdzić poprawność podanych wartości podajemy na wejscie sterowanie $u = 0$ oraz zakłócenie $z = 0$ i patrzymy na jakiej wartości się ustali $y$.

\begin{figure}[H]
\centering
% This file was created by matlab2tikz.
%
%The latest updates can be retrieved from
%  http://www.mathworks.com/matlabcentral/fileexchange/22022-matlab2tikz-matlab2tikz
%where you can also make suggestions and rate matlab2tikz.
%
\definecolor{mycolor1}{rgb}{0.00000,0.44700,0.74100}%
%
\begin{tikzpicture}

\begin{axis}[%
width=5.521in,
height=4.566in,
at={(0.758in,0.481in)},
scale only axis,
xmin=0,
xmax=300,
xlabel style={font=\color{white!15!black}},
xlabel={k},
ymin=-0.1,
ymax=1,
ylabel style={font=\color{white!15!black}},
ylabel={Y(k)},
axis background/.style={fill=white}
]
\addplot[const plot, color=mycolor1, forget plot] table[row sep=crcr] {%
1	0\\
2	0\\
3	0\\
4	0\\
5	0\\
6	0\\
7	0\\
8	0\\
9	0\\
10	0\\
11	0\\
12	0\\
13	0\\
14	0\\
15	0\\
16	0\\
17	0\\
18	0\\
19	0\\
20	0\\
21	0\\
22	0\\
23	0\\
24	0\\
25	0\\
26	0\\
27	0\\
28	0\\
29	0\\
30	0\\
31	0\\
32	0\\
33	0\\
34	0\\
35	0\\
36	0\\
37	0\\
38	0\\
39	0\\
40	0\\
41	0\\
42	0\\
43	0\\
44	0\\
45	0\\
46	0\\
47	0\\
48	0\\
49	0\\
50	0\\
51	0\\
52	0\\
53	0\\
54	0\\
55	0\\
56	0\\
57	0\\
58	0\\
59	0\\
60	0\\
61	0\\
62	0\\
63	0\\
64	0\\
65	0\\
66	0\\
67	0\\
68	0\\
69	0\\
70	0\\
71	0\\
72	0\\
73	0\\
74	0\\
75	0\\
76	0\\
77	0\\
78	0\\
79	0\\
80	0\\
81	0\\
82	0\\
83	0\\
84	0\\
85	0\\
86	0\\
87	0\\
88	0\\
89	0\\
90	0\\
91	0\\
92	0\\
93	0\\
94	0\\
95	0\\
96	0\\
97	0\\
98	0\\
99	0\\
100	0\\
101	0\\
102	0\\
103	0\\
104	0\\
105	0\\
106	0\\
107	0\\
108	0\\
109	0\\
110	0\\
111	0\\
112	0\\
113	0\\
114	0\\
115	0\\
116	0\\
117	0\\
118	0\\
119	0\\
120	0\\
121	0\\
122	0\\
123	0\\
124	0\\
125	0\\
126	0\\
127	0\\
128	0\\
129	0\\
130	0\\
131	0\\
132	0\\
133	0\\
134	0\\
135	0\\
136	0\\
137	0\\
138	0\\
139	0\\
140	0\\
141	0\\
142	0\\
143	0\\
144	0\\
145	0\\
146	0\\
147	0\\
148	0\\
149	0\\
150	0\\
151	0\\
152	0\\
153	0\\
154	0\\
155	0\\
156	0\\
157	0\\
158	0\\
159	0\\
160	0\\
161	0\\
162	0\\
163	0\\
164	0\\
165	0\\
166	0\\
167	0\\
168	0\\
169	0\\
170	0\\
171	0\\
172	0\\
173	0\\
174	0\\
175	0\\
176	0\\
177	0\\
178	0\\
179	0\\
180	0\\
181	0\\
182	0\\
183	0\\
184	0\\
185	0\\
186	0\\
187	0\\
188	0\\
189	0\\
190	0\\
191	0\\
192	0\\
193	0\\
194	0\\
195	0\\
196	0\\
197	0\\
198	0\\
199	0\\
200	0\\
201	0\\
202	0\\
203	0\\
204	0\\
205	0\\
206	0\\
207	0\\
208	0\\
209	0\\
210	0\\
211	0\\
212	0\\
213	0\\
214	0\\
215	0\\
216	0\\
217	0\\
218	0\\
219	0\\
220	0\\
221	0\\
222	0\\
223	0\\
224	0\\
225	0\\
226	0\\
227	0\\
228	0\\
229	0\\
230	0\\
231	0\\
232	0\\
233	0\\
234	0\\
235	0\\
236	0\\
237	0\\
238	0\\
239	0\\
240	0\\
241	0\\
242	0\\
243	0\\
244	0\\
245	0\\
246	0\\
247	0\\
248	0\\
249	0\\
250	0\\
251	0\\
252	0\\
253	0\\
254	0\\
255	0\\
256	0\\
257	0\\
258	0\\
259	0\\
260	0\\
261	0\\
262	0\\
263	0\\
264	0\\
265	0\\
266	0\\
267	0\\
268	0\\
269	0\\
270	0\\
271	0\\
272	0\\
273	0\\
274	0\\
275	0\\
276	0\\
277	0\\
278	0\\
279	0\\
280	0\\
281	0\\
282	0\\
283	0\\
284	0\\
285	0\\
286	0\\
287	0\\
288	0\\
289	0\\
290	0\\
291	0\\
292	0\\
293	0\\
294	0\\
295	0\\
296	0\\
297	0\\
298	0\\
299	0\\
300	0\\
};
\end{axis}
\end{tikzpicture}%
\caption{Przebieg wyjścia obiektu przy stałym wejściu i zakłóceniu: $u = z = 0$}
\end{figure}

Jak możemy obersować wyjście się ustala na poprawnej wartości, czyli na 0.

%! TEX encoding = utf8
\chapter{Odpowiedzi skokowe}

Rozważamy punkt pracy oraz 6 różnych wartości skoku, z zera do: $-0,5$, $-0,25$, $0,25$, $0,5$, $0,75$, $1,0$.

\section{Opowiedzi skokowe}

\begin{figure}[H]
\centering
% This file was created by matlab2tikz.
%
%The latest updates can be retrieved from
%  http://www.mathworks.com/matlabcentral/fileexchange/22022-matlab2tikz-matlab2tikz
%where you can also make suggestions and rate matlab2tikz.
%
\definecolor{mycolor1}{rgb}{0.00000,0.44700,0.74100}%
\definecolor{mycolor2}{rgb}{0.85000,0.32500,0.09800}%
\definecolor{mycolor3}{rgb}{0.92900,0.69400,0.12500}%
\definecolor{mycolor4}{rgb}{0.49400,0.18400,0.55600}%
\definecolor{mycolor5}{rgb}{0.46600,0.67400,0.18800}%
\definecolor{mycolor6}{rgb}{0.30100,0.74500,0.93300}%
\definecolor{mycolor7}{rgb}{0.63500,0.07800,0.18400}%
%
\begin{tikzpicture}

\begin{axis}[%
width=6.102in,
height=6.417in,
at={(1.024in,0.866in)},
scale only axis,
xmin=0,
xmax=300,
xlabel style={font=\color{white!15!black}},
xlabel={k},
ymin=-0.8,
ymax=0.1,
ylabel style={font=\color{white!15!black}},
ylabel={Y(k)},
axis background/.style={fill=white},
legend style={at={(0.97,0.03)}, anchor=south east, legend cell align=left, align=left, draw=white!15!black}
]
\addplot[const plot, color=mycolor1] table[row sep=crcr] {%
1	0\\
2	0\\
3	0\\
4	0\\
5	0\\
6	0\\
7	0\\
8	0\\
9	0\\
10	0\\
11	0\\
12	0\\
13	0\\
14	0\\
15	-0.0335675877192982\\
16	-0.112965625142105\\
17	-0.212521981792259\\
18	-0.31513883462998\\
19	-0.41036001081684\\
20	-0.492623525565041\\
21	-0.5597813733745\\
22	-0.611911428472393\\
23	-0.650410015074312\\
24	-0.677333122976111\\
25	-0.694944862934428\\
26	-0.705429889631609\\
27	-0.71072929563746\\
28	-0.712464770401887\\
29	-0.711922131286509\\
30	-0.710071688301026\\
31	-0.707608712847838\\
32	-0.70500224248527\\
33	-0.702544465392307\\
34	-0.700396013802861\\
35	-0.698624752625891\\
36	-0.697237210244591\\
37	-0.696202804510307\\
38	-0.695471602257874\\
39	-0.69498663288036\\
40	-0.694691852874273\\
41	-0.694536805208319\\
42	-0.694478892004551\\
43	-0.694484021861918\\
44	-0.694526231178164\\
45	-0.694586728534931\\
46	-0.694652681377337\\
47	-0.694715958230594\\
48	-0.694771957427408\\
49	-0.694818592565868\\
50	-0.694855462432398\\
51	-0.694883205334342\\
52	-0.694903021243572\\
53	-0.694916336792617\\
54	-0.694924585428763\\
55	-0.694929075890161\\
56	-0.69493092508991\\
57	-0.69493103538224\\
58	-0.694930100299094\\
59	-0.69492862672469\\
60	-0.694926964866447\\
61	-0.694925340176121\\
62	-0.694923883564747\\
63	-0.694922657887373\\
64	-0.69492167982743\\
65	-0.694920937075836\\
66	-0.694920401165058\\
67	-0.694920036562932\\
68	-0.694919806722385\\
69	-0.694919677775033\\
70	-0.69491962048998\\
71	-0.694919611023644\\
72	-0.694919630882301\\
73	-0.694919666419212\\
74	-0.694919708099766\\
75	-0.694919749694496\\
76	-0.69491978750165\\
77	-0.694919819657184\\
78	-0.694919845558801\\
79	-0.694919865409643\\
80	-0.69491987987422\\
81	-0.694919889832076\\
82	-0.694919896211796\\
83	-0.694919899887748\\
84	-0.694919901623475\\
85	-0.694919902047935\\
86	-0.694919901653444\\
87	-0.694919900806729\\
88	-0.694919899766789\\
89	-0.694919898705208\\
90	-0.694919897726102\\
91	-0.694919896884061\\
92	-0.694919896199279\\
93	-0.694919895669655\\
94	-0.694919895279998\\
95	-0.694919895008691\\
96	-0.69491989483223\\
97	-0.694919894728098\\
98	-0.69491989467639\\
99	-0.694919894660544\\
100	-0.694919894667477\\
101	-0.69491989468735\\
102	-0.694919894713144\\
103	-0.694919894740146\\
104	-0.694919894765443\\
105	-0.694919894787452\\
106	-0.694919894805525\\
107	-0.694919894819632\\
108	-0.694919894830109\\
109	-0.694919894837484\\
110	-0.694919894842348\\
111	-0.694919894845282\\
112	-0.694919894846802\\
113	-0.694919894847343\\
114	-0.694919894847253\\
115	-0.694919894846796\\
116	-0.69491989484616\\
117	-0.694919894845476\\
118	-0.694919894844824\\
119	-0.69491989484425\\
120	-0.694919894843774\\
121	-0.694919894843398\\
122	-0.694919894843117\\
123	-0.694919894842917\\
124	-0.694919894842784\\
125	-0.694919894842702\\
126	-0.694919894842658\\
127	-0.69491989484264\\
128	-0.69491989484264\\
129	-0.69491989484265\\
130	-0.694919894842665\\
131	-0.694919894842683\\
132	-0.6949198948427\\
133	-0.694919894842715\\
134	-0.694919894842727\\
135	-0.694919894842737\\
136	-0.694919894842745\\
137	-0.69491989484275\\
138	-0.694919894842754\\
139	-0.694919894842756\\
140	-0.694919894842757\\
141	-0.694919894842757\\
142	-0.694919894842757\\
143	-0.694919894842757\\
144	-0.694919894842757\\
145	-0.694919894842756\\
146	-0.694919894842756\\
147	-0.694919894842756\\
148	-0.694919894842755\\
149	-0.694919894842755\\
150	-0.694919894842755\\
151	-0.694919894842755\\
152	-0.694919894842755\\
153	-0.694919894842755\\
154	-0.694919894842755\\
155	-0.694919894842755\\
156	-0.694919894842755\\
157	-0.694919894842755\\
158	-0.694919894842755\\
159	-0.694919894842755\\
160	-0.694919894842755\\
161	-0.694919894842755\\
162	-0.694919894842755\\
163	-0.694919894842755\\
164	-0.694919894842755\\
165	-0.694919894842755\\
166	-0.694919894842755\\
167	-0.694919894842755\\
168	-0.694919894842755\\
169	-0.694919894842755\\
170	-0.694919894842755\\
171	-0.694919894842755\\
172	-0.694919894842755\\
173	-0.694919894842755\\
174	-0.694919894842755\\
175	-0.694919894842755\\
176	-0.694919894842755\\
177	-0.694919894842755\\
178	-0.694919894842755\\
179	-0.694919894842755\\
180	-0.694919894842755\\
181	-0.694919894842755\\
182	-0.694919894842755\\
183	-0.694919894842755\\
184	-0.694919894842755\\
185	-0.694919894842755\\
186	-0.694919894842755\\
187	-0.694919894842755\\
188	-0.694919894842755\\
189	-0.694919894842755\\
190	-0.694919894842755\\
191	-0.694919894842755\\
192	-0.694919894842755\\
193	-0.694919894842755\\
194	-0.694919894842755\\
195	-0.694919894842755\\
196	-0.694919894842755\\
197	-0.694919894842755\\
198	-0.694919894842755\\
199	-0.694919894842755\\
200	-0.694919894842755\\
201	-0.694919894842755\\
202	-0.694919894842755\\
203	-0.694919894842755\\
204	-0.694919894842755\\
205	-0.694919894842755\\
206	-0.694919894842755\\
207	-0.694919894842755\\
208	-0.694919894842755\\
209	-0.694919894842755\\
210	-0.694919894842755\\
211	-0.694919894842755\\
212	-0.694919894842755\\
213	-0.694919894842755\\
214	-0.694919894842755\\
215	-0.694919894842755\\
216	-0.694919894842755\\
217	-0.694919894842755\\
218	-0.694919894842755\\
219	-0.694919894842755\\
220	-0.694919894842755\\
221	-0.694919894842755\\
222	-0.694919894842755\\
223	-0.694919894842755\\
224	-0.694919894842755\\
225	-0.694919894842755\\
226	-0.694919894842755\\
227	-0.694919894842755\\
228	-0.694919894842755\\
229	-0.694919894842755\\
230	-0.694919894842755\\
231	-0.694919894842755\\
232	-0.694919894842755\\
233	-0.694919894842755\\
234	-0.694919894842755\\
235	-0.694919894842755\\
236	-0.694919894842755\\
237	-0.694919894842755\\
238	-0.694919894842755\\
239	-0.694919894842755\\
240	-0.694919894842755\\
241	-0.694919894842755\\
242	-0.694919894842755\\
243	-0.694919894842755\\
244	-0.694919894842755\\
245	-0.694919894842755\\
246	-0.694919894842755\\
247	-0.694919894842755\\
248	-0.694919894842755\\
249	-0.694919894842755\\
250	-0.694919894842755\\
251	-0.694919894842755\\
252	-0.694919894842755\\
253	-0.694919894842755\\
254	-0.694919894842755\\
255	-0.694919894842755\\
256	-0.694919894842755\\
257	-0.694919894842755\\
258	-0.694919894842755\\
259	-0.694919894842755\\
260	-0.694919894842755\\
261	-0.694919894842755\\
262	-0.694919894842755\\
263	-0.694919894842755\\
264	-0.694919894842755\\
265	-0.694919894842755\\
266	-0.694919894842755\\
267	-0.694919894842755\\
268	-0.694919894842755\\
269	-0.694919894842755\\
270	-0.694919894842755\\
271	-0.694919894842755\\
272	-0.694919894842755\\
273	-0.694919894842755\\
274	-0.694919894842755\\
275	-0.694919894842755\\
276	-0.694919894842755\\
277	-0.694919894842755\\
278	-0.694919894842755\\
279	-0.694919894842755\\
280	-0.694919894842755\\
281	-0.694919894842755\\
282	-0.694919894842755\\
283	-0.694919894842755\\
284	-0.694919894842755\\
285	-0.694919894842755\\
286	-0.694919894842755\\
287	-0.694919894842755\\
288	-0.694919894842755\\
289	-0.694919894842755\\
290	-0.694919894842755\\
291	-0.694919894842755\\
292	-0.694919894842755\\
293	-0.694919894842755\\
294	-0.694919894842755\\
295	-0.694919894842755\\
296	-0.694919894842755\\
297	-0.694919894842755\\
298	-0.694919894842755\\
299	-0.694919894842755\\
300	-0.694919894842755\\
};
\addlegendentry{Skok U z 0.00 do -0.50}

\addplot[const plot, color=mycolor2] table[row sep=crcr] {%
1	0\\
2	0\\
3	0\\
4	0\\
5	0\\
6	0\\
7	0\\
8	0\\
9	0\\
10	0\\
11	0\\
12	0\\
13	0\\
14	0\\
15	-0.0150905587027915\\
16	-0.0484195449261625\\
17	-0.0864115139873757\\
18	-0.121395119761953\\
19	-0.149839723783237\\
20	-0.170855068765489\\
21	-0.185040572876755\\
22	-0.193679087044566\\
23	-0.198223356168033\\
24	-0.200006749973101\\
25	-0.200110976760491\\
26	-0.19933367609057\\
27	-0.198212181266038\\
28	-0.197072885924991\\
29	-0.196086789406331\\
30	-0.195320314175638\\
31	-0.194776439892249\\
32	-0.194424976657385\\
33	-0.194222917734573\\
34	-0.194126771051117\\
35	-0.194098993056308\\
36	-0.19411046158561\\
37	-0.194140551866487\\
38	-0.194175964595706\\
39	-0.194209076957524\\
40	-0.194236282449366\\
41	-0.19425656155894\\
42	-0.194270375315654\\
43	-0.194278883192304\\
44	-0.194283439888477\\
45	-0.194285307778275\\
46	-0.194285521654744\\
47	-0.194284851357772\\
48	-0.194283820249498\\
49	-0.194282749881076\\
50	-0.194281811811171\\
51	-0.1942810757299\\
52	-0.194280548816684\\
53	-0.194280204963872\\
54	-0.194280004604477\\
55	-0.194279906864281\\
56	-0.194279876020257\\
57	-0.194279884099752\\
58	-0.194279911117332\\
59	-0.194279944058589\\
60	-0.194279975362056\\
61	-0.194280001358456\\
62	-0.194280020910386\\
63	-0.19428003434934\\
64	-0.194280042717708\\
65	-0.194280047276223\\
66	-0.194280049218616\\
67	-0.194280049533891\\
68	-0.194280048964422\\
69	-0.194280048019476\\
70	-0.194280047015419\\
71	-0.194280046123964\\
72	-0.194280045417687\\
73	-0.194280044907653\\
74	-0.194280044571604\\
75	-0.194280044373251\\
76	-0.194280044274239\\
77	-0.194280044240613\\
78	-0.194280044245527\\
79	-0.194280044269632\\
80	-0.194280044300208\\
81	-0.194280044329764\\
82	-0.194280044354582\\
83	-0.194280044373416\\
84	-0.194280044386478\\
85	-0.194280044394699\\
86	-0.19428004439925\\
87	-0.194280044401257\\
88	-0.194280044401666\\
89	-0.194280044401192\\
90	-0.194280044400329\\
91	-0.194280044399388\\
92	-0.194280044398542\\
93	-0.194280044397865\\
94	-0.194280044397372\\
95	-0.194280044397044\\
96	-0.194280044396848\\
97	-0.194280044396748\\
98	-0.194280044396711\\
99	-0.194280044396713\\
100	-0.194280044396735\\
101	-0.194280044396763\\
102	-0.194280044396791\\
103	-0.194280044396815\\
104	-0.194280044396833\\
105	-0.194280044396845\\
106	-0.194280044396854\\
107	-0.194280044396858\\
108	-0.19428004439686\\
109	-0.194280044396861\\
110	-0.19428004439686\\
111	-0.194280044396859\\
112	-0.194280044396859\\
113	-0.194280044396858\\
114	-0.194280044396857\\
115	-0.194280044396857\\
116	-0.194280044396856\\
117	-0.194280044396856\\
118	-0.194280044396856\\
119	-0.194280044396856\\
120	-0.194280044396856\\
121	-0.194280044396856\\
122	-0.194280044396856\\
123	-0.194280044396856\\
124	-0.194280044396856\\
125	-0.194280044396856\\
126	-0.194280044396856\\
127	-0.194280044396856\\
128	-0.194280044396856\\
129	-0.194280044396856\\
130	-0.194280044396856\\
131	-0.194280044396856\\
132	-0.194280044396856\\
133	-0.194280044396856\\
134	-0.194280044396856\\
135	-0.194280044396856\\
136	-0.194280044396856\\
137	-0.194280044396856\\
138	-0.194280044396856\\
139	-0.194280044396856\\
140	-0.194280044396856\\
141	-0.194280044396856\\
142	-0.194280044396856\\
143	-0.194280044396856\\
144	-0.194280044396856\\
145	-0.194280044396856\\
146	-0.194280044396856\\
147	-0.194280044396856\\
148	-0.194280044396856\\
149	-0.194280044396856\\
150	-0.194280044396856\\
151	-0.194280044396856\\
152	-0.194280044396856\\
153	-0.194280044396856\\
154	-0.194280044396856\\
155	-0.194280044396856\\
156	-0.194280044396856\\
157	-0.194280044396856\\
158	-0.194280044396856\\
159	-0.194280044396856\\
160	-0.194280044396856\\
161	-0.194280044396856\\
162	-0.194280044396856\\
163	-0.194280044396856\\
164	-0.194280044396856\\
165	-0.194280044396856\\
166	-0.194280044396856\\
167	-0.194280044396856\\
168	-0.194280044396856\\
169	-0.194280044396856\\
170	-0.194280044396856\\
171	-0.194280044396856\\
172	-0.194280044396856\\
173	-0.194280044396856\\
174	-0.194280044396856\\
175	-0.194280044396856\\
176	-0.194280044396856\\
177	-0.194280044396856\\
178	-0.194280044396856\\
179	-0.194280044396856\\
180	-0.194280044396856\\
181	-0.194280044396856\\
182	-0.194280044396856\\
183	-0.194280044396856\\
184	-0.194280044396856\\
185	-0.194280044396856\\
186	-0.194280044396856\\
187	-0.194280044396856\\
188	-0.194280044396856\\
189	-0.194280044396856\\
190	-0.194280044396856\\
191	-0.194280044396856\\
192	-0.194280044396856\\
193	-0.194280044396856\\
194	-0.194280044396856\\
195	-0.194280044396856\\
196	-0.194280044396856\\
197	-0.194280044396856\\
198	-0.194280044396856\\
199	-0.194280044396856\\
200	-0.194280044396856\\
201	-0.194280044396856\\
202	-0.194280044396856\\
203	-0.194280044396856\\
204	-0.194280044396856\\
205	-0.194280044396856\\
206	-0.194280044396856\\
207	-0.194280044396856\\
208	-0.194280044396856\\
209	-0.194280044396856\\
210	-0.194280044396856\\
211	-0.194280044396856\\
212	-0.194280044396856\\
213	-0.194280044396856\\
214	-0.194280044396856\\
215	-0.194280044396856\\
216	-0.194280044396856\\
217	-0.194280044396856\\
218	-0.194280044396856\\
219	-0.194280044396856\\
220	-0.194280044396856\\
221	-0.194280044396856\\
222	-0.194280044396856\\
223	-0.194280044396856\\
224	-0.194280044396856\\
225	-0.194280044396856\\
226	-0.194280044396856\\
227	-0.194280044396856\\
228	-0.194280044396856\\
229	-0.194280044396856\\
230	-0.194280044396856\\
231	-0.194280044396856\\
232	-0.194280044396856\\
233	-0.194280044396856\\
234	-0.194280044396856\\
235	-0.194280044396856\\
236	-0.194280044396856\\
237	-0.194280044396856\\
238	-0.194280044396856\\
239	-0.194280044396856\\
240	-0.194280044396856\\
241	-0.194280044396856\\
242	-0.194280044396856\\
243	-0.194280044396856\\
244	-0.194280044396856\\
245	-0.194280044396856\\
246	-0.194280044396856\\
247	-0.194280044396856\\
248	-0.194280044396856\\
249	-0.194280044396856\\
250	-0.194280044396856\\
251	-0.194280044396856\\
252	-0.194280044396856\\
253	-0.194280044396856\\
254	-0.194280044396856\\
255	-0.194280044396856\\
256	-0.194280044396856\\
257	-0.194280044396856\\
258	-0.194280044396856\\
259	-0.194280044396856\\
260	-0.194280044396856\\
261	-0.194280044396856\\
262	-0.194280044396856\\
263	-0.194280044396856\\
264	-0.194280044396856\\
265	-0.194280044396856\\
266	-0.194280044396856\\
267	-0.194280044396856\\
268	-0.194280044396856\\
269	-0.194280044396856\\
270	-0.194280044396856\\
271	-0.194280044396856\\
272	-0.194280044396856\\
273	-0.194280044396856\\
274	-0.194280044396856\\
275	-0.194280044396856\\
276	-0.194280044396856\\
277	-0.194280044396856\\
278	-0.194280044396856\\
279	-0.194280044396856\\
280	-0.194280044396856\\
281	-0.194280044396856\\
282	-0.194280044396856\\
283	-0.194280044396856\\
284	-0.194280044396856\\
285	-0.194280044396856\\
286	-0.194280044396856\\
287	-0.194280044396856\\
288	-0.194280044396856\\
289	-0.194280044396856\\
290	-0.194280044396856\\
291	-0.194280044396856\\
292	-0.194280044396856\\
293	-0.194280044396856\\
294	-0.194280044396856\\
295	-0.194280044396856\\
296	-0.194280044396856\\
297	-0.194280044396856\\
298	-0.194280044396856\\
299	-0.194280044396856\\
300	-0.194280044396856\\
};
\addlegendentry{Skok U z 0.00 do -0.25}

\addplot[const plot, color=mycolor3] table[row sep=crcr] {%
1	0\\
2	0\\
3	0\\
4	0\\
5	0\\
6	0\\
7	0\\
8	0\\
9	0\\
10	0\\
11	0\\
12	0\\
13	0\\
14	0\\
15	0\\
16	0\\
17	0\\
18	0\\
19	0\\
20	0\\
21	0\\
22	0\\
23	0\\
24	0\\
25	0\\
26	0\\
27	0\\
28	0\\
29	0\\
30	0\\
31	0\\
32	0\\
33	0\\
34	0\\
35	0\\
36	0\\
37	0\\
38	0\\
39	0\\
40	0\\
41	0\\
42	0\\
43	0\\
44	0\\
45	0\\
46	0\\
47	0\\
48	0\\
49	0\\
50	0\\
51	0\\
52	0\\
53	0\\
54	0\\
55	0\\
56	0\\
57	0\\
58	0\\
59	0\\
60	0\\
61	0\\
62	0\\
63	0\\
64	0\\
65	0\\
66	0\\
67	0\\
68	0\\
69	0\\
70	0\\
71	0\\
72	0\\
73	0\\
74	0\\
75	0\\
76	0\\
77	0\\
78	0\\
79	0\\
80	0\\
81	0\\
82	0\\
83	0\\
84	0\\
85	0\\
86	0\\
87	0\\
88	0\\
89	0\\
90	0\\
91	0\\
92	0\\
93	0\\
94	0\\
95	0\\
96	0\\
97	0\\
98	0\\
99	0\\
100	0\\
101	0\\
102	0\\
103	0\\
104	0\\
105	0\\
106	0\\
107	0\\
108	0\\
109	0\\
110	0\\
111	0\\
112	0\\
113	0\\
114	0\\
115	0\\
116	0\\
117	0\\
118	0\\
119	0\\
120	0\\
121	0\\
122	0\\
123	0\\
124	0\\
125	0\\
126	0\\
127	0\\
128	0\\
129	0\\
130	0\\
131	0\\
132	0\\
133	0\\
134	0\\
135	0\\
136	0\\
137	0\\
138	0\\
139	0\\
140	0\\
141	0\\
142	0\\
143	0\\
144	0\\
145	0\\
146	0\\
147	0\\
148	0\\
149	0\\
150	0\\
151	0\\
152	0\\
153	0\\
154	0\\
155	0\\
156	0\\
157	0\\
158	0\\
159	0\\
160	0\\
161	0\\
162	0\\
163	0\\
164	0\\
165	0\\
166	0\\
167	0\\
168	0\\
169	0\\
170	0\\
171	0\\
172	0\\
173	0\\
174	0\\
175	0\\
176	0\\
177	0\\
178	0\\
179	0\\
180	0\\
181	0\\
182	0\\
183	0\\
184	0\\
185	0\\
186	0\\
187	0\\
188	0\\
189	0\\
190	0\\
191	0\\
192	0\\
193	0\\
194	0\\
195	0\\
196	0\\
197	0\\
198	0\\
199	0\\
200	0\\
201	0\\
202	0\\
203	0\\
204	0\\
205	0\\
206	0\\
207	0\\
208	0\\
209	0\\
210	0\\
211	0\\
212	0\\
213	0\\
214	0\\
215	0\\
216	0\\
217	0\\
218	0\\
219	0\\
220	0\\
221	0\\
222	0\\
223	0\\
224	0\\
225	0\\
226	0\\
227	0\\
228	0\\
229	0\\
230	0\\
231	0\\
232	0\\
233	0\\
234	0\\
235	0\\
236	0\\
237	0\\
238	0\\
239	0\\
240	0\\
241	0\\
242	0\\
243	0\\
244	0\\
245	0\\
246	0\\
247	0\\
248	0\\
249	0\\
250	0\\
251	0\\
252	0\\
253	0\\
254	0\\
255	0\\
256	0\\
257	0\\
258	0\\
259	0\\
260	0\\
261	0\\
262	0\\
263	0\\
264	0\\
265	0\\
266	0\\
267	0\\
268	0\\
269	0\\
270	0\\
271	0\\
272	0\\
273	0\\
274	0\\
275	0\\
276	0\\
277	0\\
278	0\\
279	0\\
280	0\\
281	0\\
282	0\\
283	0\\
284	0\\
285	0\\
286	0\\
287	0\\
288	0\\
289	0\\
290	0\\
291	0\\
292	0\\
293	0\\
294	0\\
295	0\\
296	0\\
297	0\\
298	0\\
299	0\\
300	0\\
};
\addlegendentry{Skok U z 0.00 do 0.00}

\addplot[const plot, color=mycolor4] table[row sep=crcr] {%
1	0\\
2	0\\
3	0\\
4	0\\
5	0\\
6	0\\
7	0\\
8	0\\
9	0\\
10	0\\
11	0\\
12	0\\
13	0\\
14	0\\
15	0.00916194129720854\\
16	0.0237109142051116\\
17	0.0347854477739337\\
18	0.0414098483215953\\
19	0.0448070287362194\\
20	0.0463239521732835\\
21	0.0468966103926382\\
22	0.0470569189157674\\
23	0.0470665175232664\\
24	0.047037056606585\\
25	0.047007921724263\\
26	0.0469883933986156\\
27	0.0469775416307898\\
28	0.046972307011315\\
29	0.0469701232168774\\
30	0.046969380787595\\
31	0.046969224712116\\
32	0.0469692594342373\\
33	0.0469693282748527\\
34	0.0469693827322709\\
35	0.0469694159507636\\
36	0.0469694332427358\\
37	0.0469694410832711\\
38	0.0469694441068266\\
39	0.0469694449934194\\
40	0.0469694450807196\\
41	0.046969444948964\\
42	0.0469694448080906\\
43	0.0469694447111181\\
44	0.0469694446563247\\
45	0.0469694446295051\\
46	0.0469694446181243\\
47	0.0469694446141454\\
48	0.046969444613232\\
49	0.046969444613348\\
50	0.0469694446136714\\
51	0.0469694446139384\\
52	0.0469694446141047\\
53	0.0469694446141926\\
54	0.046969444614233\\
55	0.046969444614249\\
56	0.0469694446142538\\
57	0.0469694446142545\\
58	0.0469694446142539\\
59	0.0469694446142532\\
60	0.0469694446142527\\
61	0.0469694446142525\\
62	0.0469694446142523\\
63	0.0469694446142523\\
64	0.0469694446142522\\
65	0.0469694446142522\\
66	0.0469694446142522\\
67	0.0469694446142522\\
68	0.0469694446142522\\
69	0.0469694446142522\\
70	0.0469694446142522\\
71	0.0469694446142522\\
72	0.0469694446142522\\
73	0.0469694446142522\\
74	0.0469694446142522\\
75	0.0469694446142522\\
76	0.0469694446142522\\
77	0.0469694446142522\\
78	0.0469694446142522\\
79	0.0469694446142522\\
80	0.0469694446142522\\
81	0.0469694446142522\\
82	0.0469694446142522\\
83	0.0469694446142522\\
84	0.0469694446142522\\
85	0.0469694446142522\\
86	0.0469694446142522\\
87	0.0469694446142522\\
88	0.0469694446142522\\
89	0.0469694446142522\\
90	0.0469694446142522\\
91	0.0469694446142522\\
92	0.0469694446142522\\
93	0.0469694446142522\\
94	0.0469694446142522\\
95	0.0469694446142522\\
96	0.0469694446142522\\
97	0.0469694446142522\\
98	0.0469694446142522\\
99	0.0469694446142522\\
100	0.0469694446142522\\
101	0.0469694446142522\\
102	0.0469694446142522\\
103	0.0469694446142522\\
104	0.0469694446142522\\
105	0.0469694446142522\\
106	0.0469694446142522\\
107	0.0469694446142522\\
108	0.0469694446142522\\
109	0.0469694446142522\\
110	0.0469694446142522\\
111	0.0469694446142522\\
112	0.0469694446142522\\
113	0.0469694446142522\\
114	0.0469694446142522\\
115	0.0469694446142522\\
116	0.0469694446142522\\
117	0.0469694446142522\\
118	0.0469694446142522\\
119	0.0469694446142522\\
120	0.0469694446142522\\
121	0.0469694446142522\\
122	0.0469694446142522\\
123	0.0469694446142522\\
124	0.0469694446142522\\
125	0.0469694446142522\\
126	0.0469694446142522\\
127	0.0469694446142522\\
128	0.0469694446142522\\
129	0.0469694446142522\\
130	0.0469694446142522\\
131	0.0469694446142522\\
132	0.0469694446142522\\
133	0.0469694446142522\\
134	0.0469694446142522\\
135	0.0469694446142522\\
136	0.0469694446142522\\
137	0.0469694446142522\\
138	0.0469694446142522\\
139	0.0469694446142522\\
140	0.0469694446142522\\
141	0.0469694446142522\\
142	0.0469694446142522\\
143	0.0469694446142522\\
144	0.0469694446142522\\
145	0.0469694446142522\\
146	0.0469694446142522\\
147	0.0469694446142522\\
148	0.0469694446142522\\
149	0.0469694446142522\\
150	0.0469694446142522\\
151	0.0469694446142522\\
152	0.0469694446142522\\
153	0.0469694446142522\\
154	0.0469694446142522\\
155	0.0469694446142522\\
156	0.0469694446142522\\
157	0.0469694446142522\\
158	0.0469694446142522\\
159	0.0469694446142522\\
160	0.0469694446142522\\
161	0.0469694446142522\\
162	0.0469694446142522\\
163	0.0469694446142522\\
164	0.0469694446142522\\
165	0.0469694446142522\\
166	0.0469694446142522\\
167	0.0469694446142522\\
168	0.0469694446142522\\
169	0.0469694446142522\\
170	0.0469694446142522\\
171	0.0469694446142522\\
172	0.0469694446142522\\
173	0.0469694446142522\\
174	0.0469694446142522\\
175	0.0469694446142522\\
176	0.0469694446142522\\
177	0.0469694446142522\\
178	0.0469694446142522\\
179	0.0469694446142522\\
180	0.0469694446142522\\
181	0.0469694446142522\\
182	0.0469694446142522\\
183	0.0469694446142522\\
184	0.0469694446142522\\
185	0.0469694446142522\\
186	0.0469694446142522\\
187	0.0469694446142522\\
188	0.0469694446142522\\
189	0.0469694446142522\\
190	0.0469694446142522\\
191	0.0469694446142522\\
192	0.0469694446142522\\
193	0.0469694446142522\\
194	0.0469694446142522\\
195	0.0469694446142522\\
196	0.0469694446142522\\
197	0.0469694446142522\\
198	0.0469694446142522\\
199	0.0469694446142522\\
200	0.0469694446142522\\
201	0.0469694446142522\\
202	0.0469694446142522\\
203	0.0469694446142522\\
204	0.0469694446142522\\
205	0.0469694446142522\\
206	0.0469694446142522\\
207	0.0469694446142522\\
208	0.0469694446142522\\
209	0.0469694446142522\\
210	0.0469694446142522\\
211	0.0469694446142522\\
212	0.0469694446142522\\
213	0.0469694446142522\\
214	0.0469694446142522\\
215	0.0469694446142522\\
216	0.0469694446142522\\
217	0.0469694446142522\\
218	0.0469694446142522\\
219	0.0469694446142522\\
220	0.0469694446142522\\
221	0.0469694446142522\\
222	0.0469694446142522\\
223	0.0469694446142522\\
224	0.0469694446142522\\
225	0.0469694446142522\\
226	0.0469694446142522\\
227	0.0469694446142522\\
228	0.0469694446142522\\
229	0.0469694446142522\\
230	0.0469694446142522\\
231	0.0469694446142522\\
232	0.0469694446142522\\
233	0.0469694446142522\\
234	0.0469694446142522\\
235	0.0469694446142522\\
236	0.0469694446142522\\
237	0.0469694446142522\\
238	0.0469694446142522\\
239	0.0469694446142522\\
240	0.0469694446142522\\
241	0.0469694446142522\\
242	0.0469694446142522\\
243	0.0469694446142522\\
244	0.0469694446142522\\
245	0.0469694446142522\\
246	0.0469694446142522\\
247	0.0469694446142522\\
248	0.0469694446142522\\
249	0.0469694446142522\\
250	0.0469694446142522\\
251	0.0469694446142522\\
252	0.0469694446142522\\
253	0.0469694446142522\\
254	0.0469694446142522\\
255	0.0469694446142522\\
256	0.0469694446142522\\
257	0.0469694446142522\\
258	0.0469694446142522\\
259	0.0469694446142522\\
260	0.0469694446142522\\
261	0.0469694446142522\\
262	0.0469694446142522\\
263	0.0469694446142522\\
264	0.0469694446142522\\
265	0.0469694446142522\\
266	0.0469694446142522\\
267	0.0469694446142522\\
268	0.0469694446142522\\
269	0.0469694446142522\\
270	0.0469694446142522\\
271	0.0469694446142522\\
272	0.0469694446142522\\
273	0.0469694446142522\\
274	0.0469694446142522\\
275	0.0469694446142522\\
276	0.0469694446142522\\
277	0.0469694446142522\\
278	0.0469694446142522\\
279	0.0469694446142522\\
280	0.0469694446142522\\
281	0.0469694446142522\\
282	0.0469694446142522\\
283	0.0469694446142522\\
284	0.0469694446142522\\
285	0.0469694446142522\\
286	0.0469694446142522\\
287	0.0469694446142522\\
288	0.0469694446142522\\
289	0.0469694446142522\\
290	0.0469694446142522\\
291	0.0469694446142522\\
292	0.0469694446142522\\
293	0.0469694446142522\\
294	0.0469694446142522\\
295	0.0469694446142522\\
296	0.0469694446142522\\
297	0.0469694446142522\\
298	0.0469694446142522\\
299	0.0469694446142522\\
300	0.0469694446142522\\
};
\addlegendentry{Skok U z 0.00 do 0.25}

\addplot[const plot, color=mycolor5] table[row sep=crcr] {%
1	0\\
2	0\\
3	0\\
4	0\\
5	0\\
6	0\\
7	0\\
8	0\\
9	0\\
10	0\\
11	0\\
12	0\\
13	0\\
14	0\\
15	0.0149374122807018\\
16	0.035320856777193\\
17	0.0488293794044997\\
18	0.0561147600020654\\
19	0.0596439430779413\\
20	0.0612356364241139\\
21	0.0619148017418959\\
22	0.062190952713034\\
23	0.0622981522140596\\
24	0.0623377779031105\\
25	0.0623516109313598\\
26	0.0623560896317763\\
27	0.0623573801404703\\
28	0.06235767363587\\
29	0.06235769714626\\
30	0.062357668800254\\
31	0.0623576429737223\\
32	0.0623576273557165\\
33	0.0623576192920019\\
34	0.062357615491721\\
35	0.0623576138125338\\
36	0.0623576131082947\\
37	0.062357612826517\\
38	0.0623576127189196\\
39	0.0623576126798786\\
40	0.0623576126665644\\
41	0.062357612662397\\
42	0.0623576126612666\\
43	0.0623576126610484\\
44	0.0623576126610581\\
45	0.0623576126611003\\
46	0.0623576126611317\\
47	0.0623576126611494\\
48	0.0623576126611583\\
49	0.0623576126611623\\
50	0.0623576126611641\\
51	0.0623576126611648\\
52	0.0623576126611651\\
53	0.0623576126611652\\
54	0.0623576126611652\\
55	0.0623576126611653\\
56	0.0623576126611652\\
57	0.0623576126611652\\
58	0.0623576126611652\\
59	0.0623576126611652\\
60	0.0623576126611652\\
61	0.0623576126611652\\
62	0.0623576126611652\\
63	0.0623576126611652\\
64	0.0623576126611652\\
65	0.0623576126611652\\
66	0.0623576126611652\\
67	0.0623576126611652\\
68	0.0623576126611652\\
69	0.0623576126611652\\
70	0.0623576126611652\\
71	0.0623576126611652\\
72	0.0623576126611652\\
73	0.0623576126611652\\
74	0.0623576126611652\\
75	0.0623576126611652\\
76	0.0623576126611652\\
77	0.0623576126611652\\
78	0.0623576126611652\\
79	0.0623576126611652\\
80	0.0623576126611652\\
81	0.0623576126611652\\
82	0.0623576126611652\\
83	0.0623576126611652\\
84	0.0623576126611652\\
85	0.0623576126611652\\
86	0.0623576126611652\\
87	0.0623576126611652\\
88	0.0623576126611652\\
89	0.0623576126611652\\
90	0.0623576126611652\\
91	0.0623576126611652\\
92	0.0623576126611652\\
93	0.0623576126611652\\
94	0.0623576126611652\\
95	0.0623576126611652\\
96	0.0623576126611652\\
97	0.0623576126611652\\
98	0.0623576126611652\\
99	0.0623576126611652\\
100	0.0623576126611652\\
101	0.0623576126611652\\
102	0.0623576126611652\\
103	0.0623576126611652\\
104	0.0623576126611652\\
105	0.0623576126611652\\
106	0.0623576126611652\\
107	0.0623576126611652\\
108	0.0623576126611652\\
109	0.0623576126611652\\
110	0.0623576126611652\\
111	0.0623576126611652\\
112	0.0623576126611652\\
113	0.0623576126611652\\
114	0.0623576126611652\\
115	0.0623576126611652\\
116	0.0623576126611652\\
117	0.0623576126611652\\
118	0.0623576126611652\\
119	0.0623576126611652\\
120	0.0623576126611652\\
121	0.0623576126611652\\
122	0.0623576126611652\\
123	0.0623576126611652\\
124	0.0623576126611652\\
125	0.0623576126611652\\
126	0.0623576126611652\\
127	0.0623576126611652\\
128	0.0623576126611652\\
129	0.0623576126611652\\
130	0.0623576126611652\\
131	0.0623576126611652\\
132	0.0623576126611652\\
133	0.0623576126611652\\
134	0.0623576126611652\\
135	0.0623576126611652\\
136	0.0623576126611652\\
137	0.0623576126611652\\
138	0.0623576126611652\\
139	0.0623576126611652\\
140	0.0623576126611652\\
141	0.0623576126611652\\
142	0.0623576126611652\\
143	0.0623576126611652\\
144	0.0623576126611652\\
145	0.0623576126611652\\
146	0.0623576126611652\\
147	0.0623576126611652\\
148	0.0623576126611652\\
149	0.0623576126611652\\
150	0.0623576126611652\\
151	0.0623576126611652\\
152	0.0623576126611652\\
153	0.0623576126611652\\
154	0.0623576126611652\\
155	0.0623576126611652\\
156	0.0623576126611652\\
157	0.0623576126611652\\
158	0.0623576126611652\\
159	0.0623576126611652\\
160	0.0623576126611652\\
161	0.0623576126611652\\
162	0.0623576126611652\\
163	0.0623576126611652\\
164	0.0623576126611652\\
165	0.0623576126611652\\
166	0.0623576126611652\\
167	0.0623576126611652\\
168	0.0623576126611652\\
169	0.0623576126611652\\
170	0.0623576126611652\\
171	0.0623576126611652\\
172	0.0623576126611652\\
173	0.0623576126611652\\
174	0.0623576126611652\\
175	0.0623576126611652\\
176	0.0623576126611652\\
177	0.0623576126611652\\
178	0.0623576126611652\\
179	0.0623576126611652\\
180	0.0623576126611652\\
181	0.0623576126611652\\
182	0.0623576126611652\\
183	0.0623576126611652\\
184	0.0623576126611652\\
185	0.0623576126611652\\
186	0.0623576126611652\\
187	0.0623576126611652\\
188	0.0623576126611652\\
189	0.0623576126611652\\
190	0.0623576126611652\\
191	0.0623576126611652\\
192	0.0623576126611652\\
193	0.0623576126611652\\
194	0.0623576126611652\\
195	0.0623576126611652\\
196	0.0623576126611652\\
197	0.0623576126611652\\
198	0.0623576126611652\\
199	0.0623576126611652\\
200	0.0623576126611652\\
201	0.0623576126611652\\
202	0.0623576126611652\\
203	0.0623576126611652\\
204	0.0623576126611652\\
205	0.0623576126611652\\
206	0.0623576126611652\\
207	0.0623576126611652\\
208	0.0623576126611652\\
209	0.0623576126611652\\
210	0.0623576126611652\\
211	0.0623576126611652\\
212	0.0623576126611652\\
213	0.0623576126611652\\
214	0.0623576126611652\\
215	0.0623576126611652\\
216	0.0623576126611652\\
217	0.0623576126611652\\
218	0.0623576126611652\\
219	0.0623576126611652\\
220	0.0623576126611652\\
221	0.0623576126611652\\
222	0.0623576126611652\\
223	0.0623576126611652\\
224	0.0623576126611652\\
225	0.0623576126611652\\
226	0.0623576126611652\\
227	0.0623576126611652\\
228	0.0623576126611652\\
229	0.0623576126611652\\
230	0.0623576126611652\\
231	0.0623576126611652\\
232	0.0623576126611652\\
233	0.0623576126611652\\
234	0.0623576126611652\\
235	0.0623576126611652\\
236	0.0623576126611652\\
237	0.0623576126611652\\
238	0.0623576126611652\\
239	0.0623576126611652\\
240	0.0623576126611652\\
241	0.0623576126611652\\
242	0.0623576126611652\\
243	0.0623576126611652\\
244	0.0623576126611652\\
245	0.0623576126611652\\
246	0.0623576126611652\\
247	0.0623576126611652\\
248	0.0623576126611652\\
249	0.0623576126611652\\
250	0.0623576126611652\\
251	0.0623576126611652\\
252	0.0623576126611652\\
253	0.0623576126611652\\
254	0.0623576126611652\\
255	0.0623576126611652\\
256	0.0623576126611652\\
257	0.0623576126611652\\
258	0.0623576126611652\\
259	0.0623576126611652\\
260	0.0623576126611652\\
261	0.0623576126611652\\
262	0.0623576126611652\\
263	0.0623576126611652\\
264	0.0623576126611652\\
265	0.0623576126611652\\
266	0.0623576126611652\\
267	0.0623576126611652\\
268	0.0623576126611652\\
269	0.0623576126611652\\
270	0.0623576126611652\\
271	0.0623576126611652\\
272	0.0623576126611652\\
273	0.0623576126611652\\
274	0.0623576126611652\\
275	0.0623576126611652\\
276	0.0623576126611652\\
277	0.0623576126611652\\
278	0.0623576126611652\\
279	0.0623576126611652\\
280	0.0623576126611652\\
281	0.0623576126611652\\
282	0.0623576126611652\\
283	0.0623576126611652\\
284	0.0623576126611652\\
285	0.0623576126611652\\
286	0.0623576126611652\\
287	0.0623576126611652\\
288	0.0623576126611652\\
289	0.0623576126611652\\
290	0.0623576126611652\\
291	0.0623576126611652\\
292	0.0623576126611652\\
293	0.0623576126611652\\
294	0.0623576126611652\\
295	0.0623576126611652\\
296	0.0623576126611652\\
297	0.0623576126611652\\
298	0.0623576126611652\\
299	0.0623576126611652\\
300	0.0623576126611652\\
};
\addlegendentry{Skok U z 0.00 do 0.50}

\addplot[const plot, color=mycolor6] table[row sep=crcr] {%
1	0\\
2	0\\
3	0\\
4	0\\
5	0\\
6	0\\
7	0\\
8	0\\
9	0\\
10	0\\
11	0\\
12	0\\
13	0\\
14	0\\
15	0.0199798481468771\\
16	0.0447848083375242\\
17	0.0601297465363197\\
18	0.0681153267352949\\
19	0.0719760077764885\\
20	0.0737738156815136\\
21	0.0745938026972925\\
22	0.0749633286294141\\
23	0.0751286671851614\\
24	0.0752023261210796\\
25	0.075235055052244\\
26	0.0752495740370943\\
27	0.0752560084486156\\
28	0.0752588582538279\\
29	0.0752601199569219\\
30	0.0752606784240591\\
31	0.075260925582425\\
32	0.075261034956452\\
33	0.0752610833546389\\
34	0.0752611047701843\\
35	0.0752611142460742\\
36	0.0752611184388843\\
37	0.0752611202940678\\
38	0.0752611211149227\\
39	0.0752611214781216\\
40	0.0752611216388238\\
41	0.0752611217099285\\
42	0.0752611217413897\\
43	0.0752611217553101\\
44	0.0752611217614694\\
45	0.0752611217641946\\
46	0.0752611217654004\\
47	0.075261121765934\\
48	0.07526112176617\\
49	0.0752611217662745\\
50	0.0752611217663207\\
51	0.0752611217663411\\
52	0.0752611217663502\\
53	0.0752611217663542\\
54	0.075261121766356\\
55	0.0752611217663567\\
56	0.0752611217663571\\
57	0.0752611217663572\\
58	0.0752611217663573\\
59	0.0752611217663573\\
60	0.0752611217663573\\
61	0.0752611217663573\\
62	0.0752611217663573\\
63	0.0752611217663573\\
64	0.0752611217663573\\
65	0.0752611217663573\\
66	0.0752611217663573\\
67	0.0752611217663573\\
68	0.0752611217663573\\
69	0.0752611217663573\\
70	0.0752611217663573\\
71	0.0752611217663573\\
72	0.0752611217663573\\
73	0.0752611217663573\\
74	0.0752611217663573\\
75	0.0752611217663573\\
76	0.0752611217663573\\
77	0.0752611217663573\\
78	0.0752611217663573\\
79	0.0752611217663573\\
80	0.0752611217663573\\
81	0.0752611217663573\\
82	0.0752611217663573\\
83	0.0752611217663573\\
84	0.0752611217663573\\
85	0.0752611217663573\\
86	0.0752611217663573\\
87	0.0752611217663573\\
88	0.0752611217663573\\
89	0.0752611217663573\\
90	0.0752611217663573\\
91	0.0752611217663573\\
92	0.0752611217663573\\
93	0.0752611217663573\\
94	0.0752611217663573\\
95	0.0752611217663573\\
96	0.0752611217663573\\
97	0.0752611217663573\\
98	0.0752611217663573\\
99	0.0752611217663573\\
100	0.0752611217663573\\
101	0.0752611217663573\\
102	0.0752611217663573\\
103	0.0752611217663573\\
104	0.0752611217663573\\
105	0.0752611217663573\\
106	0.0752611217663573\\
107	0.0752611217663573\\
108	0.0752611217663573\\
109	0.0752611217663573\\
110	0.0752611217663573\\
111	0.0752611217663573\\
112	0.0752611217663573\\
113	0.0752611217663573\\
114	0.0752611217663573\\
115	0.0752611217663573\\
116	0.0752611217663573\\
117	0.0752611217663573\\
118	0.0752611217663573\\
119	0.0752611217663573\\
120	0.0752611217663573\\
121	0.0752611217663573\\
122	0.0752611217663573\\
123	0.0752611217663573\\
124	0.0752611217663573\\
125	0.0752611217663573\\
126	0.0752611217663573\\
127	0.0752611217663573\\
128	0.0752611217663573\\
129	0.0752611217663573\\
130	0.0752611217663573\\
131	0.0752611217663573\\
132	0.0752611217663573\\
133	0.0752611217663573\\
134	0.0752611217663573\\
135	0.0752611217663573\\
136	0.0752611217663573\\
137	0.0752611217663573\\
138	0.0752611217663573\\
139	0.0752611217663573\\
140	0.0752611217663573\\
141	0.0752611217663573\\
142	0.0752611217663573\\
143	0.0752611217663573\\
144	0.0752611217663573\\
145	0.0752611217663573\\
146	0.0752611217663573\\
147	0.0752611217663573\\
148	0.0752611217663573\\
149	0.0752611217663573\\
150	0.0752611217663573\\
151	0.0752611217663573\\
152	0.0752611217663573\\
153	0.0752611217663573\\
154	0.0752611217663573\\
155	0.0752611217663573\\
156	0.0752611217663573\\
157	0.0752611217663573\\
158	0.0752611217663573\\
159	0.0752611217663573\\
160	0.0752611217663573\\
161	0.0752611217663573\\
162	0.0752611217663573\\
163	0.0752611217663573\\
164	0.0752611217663573\\
165	0.0752611217663573\\
166	0.0752611217663573\\
167	0.0752611217663573\\
168	0.0752611217663573\\
169	0.0752611217663573\\
170	0.0752611217663573\\
171	0.0752611217663573\\
172	0.0752611217663573\\
173	0.0752611217663573\\
174	0.0752611217663573\\
175	0.0752611217663573\\
176	0.0752611217663573\\
177	0.0752611217663573\\
178	0.0752611217663573\\
179	0.0752611217663573\\
180	0.0752611217663573\\
181	0.0752611217663573\\
182	0.0752611217663573\\
183	0.0752611217663573\\
184	0.0752611217663573\\
185	0.0752611217663573\\
186	0.0752611217663573\\
187	0.0752611217663573\\
188	0.0752611217663573\\
189	0.0752611217663573\\
190	0.0752611217663573\\
191	0.0752611217663573\\
192	0.0752611217663573\\
193	0.0752611217663573\\
194	0.0752611217663573\\
195	0.0752611217663573\\
196	0.0752611217663573\\
197	0.0752611217663573\\
198	0.0752611217663573\\
199	0.0752611217663573\\
200	0.0752611217663573\\
201	0.0752611217663573\\
202	0.0752611217663573\\
203	0.0752611217663573\\
204	0.0752611217663573\\
205	0.0752611217663573\\
206	0.0752611217663573\\
207	0.0752611217663573\\
208	0.0752611217663573\\
209	0.0752611217663573\\
210	0.0752611217663573\\
211	0.0752611217663573\\
212	0.0752611217663573\\
213	0.0752611217663573\\
214	0.0752611217663573\\
215	0.0752611217663573\\
216	0.0752611217663573\\
217	0.0752611217663573\\
218	0.0752611217663573\\
219	0.0752611217663573\\
220	0.0752611217663573\\
221	0.0752611217663573\\
222	0.0752611217663573\\
223	0.0752611217663573\\
224	0.0752611217663573\\
225	0.0752611217663573\\
226	0.0752611217663573\\
227	0.0752611217663573\\
228	0.0752611217663573\\
229	0.0752611217663573\\
230	0.0752611217663573\\
231	0.0752611217663573\\
232	0.0752611217663573\\
233	0.0752611217663573\\
234	0.0752611217663573\\
235	0.0752611217663573\\
236	0.0752611217663573\\
237	0.0752611217663573\\
238	0.0752611217663573\\
239	0.0752611217663573\\
240	0.0752611217663573\\
241	0.0752611217663573\\
242	0.0752611217663573\\
243	0.0752611217663573\\
244	0.0752611217663573\\
245	0.0752611217663573\\
246	0.0752611217663573\\
247	0.0752611217663573\\
248	0.0752611217663573\\
249	0.0752611217663573\\
250	0.0752611217663573\\
251	0.0752611217663573\\
252	0.0752611217663573\\
253	0.0752611217663573\\
254	0.0752611217663573\\
255	0.0752611217663573\\
256	0.0752611217663573\\
257	0.0752611217663573\\
258	0.0752611217663573\\
259	0.0752611217663573\\
260	0.0752611217663573\\
261	0.0752611217663573\\
262	0.0752611217663573\\
263	0.0752611217663573\\
264	0.0752611217663573\\
265	0.0752611217663573\\
266	0.0752611217663573\\
267	0.0752611217663573\\
268	0.0752611217663573\\
269	0.0752611217663573\\
270	0.0752611217663573\\
271	0.0752611217663573\\
272	0.0752611217663573\\
273	0.0752611217663573\\
274	0.0752611217663573\\
275	0.0752611217663573\\
276	0.0752611217663573\\
277	0.0752611217663573\\
278	0.0752611217663573\\
279	0.0752611217663573\\
280	0.0752611217663573\\
281	0.0752611217663573\\
282	0.0752611217663573\\
283	0.0752611217663573\\
284	0.0752611217663573\\
285	0.0752611217663573\\
286	0.0752611217663573\\
287	0.0752611217663573\\
288	0.0752611217663573\\
289	0.0752611217663573\\
290	0.0752611217663573\\
291	0.0752611217663573\\
292	0.0752611217663573\\
293	0.0752611217663573\\
294	0.0752611217663573\\
295	0.0752611217663573\\
296	0.0752611217663573\\
297	0.0752611217663573\\
298	0.0752611217663573\\
299	0.0752611217663573\\
300	0.0752611217663573\\
};
\addlegendentry{Skok U z 0.00 do 0.75}

\addplot[const plot, color=mycolor7] table[row sep=crcr] {%
1	0\\
2	0\\
3	0\\
4	0\\
5	0\\
6	0\\
7	0\\
8	0\\
9	0\\
10	0\\
11	0\\
12	0\\
13	0\\
14	0\\
15	0.0249067777777778\\
16	0.0540264030651852\\
17	0.0713864999938143\\
18	0.0802973721942573\\
19	0.0846280344990221\\
20	0.0866849533107856\\
21	0.0876520115004241\\
22	0.0881045698142786\\
23	0.0883159045205792\\
24	0.0884144960135448\\
25	0.0884604697743512\\
26	0.0884819030636113\\
27	0.0884918944337098\\
28	0.0884965518120525\\
29	0.0884987227571013\\
30	0.0884997346904105\\
31	0.0885002063764041\\
32	0.0885004262399126\\
33	0.0885005287231659\\
34	0.0885005764928595\\
35	0.0885005987593575\\
36	0.0885006091382576\\
37	0.0885006139760891\\
38	0.0885006162311077\\
39	0.0885006172822209\\
40	0.0885006177721677\\
41	0.0885006180005426\\
42	0.0885006181069931\\
43	0.0885006181566119\\
44	0.0885006181797404\\
45	0.088500618190521\\
46	0.0885006181955462\\
47	0.0885006181978885\\
48	0.0885006181989803\\
49	0.0885006181994892\\
50	0.0885006181997264\\
51	0.088500618199837\\
52	0.0885006181998885\\
53	0.0885006181999125\\
54	0.0885006181999237\\
55	0.088500618199929\\
56	0.0885006181999314\\
57	0.0885006181999325\\
58	0.0885006181999331\\
59	0.0885006181999333\\
60	0.0885006181999334\\
61	0.0885006181999334\\
62	0.0885006181999335\\
63	0.0885006181999335\\
64	0.0885006181999335\\
65	0.0885006181999335\\
66	0.0885006181999335\\
67	0.0885006181999335\\
68	0.0885006181999335\\
69	0.0885006181999335\\
70	0.0885006181999335\\
71	0.0885006181999335\\
72	0.0885006181999335\\
73	0.0885006181999335\\
74	0.0885006181999335\\
75	0.0885006181999335\\
76	0.0885006181999335\\
77	0.0885006181999335\\
78	0.0885006181999335\\
79	0.0885006181999335\\
80	0.0885006181999335\\
81	0.0885006181999335\\
82	0.0885006181999335\\
83	0.0885006181999335\\
84	0.0885006181999335\\
85	0.0885006181999335\\
86	0.0885006181999335\\
87	0.0885006181999335\\
88	0.0885006181999335\\
89	0.0885006181999335\\
90	0.0885006181999335\\
91	0.0885006181999335\\
92	0.0885006181999335\\
93	0.0885006181999335\\
94	0.0885006181999335\\
95	0.0885006181999335\\
96	0.0885006181999335\\
97	0.0885006181999335\\
98	0.0885006181999335\\
99	0.0885006181999335\\
100	0.0885006181999335\\
101	0.0885006181999335\\
102	0.0885006181999335\\
103	0.0885006181999335\\
104	0.0885006181999335\\
105	0.0885006181999335\\
106	0.0885006181999335\\
107	0.0885006181999335\\
108	0.0885006181999335\\
109	0.0885006181999335\\
110	0.0885006181999335\\
111	0.0885006181999335\\
112	0.0885006181999335\\
113	0.0885006181999335\\
114	0.0885006181999335\\
115	0.0885006181999335\\
116	0.0885006181999335\\
117	0.0885006181999335\\
118	0.0885006181999335\\
119	0.0885006181999335\\
120	0.0885006181999335\\
121	0.0885006181999335\\
122	0.0885006181999335\\
123	0.0885006181999335\\
124	0.0885006181999335\\
125	0.0885006181999335\\
126	0.0885006181999335\\
127	0.0885006181999335\\
128	0.0885006181999335\\
129	0.0885006181999335\\
130	0.0885006181999335\\
131	0.0885006181999335\\
132	0.0885006181999335\\
133	0.0885006181999335\\
134	0.0885006181999335\\
135	0.0885006181999335\\
136	0.0885006181999335\\
137	0.0885006181999335\\
138	0.0885006181999335\\
139	0.0885006181999335\\
140	0.0885006181999335\\
141	0.0885006181999335\\
142	0.0885006181999335\\
143	0.0885006181999335\\
144	0.0885006181999335\\
145	0.0885006181999335\\
146	0.0885006181999335\\
147	0.0885006181999335\\
148	0.0885006181999335\\
149	0.0885006181999335\\
150	0.0885006181999335\\
151	0.0885006181999335\\
152	0.0885006181999335\\
153	0.0885006181999335\\
154	0.0885006181999335\\
155	0.0885006181999335\\
156	0.0885006181999335\\
157	0.0885006181999335\\
158	0.0885006181999335\\
159	0.0885006181999335\\
160	0.0885006181999335\\
161	0.0885006181999335\\
162	0.0885006181999335\\
163	0.0885006181999335\\
164	0.0885006181999335\\
165	0.0885006181999335\\
166	0.0885006181999335\\
167	0.0885006181999335\\
168	0.0885006181999335\\
169	0.0885006181999335\\
170	0.0885006181999335\\
171	0.0885006181999335\\
172	0.0885006181999335\\
173	0.0885006181999335\\
174	0.0885006181999335\\
175	0.0885006181999335\\
176	0.0885006181999335\\
177	0.0885006181999335\\
178	0.0885006181999335\\
179	0.0885006181999335\\
180	0.0885006181999335\\
181	0.0885006181999335\\
182	0.0885006181999335\\
183	0.0885006181999335\\
184	0.0885006181999335\\
185	0.0885006181999335\\
186	0.0885006181999335\\
187	0.0885006181999335\\
188	0.0885006181999335\\
189	0.0885006181999335\\
190	0.0885006181999335\\
191	0.0885006181999335\\
192	0.0885006181999335\\
193	0.0885006181999335\\
194	0.0885006181999335\\
195	0.0885006181999335\\
196	0.0885006181999335\\
197	0.0885006181999335\\
198	0.0885006181999335\\
199	0.0885006181999335\\
200	0.0885006181999335\\
201	0.0885006181999335\\
202	0.0885006181999335\\
203	0.0885006181999335\\
204	0.0885006181999335\\
205	0.0885006181999335\\
206	0.0885006181999335\\
207	0.0885006181999335\\
208	0.0885006181999335\\
209	0.0885006181999335\\
210	0.0885006181999335\\
211	0.0885006181999335\\
212	0.0885006181999335\\
213	0.0885006181999335\\
214	0.0885006181999335\\
215	0.0885006181999335\\
216	0.0885006181999335\\
217	0.0885006181999335\\
218	0.0885006181999335\\
219	0.0885006181999335\\
220	0.0885006181999335\\
221	0.0885006181999335\\
222	0.0885006181999335\\
223	0.0885006181999335\\
224	0.0885006181999335\\
225	0.0885006181999335\\
226	0.0885006181999335\\
227	0.0885006181999335\\
228	0.0885006181999335\\
229	0.0885006181999335\\
230	0.0885006181999335\\
231	0.0885006181999335\\
232	0.0885006181999335\\
233	0.0885006181999335\\
234	0.0885006181999335\\
235	0.0885006181999335\\
236	0.0885006181999335\\
237	0.0885006181999335\\
238	0.0885006181999335\\
239	0.0885006181999335\\
240	0.0885006181999335\\
241	0.0885006181999335\\
242	0.0885006181999335\\
243	0.0885006181999335\\
244	0.0885006181999335\\
245	0.0885006181999335\\
246	0.0885006181999335\\
247	0.0885006181999335\\
248	0.0885006181999335\\
249	0.0885006181999335\\
250	0.0885006181999335\\
251	0.0885006181999335\\
252	0.0885006181999335\\
253	0.0885006181999335\\
254	0.0885006181999335\\
255	0.0885006181999335\\
256	0.0885006181999335\\
257	0.0885006181999335\\
258	0.0885006181999335\\
259	0.0885006181999335\\
260	0.0885006181999335\\
261	0.0885006181999335\\
262	0.0885006181999335\\
263	0.0885006181999335\\
264	0.0885006181999335\\
265	0.0885006181999335\\
266	0.0885006181999335\\
267	0.0885006181999335\\
268	0.0885006181999335\\
269	0.0885006181999335\\
270	0.0885006181999335\\
271	0.0885006181999335\\
272	0.0885006181999335\\
273	0.0885006181999335\\
274	0.0885006181999335\\
275	0.0885006181999335\\
276	0.0885006181999335\\
277	0.0885006181999335\\
278	0.0885006181999335\\
279	0.0885006181999335\\
280	0.0885006181999335\\
281	0.0885006181999335\\
282	0.0885006181999335\\
283	0.0885006181999335\\
284	0.0885006181999335\\
285	0.0885006181999335\\
286	0.0885006181999335\\
287	0.0885006181999335\\
288	0.0885006181999335\\
289	0.0885006181999335\\
290	0.0885006181999335\\
291	0.0885006181999335\\
292	0.0885006181999335\\
293	0.0885006181999335\\
294	0.0885006181999335\\
295	0.0885006181999335\\
296	0.0885006181999335\\
297	0.0885006181999335\\
298	0.0885006181999335\\
299	0.0885006181999335\\
300	0.0885006181999335\\
};
\addlegendentry{Skok U z 0.00 do 1.00}

\end{axis}
\end{tikzpicture}%
\caption{Wykresy odpowiedzi skokowych}
\end{figure}

Jak widać wartość skoku na wyjściu jest proporcjonalna wartości skoku wejścia.




\chapter{Odpowiedzi skokowe dla DMC}

\begin{figure}[H]
\centering
% This file was created by matlab2tikz.
%
%The latest updates can be retrieved from
%  http://www.mathworks.com/matlabcentral/fileexchange/22022-matlab2tikz-matlab2tikz
%where you can also make suggestions and rate matlab2tikz.
%
\definecolor{mycolor1}{rgb}{0.00000,0.44700,0.74100}%
%
\begin{tikzpicture}

\begin{axis}[%
width=4.521in,
height=3.566in,
at={(0.758in,0.481in)},
scale only axis,
xmin=0,
xmax=250,
xlabel style={font=\color{white!15!black}},
xlabel={k},
ymin=0,
ymax=25,
ylabel style={font=\color{white!15!black}},
ylabel={Y(k)},
axis background/.style={fill=white}
]
\addplot[const plot, color=mycolor1, forget plot] table[row sep=crcr] {%
1	0.412834169999999\\
2	0.778953825762\\
3	1.22552522529589\\
4	1.73632424400836\\
5	2.2973725681653\\
6	2.89666065333908\\
7	3.67178323313577\\
8	4.45732700941061\\
9	5.24635122393158\\
10	6.03303546347107\\
11	6.81253378871436\\
12	7.58084641517885\\
13	8.33470690777239\\
14	9.07148308327445\\
15	9.78909002135876\\
16	10.4859137677517\\
17	11.1607444753708\\
18	11.8127178731484\\
19	12.4412640797858\\
20	13.046062892749\\
21	13.6270047830398\\
22	14.184156915098\\
23	14.7177335899142\\
24	15.2280705791815\\
25	15.7156028801154\\
26	16.1808454753177\\
27	16.6243767305496\\
28	17.0468241062239\\
29	17.4488518964494\\
30	17.8311507431239\\
31	18.194428702367\\
32	18.5394036669541\\
33	18.8667969717402\\
34	19.1773280297007\\
35	19.4717098644647\\
36	19.7506454213526\\
37	20.0148245531913\\
38	20.2649215897808\\
39	20.5015934110202\\
40	20.7254779535271\\
41	20.937193089265\\
42	21.1373358223465\\
43	21.3264817569342\\
44	21.505184795115\\
45	21.673977028868\\
46	21.8333687948682\\
47	21.983848864934\\
48	22.1258847485047\\
49	22.2599230866795\\
50	22.3863901201136\\
51	22.5056922154907\\
52	22.6182164374214\\
53	22.7243311544802\\
54	22.8243866697262\\
55	22.9187158674818\\
56	23.0076348693884\\
57	23.0914436938494\\
58	23.1704269139164\\
59	23.2448543094999\\
60	23.3149815105027\\
61	23.3810506280955\\
62	23.4432908718881\\
63	23.5019191512178\\
64	23.5571406591723\\
65	23.6091494383078\\
66	23.658128927316\\
67	23.7042524881435\\
68	23.7476839132784\\
69	23.7885779130991\\
70	23.8270805833273\\
71	23.8633298527562\\
72	23.8974559115258\\
73	23.9295816203016\\
74	23.9598229007815\\
75	23.9882891080122\\
76	24.0150833850333\\
77	24.0403030004047\\
78	24.064039669191\\
79	24.0863798579946\\
80	24.1074050746372\\
81	24.1271921430916\\
82	24.1458134642665\\
83	24.1633372632405\\
84	24.179827823533\\
85	24.1953457089908\\
86	24.2099479738547\\
87	24.2236883615561\\
88	24.2366174927793\\
89	24.2487830433066\\
90	24.2602299121465\\
91	24.2710003804292\\
92	24.2811342615325\\
93	24.2906690428849\\
94	24.2996400198743\\
95	24.3080804222725\\
96	24.3160215335675\\
97	24.3234928035784\\
98	24.3305219547112\\
99	24.3371350821961\\
100	24.3433567486312\\
101	24.3492100731427\\
102	24.3547168154546\\
103	24.3598974551478\\
104	24.3647712663749\\
105	24.3693563882808\\
106	24.3736698913707\\
107	24.377727840049\\
108	24.3815453515461\\
109	24.3851366514363\\
110	24.3885151259376\\
111	24.3916933711773\\
112	24.3946832395954\\
113	24.3974958836484\\
114	24.4001417969673\\
115	24.4026308531171\\
116	24.4049723420931\\
117	24.4071750046853\\
118	24.4092470648343\\
119	24.4111962600926\\
120	24.4130298703024\\
121	24.4147547445927\\
122	24.4163773267922\\
123	24.4179036793511\\
124	24.4193395058582\\
125	24.4206901722344\\
126	24.4219607266797\\
127	24.4231559184477\\
128	24.4242802155143\\
129	24.4253378212059\\
130	24.4263326898489\\
131	24.4272685414954\\
132	24.4281488757823\\
133	24.4289769849714\\
134	24.4297559662215\\
135	24.4304887331354\\
136	24.4311780266248\\
137	24.4318264251348\\
138	24.4324363542631\\
139	24.4330100958117\\
140	24.433549796303\\
141	24.434057474993\\
142	24.4345350314101\\
143	24.4349842524485\\
144	24.4354068190422\\
145	24.4358043124429\\
146	24.436178220128\\
147	24.4365299413583\\
148	24.4368607924057\\
149	24.4371720114737\\
150	24.4374647633239\\
151	24.4377401436306\\
152	24.4379991830767\\
153	24.4382428512066\\
154	24.4384720600515\\
155	24.43868766754\\
156	24.4388904807055\\
157	24.4390812587054\\
158	24.4392607156593\\
159	24.439429523321\\
160	24.4395883135904\\
161	24.4397376808774\\
162	24.4398781843261\\
163	24.4400103499059\\
164	24.4401346723795\\
165	24.4402516171543\\
166	24.4403616220233\\
167	24.4404650988038\\
168	24.4405624348776\\
169	24.4406539946413\\
170	24.4407401208696\\
171	24.440821135998\\
172	24.4408973433296\\
173	24.4409690281702\\
174	24.4410364588959\\
175	24.441099887958\\
176	24.4411595528271\\
177	24.4412156768828\\
178	24.441268470249\\
179	24.441318130581\\
180	24.4413648438049\\
181	24.4414087848139\\
182	24.4414501181226\\
183	24.4414889984832\\
184	24.4415255714645\\
185	24.4415599739969\\
186	24.441592334885\\
187	24.4416227752898\\
188	24.4416514091822\\
189	24.4416783437698\\
190	24.441703679898\\
191	24.4417275124277\\
192	24.4417499305906\\
193	24.4417710183227\\
194	24.441790854579\\
195	24.4418095136292\\
196	24.4418270653353\\
197	24.4418435754133\\
198	24.4418591056794\\
199	24.4418737142815\\
200	24.4418874559164\\
201	24.4419003820351\\
202	24.4419125410352\\
203	24.4419239784421\\
204	24.4419347370796\\
205	24.4419448572301\\
206	24.4419543767851\\
207	24.4419633313879\\
208	24.4419717545661\\
209	24.4419796778578\\
210	24.4419871309295\\
211	24.4419941416869\\
212	24.4420007363798\\
213	24.4420069397001\\
214	24.4420127748745\\
215	24.4420182637509\\
216	24.442023426881\\
217	24.4420282835966\\
218	24.4420328520824\\
219	24.4420371494439\\
220	24.4420411917712\\
221	24.4420449941998\\
222	24.4420485709669\\
223	24.4420519354647\\
224	24.4420551002906\\
225	24.4420580772946\\
226	24.4420608776231\\
227	24.4420635117613\\
228	24.442065989572\\
229	24.4420683203328\\
230	24.4420705127705\\
231	24.4420725750941\\
232	24.4420745150256\\
233	24.4420763398285\\
234	24.4420780563352\\
235	24.4420796709728\\
236	24.4420811897869\\
237	24.4420826184643\\
238	24.4420839623543\\
239	24.4420852264887\\
240	24.4420864156009\\
241	24.4420875341431\\
242	24.4420885863034\\
243	24.4420895760214\\
244	24.4420905070028\\
245	24.4420913827335\\
246	24.4420922064923\\
247	24.4420929813637\\
248	24.442093710249\\
249	24.4420943958773\\
250	24.4420950408157\\
};
\end{axis}
\end{tikzpicture}%
\caption{}
\end{figure}

%\begin{figure}[H]
%\centering
%% This file was created by matlab2tikz.
%
%The latest updates can be retrieved from
%  http://www.mathworks.com/matlabcentral/fileexchange/22022-matlab2tikz-matlab2tikz
%where you can also make suggestions and rate matlab2tikz.
%
\definecolor{mycolor1}{rgb}{0.00000,0.44700,0.74100}%
\definecolor{mycolor2}{rgb}{0.85000,0.32500,0.09800}%
%
\begin{tikzpicture}

\begin{axis}[%
width=4.521in,
height=1.493in,
at={(0.758in,2.554in)},
scale only axis,
xmin=0,
xmax=300,
xlabel style={font=\color{white!15!black}},
xlabel={k},
ymin=0,
ymax=2,
axis background/.style={fill=white},
title style={font=\bfseries},
title={Warto�� sterowania},
legend style={legend cell align=left, align=left, draw=white!15!black}
]
\addplot[const plot, color=mycolor1] table[row sep=crcr] {%
1	0\\
2	0\\
3	0\\
4	0\\
5	0\\
6	0\\
7	0\\
8	0\\
9	0\\
10	0\\
11	0\\
12	0\\
13	0\\
14	0\\
15	0\\
16	0\\
17	0\\
18	0\\
19	0\\
20	0\\
21	0\\
22	0\\
23	0\\
24	0\\
25	0\\
26	0\\
27	0\\
28	0\\
29	0\\
30	0\\
31	0\\
32	0\\
33	0\\
34	0\\
35	0\\
36	0\\
37	0\\
38	0\\
39	0\\
40	0\\
41	0\\
42	0\\
43	0\\
44	0\\
45	0\\
46	0\\
47	0\\
48	0\\
49	0\\
50	0\\
51	0\\
52	0\\
53	0\\
54	0\\
55	0\\
56	0\\
57	0\\
58	0\\
59	0\\
60	0\\
61	0\\
62	0\\
63	0\\
64	0\\
65	0\\
66	0\\
67	0\\
68	0\\
69	0\\
70	0\\
71	0\\
72	0\\
73	0\\
74	0\\
75	0\\
76	0\\
77	0\\
78	0\\
79	0\\
80	0\\
81	0\\
82	0\\
83	0\\
84	0\\
85	0\\
86	0\\
87	0\\
88	0\\
89	0\\
90	0\\
91	0\\
92	0\\
93	0\\
94	0\\
95	0\\
96	0\\
97	0\\
98	0\\
99	0\\
100	0\\
101	0\\
102	0\\
103	0\\
104	0\\
105	0\\
106	0\\
107	0\\
108	0\\
109	0\\
110	0\\
111	0\\
112	0\\
113	0\\
114	0\\
115	0\\
116	0\\
117	0\\
118	0\\
119	0\\
120	0\\
121	0\\
122	0\\
123	0\\
124	0\\
125	0\\
126	0\\
127	0\\
128	0\\
129	0\\
130	0\\
131	0\\
132	0\\
133	0\\
134	0\\
135	0\\
136	0\\
137	0\\
138	0\\
139	0\\
140	0\\
141	0\\
142	0\\
143	0\\
144	0\\
145	0\\
146	0\\
147	0\\
148	0\\
149	0\\
150	0\\
151	0\\
152	0\\
153	0\\
154	0\\
155	0\\
156	0\\
157	0\\
158	0\\
159	0\\
160	0\\
161	0\\
162	0\\
163	0\\
164	0\\
165	0\\
166	0\\
167	0\\
168	0\\
169	0\\
170	0\\
171	0\\
172	0\\
173	0\\
174	0\\
175	0\\
176	0\\
177	0\\
178	0\\
179	0\\
180	0\\
181	0\\
182	0\\
183	0\\
184	0\\
185	0\\
186	0\\
187	0\\
188	0\\
189	0\\
190	0\\
191	0\\
192	0\\
193	0\\
194	0\\
195	0\\
196	0\\
197	0\\
198	0\\
199	0\\
200	0\\
201	0\\
202	0\\
203	0\\
204	0\\
205	0\\
206	0\\
207	0\\
208	0\\
209	0\\
210	0\\
211	0\\
212	0\\
213	0\\
214	0\\
215	0\\
216	0\\
217	0\\
218	0\\
219	0\\
220	0\\
221	0\\
222	0\\
223	0\\
224	0\\
225	0\\
226	0\\
227	0\\
228	0\\
229	0\\
230	0\\
231	0\\
232	0\\
233	0\\
234	0\\
235	0\\
236	0\\
237	0\\
238	0\\
239	0\\
240	0\\
241	0\\
242	0\\
243	0\\
244	0\\
245	0\\
246	0\\
247	0\\
248	0\\
249	0\\
250	0\\
251	0\\
252	0\\
253	0\\
254	0\\
255	0\\
256	0\\
257	0\\
258	0\\
259	0\\
260	0\\
261	0\\
262	0\\
263	0\\
264	0\\
265	0\\
266	0\\
267	0\\
268	0\\
269	0\\
270	0\\
271	0\\
272	0\\
273	0\\
274	0\\
275	0\\
276	0\\
277	0\\
278	0\\
279	0\\
280	0\\
281	0\\
282	0\\
283	0\\
284	0\\
285	0\\
286	0\\
287	0\\
288	0\\
289	0\\
290	0\\
291	0\\
292	0\\
293	0\\
294	0\\
295	0\\
296	0\\
297	0\\
298	0\\
299	0\\
300	0\\
};
\addlegendentry{U(k)}

\addplot[const plot, color=mycolor2] table[row sep=crcr] {%
1	0\\
2	0\\
3	0\\
4	0\\
5	0\\
6	0\\
7	0\\
8	0\\
9	0\\
10	1\\
11	1\\
12	1\\
13	1\\
14	1\\
15	1\\
16	1\\
17	1\\
18	1\\
19	1\\
20	1\\
21	1\\
22	1\\
23	1\\
24	1\\
25	1\\
26	1\\
27	1\\
28	1\\
29	1\\
30	1\\
31	1\\
32	1\\
33	1\\
34	1\\
35	1\\
36	1\\
37	1\\
38	1\\
39	1\\
40	1\\
41	1\\
42	1\\
43	1\\
44	1\\
45	1\\
46	1\\
47	1\\
48	1\\
49	1\\
50	1\\
51	1\\
52	1\\
53	1\\
54	1\\
55	1\\
56	1\\
57	1\\
58	1\\
59	1\\
60	1\\
61	1\\
62	1\\
63	1\\
64	1\\
65	1\\
66	1\\
67	1\\
68	1\\
69	1\\
70	1\\
71	1\\
72	1\\
73	1\\
74	1\\
75	1\\
76	1\\
77	1\\
78	1\\
79	1\\
80	1\\
81	1\\
82	1\\
83	1\\
84	1\\
85	1\\
86	1\\
87	1\\
88	1\\
89	1\\
90	1\\
91	1\\
92	1\\
93	1\\
94	1\\
95	1\\
96	1\\
97	1\\
98	1\\
99	1\\
100	1\\
101	1\\
102	1\\
103	1\\
104	1\\
105	1\\
106	1\\
107	1\\
108	1\\
109	1\\
110	1\\
111	1\\
112	1\\
113	1\\
114	1\\
115	1\\
116	1\\
117	1\\
118	1\\
119	1\\
120	1\\
121	1\\
122	1\\
123	1\\
124	1\\
125	1\\
126	1\\
127	1\\
128	1\\
129	1\\
130	1\\
131	1\\
132	1\\
133	1\\
134	1\\
135	1\\
136	1\\
137	1\\
138	1\\
139	1\\
140	1\\
141	1\\
142	1\\
143	1\\
144	1\\
145	1\\
146	1\\
147	1\\
148	1\\
149	1\\
150	1\\
151	1\\
152	1\\
153	1\\
154	1\\
155	1\\
156	1\\
157	1\\
158	1\\
159	1\\
160	1\\
161	1\\
162	1\\
163	1\\
164	1\\
165	1\\
166	1\\
167	1\\
168	1\\
169	1\\
170	1\\
171	1\\
172	1\\
173	1\\
174	1\\
175	1\\
176	1\\
177	1\\
178	1\\
179	1\\
180	1\\
181	1\\
182	1\\
183	1\\
184	1\\
185	1\\
186	1\\
187	1\\
188	1\\
189	1\\
190	1\\
191	1\\
192	1\\
193	1\\
194	1\\
195	1\\
196	1\\
197	1\\
198	1\\
199	1\\
200	1\\
201	1\\
202	1\\
203	1\\
204	1\\
205	1\\
206	1\\
207	1\\
208	1\\
209	1\\
210	1\\
211	1\\
212	1\\
213	1\\
214	1\\
215	1\\
216	1\\
217	1\\
218	1\\
219	1\\
220	1\\
221	1\\
222	1\\
223	1\\
224	1\\
225	1\\
226	1\\
227	1\\
228	1\\
229	1\\
230	1\\
231	1\\
232	1\\
233	1\\
234	1\\
235	1\\
236	1\\
237	1\\
238	1\\
239	1\\
240	1\\
241	1\\
242	1\\
243	1\\
244	1\\
245	1\\
246	1\\
247	1\\
248	1\\
249	1\\
250	1\\
251	1\\
252	1\\
253	1\\
254	1\\
255	1\\
256	1\\
257	1\\
258	1\\
259	1\\
260	1\\
261	1\\
262	1\\
263	1\\
264	1\\
265	1\\
266	1\\
267	1\\
268	1\\
269	1\\
270	1\\
271	1\\
272	1\\
273	1\\
274	1\\
275	1\\
276	1\\
277	1\\
278	1\\
279	1\\
280	1\\
281	1\\
282	1\\
283	1\\
284	1\\
285	1\\
286	1\\
287	1\\
288	1\\
289	1\\
290	1\\
291	1\\
292	1\\
293	1\\
294	1\\
295	1\\
296	1\\
297	1\\
298	1\\
299	1\\
300	1\\
};
\addlegendentry{Z(k)}

\end{axis}

\begin{axis}[%
width=4.521in,
height=1.493in,
at={(0.758in,0.481in)},
scale only axis,
xmin=0,
xmax=300,
xlabel style={font=\color{white!15!black}},
xlabel={k},
ymin=0,
ymax=2,
ylabel style={font=\color{white!15!black}},
ylabel={Y(k)},
axis background/.style={fill=white},
title style={font=\bfseries},
title={Odpowied� skokowa}
]
\addplot[const plot, color=mycolor1, forget plot] table[row sep=crcr] {%
1	0\\
2	0\\
3	0\\
4	0\\
5	0\\
6	0\\
7	0\\
8	0\\
9	0\\
10	0\\
11	0\\
12	0\\
13	0.20202\\
14	0.381363772\\
15	0.5405745884792\\
16	0.681910601930065\\
17	0.807376801662065\\
18	0.918753389257166\\
19	1.01762097400741\\
20	1.10538294469291\\
21	1.18328533412576\\
22	1.25243445742761\\
23	1.31381257352359\\
24	1.36829179137883\\
25	1.41664641767974\\
26	1.45956392062001\\
27	1.49765466487875\\
28	1.53146055549941\\
29	1.56146271294626\\
30	1.58808828791264\\
31	1.61171651228869\\
32	1.63268407189206\\
33	1.6512898769727\\
34	1.66779929798444\\
35	1.68244792655289\\
36	1.69544491485312\\
37	1.70697594064711\\
38	1.71720583993623\\
39	1.7262809444819\\
40	1.73433115727287\\
41	1.74147179531066\\
42	1.7478052257929\\
43	1.75342231885168\\
44	1.75840373740865\\
45	1.7628210824046\\
46	1.76673790961448\\
47	1.77021063244259\\
48	1.77328932347903\\
49	1.77601842616608\\
50	1.77843738665143\\
51	1.78058121477559\\
52	1.78248098213803\\
53	1.7841642642963\\
54	1.78565553336142\\
55	1.78697650655121\\
56	1.78814645563942\\
57	1.78918248168534\\
58	1.79009975893699\\
59	1.7909117513645\\
60	1.791630404893\\
61	1.79226631806014\\
62	1.7928288935179\\
63	1.7933264725271\\
64	1.79376645435222\\
65	1.79415540225022\\
66	1.79449913755705\\
67	1.79480282320733\\
68	1.7950710378724\\
69	1.79530784176955\\
70	1.79551683507687\\
71	1.79570120978355\\
72	1.79586379571225\\
73	1.79600710136771\\
74	1.79613335019233\\
75	1.79624451274432\\
76	1.79634233525612\\
77	1.79642836497965\\
78	1.79650397267911\\
79	1.79657037259177\\
80	1.79662864014108\\
81	1.7966797276547\\
82	1.79672447831156\\
83	1.79676363851692\\
84	1.79679786888231\\
85	1.79682775396694\\
86	1.79685381091999\\
87	1.79687649714742\\
88	1.79689621711284\\
89	1.79691332837016\\
90	1.79692814691417\\
91	1.79694095192608\\
92	1.79695198998197\\
93	1.79696147878477\\
94	1.79696961047339\\
95	1.79697655455666\\
96	1.79698246051441\\
97	1.79698746010321\\
98	1.79699166940006\\
99	1.79699519061374\\
100	1.79699811368989\\
101	1.79700051773321\\
102	1.79700247226746\\
103	1.79700403835162\\
104	1.79700526956837\\
105	1.79700621289954\\
106	1.79700690950111\\
107	1.79700739538939\\
108	1.79700770204824\\
109	1.79700785696642\\
110	1.797007884113\\
111	1.79700780435784\\
112	1.79700763584335\\
113	1.79700739431327\\
114	1.79700709340309\\
115	1.79700674489673\\
116	1.79700635895323\\
117	1.79700594430688\\
118	1.79700550844388\\
119	1.79700505775816\\
120	1.79700459768881\\
121	1.79700413284126\\
122	1.79700366709395\\
123	1.79700320369234\\
124	1.79700274533158\\
125	1.79700229422931\\
126	1.79700185218951\\
127	1.7970014206587\\
128	1.79700100077522\\
129	1.79700059341239\\
130	1.79700019921639\\
131	1.79699981863939\\
132	1.7969994519685\\
133	1.79699909935105\\
134	1.79699876081664\\
135	1.79699843629634\\
136	1.79699812563939\\
137	1.79699782862767\\
138	1.79699754498826\\
139	1.79699727440427\\
140	1.7969970165242\\
141	1.79699677096999\\
142	1.79699653734388\\
143	1.79699631523434\\
144	1.79699610422104\\
145	1.79699590387911\\
146	1.79699571378272\\
147	1.79699553350805\\
148	1.7969953626358\\
149	1.79699520075323\\
150	1.7969950474558\\
151	1.79699490234855\\
152	1.79699476504709\\
153	1.79699463517849\\
154	1.7969945123818\\
155	1.79699439630853\\
156	1.79699428662288\\
157	1.7969941830019\\
158	1.79699408513551\\
159	1.79699399272646\\
160	1.79699390549016\\
161	1.79699382315454\\
162	1.79699374545974\\
163	1.79699367215784\\
164	1.79699360301259\\
165	1.79699353779898\\
166	1.79699347630295\\
167	1.79699341832097\\
168	1.79699336365969\\
169	1.79699331213552\\
170	1.79699326357426\\
171	1.79699321781069\\
172	1.79699317468824\\
173	1.79699313405853\\
174	1.79699309578107\\
175	1.79699305972288\\
176	1.79699302575812\\
177	1.79699299376777\\
178	1.79699296363927\\
179	1.79699293526624\\
180	1.79699290854814\\
181	1.79699288339001\\
182	1.79699285970213\\
183	1.79699283739981\\
184	1.7969928164031\\
185	1.79699279663651\\
186	1.79699277802883\\
187	1.79699276051287\\
188	1.79699274402526\\
189	1.79699272850622\\
190	1.79699271389937\\
191	1.79699270015157\\
192	1.79699268721271\\
193	1.79699267503556\\
194	1.79699266357561\\
195	1.79699265279091\\
196	1.79699264264193\\
197	1.79699263309143\\
198	1.79699262410431\\
199	1.79699261564753\\
200	1.79699260768995\\
201	1.79699260020224\\
202	1.79699259315678\\
203	1.79699258652757\\
204	1.79699258029012\\
205	1.79699257442135\\
206	1.79699256889957\\
207	1.79699256370431\\
208	1.79699255881634\\
209	1.79699255421754\\
210	1.79699254989084\\
211	1.79699254582019\\
212	1.79699254199048\\
213	1.79699253838747\\
214	1.79699253499778\\
215	1.79699253180881\\
216	1.7969925288087\\
217	1.79699252598627\\
218	1.79699252333103\\
219	1.79699252083309\\
220	1.79699251848315\\
221	1.79699251627244\\
222	1.79699251419274\\
223	1.79699251223629\\
224	1.79699251039579\\
225	1.79699250866438\\
226	1.7969925070356\\
227	1.79699250550338\\
228	1.796992504062\\
229	1.79699250270607\\
230	1.79699250143053\\
231	1.79699250023063\\
232	1.79699249910187\\
233	1.79699249804005\\
234	1.7969924970412\\
235	1.79699249610159\\
236	1.7969924952177\\
237	1.79699249438624\\
238	1.79699249360409\\
239	1.79699249286833\\
240	1.79699249217621\\
241	1.79699249152515\\
242	1.79699249091271\\
243	1.7969924903366\\
244	1.79699248979466\\
245	1.79699248928488\\
246	1.79699248880533\\
247	1.79699248835424\\
248	1.79699248792991\\
249	1.79699248753075\\
250	1.79699248715528\\
251	1.79699248680208\\
252	1.79699248646984\\
253	1.79699248615731\\
254	1.79699248586332\\
255	1.79699248558678\\
256	1.79699248532664\\
257	1.79699248508194\\
258	1.79699248485176\\
259	1.79699248463524\\
260	1.79699248443157\\
261	1.79699248423998\\
262	1.79699248405976\\
263	1.79699248389023\\
264	1.79699248373076\\
265	1.79699248358076\\
266	1.79699248343965\\
267	1.79699248330692\\
268	1.79699248318207\\
269	1.79699248306463\\
270	1.79699248295415\\
271	1.79699248285023\\
272	1.79699248275248\\
273	1.79699248266053\\
274	1.79699248257403\\
275	1.79699248249267\\
276	1.79699248241613\\
277	1.79699248234414\\
278	1.79699248227642\\
279	1.79699248221272\\
280	1.7969924821528\\
281	1.79699248209644\\
282	1.79699248204342\\
283	1.79699248199354\\
284	1.79699248194663\\
285	1.7969924819025\\
286	1.79699248186099\\
287	1.79699248182194\\
288	1.79699248178521\\
289	1.79699248175066\\
290	1.79699248171816\\
291	1.79699248168759\\
292	1.79699248165883\\
293	1.79699248163178\\
294	1.79699248160634\\
295	1.7969924815824\\
296	1.79699248155989\\
297	1.79699248153871\\
298	1.79699248151878\\
299	1.79699248150005\\
300	1.79699248148242\\
};
\end{axis}
\end{tikzpicture}%
%\caption{}
%\end{figure}

\begin{figure}[H]
\centering
% This file was created by matlab2tikz.
%
%The latest updates can be retrieved from
%  http://www.mathworks.com/matlabcentral/fileexchange/22022-matlab2tikz-matlab2tikz
%where you can also make suggestions and rate matlab2tikz.
%
\definecolor{mycolor1}{rgb}{0.00000,0.44700,0.74100}%
%
\begin{tikzpicture}

\begin{axis}[%
width=4.521in,
height=3.566in,
at={(0.758in,0.481in)},
scale only axis,
xmin=0,
xmax=300,
xlabel style={font=\color{white!15!black}},
xlabel={k},
ymin=0,
ymax=2.5,
ylabel style={font=\color{white!15!black}},
ylabel={Y(k)},
axis background/.style={fill=white}
]
\addplot[const plot, color=mycolor1, forget plot] table[row sep=crcr] {%
1	0\\
2	0\\
3	0\\
4	0\\
5	0\\
6	0\\
7	0.14788\\
8	0.287003368\\
9	0.4178875887248\\
10	0.541019681620169\\
11	0.656857981328252\\
12	0.765833815583596\\
13	0.868353086271741\\
14	0.964797759090307\\
15	1.05552726696406\\
16	1.14087983209031\\
17	1.2211737112286\\
18	1.29670836859919\\
19	1.36776558051679\\
20	1.43461047566053\\
21	1.4974925146662\\
22	1.55664641252315\\
23	1.61229300706489\\
24	1.6646400766589\\
25	1.71388311002751\\
26	1.76020603096696\\
27	1.8037818805757\\
28	1.84477345945558\\
29	1.88333393220961\\
30	1.91960739642823\\
31	1.95372941823058\\
32	1.98582753630976\\
33	2.01602173631893\\
34	2.04442489733007\\
35	2.07114321199739\\
36	2.09627658196347\\
37	2.11991898995759\\
38	2.14215884995146\\
39	2.16307933665916\\
40	2.18275869559299\\
41	2.20127053481677\\
42	2.2186840994719\\
43	2.23506453008862\\
44	2.25047310563613\\
45	2.26496747220972\\
46	2.27860185820011\\
47	2.29142727674155\\
48	2.30349171618796\\
49	2.31484031932293\\
50	2.32551555196791\\
51	2.33555736161404\\
52	2.34500332666624\\
53	2.35388879685375\\
54	2.3622470253288\\
55	2.37010929294415\\
56	2.37750502517179\\
57	2.38446190209753\\
58	2.39100596190081\\
59	2.39716169820485\\
60	2.40295215165975\\
61	2.40839899609941\\
62	2.41352261959357\\
63	2.41834220069678\\
64	2.42287578017859\\
65	2.42714032850259\\
66	2.43115180930567\\
67	2.43492523911456\\
68	2.43847474352232\\
69	2.44181361003448\\
70	2.44495433778203\\
71	2.44790868428693\\
72	2.45068770945463\\
73	2.45330181695808\\
74	2.45576079316756\\
75	2.45807384377197\\
76	2.46024962822822\\
77	2.46229629216758\\
78	2.46422149788002\\
79	2.46603245299044\\
80	2.46773593743402\\
81	2.46933832883159\\
82	2.47084562635968\\
83	2.47226347320473\\
84	2.47359717768524\\
85	2.47485173312098\\
86	2.47603183652342\\
87	2.47714190617744\\
88	2.47818609817987\\
89	2.47916832199701\\
90	2.48009225509899\\
91	2.48096135672601\\
92	2.48177888083776\\
93	2.48254788829452\\
94	2.48327125831562\\
95	2.48395169925794\\
96	2.48459175875496\\
97	2.48519383325411\\
98	2.4857601769884\\
99	2.48629291041548\\
100	2.48679402815619\\
101	2.48726540646188\\
102	2.48770881023874\\
103	2.48812589965527\\
104	2.48851823635772\\
105	2.48888728931666\\
106	2.48923444032666\\
107	2.48956098917959\\
108	2.48986815853095\\
109	2.49015709847745\\
110	2.49042889086289\\
111	2.49068455332858\\
112	2.49092504312342\\
113	2.49115126068779\\
114	2.49136405302485\\
115	2.49156421687177\\
116	2.49175250168271\\
117	2.49192961243486\\
118	2.4920962122679\\
119	2.49225292496687\\
120	2.49240033729765\\
121	2.4925390012039\\
122	2.49266943587359\\
123	2.4927921296829\\
124	2.49290754202472\\
125	2.49301610502877\\
126	2.49311822517946\\
127	2.49321428483788\\
128	2.4933046436734\\
129	2.49338964001034\\
130	2.4934695920947\\
131	2.49354479928574\\
132	2.4936155431768\\
133	2.49368208864968\\
134	2.49374468486638\\
135	2.49380356620202\\
136	2.49385895312237\\
137	2.49391105300933\\
138	2.4939600609374\\
139	2.49400616040409\\
140	2.49404952401695\\
141	2.49409031413988\\
142	2.49412868350103\\
143	2.49416477576464\\
144	2.49419872606899\\
145	2.49423066153234\\
146	2.49426070172891\\
147	2.4942889591366\\
148	2.49431553955811\\
149	2.49434054251712\\
150	2.49436406163089\\
151	2.49438618496083\\
152	2.49440699534216\\
153	2.49442657069415\\
154	2.49444498431178\\
155	2.49446230514023\\
156	2.49447859803301\\
157	2.49449392399478\\
158	2.49450834040979\\
159	2.4945219012567\\
160	2.49453465731073\\
161	2.49454665633373\\
162	2.49455794325308\\
163	2.49456856032983\\
164	2.49457854731699\\
165	2.49458794160836\\
166	2.49459677837853\\
167	2.4946050907146\\
168	2.49461290974005\\
169	2.49462026473128\\
170	2.49462718322725\\
171	2.49463369113254\\
172	2.4946398128144\\
173	2.49464557119396\\
174	2.49465098783205\\
175	2.49465608300993\\
176	2.49466087580524\\
177	2.49466538416342\\
178	2.4946696249649\\
179	2.49467361408832\\
180	2.49467736646996\\
181	2.49468089615971\\
182	2.49468421637361\\
183	2.49468733954338\\
184	2.49469027736297\\
185	2.49469304083232\\
186	2.49469564029856\\
187	2.49469808549475\\
188	2.49470038557633\\
189	2.49470254915539\\
190	2.49470458433291\\
191	2.49470649872912\\
192	2.49470829951201\\
193	2.49470999342416\\
194	2.49471158680801\\
195	2.4947130856296\\
196	2.4947144955009\\
197	2.49471582170082\\
198	2.49471706919499\\
199	2.49471824265435\\
200	2.49471934647263\\
201	2.4947203847828\\
202	2.49472136147257\\
203	2.49472228019891\\
204	2.49472314440176\\
205	2.49472395731691\\
206	2.49472472198814\\
207	2.49472544127856\\
208	2.49472611788139\\
209	2.49472675432998\\
210	2.49472735300737\\
211	2.49472791615517\\
212	2.49472844588193\\
213	2.4947289441711\\
214	2.49472941288838\\
215	2.49472985378879\\
216	2.49473026852315\\
217	2.49473065864435\\
218	2.4947310256131\\
219	2.49473137080342\\
220	2.49473169550779\\
221	2.49473200094199\\
222	2.49473228824964\\
223	2.4947325585065\\
224	2.49473281272447\\
225	2.49473305185541\\
226	2.49473327679469\\
227	2.49473348838454\\
228	2.4947336874172\\
229	2.49473387463791\\
230	2.49473405074766\\
231	2.49473421640586\\
232	2.49473437223277\\
233	2.49473451881185\\
234	2.49473465669193\\
235	2.49473478638926\\
236	2.49473490838947\\
237	2.49473502314935\\
238	2.4947351310986\\
239	2.49473523264141\\
240	2.49473532815797\\
241	2.49473541800593\\
242	2.49473550252169\\
243	2.49473558202172\\
244	2.49473565680367\\
245	2.49473572714755\\
246	2.49473579331675\\
247	2.49473585555901\\
248	2.49473591410739\\
249	2.49473596918112\\
250	2.49473602098639\\
251	2.49473606971718\\
252	2.49473611555596\\
253	2.49473615867435\\
254	2.4947361992338\\
255	2.49473623738617\\
256	2.49473627327433\\
257	2.49473630703263\\
258	2.49473633878749\\
259	2.4947363686578\\
260	2.4947363967554\\
261	2.49473642318549\\
262	2.49473644804705\\
263	2.49473647143315\\
264	2.49473649343136\\
265	2.49473651412405\\
266	2.49473653358869\\
267	2.49473655189816\\
268	2.49473656912103\\
269	2.49473658532178\\
270	2.49473660056106\\
271	2.49473661489594\\
272	2.49473662838009\\
273	2.49473664106399\\
274	2.49473665299515\\
275	2.49473666421824\\
276	2.49473667477526\\
277	2.49473668470577\\
278	2.49473669404693\\
279	2.49473670283372\\
280	2.49473671109904\\
281	2.49473671887384\\
282	2.49473672618723\\
283	2.4947367330666\\
284	2.49473673953769\\
285	2.49473674562475\\
286	2.49473675135056\\
287	2.49473675673656\\
288	2.49473676180292\\
289	2.49473676656861\\
290	2.49473677105147\\
291	2.49473677105147\\
292	2.49473677105147\\
293	2.49473677105147\\
294	2.49473677105147\\
295	2.49473677105147\\
296	2.49473677105147\\
297	2.49473677105147\\
298	2.49473677105147\\
299	2.49473677105147\\
300	2.49473677105147\\
};
\end{axis}
\end{tikzpicture}%
\caption{}
\end{figure}

\begin{figure}[H]
\centering
% This file was created by matlab2tikz.
%
%The latest updates can be retrieved from
%  http://www.mathworks.com/matlabcentral/fileexchange/22022-matlab2tikz-matlab2tikz
%where you can also make suggestions and rate matlab2tikz.
%
\definecolor{mycolor1}{rgb}{0.00000,0.44700,0.74100}%
%
\begin{tikzpicture}

\begin{axis}[%
width=4.521in,
height=3.566in,
at={(0.758in,0.481in)},
scale only axis,
xmin=0,
xmax=300,
xlabel style={font=\color{white!15!black}},
xlabel={k},
ymin=0,
ymax=1.8,
ylabel style={font=\color{white!15!black}},
ylabel={Y(k)},
axis background/.style={fill=white}
]
\addplot[const plot, color=mycolor1, forget plot] table[row sep=crcr] {%
1	0\\
2	0\\
3	0.20202\\
4	0.381363772\\
5	0.5405745884792\\
6	0.681910601930065\\
7	0.807376801662065\\
8	0.918753389257166\\
9	1.01762097400741\\
10	1.10538294469291\\
11	1.18328533412576\\
12	1.25243445742761\\
13	1.31381257352359\\
14	1.36829179137883\\
15	1.41664641767974\\
16	1.45956392062001\\
17	1.49765466487875\\
18	1.53146055549941\\
19	1.56146271294626\\
20	1.58808828791264\\
21	1.61171651228869\\
22	1.63268407189206\\
23	1.6512898769727\\
24	1.66779929798444\\
25	1.68244792655289\\
26	1.69544491485312\\
27	1.70697594064711\\
28	1.71720583993623\\
29	1.7262809444819\\
30	1.73433115727287\\
31	1.74147179531066\\
32	1.7478052257929\\
33	1.75342231885168\\
34	1.75840373740865\\
35	1.7628210824046\\
36	1.76673790961448\\
37	1.77021063244259\\
38	1.77328932347903\\
39	1.77601842616608\\
40	1.77843738665143\\
41	1.78058121477559\\
42	1.78248098213803\\
43	1.7841642642963\\
44	1.78565553336142\\
45	1.78697650655121\\
46	1.78814645563942\\
47	1.78918248168534\\
48	1.79009975893699\\
49	1.7909117513645\\
50	1.791630404893\\
51	1.79226631806014\\
52	1.7928288935179\\
53	1.7933264725271\\
54	1.79376645435222\\
55	1.79415540225022\\
56	1.79449913755705\\
57	1.79480282320733\\
58	1.7950710378724\\
59	1.79530784176955\\
60	1.79551683507687\\
61	1.79570120978355\\
62	1.79586379571225\\
63	1.79600710136771\\
64	1.79613335019233\\
65	1.79624451274432\\
66	1.79634233525612\\
67	1.79642836497965\\
68	1.79650397267911\\
69	1.79657037259177\\
70	1.79662864014108\\
71	1.7966797276547\\
72	1.79672447831156\\
73	1.79676363851692\\
74	1.79679786888231\\
75	1.79682775396694\\
76	1.79685381091999\\
77	1.79687649714742\\
78	1.79689621711284\\
79	1.79691332837016\\
80	1.79692814691417\\
81	1.79694095192608\\
82	1.79695198998197\\
83	1.79696147878477\\
84	1.79696961047339\\
85	1.79697655455666\\
86	1.79698246051441\\
87	1.79698746010321\\
88	1.79699166940006\\
89	1.79699519061374\\
90	1.79699811368989\\
91	1.79700051773321\\
92	1.79700247226746\\
93	1.79700403835162\\
94	1.79700526956837\\
95	1.79700621289954\\
96	1.79700690950111\\
97	1.79700739538939\\
98	1.79700770204824\\
99	1.79700785696642\\
100	1.797007884113\\
101	1.79700780435784\\
102	1.79700763584335\\
103	1.79700739431327\\
104	1.79700709340309\\
105	1.79700674489673\\
106	1.79700635895323\\
107	1.79700594430688\\
108	1.79700550844388\\
109	1.79700505775816\\
110	1.79700459768881\\
111	1.79700413284126\\
112	1.79700366709395\\
113	1.79700320369234\\
114	1.79700274533158\\
115	1.79700229422931\\
116	1.79700185218951\\
117	1.7970014206587\\
118	1.79700100077522\\
119	1.79700059341239\\
120	1.79700019921639\\
121	1.79699981863939\\
122	1.7969994519685\\
123	1.79699909935105\\
124	1.79699876081664\\
125	1.79699843629634\\
126	1.79699812563939\\
127	1.79699782862767\\
128	1.79699754498826\\
129	1.79699727440427\\
130	1.7969970165242\\
131	1.79699677096999\\
132	1.79699653734388\\
133	1.79699631523434\\
134	1.79699610422104\\
135	1.79699590387911\\
136	1.79699571378272\\
137	1.79699553350805\\
138	1.7969953626358\\
139	1.79699520075323\\
140	1.7969950474558\\
141	1.79699490234855\\
142	1.79699476504709\\
143	1.79699463517849\\
144	1.7969945123818\\
145	1.79699439630853\\
146	1.79699428662288\\
147	1.7969941830019\\
148	1.79699408513551\\
149	1.79699399272646\\
150	1.79699390549016\\
151	1.79699382315454\\
152	1.79699374545974\\
153	1.79699367215784\\
154	1.79699360301259\\
155	1.79699353779898\\
156	1.79699347630295\\
157	1.79699341832097\\
158	1.79699336365969\\
159	1.79699331213552\\
160	1.79699326357426\\
161	1.79699321781069\\
162	1.79699317468824\\
163	1.79699313405853\\
164	1.79699309578107\\
165	1.79699305972288\\
166	1.79699302575812\\
167	1.79699299376777\\
168	1.79699296363927\\
169	1.79699293526624\\
170	1.79699290854814\\
171	1.79699288339001\\
172	1.79699285970213\\
173	1.79699283739981\\
174	1.7969928164031\\
175	1.79699279663651\\
176	1.79699277802883\\
177	1.79699276051287\\
178	1.79699274402526\\
179	1.79699272850622\\
180	1.79699271389937\\
181	1.79699270015157\\
182	1.79699268721271\\
183	1.79699267503556\\
184	1.79699266357561\\
185	1.79699265279091\\
186	1.79699264264193\\
187	1.79699263309143\\
188	1.79699262410431\\
189	1.79699261564753\\
190	1.79699260768995\\
191	1.79699260020224\\
192	1.79699259315678\\
193	1.79699258652757\\
194	1.79699258029012\\
195	1.79699257442135\\
196	1.79699256889957\\
197	1.79699256370431\\
198	1.79699255881634\\
199	1.79699255421754\\
200	1.79699254989084\\
201	1.79699254582019\\
202	1.79699254199048\\
203	1.79699253838747\\
204	1.79699253499778\\
205	1.79699253180881\\
206	1.7969925288087\\
207	1.79699252598627\\
208	1.79699252333103\\
209	1.79699252083309\\
210	1.79699251848315\\
211	1.79699251627244\\
212	1.79699251419274\\
213	1.79699251223629\\
214	1.79699251039579\\
215	1.79699250866438\\
216	1.7969925070356\\
217	1.79699250550338\\
218	1.796992504062\\
219	1.79699250270607\\
220	1.79699250143053\\
221	1.79699250023063\\
222	1.79699249910187\\
223	1.79699249804005\\
224	1.7969924970412\\
225	1.79699249610159\\
226	1.7969924952177\\
227	1.79699249438624\\
228	1.79699249360409\\
229	1.79699249286833\\
230	1.79699249217621\\
231	1.79699249152515\\
232	1.79699249091271\\
233	1.7969924903366\\
234	1.79699248979466\\
235	1.79699248928488\\
236	1.79699248880533\\
237	1.79699248835424\\
238	1.79699248792991\\
239	1.79699248753075\\
240	1.79699248715528\\
241	1.79699248680208\\
242	1.79699248646984\\
243	1.79699248615731\\
244	1.79699248586332\\
245	1.79699248558678\\
246	1.79699248532664\\
247	1.79699248508194\\
248	1.79699248485176\\
249	1.79699248463524\\
250	1.79699248443157\\
251	1.79699248423998\\
252	1.79699248405976\\
253	1.79699248389023\\
254	1.79699248373076\\
255	1.79699248358076\\
256	1.79699248343965\\
257	1.79699248330692\\
258	1.79699248318207\\
259	1.79699248306463\\
260	1.79699248295415\\
261	1.79699248285023\\
262	1.79699248275248\\
263	1.79699248266053\\
264	1.79699248257403\\
265	1.79699248249267\\
266	1.79699248241613\\
267	1.79699248234414\\
268	1.79699248227642\\
269	1.79699248221272\\
270	1.7969924821528\\
271	1.79699248209644\\
272	1.79699248204342\\
273	1.79699248199354\\
274	1.79699248194663\\
275	1.7969924819025\\
276	1.79699248186099\\
277	1.79699248182194\\
278	1.79699248178521\\
279	1.79699248175066\\
280	1.79699248171816\\
281	1.79699248168759\\
282	1.79699248165883\\
283	1.79699248163178\\
284	1.79699248160634\\
285	1.7969924815824\\
286	1.79699248155989\\
287	1.79699248153871\\
288	1.79699248151878\\
289	1.79699248150005\\
290	1.79699248148242\\
291	1.79699248148242\\
292	1.79699248148242\\
293	1.79699248148242\\
294	1.79699248148242\\
295	1.79699248148242\\
296	1.79699248148242\\
297	1.79699248148242\\
298	1.79699248148242\\
299	1.79699248148242\\
300	1.79699248148242\\
};
\end{axis}
\end{tikzpicture}%
\caption{}
\end{figure}
%! TEX encoding = utf8
\chapter{Strojenie regulatorów}

\section{Strojenie regulatora PID}

Otrzymane metodą ZN nastawy regulatora PID: $K_p = 1.212$, $T_i = 25$, $T_d = 6$.

Wyniki symulacji o długości $n = 1000$ dla parametrów:

\begin{figure}[H]
\centering
% This file was created by matlab2tikz.
%
%The latest updates can be retrieved from
%  http://www.mathworks.com/matlabcentral/fileexchange/22022-matlab2tikz-matlab2tikz
%where you can also make suggestions and rate matlab2tikz.
%
\definecolor{mycolor1}{rgb}{0.00000,0.44700,0.74100}%
%
\begin{tikzpicture}

\begin{axis}[%
width=4.272in,
height=2.477in,
at={(0.717in,0.437in)},
scale only axis,
xmin=0,
xmax=1000,
xlabel style={font=\color{white!15!black}},
xlabel={k},
ymin=2.7,
ymax=3.3,
ylabel style={font=\color{white!15!black}},
ylabel={U(k)},
axis background/.style={fill=white}
]
\addplot[const plot, color=mycolor1, forget plot] table[row sep=crcr] {%
1	3\\
2	3\\
3	3\\
4	3\\
5	3\\
6	3\\
7	3\\
8	3\\
9	3\\
10	3\\
11	3\\
12	3.075\\
13	3\\
14	3.010908\\
15	3.021816\\
16	3.032724\\
17	3.043632\\
18	3.05454\\
19	3.065448\\
20	3.076356\\
21	3.087264\\
22	3.09191469752011\\
23	3.09741727832062\\
24	3.10835835125613\\
25	3.11750737390912\\
26	3.1250108978594\\
27	3.13100064174436\\
28	3.13559493994929\\
29	3.13890005313427\\
30	3.14101135355647\\
31	3.14201439694598\\
32	3.14250794272691\\
33	3.14294063245152\\
34	3.14278020245166\\
35	3.14177931378134\\
36	3.1402682170773\\
37	3.13852641489319\\
38	3.13678964332843\\
39	3.13525597785577\\
40	3.13409116673837\\
41	3.13343328373961\\
42	3.13335322632927\\
43	3.13383247556956\\
44	3.13486420669633\\
45	3.1365059224089\\
46	3.13880352112784\\
47	3.14175100658478\\
48	3.14530245748271\\
49	3.14938187331061\\
50	3.15389122280096\\
51	3.15871697571387\\
52	3.16373899268339\\
53	3.16884363447907\\
54	3.17392910744948\\
55	3.1788980143411\\
56	3.18365271580752\\
57	3.18810043023244\\
58	3.19216009684853\\
59	3.19576692078764\\
60	3.19887511824325\\
61	3.20145928463488\\
62	3.20351442574217\\
63	3.20505422526612\\
64	3.20610800785771\\
65	3.20671818134348\\
66	3.20693888887464\\
67	3.20683473364387\\
68	3.20647865252469\\
69	3.20594906024485\\
70	3.20532665313551\\
71	3.204691152359\\
72	3.20411820639786\\
73	3.2036766869687\\
74	3.20342658901048\\
75	3.20341754380349\\
76	3.20368775284877\\
77	3.2042632041204\\
78	3.2051572111113\\
79	3.20637037499849\\
80	3.20789102055502\\
81	3.20969609896609\\
82	3.21175251163974\\
83	3.21401877675729\\
84	3.21644693135622\\
85	3.21898455369573\\
86	3.22157681580101\\
87	3.22416851036904\\
88	3.22670600837553\\
89	3.22913909568702\\
90	3.23142262894762\\
91	3.23351795226113\\
92	3.23539402503051\\
93	3.23702822512007\\
94	3.23840680906491\\
95	3.23952503007343\\
96	3.24038693054133\\
97	3.24100483562034\\
98	3.24139857971575\\
99	3.24159450243768\\
100	3.2416242561606\\
101	3.24152347310586\\
102	3.2413303443867\\
103	3.24108416584193\\
104	3.24082390521021\\
105	3.24058684206487\\
106	3.24040732631155\\
107	3.24031569389836\\
108	3.24033737067748\\
109	3.24049218753209\\
110	3.24079392191255\\
111	3.24125007272064\\
112	3.24186186713928\\
113	3.24262448980073\\
114	3.24352751698417\\
115	3.24455553172707\\
116	3.24568889017145\\
117	3.24690460535588\\
118	3.2481773120481\\
119	3.24948027501448\\
120	3.25078640324349\\
121	3.25206923400092\\
122	3.25330385312492\\
123	3.25446772157392\\
124	3.25554138278464\\
125	3.25650903069557\\
126	3.25735892411485\\
127	3.25808363920763\\
128	3.25868015799453\\
129	3.25914979666205\\
130	3.25949798299406\\
131	3.25973389717417\\
132	3.25986999444394\\
133	3.25992143151442\\
134	3.2599054211295\\
135	3.25984054071331\\
136	3.25974602157868\\
137	3.25964104475028\\
138	3.25954406811984\\
139	3.25947220749259\\
140	3.25944069121739\\
141	3.25946240465201\\
142	3.25954753684479\\
143	3.25970333766841\\
144	3.25993398937761\\
145	3.26024059233535\\
146	3.26062126061046\\
147	3.26107131942994\\
148	3.26158359319132\\
149	3.26214877000154\\
150	3.26275582658519\\
151	3.18775582658519\\
152	3.26275582658519\\
153	3.25008041155606\\
154	3.23739527758428\\
155	3.22468807329857\\
156	3.21194754974633\\
157	3.19916391709557\\
158	3.18632913446267\\
159	3.17343712539862\\
160	3.16048391451837\\
161	3.15377810399708\\
162	3.14625809521469\\
163	3.13348055632454\\
164	3.12280047384998\\
165	3.11404825287222\\
166	3.10707344616727\\
167	3.10174270549796\\
168	3.09793787409405\\
169	3.09555421773024\\
170	3.09449879311874\\
171	3.09416247028182\\
172	3.09407943116812\\
173	3.09475080182685\\
174	3.09637395368134\\
175	3.09856511377248\\
176	3.10099918347829\\
177	3.10340136815166\\
178	3.10553990613315\\
179	3.1072197719981\\
180	3.10827724100355\\
181	3.1086191373126\\
182	3.10824555763141\\
183	3.10715010913705\\
184	3.1052694919129\\
185	3.10256058393729\\
186	3.09903964002374\\
187	3.09476854432025\\
188	3.08984355124333\\
189	3.08438612116808\\
190	3.07853551256148\\
191	3.07243917595955\\
192	3.06623897062675\\
193	3.06006505214241\\
194	3.0540423275578\\
195	3.0482939102857\\
196	3.04293493202231\\
197	3.03806491131289\\
198	3.03376275590333\\
199	3.03008380003981\\
200	3.02705839039121\\
201	3.024691933976\\
202	3.0229668082165\\
203	3.02184567058965\\
204	3.02127440295111\\
205	3.02118398469599\\
206	3.02149242843138\\
207	3.02210765743975\\
208	3.02293114543814\\
209	3.02386188364982\\
210	3.02480036480803\\
211	3.02565234519747\\
212	3.02633213883109\\
213	3.02676522460351\\
214	3.02689014813192\\
215	3.02665989847518\\
216	3.02604288512225\\
217	3.02502346671072\\
218	3.02360193182578\\
219	3.0217938900634\\
220	3.01962909424838\\
221	3.01714975642257\\
222	3.01440845351454\\
223	3.01146574760512\\
224	3.00838765362728\\
225	3.00524306241698\\
226	3.00210119328099\\
227	2.99902913827057\\
228	2.99608956746295\\
229	2.99333867039806\\
230	2.99082440450804\\
231	2.98858510911683\\
232	2.9866485263773\\
233	2.98503124976018\\
234	2.9837385987712\\
235	2.98276489991566\\
236	2.98209414157647\\
237	2.98170096275648\\
238	2.98155192884752\\
239	2.98160704023913\\
240	2.9818214126496\\
241	2.98214706296971\\
242	2.98253473203313\\
243	2.98293567652367\\
244	2.98330336631223\\
245	2.98359503043995\\
246	2.9837730037168\\
247	2.9838058354662\\
248	2.9836691318205\\
249	2.9833461131237\\
250	2.98282787843086\\
251	2.98211337961419\\
252	2.98120911783444\\
253	2.98012858467236\\
254	2.97889147859116\\
255	2.97752273422936\\
256	2.97605140707064\\
257	2.97450945924595\\
258	2.97293049366354\\
259	2.97134848343638\\
260	2.96979654177645\\
261	2.9683057742507\\
262	2.96690425067718\\
263	2.96561612816678\\
264	2.96446095012603\\
265	2.9634531387154\\
266	2.96260169061596\\
267	2.96191007830416\\
268	2.9613763516488\\
269	2.96099342777139\\
270	2.96074955095883\\
271	2.96062889916217\\
272	2.96061230940251\\
273	2.96067809134281\\
274	2.96080289643911\\
275	2.96096260947722\\
276	2.96113322990344\\
277	2.96129171209649\\
278	2.96141673649013\\
279	2.96148938709612\\
280	2.96149371532908\\
281	2.96141717491345\\
282	2.96125091786529\\
283	2.96098994689099\\
284	2.96063312483552\\
285	2.96018304686046\\
286	2.95964578566525\\
287	2.95903052413774\\
288	2.95834909320886\\
289	2.95761543530168\\
290	2.9568450155458\\
291	2.95605420384761\\
292	2.95525965096874\\
293	2.95447768100315\\
294	2.9537237211191\\
295	2.9530117872315\\
296	2.95235404149999\\
297	2.95176043432946\\
298	2.95123844001745\\
299	2.95079289148438\\
300	2.95042591577759\\
301	3.02542591577759\\
302	2.95042591577759\\
303	2.95512942908458\\
304	2.95989499919844\\
305	2.96471346489173\\
306	2.96957436038736\\
307	2.97446633800435\\
308	2.97937760437466\\
309	2.98429635417871\\
310	2.98921118611112\\
311	2.98783007750046\\
312	2.98723815650551\\
313	2.99254644075\\
314	2.99697245873784\\
315	3.00058131129338\\
316	3.0034329483037\\
317	3.00558286939169\\
318	3.00708269150639\\
319	3.00798059424612\\
320	3.00832165520727\\
321	3.00867215125167\\
322	3.00945645971642\\
323	3.01007893769673\\
324	3.01015599987653\\
325	3.00986482751503\\
326	3.00935714308678\\
327	3.00876235951009\\
328	3.00819032895181\\
329	3.00773374925175\\
330	3.00747027883543\\
331	3.00742068136657\\
332	3.00752432641724\\
333	3.00774217487303\\
334	3.00811769827039\\
335	3.00870656015214\\
336	3.00953357578893\\
337	3.01059919377235\\
338	3.01188491978975\\
339	3.01335783957974\\
340	3.01497437478075\\
341	3.0166870331934\\
342	3.01845593571681\\
343	3.02025273517867\\
344	3.02205101974527\\
345	3.02381857986731\\
346	3.02551944876457\\
347	3.02711862163017\\
348	3.0285854193973\\
349	3.02989577412236\\
350	3.03103366466919\\
351	3.03199158656542\\
352	3.03276950256207\\
353	3.03337265675386\\
354	3.03381007126073\\
355	3.03409457691489\\
356	3.03424328769332\\
357	3.03427748809872\\
358	3.03422188393634\\
359	3.03410346517824\\
360	3.0339501673144\\
361	3.03378949337594\\
362	3.0336473005721\\
363	3.03354695047589\\
364	3.03350882373911\\
365	3.03354998629113\\
366	3.03368383577379\\
367	3.03391974322731\\
368	3.03426278924895\\
369	3.03471366890098\\
370	3.03526879856596\\
371	3.0359206272797\\
372	3.0366581253909\\
373	3.0374673932672\\
374	3.03833232196624\\
375	3.03923526125113\\
376	3.04015768515945\\
377	3.04108085840625\\
378	3.04198649716753\\
379	3.04285740429348\\
380	3.0436780519899\\
381	3.04443508433414\\
382	3.045117716226\\
383	3.04571801407145\\
384	3.04623105472377\\
385	3.04665496838597\\
386	3.04699087484521\\
387	3.04724272194818\\
388	3.04741703459842\\
389	3.04752258388925\\
390	3.04756998887373\\
391	3.04757126675004\\
392	3.04753934997139\\
393	3.04748759026387\\
394	3.04742926929756\\
395	3.04737713391996\\
396	3.04734297117707\\
397	3.04733723566256\\
398	3.0473687393843\\
399	3.04744441213782\\
400	3.04756913802734\\
401	3.0477456711302\\
402	3.04797463039486\\
403	3.04825457086069\\
404	3.04858212543498\\
405	3.0489522090097\\
406	3.0493582748072\\
407	3.04979261152713\\
408	3.0502466690488\\
409	3.05071140004138\\
410	3.05117760481453\\
411	3.05163626710853\\
412	3.05207886928934\\
413	3.05249767657633\\
414	3.05288598145428\\
415	3.05323830123165\\
416	3.05355052370736\\
417	3.05381999799186\\
418	3.05404556960575\\
419	3.05422756097972\\
420	3.05436770034909\\
421	3.05446900372524\\
422	3.05453561609009\\
423	3.05457261915414\\
424	3.05458581390707\\
425	3.05458148674669\\
426	3.05456616818777\\
427	3.05454639303323\\
428	3.05452847045843\\
429	3.05451827174703\\
430	3.05452104246284\\
431	3.05454124469118\\
432	3.05458243368165\\
433	3.0546471718239\\
434	3.0547369814406\\
435	3.05485233644112\\
436	3.05499269149694\\
437	3.05515654612345\\
438	3.05534153992342\\
439	3.05554457429986\\
440	3.05576195520501\\
441	3.05598955097505\\
442	3.05622295901606\\
443	3.05645767505453\\
444	3.05668925883931\\
445	3.05691349056465\\
446	3.0571265128544\\
447	3.05732495387699\\
448	3.05750602801744\\
449	3.05766761147949\\
450	3.05780829119128\\
451	3.05792738640409\\
452	3.05802494336783\\
453	3.05810170440643\\
454	3.0581590535687\\
455	3.0581989417701\\
456	3.05822379494568\\
457	3.0582364091894\\
458	3.05823983714929\\
459	3.05823727007856\\
460	3.05823191991154\\
461	3.05822690554843\\
462	3.05822514720747\\
463	3.05822927225404\\
464	3.05824153536584\\
465	3.05826375526372\\
466	3.05829726955641\\
467	3.0583429085391\\
468	3.05840098807814\\
469	3.05847132103091\\
470	3.05855324601575\\
471	3.05864567178067\\
472	3.05874713494004\\
473	3.05885586846791\\
474	3.05896987806477\\
475	3.05908702335669\\
476	3.05920510084231\\
477	3.05932192557088\\
478	3.05943540870673\\
479	3.05954362840123\\
480	3.05964489173938\\
481	3.05973778593886\\
482	3.05982121743766\\
483	3.0598944379933\\
484	3.05995705741474\\
485	3.06000904303835\\
486	3.06005070652486\\
487	3.06008267897993\\
488	3.06010587577328\\
489	3.06012145273854\\
490	3.06013075567153\\
491	3.06013526520137\\
492	3.0601365391858\\
493	3.06013615477941\\
494	3.0601356522444\\
495	3.06013648242495\\
496	3.06013995959468\\
497	3.06014722112486\\
498	3.0601591951173\\
499	3.06017657681548\\
500	3.06019981426158\\
501	3.13519981426158\\
502	3.06019981426158\\
503	3.06145281489043\\
504	3.06271102562167\\
505	3.06397375714421\\
506	3.06524016564909\\
507	3.06650928973373\\
508	3.0677800899163\\
509	3.06905148924817\\
510	3.07032241353731\\
511	3.06533697089846\\
512	3.06120600206443\\
513	3.06332133103454\\
514	3.06514756309781\\
515	3.06670883872709\\
516	3.06802701345722\\
517	3.06912190169898\\
518	3.07001149167258\\
519	3.07071213379151\\
520	3.07123870484846\\
521	3.0721265976191\\
522	3.07376745605835\\
523	3.07551068804718\\
524	3.07687998815798\\
525	3.07795172960008\\
526	3.07879172736933\\
527	3.07945657273642\\
528	3.07999480544429\\
529	3.08044794327634\\
530	3.08085138645092\\
531	3.08119167421284\\
532	3.08138146495911\\
533	3.08136384872773\\
534	3.08117740274781\\
535	3.08088843685575\\
536	3.08054606989055\\
537	3.08018534561367\\
538	3.07982986338876\\
539	3.07949399284336\\
540	3.07918473233858\\
541	3.07890689529885\\
542	3.07867332010569\\
543	3.07850751278987\\
544	3.07843242947433\\
545	3.07846071307021\\
546	3.07859482002796\\
547	3.07883040396058\\
548	3.07915894073597\\
549	3.0795697354732\\
550	3.08005142874698\\
551	3.15505142874698\\
552	3.08005142874698\\
553	3.08188739573397\\
554	3.08374182225916\\
555	3.08559715505738\\
556	3.08743612502307\\
557	3.08924306556534\\
558	3.09100463777861\\
559	3.09271012447919\\
560	3.09435142008839\\
561	3.08971070436468\\
562	3.08593673724441\\
563	3.08837132731893\\
564	3.09043472383046\\
565	3.09215663880269\\
566	3.09356662301866\\
567	3.09469396703591\\
568	3.09556744663615\\
569	3.09621500787634\\
570	3.09666345551231\\
571	3.09745646575836\\
572	3.09898709353326\\
573	3.10060720921155\\
574	3.10185040331946\\
575	3.10280516342765\\
576	3.10354735634633\\
577	3.10414143496713\\
578	3.10464155607561\\
579	3.1050926354\\
580	3.10553135109211\\
581	3.10594385695111\\
582	3.10624176138622\\
583	3.10636694787428\\
584	3.10635593821051\\
585	3.10627143348261\\
586	3.10615751073673\\
587	3.10604319515026\\
588	3.10594552192289\\
589	3.10587214746192\\
590	3.10582356079234\\
591	3.10579854782488\\
592	3.10580453745707\\
593	3.10586024896688\\
594	3.10598449749623\\
595	3.10618656635026\\
596	3.10646650491181\\
597	3.10681862321267\\
598	3.10723414854235\\
599	3.10770319607028\\
600	3.10821618275658\\
601	3.18321618275658\\
602	3.10821618275658\\
603	3.11002077583757\\
604	3.11183077343174\\
605	3.11363306338932\\
606	3.11541490744774\\
607	3.11716509570219\\
608	3.11887451004634\\
609	3.12053627011361\\
610	3.12214559733615\\
611	3.11748797018653\\
612	3.1137129363882\\
613	3.11616320963362\\
614	3.11826085122342\\
615	3.12003545696909\\
616	3.12151578090694\\
617	3.12272969068594\\
618	3.12370398822883\\
619	3.12446418791554\\
620	3.12503431178895\\
621	3.12595496936073\\
622	3.12761609498234\\
623	3.12936657202842\\
624	3.13073703639327\\
625	3.13181313492648\\
626	3.13266820214814\\
627	3.13336454294547\\
628	3.13395460999949\\
629	3.13448210052206\\
630	3.13498298251103\\
631	3.13544321568774\\
632	3.13577471390992\\
633	3.13592011750745\\
634	3.13591710683991\\
635	3.13582989957819\\
636	3.13570438526176\\
637	3.13557161936429\\
638	3.13545080299777\\
639	3.13535181081016\\
640	3.13527732095108\\
641	3.13522820219691\\
642	3.13521378993483\\
643	3.1352544743099\\
644	3.13537045894879\\
645	3.13557209748967\\
646	3.13586016680436\\
647	3.13622934977715\\
648	3.1366708935283\\
649	3.1371745943572\\
650	3.137730237578\\
651	3.212730237578\\
652	3.137730237578\\
653	3.13959516945804\\
654	3.14146869089874\\
655	3.14333617255244\\
656	3.14518334645507\\
657	3.1469975155383\\
658	3.14876816839193\\
659	3.15048717108546\\
660	3.15214867022607\\
661	3.14754142999091\\
662	3.14381844742349\\
663	3.14631720388809\\
664	3.14845496922323\\
665	3.15026157142016\\
666	3.1517662137806\\
667	3.15299741030683\\
668	3.15398277974671\\
669	3.15474879179602\\
670	3.15532052635449\\
671	3.15623936858984\\
672	3.15789541343266\\
673	3.15963785270828\\
674	3.16099841495546\\
675	3.16206410638927\\
676	3.16290945317361\\
677	3.16359776647365\\
678	3.16418230720535\\
679	3.16470737559513\\
680	3.16520933603964\\
681	3.16567437135731\\
682	3.16601453357416\\
683	3.16617255972764\\
684	3.16618611191324\\
685	3.16611919995922\\
686	3.16601732434958\\
687	3.16591101340348\\
688	3.16581884471977\\
689	3.16575001277405\\
690	3.16570649586229\\
691	3.16568847441851\\
692	3.16570462560841\\
693	3.16577471798653\\
694	3.16591837895299\\
695	3.16614545138999\\
696	3.16645629061387\\
697	3.16684526479941\\
698	3.16730342267442\\
699	3.16782048073467\\
700	3.16838625914457\\
701	3.09338625914457\\
702	3.16838625914457\\
703	3.16782886541548\\
704	3.16727791212584\\
705	3.16671920675602\\
706	3.16613895561469\\
707	3.16552495505121\\
708	3.1648671872939\\
709	3.16415799398909\\
710	3.16339196261961\\
711	3.168873442739\\
712	3.17353766222743\\
713	3.1719306143168\\
714	3.17053286916473\\
715	3.1693251316753\\
716	3.16829229151701\\
717	3.16742288912736\\
718	3.16670848766453\\
719	3.166143040031\\
720	3.16572230771118\\
721	3.16491709891776\\
722	3.16333377792779\\
723	3.16162265196529\\
724	3.16026923200782\\
725	3.15920950604746\\
726	3.15838806703418\\
727	3.15775667385483\\
728	3.15727303733162\\
729	3.15689981858324\\
730	3.15660381668802\\
731	3.15639922425571\\
732	3.15637380413968\\
733	3.15658510712593\\
734	3.15699447226147\\
735	3.15753382921342\\
736	3.1581506722323\\
737	3.1588053994779\\
738	3.15946910943745\\
739	3.16012177749259\\
740	3.16075074559093\\
741	3.16134580502796\\
742	3.16188909674214\\
743	3.16235243474902\\
744	3.16270853016684\\
745	3.16294091497603\\
746	3.16304403413882\\
747	3.16302002187857\\
748	3.16287615310601\\
749	3.16262284184685\\
750	3.16227208172636\\
751	3.08727208172636\\
752	3.16227208172636\\
753	3.15928679446798\\
754	3.15626606362772\\
755	3.15323104783297\\
756	3.15020287585127\\
757	3.14720118194329\\
758	3.1442432277808\\
759	3.14134346168451\\
760	3.13851339794849\\
761	3.14198266487555\\
762	3.14461068120361\\
763	3.141148751265\\
764	3.13825838466042\\
765	3.13589552010942\\
766	3.13401716097623\\
767	3.1325813474225\\
768	3.13154731981568\\
769	3.13087577518039\\
770	3.13052914835221\\
771	3.12995285470545\\
772	3.12874241550991\\
773	3.12752724346137\\
774	3.1267425598046\\
775	3.12626611806007\\
776	3.12599338580103\\
777	3.12583572857396\\
778	3.1257187585803\\
779	3.12558080921382\\
780	3.12537151337651\\
781	3.12509377663692\\
782	3.12482760491158\\
783	3.1246257056599\\
784	3.12445048299082\\
785	3.12424333164525\\
786	3.1239687938606\\
787	3.12360984839411\\
788	3.12316390302639\\
789	3.12263940207017\\
790	3.12205297362487\\
791	3.12142343835362\\
792	3.12076102209257\\
793	3.12006428413587\\
794	3.11933077558619\\
795	3.11856599722876\\
796	3.1177824730391\\
797	3.11699582174616\\
798	3.11622186683859\\
799	3.11547458698401\\
800	3.11476473993689\\
801	3.03976473993689\\
802	3.11476473993689\\
803	3.11298749134232\\
804	3.11126509306891\\
805	3.1096002694755\\
806	3.10799409146334\\
807	3.10644530420699\\
808	3.10495025808806\\
809	3.10350324041508\\
810	3.10209705258123\\
811	3.10692547650091\\
812	3.11082719975435\\
813	3.10846188736298\\
814	3.10640467270728\\
815	3.104621571134\\
816	3.10308150359425\\
817	3.10175633298785\\
818	3.10062096008357\\
819	3.09965339332551\\
820	3.09883474284761\\
821	3.09763169779017\\
822	3.09566356323048\\
823	3.09359060178071\\
824	3.09189057006018\\
825	3.09048607842145\\
826	3.08931189231681\\
827	3.08831334218385\\
828	3.0874448710455\\
829	3.08666870373146\\
830	3.08595363402457\\
831	3.0853171008708\\
832	3.08484901260959\\
833	3.0846069639471\\
834	3.08455212549116\\
835	3.08461793080623\\
836	3.08475506186629\\
837	3.0849280905522\\
838	3.08511266881321\\
839	3.08529319639115\\
840	3.08546090322873\\
841	3.08560868804512\\
842	3.08572108280705\\
843	3.08577190288435\\
844	3.08573569123028\\
845	3.08559753734652\\
846	3.08535292320696\\
847	3.08500433028401\\
848	3.08455864029805\\
849	3.08402517622941\\
850	3.08341425478646\\
851	3.00841425478646\\
852	3.08341425478646\\
853	3.07294312887672\\
854	3.06244683482349\\
855	3.05194379266889\\
856	3.04145238840538\\
857	3.03098960432\\
858	3.02057024571216\\
859	3.01020659696133\\
860	2.99990837738981\\
861	2.99588362490778\\
862	2.99097465135577\\
863	2.9805620519916\\
864	2.97185025457337\\
865	2.96470350735253\\
866	2.95899684666735\\
867	2.9546151171762\\
868	2.95145225815287\\
869	2.94941075028225\\
870	2.94840115036627\\
871	2.94782433496797\\
872	2.94723911919567\\
873	2.94719103555727\\
874	2.94794424633897\\
875	2.9491850046186\\
876	2.95064747230591\\
877	2.9521076164839\\
878	2.95337781837686\\
879	2.95430209323375\\
880	2.9547518430924\\
881	2.95466524262684\\
882	2.95406859837614\\
883	2.95297425700817\\
884	2.95132741562658\\
885	2.94908204008115\\
886	2.94624165416043\\
887	2.94284794872663\\
888	2.93897133455722\\
889	2.93470315442242\\
890	2.93014931073949\\
891	2.92542149793806\\
892	2.92062428646171\\
893	2.91585010476668\\
894	2.91118711196707\\
895	2.90672426970628\\
896	2.9025465999559\\
897	2.89872856358666\\
898	2.89532965986284\\
899	2.89239176275831\\
900	2.88993779593485\\
901	2.88797172076856\\
902	2.88648026300909\\
903	2.88543590430072\\
904	2.88479933946818\\
905	2.88452066485596\\
906	2.88454044526746\\
907	2.88479160251994\\
908	2.88520202331239\\
909	2.88569751072183\\
910	2.88620480692082\\
911	2.88665447107435\\
912	2.88698338065197\\
913	2.88713664903848\\
914	2.88706895610394\\
915	2.88674549093259\\
916	2.88614265087793\\
917	2.8852484594664\\
918	2.88406260245168\\
919	2.88259602848338\\
920	2.88087011695082\\
921	2.87891545414084\\
922	2.87677029114531\\
923	2.87447878586537\\
924	2.87208913918455\\
925	2.86965170967192\\
926	2.86721715636229\\
927	2.86483464682849\\
928	2.86255017612619\\
929	2.86040505099048\\
930	2.85843459341973\\
931	2.85666711000462\\
932	2.85512316057121\\
933	2.85381514307821\\
934	2.85274719357044\\
935	2.85191538489661\\
936	2.85130819897089\\
937	2.85090724290627\\
938	2.8506881754928\\
939	2.85062180550956\\
940	2.85067531805184\\
941	2.85081358078429\\
942	2.85100047971548\\
943	2.85120023427928\\
944	2.8513786444226\\
945	2.85150422768182\\
946	2.85154921093721\\
947	2.85149034869966\\
948	2.85130954697637\\
949	2.85099427904018\\
950	2.8505377869117\\
951	2.84993906998779\\
952	2.84920266977234\\
953	2.8483382667417\\
954	2.84736011163287\\
955	2.84628631854368\\
956	2.84513805097949\\
957	2.84393863433982\\
958	2.84271262939098\\
959	2.84148490111996\\
960	2.84027971608476\\
961	2.83911989903294\\
962	2.8380260762344\\
963	2.83701602878781\\
964	2.83610417428219\\
965	2.8353011898349\\
966	2.83461378392095\\
967	2.83404461879042\\
968	2.83359237984893\\
969	2.83325198333069\\
970	2.83301490907229\\
971	2.83286964132577\\
972	2.83280219743544\\
973	2.83279672192698\\
974	2.83283612217011\\
975	2.83290272129653\\
976	2.83297890446533\\
977	2.83304773581477\\
978	2.83309352543978\\
979	2.83310232838221\\
980	2.8330623607927\\
981	2.83296432198533\\
982	2.83280161491715\\
983	2.83257046154176\\
984	2.83226991336511\\
985	2.83190176123811\\
986	2.83147035182781\\
987	2.83098232120878\\
988	2.83044625851794\\
989	2.82987231455237\\
990	2.82927177151618\\
991	2.82865659081804\\
992	2.82803895588793\\
993	2.82743082644459\\
994	2.82684351954684\\
995	2.82628733116696\\
996	2.82577121000741\\
997	2.82530249293524\\
998	2.82488670882656\\
999	2.82452745490001\\
1000	2.8242263468738\\
};
\end{axis}
\end{tikzpicture}%
\caption{Sterowanie PID dla parametrów ZN}
\end{figure}

\begin{figure}[H]
\centering
% This file was created by matlab2tikz.
%
%The latest updates can be retrieved from
%  http://www.mathworks.com/matlabcentral/fileexchange/22022-matlab2tikz-matlab2tikz
%where you can also make suggestions and rate matlab2tikz.
%
\definecolor{mycolor1}{rgb}{0.00000,0.44700,0.74100}%
\definecolor{mycolor2}{rgb}{0.85000,0.32500,0.09800}%
%
\begin{tikzpicture}

\begin{axis}[%
width=4.272in,
height=2.477in,
at={(0.717in,0.437in)},
scale only axis,
xmin=0,
xmax=1000,
xlabel style={font=\color{white!15!black}},
xlabel={k},
ymin=0.5,
ymax=1.4,
ylabel style={font=\color{white!15!black}},
ylabel={Y(k)},
axis background/.style={fill=white},
legend style={legend cell align=left, align=left, draw=white!15!black}
]
\addplot[const plot, color=mycolor1] table[row sep=crcr] {%
1	0.9\\
2	0.9\\
3	0.9\\
4	0.9\\
5	0.9\\
6	0.9\\
7	0.9\\
8	0.9\\
9	0.9\\
10	0.9\\
11	0.9\\
12	0.9\\
13	0.9\\
14	0.9\\
15	0.9\\
16	0.9\\
17	0.9\\
18	0.9\\
19	0.9\\
20	0.9\\
21	0.9\\
22	0.9003968325\\
23	0.90110505465075\\
24	0.901754499448244\\
25	0.902462381771257\\
26	0.903327432799628\\
27	0.904432123866198\\
28	0.90584465078664\\
29	0.907620702973639\\
30	0.909805039263915\\
31	0.912432890235655\\
32	0.915498096835601\\
33	0.918965853372063\\
34	0.922837193061285\\
35	0.927128134502026\\
36	0.931831863484165\\
37	0.936923734431968\\
38	0.942365468215513\\
39	0.948108657807613\\
40	0.954097679116409\\
41	0.960272091866446\\
42	0.966571366665219\\
43	0.97294065568034\\
44	0.979329694247744\\
45	0.985688442749354\\
46	0.991967881999957\\
47	0.998123300499065\\
48	1.00411656075767\\
49	1.00991755516541\\
50	1.01550502606484\\
51	1.0208668943165\\
52	1.0259999844607\\
53	1.03090892615053\\
54	1.03560478986773\\
55	1.04010429107047\\
56	1.04442934161618\\
57	1.04860630459408\\
58	1.05266486491142\\
59	1.05663669420945\\
60	1.0605540422755\\
61	1.06444835015091\\
62	1.06834897017567\\
63	1.07228209848007\\
64	1.07626998780362\\
65	1.08033038969881\\
66	1.08447613243709\\
67	1.08871482012883\\
68	1.09304869808435\\
69	1.09747471797913\\
70	1.10198480880132\\
71	1.10656634102768\\
72	1.11120275821343\\
73	1.11587433707758\\
74	1.12055902663016\\
75	1.12523332003065\\
76	1.12987312861239\\
77	1.13445463865313\\
78	1.13895513044294\\
79	1.14335373448336\\
80	1.14763209849147\\
81	1.15177494149719\\
82	1.1557704762483\\
83	1.15961068776104\\
84	1.16329146381875\\
85	1.16681258108326\\
86	1.17017755613573\\
87	1.17339337378769\\
88	1.17647010700276\\
89	1.17942044510106\\
90	1.18225914955358\\
91	1.18500245899432\\
92	1.18766746662559\\
93	1.19027149370736\\
94	1.19283148214764\\
95	1.19536342734718\\
96	1.1978818696875\\
97	1.20039945985026\\
98	1.20292660984184\\
99	1.20547123822702\\
100	1.2080386145976\\
101	1.21063130471987\\
102	1.21324921421263\\
103	1.21588972514576\\
104	1.21854791678215\\
105	1.22121685897705\\
106	1.22388796462404\\
107	1.22655138605161\\
108	1.2291964394166\\
109	1.23181204087031\\
110	1.23438713855422\\
111	1.2369111252862\\
112	1.23937421809208\\
113	1.24176779247258\\
114	1.24408466140287\\
115	1.24631929145305\\
116	1.24846795098764\\
117	1.2505287880408\\
118	1.2525018380675\\
119	1.25438896424786\\
120	1.25619373529506\\
121	1.25792124771947\\
122	1.25957790117795\\
123	1.26117113684122\\
124	1.26270914961122\\
125	1.26420058549763\\
126	1.26565423551553\\
127	1.26707873711107\\
128	1.26848229338989\\
129	1.2698724193567\\
130	1.27125572302791\\
131	1.27263772771127\\
132	1.27402274002252\\
133	1.27541376639511\\
134	1.27681247900324\\
135	1.27821923022645\\
136	1.27963311309799\\
137	1.28105206365459\\
138	1.28247299979045\\
139	1.28389199015027\\
140	1.285304445803\\
141	1.28670532693611\\
142	1.2880893566037\\
143	1.28945123364788\\
144	1.29078583727327\\
145	1.29208841636796\\
146	1.29335475749583\\
147	1.29458132649851\\
148	1.29576537979462\\
149	1.29690504270447\\
150	1.29799935341126\\
151	1.29904827244758\\
152	1.30005265882428\\
153	1.30101421505558\\
154	1.30193540434301\\
155	1.30281934403149\\
156	1.30366968011989\\
157	1.30449044807919\\
158	1.30528592549637\\
159	1.30606048211984\\
160	1.30681843273953\\
161	1.30716369682455\\
162	1.30717941725438\\
163	1.30722458102453\\
164	1.30716381414196\\
165	1.30688317044968\\
166	1.30628742896576\\
167	1.30529770217837\\
168	1.30384932587905\\
169	1.3018900031321\\
170	1.2993781767954\\
171	1.29631499574341\\
172	1.2927319090087\\
173	1.2886269424304\\
174	1.28398656026348\\
175	1.2788234391972\\
176	1.27317069285459\\
177	1.26707704408256\\
178	1.26060280915345\\
179	1.25381657523541\\
180	1.24679246780261\\
181	1.23960513258249\\
182	1.23232366515491\\
183	1.22501234716547\\
184	1.21773453022157\\
185	1.21055133104061\\
186	1.20351794537139\\
187	1.19668112903149\\
188	1.19007760151277\\
189	1.18373317014001\\
190	1.17766240879857\\
191	1.1718689880777\\
192	1.16634687067229\\
193	1.16108181655011\\
194	1.15605237449859\\
195	1.15123058867484\\
196	1.14658305274622\\
197	1.14207237455282\\
198	1.13765885013809\\
199	1.13330219926585\\
200	1.12896325677001\\
201	1.12460552784681\\
202	1.12019649736667\\
203	1.1157086215278\\
204	1.11112004815648\\
205	1.10641515354477\\
206	1.10158490527865\\
207	1.09662700428\\
208	1.0915457746403\\
209	1.08635180023695\\
210	1.08106132727517\\
211	1.07569546592208\\
212	1.07027923752917\\
213	1.0648405244653\\
214	1.05940897646577\\
215	1.05401491185706\\
216	1.04868824109481\\
217	1.04345744096653\\
218	1.03834861185122\\
219	1.0333846504378\\
220	1.02858456628647\\
221	1.02396296422117\\
222	1.01952970649721\\
223	1.01528975940691\\
224	1.01124321991638\\
225	1.00738551114763\\
226	1.00370773142582\\
227	1.00019713861747\\
228	0.99683774830098\\
229	0.993611021036291\\
230	0.990496611290706\\
231	0.987473148899105\\
232	0.984519023523776\\
233	0.981613143560161\\
234	0.978735643274665\\
235	0.975868515342502\\
236	0.97299614989132\\
237	0.970105765306768\\
238	0.967187720337565\\
239	0.964235701468817\\
240	0.961246784068318\\
241	0.958221370319641\\
242	0.955163011260938\\
243	0.952078124157511\\
244	0.948975619765714\\
245	0.945866456655265\\
246	0.94276314158254\\
247	0.939679195955599\\
248	0.936628608741493\\
249	0.93362529577704\\
250	0.930682584392817\\
251	0.927812740593989\\
252	0.92502655382839\\
253	0.922332991702936\\
254	0.919738933997245\\
255	0.917248992097394\\
256	0.91486541666791\\
257	0.912588093123838\\
258	0.910414621370712\\
259	0.908340473446145\\
260	0.906359220206614\\
261	0.904462816127511\\
262	0.902641929679672\\
263	0.900886305650784\\
264	0.8991851452154\\
265	0.897527489522673\\
266	0.895902593045374\\
267	0.894300273878\\
268	0.892711229530689\\
269	0.891127308472649\\
270	0.889541729658373\\
271	0.887949244441043\\
272	0.886346237555857\\
273	0.884730766156806\\
274	0.88310253813116\\
275	0.881462833019366\\
276	0.879814370764858\\
277	0.878161135149582\\
278	0.876508160089889\\
279	0.874861287940593\\
280	0.873226909562449\\
281	0.871611696144242\\
282	0.870022332642392\\
283	0.868465262228174\\
284	0.866946450346648\\
285	0.865471175933068\\
286	0.864043856050893\\
287	0.862667908765514\\
288	0.861345657508032\\
289	0.860078278573851\\
290	0.858865791800825\\
291	0.857707092937426\\
292	0.856600024794666\\
293	0.855541483021535\\
294	0.854527551290118\\
295	0.853553659852368\\
296	0.852614760855281\\
297	0.851705513484875\\
298	0.850820471952338\\
299	0.84995426952895\\
300	0.849101792262645\\
301	0.848258336643261\\
302	0.847419746294486\\
303	0.846582523721994\\
304	0.84574391419954\\
305	0.844901959986352\\
306	0.844055524198045\\
307	0.843204284758678\\
308	0.842348699905252\\
309	0.841489947663514\\
310	0.840629842535724\\
311	0.840169094662738\\
312	0.840027359635076\\
313	0.839805903831571\\
314	0.839567209155041\\
315	0.83936480474152\\
316	0.839244206365335\\
317	0.839243750798437\\
318	0.839395340436324\\
319	0.839725111766168\\
320	0.840254039536307\\
321	0.840965251242341\\
322	0.841815990385479\\
323	0.842804380409815\\
324	0.843952805469557\\
325	0.845270429284305\\
326	0.846755821961687\\
327	0.848399199457051\\
328	0.850184327694026\\
329	0.852090136165507\\
330	0.854092079575684\\
331	0.856166053544525\\
332	0.858293559373643\\
333	0.860459927998672\\
334	0.862648812483973\\
335	0.864841570203845\\
336	0.867019426091502\\
337	0.869165057096226\\
338	0.871263707431921\\
339	0.873303926457666\\
340	0.875278006001206\\
341	0.877181949883109\\
342	0.879014709072526\\
343	0.880777223160516\\
344	0.882472141796327\\
345	0.884104055543816\\
346	0.885679576426447\\
347	0.887207109824139\\
348	0.888696427994394\\
349	0.890158129663874\\
350	0.891603049252957\\
351	0.893041681803541\\
352	0.894483719627462\\
353	0.895937764786653\\
354	0.897411160909335\\
355	0.898909838051168\\
356	0.900438141685211\\
357	0.901998684545975\\
358	0.90359226015397\\
359	0.905217837966332\\
360	0.906872646603919\\
361	0.908552340683483\\
362	0.910251234068291\\
363	0.911962570966477\\
364	0.913678808404027\\
365	0.915391899148136\\
366	0.917093575864466\\
367	0.918775636276858\\
368	0.920430222821914\\
369	0.922050086094013\\
370	0.923628820126828\\
371	0.925161058368271\\
372	0.926642621853169\\
373	0.928070615443676\\
374	0.929443472725517\\
375	0.930760953011266\\
376	0.93202409432847\\
377	0.933235125828997\\
378	0.934397343260897\\
379	0.935514952183391\\
380	0.936592885001724\\
381	0.937636599219598\\
382	0.938651865245467\\
383	0.93964455239024\\
384	0.940620421246598\\
385	0.941584929632738\\
386	0.942543058110608\\
387	0.943499160027043\\
388	0.944456840079157\\
389	0.945418864441198\\
390	0.946387104412926\\
391	0.947362514345214\\
392	0.948345143313419\\
393	0.949334178730906\\
394	0.950328018934164\\
395	0.951324370821061\\
396	0.9523203679239\\
397	0.953312703830763\\
398	0.954297775593714\\
399	0.955271831656608\\
400	0.956231118895027\\
401	0.957172023592017\\
402	0.95809120157877\\
403	0.958985693341087\\
404	0.959853020607133\\
405	0.960691261752121\\
406	0.961499104236219\\
407	0.962275873190419\\
408	0.9630215361472\\
409	0.963736684751766\\
410	0.96442249506354\\
411	0.965080668745633\\
412	0.965713358022552\\
413	0.966323077745515\\
414	0.966912608226528\\
415	0.967484892678643\\
416	0.968042933129712\\
417	0.968589688566427\\
418	0.969127978825942\\
419	0.969660397399219\\
420	0.970189235861091\\
421	0.970716422116399\\
422	0.971243474070611\\
423	0.971771469719207\\
424	0.972301034025811\\
425	0.972832342347339\\
426	0.973365139586878\\
427	0.973898773731139\\
428	0.974432241975527\\
429	0.974964247269264\\
430	0.975493262835007\\
431	0.976017602037695\\
432	0.97653549089826\\
433	0.977045140567746\\
434	0.97754481719178\\
435	0.978032906796037\\
436	0.978507973099932\\
437	0.978968806504891\\
438	0.979414462890818\\
439	0.979844291273657\\
440	0.980257949812176\\
441	0.980655410087131\\
442	0.98103694999534\\
443	0.981403135990794\\
444	0.981754795752179\\
445	0.982092982650503\\
446	0.982418933623735\\
447	0.982734022231702\\
448	0.983039708760909\\
449	0.983337489274906\\
450	0.983628845463454\\
451	0.983915197037351\\
452	0.98419785825193\\
453	0.984477999929074\\
454	0.984756618094831\\
455	0.985034510067923\\
456	0.985312258534895\\
457	0.985590223841668\\
458	0.985868544430038\\
459	0.986147145061628\\
460	0.98642575221052\\
461	0.986703915777452\\
462	0.986981036089827\\
463	0.987256395007808\\
464	0.987529189860735\\
465	0.987798568891391\\
466	0.988063666887806\\
467	0.988323639731228\\
468	0.988577696680858\\
469	0.988825129345951\\
470	0.989065336457574\\
471	0.989297843738771\\
472	0.989522318375325\\
473	0.989738577801861\\
474	0.989946592731781\\
475	0.990146484566763\\
476	0.990338517515298\\
477	0.990523085923603\\
478	0.990700697470936\\
479	0.990871953000652\\
480	0.991037523845198\\
481	0.991198127556078\\
482	0.991354502968001\\
483	0.99150738551097\\
484	0.991657483636728\\
485	0.991805457149996\\
486	0.99195189813403\\
487	0.992097315038895\\
488	0.992242120364734\\
489	0.992386622226472\\
490	0.992531019936621\\
491	0.992675403594489\\
492	0.992819757528406\\
493	0.992963967307301\\
494	0.993107829923245\\
495	0.993251066650795\\
496	0.993393338014796\\
497	0.993534260247401\\
498	0.993673422588261\\
499	0.993810404779147\\
500	0.993944794124629\\
501	0.99407620153232\\
502	0.994204276007057\\
503	0.994328717150409\\
504	0.994449285306534\\
505	0.994565809094043\\
506	0.994678190167261\\
507	0.994786405155133\\
508	0.994890504828404\\
509	0.994990610641852\\
510	0.995086908885376\\
511	0.995576320281363\\
512	0.996373384114822\\
513	0.997056856627257\\
514	0.997652839785233\\
515	0.998184012720492\\
516	0.998670007498053\\
517	0.999127744934178\\
518	0.999571734880995\\
519	1.00001434495842\\
520	1.00046604130442\\
521	1.00090250944675\\
522	1.00127650821095\\
523	1.0015858198777\\
524	1.00185870519926\\
525	1.00211644344964\\
526	1.00237451129508\\
527	1.00264359443392\\
528	1.00293045372561\\
529	1.00323866481912\\
530	1.00356924790285\\
531	1.00392396324233\\
532	1.00430993647707\\
533	1.00473730119126\\
534	1.00521289410927\\
535	1.00573874175239\\
536	1.00631352098964\\
537	1.00693371374279\\
538	1.00759450858742\\
539	1.00829049389687\\
540	1.00901618023572\\
541	1.00976615340582\\
542	1.01053456799302\\
543	1.01131451329682\\
544	1.01209814152972\\
545	1.01287740668452\\
546	1.0136447383078\\
547	1.01439344764917\\
548	1.01511793444369\\
549	1.01581374868606\\
550	1.01647755031866\\
551	1.01710701960353\\
552	1.01770080634101\\
553	1.01825857732303\\
554	1.01878110040905\\
555	1.01927025124573\\
556	1.01972890636205\\
557	1.02016075941615\\
558	1.02057010446154\\
559	1.02096161646853\\
560	1.02134014894019\\
561	1.02210452711472\\
562	1.0231686533107\\
563	1.02411388033609\\
564	1.02497152632599\\
565	1.02576872843422\\
566	1.02652873592204\\
567	1.02727119274802\\
568	1.02801241990539\\
569	1.02876570092291\\
570	1.02954156963095\\
571	1.0303152279108\\
572	1.0310388789943\\
573	1.0317096173736\\
574	1.0323544237142\\
575	1.03299261216525\\
576	1.03363719623706\\
577	1.03429608179113\\
578	1.03497310437641\\
579	1.03566892538143\\
580	1.03638179985561\\
581	1.03711097021104\\
582	1.03786131833729\\
583	1.03864100027169\\
584	1.03945517760363\\
585	1.04030458173921\\
586	1.04118703782354\\
587	1.0420986482806\\
588	1.04303469293783\\
589	1.04399029493723\\
590	1.04496089494533\\
591	1.04594234152182\\
592	1.04693031770527\\
593	1.04791964447353\\
594	1.04890434954974\\
595	1.04987834266631\\
596	1.05083601969436\\
597	1.05177259750311\\
598	1.05268425308141\\
599	1.05356812540787\\
600	1.05442222602796\\
601	1.05524531299143\\
602	1.05603681669851\\
603	1.05679687588109\\
604	1.05752642048665\\
605	1.05822718673863\\
606	1.05890162816246\\
607	1.05955275920072\\
608	1.06018397437369\\
609	1.06079887163916\\
610	1.06140109777379\\
611	1.06238815529502\\
612	1.06367267808548\\
613	1.06483479509767\\
614	1.06590458609383\\
615	1.06690806065741\\
616	1.06786748757226\\
617	1.06880170721111\\
618	1.06972643631653\\
619	1.0706545679902\\
620	1.07159646572185\\
621	1.07252738232499\\
622	1.07339977939622\\
623	1.07421119225364\\
624	1.07498920795452\\
625	1.07575389159591\\
626	1.07651912025515\\
627	1.07729374153833\\
628	1.07808257524797\\
629	1.07888727392648\\
630	1.07970705637811\\
631	1.08054206928726\\
632	1.08139801088139\\
633	1.08228374254525\\
634	1.08320500228119\\
635	1.08416295539202\\
636	1.08515571003377\\
637	1.08617949822132\\
638	1.08722957906351\\
639	1.0883009130564\\
640	1.08938864940792\\
641	1.09048823359354\\
642	1.09159485441299\\
643	1.09270277022996\\
644	1.09380540185966\\
645	1.09489603238126\\
646	1.09596843550646\\
647	1.09701723353714\\
648	1.09803805764719\\
649	1.09902756828703\\
650	1.0999833811087\\
651	1.10090395264316\\
652	1.10178851400379\\
653	1.10263711072739\\
654	1.10345068556976\\
655	1.10423108971657\\
656	1.10498098649525\\
657	1.10570368460722\\
658	1.10640294432696\\
659	1.10708278588701\\
660	1.10774731848689\\
661	1.10879426684647\\
662	1.11013625476138\\
663	1.11135373743873\\
664	1.11247740266507\\
665	1.11353380841378\\
666	1.11454570224341\\
667	1.11553232585545\\
668	1.11650971445948\\
669	1.11749099392267\\
670	1.1184866745911\\
671	1.11947209456598\\
672	1.12039978281921\\
673	1.12126731604091\\
674	1.12210224887935\\
675	1.12292452910398\\
676	1.12374784895551\\
677	1.12458082097511\\
678	1.12542799649262\\
679	1.12629074219102\\
680	1.1271679885211\\
681	1.12805960294725\\
682	1.12897101871664\\
683	1.12991084840482\\
684	1.13088460379443\\
685	1.13189325717508\\
686	1.13293476610728\\
687	1.13400526012467\\
688	1.13509994650245\\
689	1.13621378429167\\
690	1.13734196898248\\
691	1.13848003555344\\
692	1.13962330012502\\
693	1.14076618074463\\
694	1.14190228463933\\
695	1.14302510151463\\
696	1.1441286243791\\
697	1.1452076994052\\
698	1.14625817813512\\
699	1.1472769302977\\
700	1.14826176300142\\
701	1.1492113007963\\
702	1.15012491499655\\
703	1.15100276034006\\
704	1.15184585571573\\
705	1.1526560944107\\
706	1.15343614802474\\
707	1.15418930110183\\
708	1.15491925987001\\
709	1.15562996417528\\
710	1.15632542085964\\
711	1.15660953296027\\
712	1.15656562739792\\
713	1.15661695467116\\
714	1.15674276891108\\
715	1.15692504397472\\
716	1.15714802462289\\
717	1.15739784291573\\
718	1.15766220131747\\
719	1.15793011813117\\
720	1.15819172753006\\
721	1.15847150016088\\
722	1.15881694305253\\
723	1.15923053228696\\
724	1.15968365629161\\
725	1.1601539234925\\
726	1.16062416814294\\
727	1.16108161656035\\
728	1.16151718832924\\
729	1.16192490959913\\
730	1.1623014186716\\
731	1.1626427627249\\
732	1.16293978402222\\
733	1.16318045408291\\
734	1.16335621669888\\
735	1.16346359664284\\
736	1.16350282346908\\
737	1.16347672519449\\
738	1.16338984362053\\
739	1.16324773134106\\
740	1.163056397555\\
741	1.16282210804564\\
742	1.16255184080787\\
743	1.16225387495435\\
744	1.16193762008565\\
745	1.16161282781298\\
746	1.16128885812461\\
747	1.16097420775358\\
748	1.16067623811486\\
749	1.16040105304662\\
750	1.16015348687908\\
751	1.15993715228771\\
752	1.15975446034756\\
753	1.15960655226226\\
754	1.15949320284716\\
755	1.15941280941183\\
756	1.15936250305687\\
757	1.15933834616354\\
758	1.15933557188196\\
759	1.15934883425454\\
760	1.15937244746529\\
761	1.15900607376211\\
762	1.15833390341491\\
763	1.157766928442\\
764	1.15726160581972\\
765	1.15678047243207\\
766	1.15629166314418\\
767	1.15576845944938\\
768	1.15518885488097\\
769	1.15453513065448\\
770	1.15379343983219\\
771	1.15298631693775\\
772	1.15216029934633\\
773	1.15131842387498\\
774	1.15043591462256\\
775	1.14949773433263\\
776	1.14849672806443\\
777	1.14743201403219\\
778	1.14630759444943\\
779	1.14513116345229\\
780	1.14391309205334\\
781	1.14266282588095\\
782	1.14138405286393\\
783	1.14007687579086\\
784	1.13874381743319\\
785	1.13739096119269\\
786	1.13602619733521\\
787	1.13465790136206\\
788	1.13329396838876\\
789	1.13194113837472\\
790	1.13060455643133\\
791	1.12928774967432\\
792	1.12799329248697\\
793	1.12672361640726\\
794	1.12548108141797\\
795	1.12426747371202\\
796	1.12308360208697\\
797	1.12192917631291\\
798	1.12080287991332\\
799	1.11970256885595\\
800	1.11862554336144\\
801	1.11756883382064\\
802	1.1165294097582\\
803	1.11550425217048\\
804	1.11449035090926\\
805	1.11348473943144\\
806	1.11248460083501\\
807	1.11148740806793\\
808	1.11049105718672\\
809	1.10949396858202\\
810	1.1084951429369\\
811	1.10710085710149\\
812	1.10540272560453\\
813	1.10382439704228\\
814	1.10233945785245\\
815	1.10092555673665\\
816	1.09956393938571\\
817	1.09823901149092\\
818	1.09693792473891\\
819	1.09565018629839\\
820	1.09436729538197\\
821	1.09311522578388\\
822	1.09194203083433\\
823	1.09085003884519\\
824	1.08981097364837\\
825	1.08880358084663\\
826	1.0878123445037\\
827	1.08682639548319\\
828	1.08583858950236\\
829	1.08484473542966\\
830	1.08384295586982\\
831	1.08283042539599\\
832	1.08179885254763\\
833	1.08073696622068\\
834	1.0796368794415\\
835	1.07849559783094\\
836	1.07731355050408\\
837	1.07609344275047\\
838	1.074839372482\\
839	1.07355616104324\\
840	1.07224885647259\\
841	1.07092260231086\\
842	1.06958315289201\\
843	1.0682374967167\\
844	1.0668937037547\\
845	1.06556016022923\\
846	1.06424487209224\\
847	1.06295503833647\\
848	1.06169682351523\\
849	1.0604752737494\\
850	1.05929433277565\\
851	1.05815690557293\\
852	1.05706488288663\\
853	1.05601906989174\\
854	1.05501908305764\\
855	1.05406332999805\\
856	1.05314910792501\\
857	1.0522727830289\\
858	1.0514300064048\\
859	1.05061593612723\\
860	1.04982544564154\\
861	1.04866006194682\\
862	1.04720675699953\\
863	1.04583915201055\\
864	1.04444483292963\\
865	1.04292876302134\\
866	1.04121139574632\\
867	1.0392269724676\\
868	1.03692197637707\\
869	1.03425372271224\\
870	1.03118907127479\\
871	1.02773605986978\\
872	1.0239306788736\\
873	1.0197720691455\\
874	1.01524322831835\\
875	1.0103489406592\\
876	1.00511104002655\\
877	0.999564401455219\\
878	0.993753565771211\\
879	0.987729914538255\\
880	0.981549323160261\\
881	0.975267491456203\\
882	0.96893428242139\\
883	0.962594984024355\\
884	0.956294853322811\\
885	0.95007849318461\\
886	0.943986691587205\\
887	0.93805423455346\\
888	0.932308497231893\\
889	0.926768649650635\\
890	0.921445340872772\\
891	0.916340976806125\\
892	0.911450808238994\\
893	0.906764263645407\\
894	0.902265688195684\\
895	0.897934711556627\\
896	0.893746895133604\\
897	0.889674755244686\\
898	0.885688990620384\\
899	0.881759786331988\\
900	0.877858101313884\\
901	0.873956855501901\\
902	0.870031912293237\\
903	0.866062790490992\\
904	0.862033159520349\\
905	0.857931214455777\\
906	0.853749947312385\\
907	0.849487270131541\\
908	0.845145955558932\\
909	0.84073338738237\\
910	0.836261131649344\\
911	0.831744352136894\\
912	0.827201107038398\\
913	0.822651574187706\\
914	0.818117248789103\\
915	0.813620141617379\\
916	0.809181994337698\\
917	0.804823529652822\\
918	0.800563758943257\\
919	0.7964193716131\\
920	0.792404228186671\\
921	0.788528974721117\\
922	0.784800789892824\\
923	0.781223268538993\\
924	0.777796437951791\\
925	0.774516897955673\\
926	0.771378073162333\\
927	0.76837056416608\\
928	0.765482582427319\\
929	0.762700451198557\\
930	0.760009152682474\\
931	0.757392900138492\\
932	0.754835713139305\\
933	0.752321974791951\\
934	0.749836951486375\\
935	0.747367258334212\\
936	0.744901256450578\\
937	0.742429371293228\\
938	0.739944324358326\\
939	0.737441273705462\\
940	0.73491786205511\\
941	0.732374174491128\\
942	0.729812610978711\\
943	0.72723768181921\\
944	0.724655736650958\\
945	0.722074639548534\\
946	0.719503404120612\\
947	0.716951803273966\\
948	0.714429968540231\\
949	0.711947993588959\\
950	0.709515555801358\\
951	0.707141568583377\\
952	0.704833875497243\\
953	0.702598995349443\\
954	0.700441925171891\\
955	0.698366005668534\\
956	0.696372851274094\\
957	0.694462344581607\\
958	0.692632692624408\\
959	0.690880540417061\\
960	0.689201135328303\\
961	0.687588534328345\\
962	0.686035844964175\\
963	0.684535490099561\\
964	0.683079486027715\\
965	0.68165972352496\\
966	0.680268241748796\\
967	0.678897485564674\\
968	0.677540537872084\\
969	0.676191319743302\\
970	0.67484475263212\\
971	0.673496878496058\\
972	0.672144935342448\\
973	0.670787387394415\\
974	0.669423910716804\\
975	0.6680553366876\\
976	0.666683557096194\\
977	0.66531139585244\\
978	0.663942453265141\\
979	0.662580929570544\\
980	0.661231434845904\\
981	0.659898792625496\\
982	0.658587844451669\\
983	0.657303262255805\\
984	0.656049374895924\\
985	0.654830014408904\\
986	0.653648386601519\\
987	0.652506969545916\\
988	0.651407442404593\\
989	0.650350645831398\\
990	0.649336574021993\\
991	0.648364397360974\\
992	0.647432513571419\\
993	0.646538624349309\\
994	0.645679833687972\\
995	0.644852763487643\\
996	0.644053681617327\\
997	0.643278637357637\\
998	0.642523599104896\\
999	0.641784589352077\\
1000	0.641057812268653\\
};
\addlegendentry{Y}

\addplot[const plot, color=mycolor2] table[row sep=crcr] {%
1	0.9\\
2	0.9\\
3	0.9\\
4	0.9\\
5	0.9\\
6	0.9\\
7	0.9\\
8	0.9\\
9	0.9\\
10	0.9\\
11	0.9\\
12	1.35\\
13	1.35\\
14	1.35\\
15	1.35\\
16	1.35\\
17	1.35\\
18	1.35\\
19	1.35\\
20	1.35\\
21	1.35\\
22	1.35\\
23	1.35\\
24	1.35\\
25	1.35\\
26	1.35\\
27	1.35\\
28	1.35\\
29	1.35\\
30	1.35\\
31	1.35\\
32	1.35\\
33	1.35\\
34	1.35\\
35	1.35\\
36	1.35\\
37	1.35\\
38	1.35\\
39	1.35\\
40	1.35\\
41	1.35\\
42	1.35\\
43	1.35\\
44	1.35\\
45	1.35\\
46	1.35\\
47	1.35\\
48	1.35\\
49	1.35\\
50	1.35\\
51	1.35\\
52	1.35\\
53	1.35\\
54	1.35\\
55	1.35\\
56	1.35\\
57	1.35\\
58	1.35\\
59	1.35\\
60	1.35\\
61	1.35\\
62	1.35\\
63	1.35\\
64	1.35\\
65	1.35\\
66	1.35\\
67	1.35\\
68	1.35\\
69	1.35\\
70	1.35\\
71	1.35\\
72	1.35\\
73	1.35\\
74	1.35\\
75	1.35\\
76	1.35\\
77	1.35\\
78	1.35\\
79	1.35\\
80	1.35\\
81	1.35\\
82	1.35\\
83	1.35\\
84	1.35\\
85	1.35\\
86	1.35\\
87	1.35\\
88	1.35\\
89	1.35\\
90	1.35\\
91	1.35\\
92	1.35\\
93	1.35\\
94	1.35\\
95	1.35\\
96	1.35\\
97	1.35\\
98	1.35\\
99	1.35\\
100	1.35\\
101	1.35\\
102	1.35\\
103	1.35\\
104	1.35\\
105	1.35\\
106	1.35\\
107	1.35\\
108	1.35\\
109	1.35\\
110	1.35\\
111	1.35\\
112	1.35\\
113	1.35\\
114	1.35\\
115	1.35\\
116	1.35\\
117	1.35\\
118	1.35\\
119	1.35\\
120	1.35\\
121	1.35\\
122	1.35\\
123	1.35\\
124	1.35\\
125	1.35\\
126	1.35\\
127	1.35\\
128	1.35\\
129	1.35\\
130	1.35\\
131	1.35\\
132	1.35\\
133	1.35\\
134	1.35\\
135	1.35\\
136	1.35\\
137	1.35\\
138	1.35\\
139	1.35\\
140	1.35\\
141	1.35\\
142	1.35\\
143	1.35\\
144	1.35\\
145	1.35\\
146	1.35\\
147	1.35\\
148	1.35\\
149	1.35\\
150	1.35\\
151	0.8\\
152	0.8\\
153	0.8\\
154	0.8\\
155	0.8\\
156	0.8\\
157	0.8\\
158	0.8\\
159	0.8\\
160	0.8\\
161	0.8\\
162	0.8\\
163	0.8\\
164	0.8\\
165	0.8\\
166	0.8\\
167	0.8\\
168	0.8\\
169	0.8\\
170	0.8\\
171	0.8\\
172	0.8\\
173	0.8\\
174	0.8\\
175	0.8\\
176	0.8\\
177	0.8\\
178	0.8\\
179	0.8\\
180	0.8\\
181	0.8\\
182	0.8\\
183	0.8\\
184	0.8\\
185	0.8\\
186	0.8\\
187	0.8\\
188	0.8\\
189	0.8\\
190	0.8\\
191	0.8\\
192	0.8\\
193	0.8\\
194	0.8\\
195	0.8\\
196	0.8\\
197	0.8\\
198	0.8\\
199	0.8\\
200	0.8\\
201	0.8\\
202	0.8\\
203	0.8\\
204	0.8\\
205	0.8\\
206	0.8\\
207	0.8\\
208	0.8\\
209	0.8\\
210	0.8\\
211	0.8\\
212	0.8\\
213	0.8\\
214	0.8\\
215	0.8\\
216	0.8\\
217	0.8\\
218	0.8\\
219	0.8\\
220	0.8\\
221	0.8\\
222	0.8\\
223	0.8\\
224	0.8\\
225	0.8\\
226	0.8\\
227	0.8\\
228	0.8\\
229	0.8\\
230	0.8\\
231	0.8\\
232	0.8\\
233	0.8\\
234	0.8\\
235	0.8\\
236	0.8\\
237	0.8\\
238	0.8\\
239	0.8\\
240	0.8\\
241	0.8\\
242	0.8\\
243	0.8\\
244	0.8\\
245	0.8\\
246	0.8\\
247	0.8\\
248	0.8\\
249	0.8\\
250	0.8\\
251	0.8\\
252	0.8\\
253	0.8\\
254	0.8\\
255	0.8\\
256	0.8\\
257	0.8\\
258	0.8\\
259	0.8\\
260	0.8\\
261	0.8\\
262	0.8\\
263	0.8\\
264	0.8\\
265	0.8\\
266	0.8\\
267	0.8\\
268	0.8\\
269	0.8\\
270	0.8\\
271	0.8\\
272	0.8\\
273	0.8\\
274	0.8\\
275	0.8\\
276	0.8\\
277	0.8\\
278	0.8\\
279	0.8\\
280	0.8\\
281	0.8\\
282	0.8\\
283	0.8\\
284	0.8\\
285	0.8\\
286	0.8\\
287	0.8\\
288	0.8\\
289	0.8\\
290	0.8\\
291	0.8\\
292	0.8\\
293	0.8\\
294	0.8\\
295	0.8\\
296	0.8\\
297	0.8\\
298	0.8\\
299	0.8\\
300	0.8\\
301	1\\
302	1\\
303	1\\
304	1\\
305	1\\
306	1\\
307	1\\
308	1\\
309	1\\
310	1\\
311	1\\
312	1\\
313	1\\
314	1\\
315	1\\
316	1\\
317	1\\
318	1\\
319	1\\
320	1\\
321	1\\
322	1\\
323	1\\
324	1\\
325	1\\
326	1\\
327	1\\
328	1\\
329	1\\
330	1\\
331	1\\
332	1\\
333	1\\
334	1\\
335	1\\
336	1\\
337	1\\
338	1\\
339	1\\
340	1\\
341	1\\
342	1\\
343	1\\
344	1\\
345	1\\
346	1\\
347	1\\
348	1\\
349	1\\
350	1\\
351	1\\
352	1\\
353	1\\
354	1\\
355	1\\
356	1\\
357	1\\
358	1\\
359	1\\
360	1\\
361	1\\
362	1\\
363	1\\
364	1\\
365	1\\
366	1\\
367	1\\
368	1\\
369	1\\
370	1\\
371	1\\
372	1\\
373	1\\
374	1\\
375	1\\
376	1\\
377	1\\
378	1\\
379	1\\
380	1\\
381	1\\
382	1\\
383	1\\
384	1\\
385	1\\
386	1\\
387	1\\
388	1\\
389	1\\
390	1\\
391	1\\
392	1\\
393	1\\
394	1\\
395	1\\
396	1\\
397	1\\
398	1\\
399	1\\
400	1\\
401	1\\
402	1\\
403	1\\
404	1\\
405	1\\
406	1\\
407	1\\
408	1\\
409	1\\
410	1\\
411	1\\
412	1\\
413	1\\
414	1\\
415	1\\
416	1\\
417	1\\
418	1\\
419	1\\
420	1\\
421	1\\
422	1\\
423	1\\
424	1\\
425	1\\
426	1\\
427	1\\
428	1\\
429	1\\
430	1\\
431	1\\
432	1\\
433	1\\
434	1\\
435	1\\
436	1\\
437	1\\
438	1\\
439	1\\
440	1\\
441	1\\
442	1\\
443	1\\
444	1\\
445	1\\
446	1\\
447	1\\
448	1\\
449	1\\
450	1\\
451	1\\
452	1\\
453	1\\
454	1\\
455	1\\
456	1\\
457	1\\
458	1\\
459	1\\
460	1\\
461	1\\
462	1\\
463	1\\
464	1\\
465	1\\
466	1\\
467	1\\
468	1\\
469	1\\
470	1\\
471	1\\
472	1\\
473	1\\
474	1\\
475	1\\
476	1\\
477	1\\
478	1\\
479	1\\
480	1\\
481	1\\
482	1\\
483	1\\
484	1\\
485	1\\
486	1\\
487	1\\
488	1\\
489	1\\
490	1\\
491	1\\
492	1\\
493	1\\
494	1\\
495	1\\
496	1\\
497	1\\
498	1\\
499	1\\
500	1\\
501	1.05\\
502	1.05\\
503	1.05\\
504	1.05\\
505	1.05\\
506	1.05\\
507	1.05\\
508	1.05\\
509	1.05\\
510	1.05\\
511	1.05\\
512	1.05\\
513	1.05\\
514	1.05\\
515	1.05\\
516	1.05\\
517	1.05\\
518	1.05\\
519	1.05\\
520	1.05\\
521	1.05\\
522	1.05\\
523	1.05\\
524	1.05\\
525	1.05\\
526	1.05\\
527	1.05\\
528	1.05\\
529	1.05\\
530	1.05\\
531	1.05\\
532	1.05\\
533	1.05\\
534	1.05\\
535	1.05\\
536	1.05\\
537	1.05\\
538	1.05\\
539	1.05\\
540	1.05\\
541	1.05\\
542	1.05\\
543	1.05\\
544	1.05\\
545	1.05\\
546	1.05\\
547	1.05\\
548	1.05\\
549	1.05\\
550	1.05\\
551	1.1\\
552	1.1\\
553	1.1\\
554	1.1\\
555	1.1\\
556	1.1\\
557	1.1\\
558	1.1\\
559	1.1\\
560	1.1\\
561	1.1\\
562	1.1\\
563	1.1\\
564	1.1\\
565	1.1\\
566	1.1\\
567	1.1\\
568	1.1\\
569	1.1\\
570	1.1\\
571	1.1\\
572	1.1\\
573	1.1\\
574	1.1\\
575	1.1\\
576	1.1\\
577	1.1\\
578	1.1\\
579	1.1\\
580	1.1\\
581	1.1\\
582	1.1\\
583	1.1\\
584	1.1\\
585	1.1\\
586	1.1\\
587	1.1\\
588	1.1\\
589	1.1\\
590	1.1\\
591	1.1\\
592	1.1\\
593	1.1\\
594	1.1\\
595	1.1\\
596	1.1\\
597	1.1\\
598	1.1\\
599	1.1\\
600	1.1\\
601	1.15\\
602	1.15\\
603	1.15\\
604	1.15\\
605	1.15\\
606	1.15\\
607	1.15\\
608	1.15\\
609	1.15\\
610	1.15\\
611	1.15\\
612	1.15\\
613	1.15\\
614	1.15\\
615	1.15\\
616	1.15\\
617	1.15\\
618	1.15\\
619	1.15\\
620	1.15\\
621	1.15\\
622	1.15\\
623	1.15\\
624	1.15\\
625	1.15\\
626	1.15\\
627	1.15\\
628	1.15\\
629	1.15\\
630	1.15\\
631	1.15\\
632	1.15\\
633	1.15\\
634	1.15\\
635	1.15\\
636	1.15\\
637	1.15\\
638	1.15\\
639	1.15\\
640	1.15\\
641	1.15\\
642	1.15\\
643	1.15\\
644	1.15\\
645	1.15\\
646	1.15\\
647	1.15\\
648	1.15\\
649	1.15\\
650	1.15\\
651	1.2\\
652	1.2\\
653	1.2\\
654	1.2\\
655	1.2\\
656	1.2\\
657	1.2\\
658	1.2\\
659	1.2\\
660	1.2\\
661	1.2\\
662	1.2\\
663	1.2\\
664	1.2\\
665	1.2\\
666	1.2\\
667	1.2\\
668	1.2\\
669	1.2\\
670	1.2\\
671	1.2\\
672	1.2\\
673	1.2\\
674	1.2\\
675	1.2\\
676	1.2\\
677	1.2\\
678	1.2\\
679	1.2\\
680	1.2\\
681	1.2\\
682	1.2\\
683	1.2\\
684	1.2\\
685	1.2\\
686	1.2\\
687	1.2\\
688	1.2\\
689	1.2\\
690	1.2\\
691	1.2\\
692	1.2\\
693	1.2\\
694	1.2\\
695	1.2\\
696	1.2\\
697	1.2\\
698	1.2\\
699	1.2\\
700	1.2\\
701	1.15\\
702	1.15\\
703	1.15\\
704	1.15\\
705	1.15\\
706	1.15\\
707	1.15\\
708	1.15\\
709	1.15\\
710	1.15\\
711	1.15\\
712	1.15\\
713	1.15\\
714	1.15\\
715	1.15\\
716	1.15\\
717	1.15\\
718	1.15\\
719	1.15\\
720	1.15\\
721	1.15\\
722	1.15\\
723	1.15\\
724	1.15\\
725	1.15\\
726	1.15\\
727	1.15\\
728	1.15\\
729	1.15\\
730	1.15\\
731	1.15\\
732	1.15\\
733	1.15\\
734	1.15\\
735	1.15\\
736	1.15\\
737	1.15\\
738	1.15\\
739	1.15\\
740	1.15\\
741	1.15\\
742	1.15\\
743	1.15\\
744	1.15\\
745	1.15\\
746	1.15\\
747	1.15\\
748	1.15\\
749	1.15\\
750	1.15\\
751	1.05\\
752	1.05\\
753	1.05\\
754	1.05\\
755	1.05\\
756	1.05\\
757	1.05\\
758	1.05\\
759	1.05\\
760	1.05\\
761	1.05\\
762	1.05\\
763	1.05\\
764	1.05\\
765	1.05\\
766	1.05\\
767	1.05\\
768	1.05\\
769	1.05\\
770	1.05\\
771	1.05\\
772	1.05\\
773	1.05\\
774	1.05\\
775	1.05\\
776	1.05\\
777	1.05\\
778	1.05\\
779	1.05\\
780	1.05\\
781	1.05\\
782	1.05\\
783	1.05\\
784	1.05\\
785	1.05\\
786	1.05\\
787	1.05\\
788	1.05\\
789	1.05\\
790	1.05\\
791	1.05\\
792	1.05\\
793	1.05\\
794	1.05\\
795	1.05\\
796	1.05\\
797	1.05\\
798	1.05\\
799	1.05\\
800	1.05\\
801	1\\
802	1\\
803	1\\
804	1\\
805	1\\
806	1\\
807	1\\
808	1\\
809	1\\
810	1\\
811	1\\
812	1\\
813	1\\
814	1\\
815	1\\
816	1\\
817	1\\
818	1\\
819	1\\
820	1\\
821	1\\
822	1\\
823	1\\
824	1\\
825	1\\
826	1\\
827	1\\
828	1\\
829	1\\
830	1\\
831	1\\
832	1\\
833	1\\
834	1\\
835	1\\
836	1\\
837	1\\
838	1\\
839	1\\
840	1\\
841	1\\
842	1\\
843	1\\
844	1\\
845	1\\
846	1\\
847	1\\
848	1\\
849	1\\
850	1\\
851	0.6\\
852	0.6\\
853	0.6\\
854	0.6\\
855	0.6\\
856	0.6\\
857	0.6\\
858	0.6\\
859	0.6\\
860	0.6\\
861	0.6\\
862	0.6\\
863	0.6\\
864	0.6\\
865	0.6\\
866	0.6\\
867	0.6\\
868	0.6\\
869	0.6\\
870	0.6\\
871	0.6\\
872	0.6\\
873	0.6\\
874	0.6\\
875	0.6\\
876	0.6\\
877	0.6\\
878	0.6\\
879	0.6\\
880	0.6\\
881	0.6\\
882	0.6\\
883	0.6\\
884	0.6\\
885	0.6\\
886	0.6\\
887	0.6\\
888	0.6\\
889	0.6\\
890	0.6\\
891	0.6\\
892	0.6\\
893	0.6\\
894	0.6\\
895	0.6\\
896	0.6\\
897	0.6\\
898	0.6\\
899	0.6\\
900	0.6\\
901	0.6\\
902	0.6\\
903	0.6\\
904	0.6\\
905	0.6\\
906	0.6\\
907	0.6\\
908	0.6\\
909	0.6\\
910	0.6\\
911	0.6\\
912	0.6\\
913	0.6\\
914	0.6\\
915	0.6\\
916	0.6\\
917	0.6\\
918	0.6\\
919	0.6\\
920	0.6\\
921	0.6\\
922	0.6\\
923	0.6\\
924	0.6\\
925	0.6\\
926	0.6\\
927	0.6\\
928	0.6\\
929	0.6\\
930	0.6\\
931	0.6\\
932	0.6\\
933	0.6\\
934	0.6\\
935	0.6\\
936	0.6\\
937	0.6\\
938	0.6\\
939	0.6\\
940	0.6\\
941	0.6\\
942	0.6\\
943	0.6\\
944	0.6\\
945	0.6\\
946	0.6\\
947	0.6\\
948	0.6\\
949	0.6\\
950	0.6\\
951	0.6\\
952	0.6\\
953	0.6\\
954	0.6\\
955	0.6\\
956	0.6\\
957	0.6\\
958	0.6\\
959	0.6\\
960	0.6\\
961	0.6\\
962	0.6\\
963	0.6\\
964	0.6\\
965	0.6\\
966	0.6\\
967	0.6\\
968	0.6\\
969	0.6\\
970	0.6\\
971	0.6\\
972	0.6\\
973	0.6\\
974	0.6\\
975	0.6\\
976	0.6\\
977	0.6\\
978	0.6\\
979	0.6\\
980	0.6\\
981	0.6\\
982	0.6\\
983	0.6\\
984	0.6\\
985	0.6\\
986	0.6\\
987	0.6\\
988	0.6\\
989	0.6\\
990	0.6\\
991	0.6\\
992	0.6\\
993	0.6\\
994	0.6\\
995	0.6\\
996	0.6\\
997	0.6\\
998	0.6\\
999	0.6\\
1000	0.6\\
};
\addlegendentry{Yzad}

\end{axis}
\end{tikzpicture}%
\caption{Śledzenie wartości zadanej dla parametrów ZN}
\end{figure}

Wskaźnik jakości regulacji:

\begin{equation}
E = 36,0711
\end{equation}

Zmieniając $T_i$ i $T_d$ na $15$ i $4$ odpowiednio:

\begin{figure}[H]
\centering
% This file was created by matlab2tikz.
%
%The latest updates can be retrieved from
%  http://www.mathworks.com/matlabcentral/fileexchange/22022-matlab2tikz-matlab2tikz
%where you can also make suggestions and rate matlab2tikz.
%
\definecolor{mycolor1}{rgb}{0.00000,0.44700,0.74100}%
%
\begin{tikzpicture}

\begin{axis}[%
width=4.272in,
height=2.477in,
at={(0.717in,0.437in)},
scale only axis,
xmin=0,
xmax=1000,
xlabel style={font=\color{white!15!black}},
xlabel={k},
ymin=2.7,
ymax=3.3,
ylabel style={font=\color{white!15!black}},
ylabel={U(k)},
axis background/.style={fill=white}
]
\addplot[const plot, color=mycolor1, forget plot] table[row sep=crcr] {%
1	3\\
2	3\\
3	3\\
4	3\\
5	3\\
6	3\\
7	3\\
8	3\\
9	3\\
10	3\\
11	3\\
12	3.075\\
13	3\\
14	3.01818\\
15	3.03636\\
16	3.05454\\
17	3.07272\\
18	3.0909\\
19	3.10908\\
20	3.12725999999999\\
21	3.14543999999999\\
22	3.15928333507349\\
23	3.17355539765267\\
24	3.1910399297512\\
25	3.20649030545277\\
26	3.21999621778548\\
27	3.23163838589503\\
28	3.24148942490702\\
29	3.24961463326127\\
30	3.25607270522973\\
31	3.26091637561713\\
32	3.2644437584983\\
33	3.26689853931449\\
34	3.26809019539458\\
35	3.26798668177083\\
36	3.26683749219293\\
37	3.26486990230078\\
38	3.26229161921763\\
39	3.25929313025485\\
40	3.25604978372852\\
41	3.25272363136267\\
42	3.24945055936301\\
43	3.24633153112004\\
44	3.24346248815398\\
45	3.24094931584147\\
46	3.23888260096871\\
47	3.23732436315418\\
48	3.23631178426299\\
49	3.23586041412742\\
50	3.23596692089787\\
51	3.23661144588014\\
52	3.23776045363396\\
53	3.2393706300111\\
54	3.24139133035207\\
55	3.24376440325935\\
56	3.24642465505088\\
57	3.24930243257214\\
58	3.25232659433024\\
59	3.25542696040759\\
60	3.25853632672065\\
61	3.26159211652254\\
62	3.26453768265446\\
63	3.26732319166278\\
64	3.26990613713101\\
65	3.27225173079995\\
66	3.27433328344434\\
67	3.27613241821397\\
68	3.27763898535544\\
69	3.27885069982907\\
70	3.27977256644732\\
71	3.28041614477064\\
72	3.28079869847567\\
73	3.28094227488736\\
74	3.28087275920979\\
75	3.2806189290163\\
76	3.28021151343921\\
77	3.27968226441559\\
78	3.27906306334473\\
79	3.27838509155555\\
80	3.27767808701557\\
81	3.2769697024312\\
82	3.27628497408613\\
83	3.27564590559634\\
84	3.27507116584204\\
85	3.27457589679156\\
86	3.2741716257836\\
87	3.27386627669754\\
88	3.27366427319752\\
89	3.27356672485382\\
90	3.27357168470954\\
91	3.27367446543794\\
92	3.2738680005728\\
93	3.27414323723773\\
94	3.27448954727475\\
95	3.27489514456272\\
96	3.27534749739296\\
97	3.2758337258396\\
98	3.27634097513778\\
99	3.27685675729524\\
100	3.27736925456227\\
101	3.27786757991313\\
102	3.27834199125828\\
103	3.27878405763758\\
104	3.27918677708181\\
105	3.27954464712799\\
106	3.27985369011035\\
107	3.2801114363221\\
108	3.28031686896467\\
109	3.28047033546999\\
110	3.28057343028871\\
111	3.2806288545719\\
112	3.28064025833672\\
113	3.28061207070337\\
114	3.28054932363711\\
115	3.28045747434372\\
116	3.2803422310709\\
117	3.28020938658443\\
118	3.28006466303543\\
119	3.2799135713325\\
120	3.27976128749867\\
121	3.27961254784497\\
122	3.2794715641494\\
123	3.27934195940832\\
124	3.27922672414223\\
125	3.27912819270236\\
126	3.27904803854802\\
127	3.27898728705503\\
128	3.27894634407718\\
129	3.27892503821856\\
130	3.27892267458553\\
131	3.27893809767122\\
132	3.27896976098103\\
133	3.27901580102892\\
134	3.27907411341595\\
135	3.27914242883742\\
136	3.27921838704447\\
137	3.27929960700285\\
138	3.27938375173556\\
139	3.27946858659993\\
140	3.279552030024\\
141	3.27963219600442\\
142	3.27970742794036\\
143	3.27977632363947\\
144	3.27983775157531\\
145	3.2798908586976\\
146	3.279935070292\\
147	3.27997008255229\\
148	3.27999584866321\\
149	3.28001255929547\\
150	3.2800206184856\\
151	3.2050206184856\\
152	3.2800206184856\\
153	3.25778685091219\\
154	3.23554759454058\\
155	3.21330386760868\\
156	3.19105671603973\\
157	3.16880718417757\\
158	3.14655628796322\\
159	3.124304991052\\
160	3.10205418425231\\
161	3.08414133332687\\
162	3.0658013205565\\
163	3.04448463297295\\
164	3.02564928721542\\
165	3.00918677080369\\
166	2.99499934719576\\
167	2.98299900353603\\
168	2.9731065013009\\
169	2.96525052007807\\
170	2.95936688555538\\
171	2.95514711807443\\
172	2.95233816520567\\
173	2.95110858058154\\
174	2.95144487506602\\
175	2.95304398246442\\
176	2.95562972143215\\
177	2.95894959813855\\
178	2.96277197248644\\
179	2.96688354767824\\
180	2.97108714724617\\
181	2.975214246776\\
182	2.97913343091441\\
183	2.98272100604137\\
184	2.98584791165664\\
185	2.98840646775917\\
186	2.99032351168644\\
187	2.99155589665657\\
188	2.9920866219755\\
189	2.99192151245048\\
190	2.99108637480931\\
191	2.98962372957668\\
192	2.987588556416\\
193	2.98504549749928\\
194	2.98206856859452\\
195	2.97874006381598\\
196	2.97514729736605\\
197	2.97137903911673\\
198	2.96752257697865\\
199	2.96366130305151\\
200	2.95987273581276\\
201	2.95622695222705\\
202	2.95278548796039\\
203	2.94960065189385\\
204	2.94671501047289\\
205	2.94416094363322\\
206	2.94196043850123\\
207	2.94012524666937\\
208	2.93865737036481\\
209	2.9375498001791\\
210	2.93678744190031\\
211	2.93634817957529\\
212	2.93620402266442\\
213	2.93632228757115\\
214	2.93666678314916\\
215	2.93719899009615\\
216	2.93787922018937\\
217	2.93866772539827\\
218	2.93952572315878\\
219	2.94041631168594\\
220	2.94130525785713\\
221	2.94216164705387\\
222	2.94295839031017\\
223	2.94367258962448\\
224	2.94428576631684\\
225	2.94478395895041\\
226	2.94515769805167\\
227	2.94540186671765\\
228	2.94551545909249\\
229	2.94550125126397\\
230	2.94536540071244\\
231	2.94511699113327\\
232	2.94476753944138\\
233	2.94433048115775\\
234	2.94382064930825\\
235	2.94325376066951\\
236	2.94264592186521\\
237	2.94201316642904\\
238	2.94137103238387\\
239	2.94073418810429\\
240	2.94011611231248\\
241	2.93952883210977\\
242	2.93898272105569\\
243	2.93848635753578\\
244	2.93804644205969\\
245	2.93766777072729\\
246	2.93735326089608\\
247	2.93710402406982\\
248	2.93691948021039\\
249	2.93679750706296\\
250	2.93673461768866\\
251	2.93672615921803\\
252	2.93676652586074\\
253	2.93684937941495\\
254	2.93696787088883\\
255	2.93711485735066\\
256	2.93728310873702\\
257	2.93746550004522\\
258	2.93765518509312\\
259	2.93784574882496\\
260	2.93803133595077\\
261	2.93820675450871\\
262	2.93836755371211\\
263	2.93851007616824\\
264	2.93863148521807\\
265	2.93872976873385\\
266	2.93880372121505\\
267	2.93885290643714\\
268	2.93887760322889\\
269	2.93887873718173\\
270	2.93885780123037\\
271	2.93881676809204\\
272	2.93875799751728\\
273	2.93868414119692\\
274	2.9385980479961\\
275	2.93850267195669\\
276	2.93840098523587\\
277	2.93829589784068\\
278	2.93819018568744\\
279	2.93808642817171\\
280	2.93798695608833\\
281	2.93789381040167\\
282	2.93780871204225\\
283	2.93773304260313\\
284	2.93766783553628\\
285	2.93761377720783\\
286	2.93757121696714\\
287	2.93754018521952\\
288	2.93752041836701\\
289	2.93751138939647\\
290	2.93751234284793\\
291	2.93752233288668\\
292	2.93754026322706\\
293	2.93756492771151\\
294	2.93759505042988\\
295	2.93762932436874\\
296	2.93766644770204\\
297	2.93770515696976\\
298	2.93774425653454\\
299	2.9377826438543\\
300	2.93781933025571\\
301	3.01281933025571\\
302	2.93781933025571\\
303	2.94592633395756\\
304	2.95402907287628\\
305	2.96212727714449\\
306	2.9702208202183\\
307	2.97830971013865\\
308	2.98639407801522\\
309	2.99447416426235\\
310	3.00255030313122\\
311	3.00628821444126\\
312	3.01045535075978\\
313	3.01841785586235\\
314	3.02545417467115\\
315	3.03160667195993\\
316	3.03691353265514\\
317	3.04140915202214\\
318	3.04512448922803\\
319	3.04808738803382\\
320	3.05032286797598\\
321	3.05210403043972\\
322	3.05365188346841\\
323	3.05472177981323\\
324	3.05516895751109\\
325	3.05511023837836\\
326	3.0546518434535\\
327	3.05389064794574\\
328	3.05291529573528\\
329	3.05180718868435\\
330	3.05064136436297\\
331	3.0494727816791\\
332	3.04832682869721\\
333	3.04723051946093\\
334	3.0462321363074\\
335	3.04537972999637\\
336	3.04470752339454\\
337	3.04423772382595\\
338	3.04398208336191\\
339	3.04394323957163\\
340	3.04411586525945\\
341	3.04448849019289\\
342	3.04504652262122\\
343	3.04577383911337\\
344	3.0466514578858\\
345	3.04765644013381\\
346	3.04876279848872\\
347	3.04994299778011\\
348	3.05116919644631\\
349	3.05241427051146\\
350	3.05365265591911\\
351	3.05486099126621\\
352	3.05601846550385\\
353	3.05710690212399\\
354	3.0581108388895\\
355	3.059017749334\\
356	3.05981827095312\\
357	3.06050629590489\\
358	3.06107891305455\\
359	3.06153623448734\\
360	3.06188113341998\\
361	3.06211891806702\\
362	3.06225697124741\\
363	3.06230438747948\\
364	3.06227162116957\\
365	3.06217013632639\\
366	3.06201204855726\\
367	3.06180976629106\\
368	3.061575646423\\
369	3.06132167764292\\
370	3.06105920083235\\
371	3.06079867271744\\
372	3.06054947575337\\
373	3.06031977375731\\
374	3.06011641044412\\
375	3.05994484808529\\
376	3.05980914480946\\
377	3.05971196933654\\
378	3.05965465082806\\
379	3.05963726012856\\
380	3.05965871764796\\
381	3.05971692253847\\
382	3.05980889759576\\
383	3.05993094445939\\
384	3.06007880413686\\
385	3.06024781841574\\
386	3.06043308815301\\
387	3.06062962473943\\
388	3.06083249137135\\
389	3.06103693121696\\
390	3.06123848012737\\
391	3.06143306216723\\
392	3.06161706687778\\
393	3.06178740779783\\
394	3.06194156231869\\
395	3.06207759342189\\
396	3.06219415424832\\
397	3.06229047679254\\
398	3.06236634631429\\
399	3.06242206330491\\
400	3.06245839502968\\
401	3.06247651878107\\
402	3.06247795902271\\
403	3.06246452058201\\
404	3.06243821996965\\
405	3.06240121677586\\
406	3.06235574692756\\
407	3.06230405939427\\
408	3.06224835771132\\
409	3.06219074745224\\
410	3.06213319053423\\
411	3.06207746698914\\
412	3.06202514458451\\
413	3.06197755644114\\
414	3.06193578657245\\
415	3.06190066306987\\
416	3.06187275848213\\
417	3.06185239678615\\
418	3.06183966622543\\
419	3.06183443719836\\
420	3.06183638431438\\
421	3.06184501169999\\
422	3.06185968062723\\
423	3.06187963855335\\
424	3.06190404869872\\
425	3.06193201934831\\
426	3.06196263213642\\
427	3.06199496866225\\
428	3.06202813488116\\
429	3.06206128282023\\
430	3.06209362927398\\
431	3.06212447124322\\
432	3.06215319798422\\
433	3.06217929963542\\
434	3.06220237248025\\
435	3.06222212098788\\
436	3.06223835684596\\
437	3.06225099526011\\
438	3.06226004884354\\
439	3.06226561945615\\
440	3.06226788837652\\
441	3.06226710520155\\
442	3.06226357586947\\
443	3.06225765019179\\
444	3.06224970926072\\
445	3.06224015307164\\
446	3.06222938866629\\
447	3.06221781906362\\
448	3.06220583320249\\
449	3.06219379707549\\
450	3.06218204618724\\
451	3.06217087942493\\
452	3.06216055438454\\
453	3.06215128415493\\
454	3.0621432355235\\
455	3.06213652853292\\
456	3.06213123728932\\
457	3.0621273918976\\
458	3.06212498138037\\
459	3.06212395742349\\
460	3.06212423878221\\
461	3.06212571617867\\
462	3.06212825752251\\
463	3.0621317132918\\
464	3.06213592192099\\
465	3.06214071505467\\
466	3.06214592254146\\
467	3.06215137705933\\
468	3.06215691828221\\
469	3.06216239651732\\
470	3.06216767576219\\
471	3.06217263614958\\
472	3.06217717576736\\
473	3.06218121185765\\
474	3.06218468141522\\
475	3.06218754121947\\
476	3.06218976734589\\
477	3.06219135421311\\
478	3.06219231322867\\
479	3.06219267110219\\
480	3.06219246789758\\
481	3.06219175489672\\
482	3.06219059234594\\
483	3.06218904715404\\
484	3.06218719060578\\
485	3.06218509614947\\
486	3.06218283731021\\
487	3.06218048577289\\
488	3.06217810967112\\
489	3.06217577210958\\
490	3.06217352993927\\
491	3.06217143279659\\
492	3.06216952240969\\
493	3.06216783216808\\
494	3.06216638694494\\
495	3.062165203156\\
496	3.06216428903394\\
497	3.06216364509316\\
498	3.0621632647574\\
499	3.06216313512007\\
500	3.06216323780677\\
501	3.13716323780677\\
502	3.06216323780677\\
503	3.06418388676137\\
504	3.06620465897893\\
505	3.0682255231491\\
506	3.07024644816705\\
507	3.07226740392916\\
508	3.07428836203663\\
509	3.07630929639631\\
510	3.07833018371154\\
511	3.07601435697931\\
512	3.0741272187864\\
513	3.07638688421237\\
514	3.07838872013186\\
515	3.08014576401641\\
516	3.08166976743027\\
517	3.08297131978811\\
518	3.08405996035077\\
519	3.08494427956019\\
520	3.08563201071519\\
521	3.08638086735711\\
522	3.08739828138896\\
523	3.08840718728643\\
524	3.08919499431464\\
525	3.08979867727777\\
526	3.09025168968618\\
527	3.09058437350276\\
528	3.09082432312032\\
529	3.0909967085379\\
530	3.09112456217652\\
531	3.09121453412909\\
532	3.09124691893345\\
533	3.09120760658072\\
534	3.0911105457264\\
535	3.09097850680696\\
536	3.0908292853092\\
537	3.09067635232563\\
538	3.09052941757295\\
539	3.09039491597362\\
540	3.09027642755198\\
541	3.09017587758008\\
542	3.09009603131972\\
543	3.0900415714634\\
544	3.09001708401363\\
545	3.09002502252451\\
546	3.09006561188602\\
547	3.09013746761301\\
548	3.09023811401672\\
549	3.090364416768\\
550	3.09051294316223\\
551	3.16551294316223\\
552	3.09051294316223\\
553	3.09272701911382\\
554	3.09494867920731\\
555	3.0971737846205\\
556	3.09939827384226\\
557	3.10161836150776\\
558	3.10383066540682\\
559	3.10603227579124\\
560	3.10822077865991\\
561	3.10606725242576\\
562	3.10434521996393\\
563	3.10676125954346\\
564	3.10889991427719\\
565	3.11077427753534\\
566	3.11239657844722\\
567	3.11377826839627\\
568	3.11493008026695\\
569	3.11586206934175\\
570	3.1165836428204\\
571	3.11735377946013\\
572	3.11838021487796\\
573	3.11938638827298\\
574	3.12016144035267\\
575	3.1207445830371\\
576	3.12117137542868\\
577	3.12147409967034\\
578	3.12168209703016\\
579	3.12182207014861\\
580	3.12191835606123\\
581	3.12197870667473\\
582	3.12198440467448\\
583	3.12192226209796\\
584	3.12180700576269\\
585	3.12166195143967\\
586	3.12150519799175\\
587	3.12135030539387\\
588	3.12120688686051\\
589	3.12108112547508\\
590	3.12097622433512\\
591	3.1208936345887\\
592	3.12083557007534\\
593	3.12080609601992\\
594	3.12080912748074\\
595	3.12084641494886\\
596	3.12091747391819\\
597	3.12102022776579\\
598	3.12115154538439\\
599	3.12130768916946\\
600	3.12148468682843\\
601	3.19648468682843\\
602	3.12148468682843\\
603	3.12372062829973\\
604	3.1259613865065\\
605	3.12820267766251\\
606	3.13044037640435\\
607	3.13267071020184\\
608	3.13489037993714\\
609	3.13709662130474\\
610	3.13928721919621\\
611	3.13713503491969\\
612	3.13541512745085\\
613	3.13783272333758\\
614	3.13997130803307\\
615	3.14184432101705\\
616	3.14346433259457\\
617	3.14484312263557\\
618	3.14599173296958\\
619	3.14692050247179\\
620	3.14763909189084\\
621	3.14840661089671\\
622	3.14943073697928\\
623	3.15043485120172\\
624	3.15120817669746\\
625	3.15179004457453\\
626	3.15221609184184\\
627	3.15251864025161\\
628	3.15272703594656\\
629	3.15286795572535\\
630	3.15296568441099\\
631	3.15302790502105\\
632	3.15303583278596\\
633	3.15297622370095\\
634	3.15286375104391\\
635	3.15272166939314\\
636	3.1525680092677\\
637	3.1524162597142\\
638	3.15227596442258\\
639	3.15215324174503\\
640	3.15205123762436\\
641	3.15197135551451\\
642	3.15191577127347\\
643	3.15188852056247\\
644	3.15189349605762\\
645	3.15193243296833\\
646	3.15200483907813\\
647	3.15210863780387\\
648	3.15224070553942\\
649	3.15239731897286\\
650	3.15257452592739\\
651	3.22757452592739\\
652	3.15257452592739\\
653	3.15481005686171\\
654	3.15705025962352\\
655	3.15929088301751\\
656	3.16152783413318\\
657	3.16375737197531\\
658	3.16597622730101\\
659	3.16818166335917\\
660	3.17037148972956\\
661	3.16821858761264\\
662	3.16649802049327\\
663	3.16891503194443\\
664	3.17105313118425\\
665	3.17292576442164\\
666	3.17454550519685\\
667	3.17592413332801\\
668	3.17707268758099\\
669	3.17800150108993\\
670	3.17872022656593\\
671	3.17948796384535\\
672	3.18051237998031\\
673	3.18151684509427\\
674	3.18229056978592\\
675	3.18287287168045\\
676	3.18329937448903\\
677	3.18360238718175\\
678	3.18381124392578\\
679	3.18395261059805\\
680	3.18405076235348\\
681	3.18411337394061\\
682	3.18412165376736\\
683	3.18406235243185\\
684	3.18395013928247\\
685	3.18380826650253\\
686	3.18365476371806\\
687	3.18350312047642\\
688	3.18336288222785\\
689	3.18324017018929\\
690	3.18313813410235\\
691	3.18305818197414\\
692	3.18300249478947\\
693	3.18297511373743\\
694	3.18297993725743\\
695	3.18301870639359\\
696	3.18309093467682\\
697	3.18319455103985\\
698	3.18332643703598\\
699	3.18348287405178\\
700	3.1836599140652\\
701	3.1086599140652\\
702	3.1836599140652\\
703	3.181855325186\\
704	3.18005542787664\\
705	3.17825597186081\\
706	3.17645286452442\\
707	3.17464236459044\\
708	3.17282120201189\\
709	3.17098663877652\\
710	3.16913648281741\\
711	3.17161693433186\\
712	3.17367226265268\\
713	3.17157142104711\\
714	3.16970711541176\\
715	3.16806670341545\\
716	3.16663924874393\\
717	3.16541535215383\\
718	3.16438697963411\\
719	3.16354729452496\\
720	3.16289049865479\\
721	3.16216028276214\\
722	3.16114946699055\\
723	3.16013557113051\\
724	3.15933301201175\\
725	3.15870718562443\\
726	3.1582268363195\\
727	3.15786361583148\\
728	3.15759169467945\\
729	3.15738742182927\\
730	3.15722902822336\\
731	3.15711090637515\\
732	3.15705369099118\\
733	3.15707236559546\\
734	3.15715371235498\\
735	3.15727544955314\\
736	3.15742001965333\\
737	3.15757397088407\\
738	3.15772742745258\\
739	3.15787363655595\\
740	3.15800858168841\\
741	3.15812981244654\\
742	3.15823397010027\\
743	3.15831572024996\\
744	3.15836977842288\\
745	3.15839296727759\\
746	3.15838433925667\\
747	3.15834458123642\\
748	3.15827551567526\\
749	3.15817968293044\\
750	3.15805999167226\\
751	3.08305999167226\\
752	3.15805999167226\\
753	3.15384748901815\\
754	3.14962447249787\\
755	3.14539496696323\\
756	3.14116300156456\\
757	3.13693240568458\\
758	3.13270667471721\\
759	3.12848889213943\\
760	3.12428169671773\\
761	3.12441582282422\\
762	3.12411929695867\\
763	3.11980100006546\\
764	3.11598060367966\\
765	3.11263578851134\\
766	3.10974605166408\\
767	3.10729251929232\\
768	3.10525779633867\\
769	3.10362584375501\\
770	3.10238187555046\\
771	3.10126198405073\\
772	3.10005383249602\\
773	3.09902309369567\\
774	3.09835812781653\\
775	3.09799326289157\\
776	3.09786877550906\\
777	3.09793023834539\\
778	3.09812793990053\\
779	3.09841636686734\\
780	3.09875374125946\\
781	3.09911607786738\\
782	3.09950693461202\\
783	3.0999256802395\\
784	3.10034603719357\\
785	3.10073614802967\\
786	3.1010723157384\\
787	3.10133794534454\\
788	3.10152262530733\\
789	3.10162133115936\\
790	3.10163373610568\\
791	3.10156277844039\\
792	3.10141197154911\\
793	3.10118414238041\\
794	3.10088320183554\\
795	3.10051584637755\\
796	3.10009129387624\\
797	3.09962034676259\\
798	3.09911461249478\\
799	3.09858585706474\\
800	3.0980454707139\\
801	3.0230454707139\\
802	3.0980454707139\\
803	3.09551036725303\\
804	3.09300094001939\\
805	3.09052432554039\\
806	3.08808647170253\\
807	3.0856919781166\\
808	3.08334403602529\\
809	3.08104444683006\\
810	3.07879370203392\\
811	3.08089646804434\\
812	3.08255791189342\\
813	3.08009107257592\\
814	3.07792715913701\\
815	3.07604857749722\\
816	3.07443855603176\\
817	3.07308117109742\\
818	3.07196138932417\\
819	3.07106511577157\\
820	3.0703792397174\\
821	3.06964272674201\\
822	3.06864840306822\\
823	3.06767336155104\\
824	3.06692695406528\\
825	3.06636789786401\\
826	3.06595913224217\\
827	3.06566741214978\\
828	3.06546293252564\\
829	3.06531897895874\\
830	3.06521160170661\\
831	3.06513370573289\\
832	3.06510468417924\\
833	3.06513828089807\\
834	3.06522033313495\\
835	3.06532824384766\\
836	3.06544478123466\\
837	3.06555733474998\\
838	3.06565726412121\\
839	3.06573933141387\\
840	3.06580120739385\\
841	3.06584221208826\\
842	3.06586079138556\\
843	3.06585344132044\\
844	3.06581672182579\\
845	3.06574926697864\\
846	3.06565184352616\\
847	3.06552670050487\\
848	3.0653770311493\\
849	3.06520653034654\\
850	3.0650190330565\\
851	2.9900190330565\\
852	3.0650190330565\\
853	3.04864260539933\\
854	3.0322640043045\\
855	3.01588710839536\\
856	2.99951561977009\\
857	2.98315288130692\\
858	2.96680176772084\\
859	2.95046463502666\\
860	2.93414331570686\\
861	2.92216420266549\\
862	2.90975176387691\\
863	2.89401852289708\\
864	2.88011774736357\\
865	2.86796875290447\\
866	2.8574984969627\\
867	2.84864084648191\\
868	2.84133593276998\\
869	2.8355295787995\\
870	2.83117278678954\\
871	2.827971191951\\
872	2.82568554996108\\
873	2.82451691504569\\
874	2.82451822611928\\
875	2.82546433784754\\
876	2.82715016671122\\
877	2.82938834733955\\
878	2.83200715162871\\
879	2.83484864039277\\
880	2.83776702123227\\
881	2.8406416500937\\
882	2.84338582776702\\
883	2.84591658273513\\
884	2.84813885706858\\
885	2.84997009943684\\
886	2.85135360874691\\
887	2.85225515776283\\
888	2.85266008531448\\
889	2.8525707965863\\
890	2.85200461856869\\
891	2.85099112823653\\
892	2.84956841203262\\
893	2.84778078163584\\
894	2.84567917104905\\
895	2.84332096549282\\
896	2.84076772907512\\
897	2.83808250561286\\
898	2.83532759263768\\
899	2.83256271155662\\
900	2.82984350824506\\
901	2.82722037643869\\
902	2.82473767721148\\
903	2.82243330912972\\
904	2.8203383778919\\
905	2.81847684726806\\
906	2.81686532497574\\
907	2.81551311725546\\
908	2.81442253670579\\
909	2.81358940451444\\
910	2.81300369942501\\
911	2.81265031233217\\
912	2.81250986366578\\
913	2.81255954139205\\
914	2.81277393637241\\
915	2.81312587334599\\
916	2.81358723332569\\
917	2.81412974712076\\
918	2.81472573404441\\
919	2.81534876508776\\
920	2.81597423650113\\
921	2.81657984505706\\
922	2.81714596108561\\
923	2.81765590002356\\
924	2.81809609653902\\
925	2.81845618620316\\
926	2.81872899954087\\
927	2.8189104742488\\
928	2.81899949351337\\
929	2.81899766044197\\
930	2.81890901996749\\
931	2.81873974023382\\
932	2.81849776555293\\
933	2.81819245260577\\
934	2.81783420074616\\
935	2.81743408628615\\
936	2.81700350968887\\
937	2.81655386366061\\
938	2.81609622908365\\
939	2.81564110450678\\
940	2.81519817356102\\
941	2.81477611327633\\
942	2.81438244491057\\
943	2.81402342761837\\
944	2.81370399412925\\
945	2.81342772659325\\
946	2.81319686988672\\
947	2.81301237894001\\
948	2.81287399605048\\
949	2.81278035368746\\
950	2.81272909799069\\
951	2.81271702801218\\
952	2.81274024574684\\
953	2.81279431212828\\
954	2.81287440441456\\
955	2.81297547073602\\
956	2.81309237800306\\
957	2.81322004985888\\
958	2.81335359189367\\
959	2.81348840189773\\
960	2.81362026350436\\
961	2.81374542214418\\
962	2.81386064278402\\
963	2.8139632494442\\
964	2.81405114696447\\
965	2.81412282591363\\
966	2.81417735190336\\
967	2.81421434086959\\
968	2.81423392212147\\
969	2.81423669112868\\
970	2.81422365412315\\
971	2.81419616663375\\
972	2.81415586805558\\
973	2.81410461428609\\
974	2.81404441034211\\
975	2.81397734471477\\
976	2.81390552702843\\
977	2.81383103035436\\
978	2.81375583929696\\
979	2.81368180472728\\
980	2.81361060579323\\
981	2.81354371959396\\
982	2.81348239867376\\
983	2.81342765627453\\
984	2.81338025908795\\
985	2.81334072707471\\
986	2.81330933976942\\
987	2.81328614836873\\
988	2.81327099280749\\
989	2.81326352296343\\
990	2.81326322309428\\
991	2.81326943860083\\
992	2.81328140422415\\
993	2.81329827282106\\
994	2.81331914391816\\
995	2.81334309131623\\
996	2.81336918910225\\
997	2.81339653552076\\
998	2.81342427425759\\
999	2.81345161279367\\
1000	2.81347783759117\\
};
\end{axis}
\end{tikzpicture}%
\caption{Sterowanie PID dla parametrów $K = 1,212$, $T_i = 15$, $T_d = 4$}
\end{figure}

\begin{figure}[H]
\centering
% This file was created by matlab2tikz.
%
%The latest updates can be retrieved from
%  http://www.mathworks.com/matlabcentral/fileexchange/22022-matlab2tikz-matlab2tikz
%where you can also make suggestions and rate matlab2tikz.
%
\definecolor{mycolor1}{rgb}{0.00000,0.44700,0.74100}%
\definecolor{mycolor2}{rgb}{0.85000,0.32500,0.09800}%
%
\begin{tikzpicture}

\begin{axis}[%
width=4.272in,
height=2.477in,
at={(0.717in,0.437in)},
scale only axis,
xmin=0,
xmax=1000,
xlabel style={font=\color{white!15!black}},
xlabel={k},
ymin=0.5,
ymax=1.4,
ylabel style={font=\color{white!15!black}},
ylabel={Y(k)},
axis background/.style={fill=white},
legend style={legend cell align=left, align=left, draw=white!15!black}
]
\addplot[const plot, color=mycolor1] table[row sep=crcr] {%
1	0.9\\
2	0.9\\
3	0.9\\
4	0.9\\
5	0.9\\
6	0.9\\
7	0.9\\
8	0.9\\
9	0.9\\
10	0.9\\
11	0.9\\
12	0.9\\
13	0.9\\
14	0.9\\
15	0.9\\
16	0.9\\
17	0.9\\
18	0.9\\
19	0.9\\
20	0.9\\
21	0.9\\
22	0.9003968325\\
23	0.90110505465075\\
24	0.901792976327444\\
25	0.902646481628594\\
26	0.903821675824292\\
27	0.90544848711148\\
28	0.907633879431284\\
29	0.910464715877318\\
30	0.914010308352824\\
31	0.918324685633407\\
32	0.923425663098293\\
33	0.929303840647585\\
34	0.935965438923209\\
35	0.943416013287854\\
36	0.951634745712284\\
37	0.960579528545015\\
38	0.970191311315836\\
39	0.980397805132888\\
40	0.991116627926128\\
41	1.00225796377292\\
42	1.01372812744431\\
43	1.02543354268672\\
44	1.03728153540572\\
45	1.04917966462057\\
46	1.06103743783932\\
47	1.07276903997461\\
48	1.08429539600579\\
49	1.09554568280648\\
50	1.10645838832648\\
51	1.1169820014422\\
52	1.12707532624939\\
53	1.13670736608879\\
54	1.14585695891761\\
55	1.15451243356308\\
56	1.16267122714802\\
57	1.1703392756374\\
58	1.17753016980654\\
59	1.18426415389599\\
60	1.19056702825782\\
61	1.19646900406982\\
62	1.20200355171508\\
63	1.20720628606024\\
64	1.21211392252238\\
65	1.21676331098508\\
66	1.22119054329472\\
67	1.22543014467057\\
68	1.22951436988186\\
69	1.23347262033773\\
70	1.23733099005462\\
71	1.24111194231149\\
72	1.24483411402396\\
73	1.24851224055036\\
74	1.25215718994361\\
75	1.2557760941581\\
76	1.2593725653644\\
77	1.26294698587274\\
78	1.26649685905318\\
79	1.2700172072387\\
80	1.27350100195094\\
81	1.27693961193569\\
82	1.28032325521044\\
83	1.28364144248476\\
84	1.28688340083927\\
85	1.29003846826683\\
86	1.29309645134434\\
87	1.29604793982105\\
88	1.29888457339305\\
89	1.30159925748491\\
90	1.30418632645296\\
91	1.30664165417363\\
92	1.30896271341634\\
93	1.31114858667364\\
94	1.31319993219844\\
95	1.31511890985804\\
96	1.31690907206215\\
97	1.31857522547928\\
98	1.32012326954189\\
99	1.32156001786129\\
100	1.32289300862979\\
101	1.32413030988634\\
102	1.32528032517879\\
103	1.32635160469043\\
104	1.32735266633564\\
105	1.32829183069443\\
106	1.32917707297408\\
107	1.33001589448023\\
108	1.33081521536968\\
109	1.33158128975868\\
110	1.33231964359104\\
111	1.33303503504382\\
112	1.33373143667863\\
113	1.33441203804415\\
114	1.33507926700861\\
115	1.33573482775472\\
116	1.33637975310617\\
117	1.33701446867403\\
118	1.33763886621073\\
119	1.33825238353507\\
120	1.3388540884376\\
121	1.33944276408553\\
122	1.34001699361179\\
123	1.34057524178494\\
124	1.34111593190635\\
125	1.34163751635832\\
126	1.34213853952223\\
127	1.34261769209022\\
128	1.34307385609862\\
129	1.34350614030807\\
130	1.34391390583757\\
131	1.34429678222091\\
132	1.34465467428926\\
133	1.34498776048878\\
134	1.34529648341476\\
135	1.34558153348147\\
136	1.34584382674907\\
137	1.34608447799663\\
138	1.34630477016301\\
139	1.34650612127905\\
140	1.34669004998619\\
141	1.34685814068223\\
142	1.34701200925797\\
143	1.34715327029247\\
144	1.34728350646447\\
145	1.34740424081598\\
146	1.34751691237662\\
147	1.3476228555266\\
148	1.3477232833471\\
149	1.34781927508116\\
150	1.34791176771032\\
151	1.34800155154383\\
152	1.3480892696202\\
153	1.34817542063768\\
154	1.34826036506068\\
155	1.34834433399507\\
156	1.34842744038619\\
157	1.34850969206888\\
158	1.34859100618924\\
159	1.34867122452092\\
160	1.34875012921456\\
161	1.34843062605893\\
162	1.34779790638695\\
163	1.34716201046868\\
164	1.34629858865274\\
165	1.34501935835889\\
166	1.34316774735261\\
167	1.34061500731763\\
168	1.33725674965034\\
169	1.33300986015298\\
170	1.32780975360351\\
171	1.32163087879629\\
172	1.31447733980858\\
173	1.30634095742823\\
174	1.29721934245738\\
175	1.2871417498446\\
176	1.27616294238912\\
177	1.26435794461572\\
178	1.25181757257629\\
179	1.23864463902656\\
180	1.22495074555071\\
181	1.21085225716361\\
182	1.19646593931354\\
183	1.18190777447857\\
184	1.16729311278276\\
185	1.1527343604117\\
186	1.13833773744747\\
187	1.12420083605252\\
188	1.11041083998539\\
189	1.09704328720659\\
190	1.08416127526453\\
191	1.07181510132204\\
192	1.06004237942099\\
193	1.04886844732197\\
194	1.03830679929219\\
195	1.02835961410069\\
196	1.01901856346322\\
197	1.01026589434154\\
198	1.00207569286305\\
199	0.99441525678027\\
200	0.987246519257727\\
201	0.980527475425177\\
202	0.974213563276706\\
203	0.968258961047482\\
204	0.962617790148347\\
205	0.957245223310229\\
206	0.952098482771791\\
207	0.947137703948856\\
208	0.942326646464351\\
209	0.937633244101546\\
210	0.933029992505431\\
211	0.928494179027628\\
212	0.924007963969099\\
213	0.919558326530308\\
214	0.915136890559486\\
215	0.91073964485583\\
216	0.906366572795652\\
217	0.90202120746144\\
218	0.897710129990085\\
219	0.893442429459254\\
220	0.889229142305992\\
221	0.885082688289417\\
222	0.881016318529722\\
223	0.877043589281272\\
224	0.873177873015803\\
225	0.869431916351998\\
226	0.865817452457291\\
227	0.862344873654896\\
228	0.859022968003598\\
229	0.855858721625072\\
230	0.852857186634102\\
231	0.850021412761617\\
232	0.847352439202266\\
233	0.844849341904303\\
234	0.842509330469255\\
235	0.840327888037158\\
236	0.838298946976964\\
237	0.8364150928618\\
238	0.834667789078279\\
239	0.833047614494705\\
240	0.831544506883561\\
241	0.830148005238913\\
242	0.828847484722811\\
243	0.827632378686407\\
244	0.826492383009918\\
245	0.825417638859881\\
246	0.824398890844884\\
247	0.823427618437965\\
248	0.822496139403636\\
249	0.82159768479983\\
250	0.820726445901373\\
251	0.819877594095656\\
252	0.8190472754199\\
253	0.818232581933204\\
254	0.817431502539435\\
255	0.816642856196359\\
256	0.815866210662862\\
257	0.815101790052719\\
258	0.814350374485473\\
259	0.813613195059669\\
260	0.81289182722976\\
261	0.812188085455559\\
262	0.811503921723339\\
263	0.810841330222326\\
264	0.810202260111545\\
265	0.809588537941525\\
266	0.809001800914883\\
267	0.808443441789924\\
268	0.807914565861989\\
269	0.807415960107068\\
270	0.806948074248767\\
271	0.806511013219192\\
272	0.806104540231688\\
273	0.805728089472016\\
274	0.805380787246615\\
275	0.805061480302851\\
276	0.80476876995621\\
277	0.804501050621638\\
278	0.80425655134821\\
279	0.804033378994675\\
280	0.803829561754153\\
281	0.803643091834832\\
282	0.803471966225085\\
283	0.80331422461087\\
284	0.803167983665529\\
285	0.803031467092108\\
286	0.802903030961207\\
287	0.802781184048775\\
288	0.802664603033969\\
289	0.802552142563663\\
290	0.802442840324401\\
291	0.802335917381945\\
292	0.802230774151343\\
293	0.802126982445107\\
294	0.802024274113195\\
295	0.801922526835577\\
296	0.801821747656825\\
297	0.801722054862931\\
298	0.801623658794884\\
299	0.801526842172671\\
300	0.801431940469357\\
301	0.801339322829474\\
302	0.801249373971504\\
303	0.801162477452665\\
304	0.801079000607945\\
305	0.800999281406322\\
306	0.800923617397467\\
307	0.800852256853822\\
308	0.800785392147407\\
309	0.80072315533958\\
310	0.800665615906437\\
311	0.801009432405365\\
312	0.801668802419266\\
313	0.802258407929274\\
314	0.80286868743879\\
315	0.803575929057992\\
316	0.804443966475922\\
317	0.805525694042367\\
318	0.806864419105877\\
319	0.808495067966452\\
320	0.810445260190617\\
321	0.812713329298985\\
322	0.815276397923495\\
323	0.818135464426882\\
324	0.821303566983216\\
325	0.824780441254627\\
326	0.828554991320856\\
327	0.832607407208244\\
328	0.836910973889249\\
329	0.84143361130848\\
330	0.846139180280623\\
331	0.850989911101292\\
332	0.855949432944834\\
333	0.860982601805241\\
334	0.86605354590501\\
335	0.871125901558076\\
336	0.876164268406323\\
337	0.881135331169228\\
338	0.886008704426848\\
339	0.890757548631504\\
340	0.895358998344706\\
341	0.899794360797135\\
342	0.904049000146178\\
343	0.908112074305717\\
344	0.911976408823433\\
345	0.915638471955449\\
346	0.919098255955204\\
347	0.922359021390467\\
348	0.925426944610205\\
349	0.928310700481362\\
350	0.931021005858244\\
351	0.93357014816443\\
352	0.935971529506032\\
353	0.93823925036363\\
354	0.940387730347073\\
355	0.942431350263339\\
356	0.944384113805218\\
357	0.946259340480476\\
358	0.948069400997142\\
359	0.949825501867965\\
360	0.951537522588247\\
361	0.953213905921664\\
362	0.9548615988465\\
363	0.956486038898269\\
364	0.958091179825445\\
365	0.959679551891123\\
366	0.961252353397458\\
367	0.962809569644816\\
368	0.964350114472765\\
369	0.96587198878026\\
370	0.967372450132479\\
371	0.968848187619044\\
372	0.970295496494612\\
373	0.971710447794478\\
374	0.973089048938837\\
375	0.974427392076537\\
376	0.975721787466361\\
377	0.976968879664437\\
378	0.978165744813555\\
379	0.979309967933086\\
380	0.980399699740986\\
381	0.981433693157042\\
382	0.982411320204611\\
383	0.983332570516038\\
384	0.98419803303441\\
385	0.985008862792712\\
386	0.985766734861524\\
387	0.986473787706858\\
388	0.987132558294246\\
389	0.987745911308887\\
390	0.988316964831212\\
391	0.988849014714398\\
392	0.989345459761372\\
393	0.989809729602797\\
394	0.990245216946151\\
395	0.990655215611379\\
396	0.99104286550074\\
397	0.991411105377096\\
398	0.991762634051733\\
399	0.992099880315799\\
400	0.992424981694512\\
401	0.992739771866375\\
402	0.993045776375933\\
403	0.99334421608236\\
404	0.993636017630296\\
405	0.993921830105389\\
406	0.994202046945357\\
407	0.994476832117392\\
408	0.994746149543269\\
409	0.995009794752803\\
410	0.995267427772114\\
411	0.995518606302736\\
412	0.995762818317789\\
413	0.995999513288717\\
414	0.996228131356646\\
415	0.996448129872632\\
416	0.996659006847055\\
417	0.996860320966798\\
418	0.9970517079563\\
419	0.99723289317215\\
420	0.997403700427938\\
421	0.997564057144436\\
422	0.997713996007861\\
423	0.997853653394707\\
424	0.99798326488428\\
425	0.998103158229202\\
426	0.998213744189543\\
427	0.998315505658072\\
428	0.998408985513067\\
429	0.998494773631901\\
430	0.998573493484367\\
431	0.998645788700735\\
432	0.998712309977194\\
433	0.99877370264225\\
434	0.998830595163324\\
435	0.99888358882487\\
436	0.998933248759328\\
437	0.998980096461647\\
438	0.999024603868284\\
439	0.999067189033805\\
440	0.999108213393495\\
441	0.999147980559674\\
442	0.999186736563474\\
443	0.999224671423136\\
444	0.999261921894839\\
445	0.999298575242889\\
446	0.999334673852745\\
447	0.999370220502713\\
448	0.999405184107965\\
449	0.999439505753393\\
450	0.999473104839233\\
451	0.999505885174797\\
452	0.999537740870424\\
453	0.999568561895227\\
454	0.999598239187677\\
455	0.999626669226933\\
456	0.999653757994291\\
457	0.999679424275753\\
458	0.99970360227776\\
459	0.999726243548265\\
460	0.999747318213877\\
461	0.999766815560692\\
462	0.999784744001079\\
463	0.999801130481108\\
464	0.999816019393256\\
465	0.999829471066452\\
466	0.999841559910491\\
467	0.999852372294374\\
468	0.999862004238387\\
469	0.99987055899789\\
470	0.999878144613066\\
471	0.999884871493521\\
472	0.999890850099949\\
473	0.99989618877728\\
474	0.999900991785206\\
475	0.999905357562963\\
476	0.999909377255975\\
477	0.999913133522795\\
478	0.999916699631873\\
479	0.999920138849216\\
480	0.999923504110326\\
481	0.999926837962792\\
482	0.999930172759977\\
483	0.999933531081192\\
484	0.999936926349862\\
485	0.999940363618286\\
486	0.999943840485815\\
487	0.999947348116464\\
488	0.999950872322155\\
489	0.999954394678819\\
490	0.999957893644386\\
491	0.999961345650168\\
492	0.999964726140156\\
493	0.999968010536166\\
494	0.999971175110496\\
495	0.999974197751661\\
496	0.999977058612702\\
497	0.999979740635494\\
498	0.999982229948198\\
499	0.999984516136535\\
500	0.999986592392781\\
501	0.999988455549186\\
502	0.999990106005001\\
503	0.999991547558265\\
504	0.999992787155045\\
505	0.999993834569912\\
506	0.999994702032071\\
507	0.999995403811769\\
508	0.999995955781397\\
509	0.999996374965183\\
510	0.99999667909049\\
511	1.00039371700121\\
512	1.00110205978487\\
513	1.00170452890047\\
514	1.00223442784612\\
515	1.00272038028179\\
516	1.00318686955165\\
517	1.0036547213546\\
518	1.00414153526502\\
519	1.0046620702517\\
520	1.00522858884009\\
521	1.00582821847735\\
522	1.00643043429556\\
523	1.00703349693446\\
524	1.00765531343145\\
525	1.00830831897024\\
526	1.00900037010485\\
527	1.00973551670104\\
528	1.0105146676046\\
529	1.0113361632921\\
530	1.0121962672022\\
531	1.01309091283139\\
532	1.01401816767096\\
533	1.01497748771482\\
534	1.01596699536437\\
535	1.01698282948864\\
536	1.01801983858944\\
537	1.01907214104967\\
538	1.02013357339064\\
539	1.02119804450322\\
540	1.02225981124717\\
541	1.02331361186821\\
542	1.02435455430727\\
543	1.02537791714391\\
544	1.02637915665831\\
545	1.02735409872487\\
546	1.02829911650706\\
547	1.0292112294243\\
548	1.03008814115479\\
549	1.03092823124144\\
550	1.03173051217918\\
551	1.03249456604228\\
552	1.03322048342288\\
553	1.03390882289695\\
554	1.0345605828393\\
555	1.0351771630118\\
556	1.03576030705612\\
557	1.03631203200471\\
558	1.03683455310695\\
559	1.03733021005661\\
560	1.03780139894673\\
561	1.03864646009087\\
562	1.03978062933444\\
563	1.04078845569021\\
564	1.04170584153618\\
565	1.04256377732673\\
566	1.04338885994869\\
567	1.04420376095814\\
568	1.04502765111153\\
569	1.04587658622872\\
570	1.04676385836263\\
571	1.04767742089744\\
572	1.04858747919757\\
573	1.04949291812835\\
574	1.05041206033849\\
575	1.05135752372197\\
576	1.05233715226248\\
577	1.05335482517619\\
578	1.05441115872375\\
579	1.05550411329506\\
580	1.05662951687788\\
581	1.05778283856339\\
582	1.05896166786602\\
583	1.06016497857712\\
584	1.06139042209532\\
585	1.06263369927022\\
586	1.06388927360866\\
587	1.06515094659502\\
588	1.06641231640585\\
589	1.06766713837913\\
590	1.06890960306704\\
591	1.07013446895497\\
592	1.07133694801071\\
593	1.07251250283952\\
594	1.07365684893903\\
595	1.07476614009545\\
596	1.07583713770013\\
597	1.07686729990107\\
598	1.0778548089296\\
599	1.07879855167543\\
600	1.07969806582385\\
601	1.08055346597064\\
602	1.08136537272994\\
603	1.0821348631683\\
604	1.08286343444368\\
605	1.08355295816932\\
606	1.08420561677659\\
607	1.08482382811431\\
608	1.08541016664575\\
609	1.08596728733359\\
610	1.08649785649315\\
611	1.08740029874762\\
612	1.08858979157341\\
613	1.08965096349308\\
614	1.09061989152446\\
615	1.09152769606599\\
616	1.0924010642635\\
617	1.09326272354918\\
618	1.09413187163108\\
619	1.09502456783777\\
620	1.09595408966757\\
621	1.09690837119914\\
622	1.09785761164597\\
623	1.09880070141734\\
624	1.09975596619931\\
625	1.10073602132407\\
626	1.10174870703268\\
627	1.10279790126932\\
628	1.10388422439776\\
629	1.10500564847573\\
630	1.10615802224048\\
631	1.1073368450397\\
632	1.10853974483626\\
633	1.10976573995174\\
634	1.11101253131745\\
635	1.11227587427123\\
636	1.11355029165787\\
637	1.1148296486198\\
638	1.11610761046075\\
639	1.11737800203711\\
640	1.11863508459076\\
641	1.11987368720899\\
642	1.12108909123382\\
643	1.12227682652874\\
644	1.12343267306638\\
645	1.12455284575047\\
646	1.12563416319222\\
647	1.12667413638053\\
648	1.12767099562436\\
649	1.12862367086563\\
650	1.12953173769107\\
651	1.130395343462\\
652	1.13121513656646\\
653	1.13199221710025\\
654	1.13272810082519\\
655	1.13342467390489\\
656	1.13408412968347\\
657	1.13470889373864\\
658	1.1353015455611\\
659	1.13586474293777\\
660	1.13640115330537\\
661	1.13730920133227\\
662	1.13850406499274\\
663	1.1395703725617\\
664	1.14054419953204\\
665	1.14145666520854\\
666	1.14233445634978\\
667	1.14320030092966\\
668	1.14407339829573\\
669	1.14496981062909\\
670	1.14590281955634\\
671	1.1468603645655\\
672	1.14781265145131\\
673	1.14875857826562\\
674	1.14971647939645\\
675	1.15069897987251\\
676	1.15171393046093\\
677	1.15276522028194\\
678	1.15385348133693\\
679	1.15497669759129\\
680	1.1561307297708\\
681	1.15731108911111\\
682	1.15851541519744\\
683	1.15974273756478\\
684	1.16099076781911\\
685	1.16225527131521\\
686	1.16353078015669\\
687	1.16481116791043\\
688	1.16609010741786\\
689	1.16736143015747\\
690	1.16861940307084\\
691	1.1698588600365\\
692	1.17107508631148\\
693	1.17226361484566\\
694	1.17342022792995\\
695	1.17454114208971\\
696	1.17562317694168\\
697	1.17666384395119\\
698	1.17766137346496\\
699	1.17861469511554\\
700	1.17952338392319\\
701	1.18038758651215\\
702	1.18120795044366\\
703	1.18198557497114\\
704	1.18272197506439\\
705	1.1834190362023\\
706	1.184078951199\\
707	1.18470414529443\\
708	1.18529719786099\\
709	1.18586076680438\\
710	1.18639751992523\\
711	1.18651221831475\\
712	1.18629092785924\\
713	1.18615266683395\\
714	1.18606693990674\\
715	1.18600765286384\\
716	1.18595255672064\\
717	1.18588275575563\\
718	1.18578227434336\\
719	1.18563767719582\\
720	1.18543773756889\\
721	1.18519615310595\\
722	1.18494418921743\\
723	1.1846842295491\\
724	1.18439879803255\\
725	1.18407564975217\\
726	1.18370691998225\\
727	1.18328839333299\\
728	1.18281887737155\\
729	1.18229966683755\\
730	1.18173408620468\\
731	1.18112576964288\\
732	1.18047621133567\\
733	1.17978551836803\\
734	1.17905514733567\\
735	1.17828857620041\\
736	1.17749063122264\\
737	1.17666694138931\\
738	1.17582349986878\\
739	1.17496631501848\\
740	1.17410113607155\\
741	1.1732333177918\\
742	1.17236792827358\\
743	1.17150994331535\\
744	1.17066423386293\\
745	1.16983536700897\\
746	1.16902741933029\\
747	1.16824386756392\\
748	1.16748753947557\\
749	1.16676061088764\\
750	1.16606463744184\\
751	1.16540060740034\\
752	1.16476899294972\\
753	1.16416978188346\\
754	1.16360249786905\\
755	1.16306623195307\\
756	1.16255969430794\\
757	1.16208128022535\\
758	1.16162914211492\\
759	1.16120126141339\\
760	1.16079551601878\\
761	1.16001365145738\\
762	1.15894037457367\\
763	1.15798052877069\\
764	1.15707931156364\\
765	1.15618992529156\\
766	1.15527268133295\\
767	1.15429419618655\\
768	1.1532266686384\\
769	1.15204722902929\\
770	1.15073735308466\\
771	1.1493052366589\\
772	1.14777803356319\\
773	1.14615587666895\\
774	1.14442199873985\\
775	1.14256794553579\\
776	1.14059212541136\\
777	1.13849855678803\\
778	1.1362957892123\\
779	1.13399597668459\\
780	1.13161408450484\\
781	1.12916588880959\\
782	1.12666530645259\\
783	1.12412493653425\\
784	1.12155850920695\\
785	1.11898121798116\\
786	1.11640875182568\\
787	1.11385653142127\\
788	1.11133911657954\\
789	1.10886975653239\\
790	1.10646005885781\\
791	1.10411983289513\\
792	1.10185720361568\\
793	1.09967883431872\\
794	1.09758996761927\\
795	1.09559431135825\\
796	1.09369396722249\\
797	1.0918894590782\\
798	1.09017983527501\\
799	1.08856282402788\\
800	1.08703502504983\\
801	1.08559211957906\\
802	1.08422907321397\\
803	1.08294031123552\\
804	1.08171987260122\\
805	1.08056156278147\\
806	1.07945911175469\\
807	1.0784063290746\\
808	1.07739724665759\\
809	1.07642624295296\\
810	1.07548814451595\\
811	1.07418433515173\\
812	1.07260125407848\\
813	1.07115400700819\\
814	1.06980490870859\\
815	1.06852177024251\\
816	1.06727732297586\\
817	1.06604868700549\\
818	1.06481687942964\\
819	1.06356635924957\\
820	1.06228460667134\\
821	1.06098451533583\\
822	1.05969655059005\\
823	1.05842234209261\\
824	1.05714412786465\\
825	1.05584994110756\\
826	1.05453261466811\\
827	1.05318891719762\\
828	1.05181880653853\\
829	1.05042478747156\\
830	1.04901136229311\\
831	1.04758324662967\\
832	1.04614290411372\\
833	1.04469130814368\\
834	1.04323065994807\\
835	1.04176501379597\\
836	1.04029956211728\\
837	1.03884006510026\\
838	1.03739240207448\\
839	1.03596222500649\\
840	1.03455469707677\\
841	1.03317437778697\\
842	1.03182535416059\\
843	1.0305114573318\\
844	1.02923627138778\\
845	1.02800295817699\\
846	1.02681409807654\\
847	1.02567161037729\\
848	1.02457673432857\\
849	1.02353005532998\\
850	1.02253156367392\\
851	1.02158073130043\\
852	1.02067658363713\\
853	1.01981774850737\\
854	1.01900249060918\\
855	1.01822875428551\\
856	1.01749422339825\\
857	1.01679639213189\\
858	1.01613263848402\\
859	1.01550029452341\\
860	1.0148967093382\\
861	1.01392353205634\\
862	1.01266568549135\\
863	1.01146379194913\\
864	1.01014830344722\\
865	1.00857659232235\\
866	1.00662974385191\\
867	1.00420968986861\\
868	1.00123664753952\\
869	0.997646831740664\\
870	0.993390413119296\\
871	0.988452581380834\\
872	0.982844607621753\\
873	0.9765605717512\\
874	0.969592876250273\\
875	0.961957909878341\\
876	0.953691446749461\\
877	0.944844706977621\\
878	0.935480994812871\\
879	0.925672840303994\\
880	0.915499579409624\\
881	0.90504399207581\\
882	0.894388501716754\\
883	0.883614569599948\\
884	0.872803608165325\\
885	0.862035541126998\\
886	0.851386310405085\\
887	0.840925987841813\\
888	0.83071738761177\\
889	0.820815090658765\\
890	0.811264805808438\\
891	0.802103080205753\\
892	0.793357418552412\\
893	0.785046631350171\\
894	0.777181139683411\\
895	0.769763292308008\\
896	0.762787884762076\\
897	0.756242894860035\\
898	0.750110364153481\\
899	0.744367369391171\\
900	0.738987040002324\\
901	0.733939583114274\\
902	0.729193275221195\\
903	0.724715388508976\\
904	0.720473046675956\\
905	0.716434016746523\\
906	0.712567428819872\\
907	0.708844404630999\\
908	0.705238579707156\\
909	0.701726511398523\\
910	0.698287970715558\\
911	0.694906120329759\\
912	0.691567585169746\\
913	0.68826242557543\\
914	0.684984024338255\\
915	0.681728898155307\\
916	0.678496443502662\\
917	0.675288627904062\\
918	0.672109638909076\\
919	0.66896550375867\\
920	0.665863692647641\\
921	0.662812717896865\\
922	0.659821740332846\\
923	0.65690019280369\\
924	0.654057429206078\\
925	0.651302405904297\\
926	0.648643401084628\\
927	0.646087776293423\\
928	0.643641783046377\\
929	0.641310415988031\\
930	0.639097312702113\\
931	0.63700469899082\\
932	0.635033377299871\\
933	0.633182755000572\\
934	0.631450908469906\\
935	0.629834678331045\\
936	0.628329790807601\\
937	0.626930999886161\\
938	0.625632244867703\\
939	0.624426817919554\\
940	0.623307536410482\\
941	0.622266915109787\\
942	0.621297333739002\\
943	0.62039119586067\\
944	0.619541075649477\\
945	0.618739849693879\\
946	0.617980811601001\\
947	0.617257767806917\\
948	0.616565113613426\\
949	0.615897889067111\\
950	0.615251814853368\\
951	0.614623308885504\\
952	0.614009484717019\\
953	0.613408133286624\\
954	0.612817689815588\\
955	0.612237187913612\\
956	0.611666203112794\\
957	0.611104788141406\\
958	0.610553402273809\\
959	0.610012837054638\\
960	0.609484140600628\\
961	0.6089685425389\\
962	0.608467381454282\\
963	0.607982036498353\\
964	0.607513864568163\\
965	0.607064144201329\\
966	0.606634027064722\\
967	0.606224497643939\\
968	0.605836341477269\\
969	0.605470122027138\\
970	0.605126166049681\\
971	0.604804557113496\\
972	0.604505136735531\\
973	0.604227512447966\\
974	0.603971071986485\\
975	0.603735002698274\\
976	0.60351831520709\\
977	0.603319870341987\\
978	0.603138408333929\\
979	0.602972579308368\\
980	0.602820974149068\\
981	0.602682154875911\\
982	0.602554683763622\\
983	0.602437150525799\\
984	0.602328196995612\\
985	0.602226538847511\\
986	0.602130984019804\\
987	0.602040447612789\\
988	0.601953963148308\\
989	0.601870690181603\\
990	0.601789918352816\\
991	0.601711068051767\\
992	0.601633687944194\\
993	0.601557449669557\\
994	0.601482140069199\\
995	0.601407651338948\\
996	0.601333969522305\\
997	0.601261161769759\\
998	0.601189362787216\\
999	0.601118760883205\\
1000	0.601049584001516\\
};
\addlegendentry{Y}

\addplot[const plot, color=mycolor2] table[row sep=crcr] {%
1	0.9\\
2	0.9\\
3	0.9\\
4	0.9\\
5	0.9\\
6	0.9\\
7	0.9\\
8	0.9\\
9	0.9\\
10	0.9\\
11	0.9\\
12	1.35\\
13	1.35\\
14	1.35\\
15	1.35\\
16	1.35\\
17	1.35\\
18	1.35\\
19	1.35\\
20	1.35\\
21	1.35\\
22	1.35\\
23	1.35\\
24	1.35\\
25	1.35\\
26	1.35\\
27	1.35\\
28	1.35\\
29	1.35\\
30	1.35\\
31	1.35\\
32	1.35\\
33	1.35\\
34	1.35\\
35	1.35\\
36	1.35\\
37	1.35\\
38	1.35\\
39	1.35\\
40	1.35\\
41	1.35\\
42	1.35\\
43	1.35\\
44	1.35\\
45	1.35\\
46	1.35\\
47	1.35\\
48	1.35\\
49	1.35\\
50	1.35\\
51	1.35\\
52	1.35\\
53	1.35\\
54	1.35\\
55	1.35\\
56	1.35\\
57	1.35\\
58	1.35\\
59	1.35\\
60	1.35\\
61	1.35\\
62	1.35\\
63	1.35\\
64	1.35\\
65	1.35\\
66	1.35\\
67	1.35\\
68	1.35\\
69	1.35\\
70	1.35\\
71	1.35\\
72	1.35\\
73	1.35\\
74	1.35\\
75	1.35\\
76	1.35\\
77	1.35\\
78	1.35\\
79	1.35\\
80	1.35\\
81	1.35\\
82	1.35\\
83	1.35\\
84	1.35\\
85	1.35\\
86	1.35\\
87	1.35\\
88	1.35\\
89	1.35\\
90	1.35\\
91	1.35\\
92	1.35\\
93	1.35\\
94	1.35\\
95	1.35\\
96	1.35\\
97	1.35\\
98	1.35\\
99	1.35\\
100	1.35\\
101	1.35\\
102	1.35\\
103	1.35\\
104	1.35\\
105	1.35\\
106	1.35\\
107	1.35\\
108	1.35\\
109	1.35\\
110	1.35\\
111	1.35\\
112	1.35\\
113	1.35\\
114	1.35\\
115	1.35\\
116	1.35\\
117	1.35\\
118	1.35\\
119	1.35\\
120	1.35\\
121	1.35\\
122	1.35\\
123	1.35\\
124	1.35\\
125	1.35\\
126	1.35\\
127	1.35\\
128	1.35\\
129	1.35\\
130	1.35\\
131	1.35\\
132	1.35\\
133	1.35\\
134	1.35\\
135	1.35\\
136	1.35\\
137	1.35\\
138	1.35\\
139	1.35\\
140	1.35\\
141	1.35\\
142	1.35\\
143	1.35\\
144	1.35\\
145	1.35\\
146	1.35\\
147	1.35\\
148	1.35\\
149	1.35\\
150	1.35\\
151	0.8\\
152	0.8\\
153	0.8\\
154	0.8\\
155	0.8\\
156	0.8\\
157	0.8\\
158	0.8\\
159	0.8\\
160	0.8\\
161	0.8\\
162	0.8\\
163	0.8\\
164	0.8\\
165	0.8\\
166	0.8\\
167	0.8\\
168	0.8\\
169	0.8\\
170	0.8\\
171	0.8\\
172	0.8\\
173	0.8\\
174	0.8\\
175	0.8\\
176	0.8\\
177	0.8\\
178	0.8\\
179	0.8\\
180	0.8\\
181	0.8\\
182	0.8\\
183	0.8\\
184	0.8\\
185	0.8\\
186	0.8\\
187	0.8\\
188	0.8\\
189	0.8\\
190	0.8\\
191	0.8\\
192	0.8\\
193	0.8\\
194	0.8\\
195	0.8\\
196	0.8\\
197	0.8\\
198	0.8\\
199	0.8\\
200	0.8\\
201	0.8\\
202	0.8\\
203	0.8\\
204	0.8\\
205	0.8\\
206	0.8\\
207	0.8\\
208	0.8\\
209	0.8\\
210	0.8\\
211	0.8\\
212	0.8\\
213	0.8\\
214	0.8\\
215	0.8\\
216	0.8\\
217	0.8\\
218	0.8\\
219	0.8\\
220	0.8\\
221	0.8\\
222	0.8\\
223	0.8\\
224	0.8\\
225	0.8\\
226	0.8\\
227	0.8\\
228	0.8\\
229	0.8\\
230	0.8\\
231	0.8\\
232	0.8\\
233	0.8\\
234	0.8\\
235	0.8\\
236	0.8\\
237	0.8\\
238	0.8\\
239	0.8\\
240	0.8\\
241	0.8\\
242	0.8\\
243	0.8\\
244	0.8\\
245	0.8\\
246	0.8\\
247	0.8\\
248	0.8\\
249	0.8\\
250	0.8\\
251	0.8\\
252	0.8\\
253	0.8\\
254	0.8\\
255	0.8\\
256	0.8\\
257	0.8\\
258	0.8\\
259	0.8\\
260	0.8\\
261	0.8\\
262	0.8\\
263	0.8\\
264	0.8\\
265	0.8\\
266	0.8\\
267	0.8\\
268	0.8\\
269	0.8\\
270	0.8\\
271	0.8\\
272	0.8\\
273	0.8\\
274	0.8\\
275	0.8\\
276	0.8\\
277	0.8\\
278	0.8\\
279	0.8\\
280	0.8\\
281	0.8\\
282	0.8\\
283	0.8\\
284	0.8\\
285	0.8\\
286	0.8\\
287	0.8\\
288	0.8\\
289	0.8\\
290	0.8\\
291	0.8\\
292	0.8\\
293	0.8\\
294	0.8\\
295	0.8\\
296	0.8\\
297	0.8\\
298	0.8\\
299	0.8\\
300	0.8\\
301	1\\
302	1\\
303	1\\
304	1\\
305	1\\
306	1\\
307	1\\
308	1\\
309	1\\
310	1\\
311	1\\
312	1\\
313	1\\
314	1\\
315	1\\
316	1\\
317	1\\
318	1\\
319	1\\
320	1\\
321	1\\
322	1\\
323	1\\
324	1\\
325	1\\
326	1\\
327	1\\
328	1\\
329	1\\
330	1\\
331	1\\
332	1\\
333	1\\
334	1\\
335	1\\
336	1\\
337	1\\
338	1\\
339	1\\
340	1\\
341	1\\
342	1\\
343	1\\
344	1\\
345	1\\
346	1\\
347	1\\
348	1\\
349	1\\
350	1\\
351	1\\
352	1\\
353	1\\
354	1\\
355	1\\
356	1\\
357	1\\
358	1\\
359	1\\
360	1\\
361	1\\
362	1\\
363	1\\
364	1\\
365	1\\
366	1\\
367	1\\
368	1\\
369	1\\
370	1\\
371	1\\
372	1\\
373	1\\
374	1\\
375	1\\
376	1\\
377	1\\
378	1\\
379	1\\
380	1\\
381	1\\
382	1\\
383	1\\
384	1\\
385	1\\
386	1\\
387	1\\
388	1\\
389	1\\
390	1\\
391	1\\
392	1\\
393	1\\
394	1\\
395	1\\
396	1\\
397	1\\
398	1\\
399	1\\
400	1\\
401	1\\
402	1\\
403	1\\
404	1\\
405	1\\
406	1\\
407	1\\
408	1\\
409	1\\
410	1\\
411	1\\
412	1\\
413	1\\
414	1\\
415	1\\
416	1\\
417	1\\
418	1\\
419	1\\
420	1\\
421	1\\
422	1\\
423	1\\
424	1\\
425	1\\
426	1\\
427	1\\
428	1\\
429	1\\
430	1\\
431	1\\
432	1\\
433	1\\
434	1\\
435	1\\
436	1\\
437	1\\
438	1\\
439	1\\
440	1\\
441	1\\
442	1\\
443	1\\
444	1\\
445	1\\
446	1\\
447	1\\
448	1\\
449	1\\
450	1\\
451	1\\
452	1\\
453	1\\
454	1\\
455	1\\
456	1\\
457	1\\
458	1\\
459	1\\
460	1\\
461	1\\
462	1\\
463	1\\
464	1\\
465	1\\
466	1\\
467	1\\
468	1\\
469	1\\
470	1\\
471	1\\
472	1\\
473	1\\
474	1\\
475	1\\
476	1\\
477	1\\
478	1\\
479	1\\
480	1\\
481	1\\
482	1\\
483	1\\
484	1\\
485	1\\
486	1\\
487	1\\
488	1\\
489	1\\
490	1\\
491	1\\
492	1\\
493	1\\
494	1\\
495	1\\
496	1\\
497	1\\
498	1\\
499	1\\
500	1\\
501	1.05\\
502	1.05\\
503	1.05\\
504	1.05\\
505	1.05\\
506	1.05\\
507	1.05\\
508	1.05\\
509	1.05\\
510	1.05\\
511	1.05\\
512	1.05\\
513	1.05\\
514	1.05\\
515	1.05\\
516	1.05\\
517	1.05\\
518	1.05\\
519	1.05\\
520	1.05\\
521	1.05\\
522	1.05\\
523	1.05\\
524	1.05\\
525	1.05\\
526	1.05\\
527	1.05\\
528	1.05\\
529	1.05\\
530	1.05\\
531	1.05\\
532	1.05\\
533	1.05\\
534	1.05\\
535	1.05\\
536	1.05\\
537	1.05\\
538	1.05\\
539	1.05\\
540	1.05\\
541	1.05\\
542	1.05\\
543	1.05\\
544	1.05\\
545	1.05\\
546	1.05\\
547	1.05\\
548	1.05\\
549	1.05\\
550	1.05\\
551	1.1\\
552	1.1\\
553	1.1\\
554	1.1\\
555	1.1\\
556	1.1\\
557	1.1\\
558	1.1\\
559	1.1\\
560	1.1\\
561	1.1\\
562	1.1\\
563	1.1\\
564	1.1\\
565	1.1\\
566	1.1\\
567	1.1\\
568	1.1\\
569	1.1\\
570	1.1\\
571	1.1\\
572	1.1\\
573	1.1\\
574	1.1\\
575	1.1\\
576	1.1\\
577	1.1\\
578	1.1\\
579	1.1\\
580	1.1\\
581	1.1\\
582	1.1\\
583	1.1\\
584	1.1\\
585	1.1\\
586	1.1\\
587	1.1\\
588	1.1\\
589	1.1\\
590	1.1\\
591	1.1\\
592	1.1\\
593	1.1\\
594	1.1\\
595	1.1\\
596	1.1\\
597	1.1\\
598	1.1\\
599	1.1\\
600	1.1\\
601	1.15\\
602	1.15\\
603	1.15\\
604	1.15\\
605	1.15\\
606	1.15\\
607	1.15\\
608	1.15\\
609	1.15\\
610	1.15\\
611	1.15\\
612	1.15\\
613	1.15\\
614	1.15\\
615	1.15\\
616	1.15\\
617	1.15\\
618	1.15\\
619	1.15\\
620	1.15\\
621	1.15\\
622	1.15\\
623	1.15\\
624	1.15\\
625	1.15\\
626	1.15\\
627	1.15\\
628	1.15\\
629	1.15\\
630	1.15\\
631	1.15\\
632	1.15\\
633	1.15\\
634	1.15\\
635	1.15\\
636	1.15\\
637	1.15\\
638	1.15\\
639	1.15\\
640	1.15\\
641	1.15\\
642	1.15\\
643	1.15\\
644	1.15\\
645	1.15\\
646	1.15\\
647	1.15\\
648	1.15\\
649	1.15\\
650	1.15\\
651	1.2\\
652	1.2\\
653	1.2\\
654	1.2\\
655	1.2\\
656	1.2\\
657	1.2\\
658	1.2\\
659	1.2\\
660	1.2\\
661	1.2\\
662	1.2\\
663	1.2\\
664	1.2\\
665	1.2\\
666	1.2\\
667	1.2\\
668	1.2\\
669	1.2\\
670	1.2\\
671	1.2\\
672	1.2\\
673	1.2\\
674	1.2\\
675	1.2\\
676	1.2\\
677	1.2\\
678	1.2\\
679	1.2\\
680	1.2\\
681	1.2\\
682	1.2\\
683	1.2\\
684	1.2\\
685	1.2\\
686	1.2\\
687	1.2\\
688	1.2\\
689	1.2\\
690	1.2\\
691	1.2\\
692	1.2\\
693	1.2\\
694	1.2\\
695	1.2\\
696	1.2\\
697	1.2\\
698	1.2\\
699	1.2\\
700	1.2\\
701	1.15\\
702	1.15\\
703	1.15\\
704	1.15\\
705	1.15\\
706	1.15\\
707	1.15\\
708	1.15\\
709	1.15\\
710	1.15\\
711	1.15\\
712	1.15\\
713	1.15\\
714	1.15\\
715	1.15\\
716	1.15\\
717	1.15\\
718	1.15\\
719	1.15\\
720	1.15\\
721	1.15\\
722	1.15\\
723	1.15\\
724	1.15\\
725	1.15\\
726	1.15\\
727	1.15\\
728	1.15\\
729	1.15\\
730	1.15\\
731	1.15\\
732	1.15\\
733	1.15\\
734	1.15\\
735	1.15\\
736	1.15\\
737	1.15\\
738	1.15\\
739	1.15\\
740	1.15\\
741	1.15\\
742	1.15\\
743	1.15\\
744	1.15\\
745	1.15\\
746	1.15\\
747	1.15\\
748	1.15\\
749	1.15\\
750	1.15\\
751	1.05\\
752	1.05\\
753	1.05\\
754	1.05\\
755	1.05\\
756	1.05\\
757	1.05\\
758	1.05\\
759	1.05\\
760	1.05\\
761	1.05\\
762	1.05\\
763	1.05\\
764	1.05\\
765	1.05\\
766	1.05\\
767	1.05\\
768	1.05\\
769	1.05\\
770	1.05\\
771	1.05\\
772	1.05\\
773	1.05\\
774	1.05\\
775	1.05\\
776	1.05\\
777	1.05\\
778	1.05\\
779	1.05\\
780	1.05\\
781	1.05\\
782	1.05\\
783	1.05\\
784	1.05\\
785	1.05\\
786	1.05\\
787	1.05\\
788	1.05\\
789	1.05\\
790	1.05\\
791	1.05\\
792	1.05\\
793	1.05\\
794	1.05\\
795	1.05\\
796	1.05\\
797	1.05\\
798	1.05\\
799	1.05\\
800	1.05\\
801	1\\
802	1\\
803	1\\
804	1\\
805	1\\
806	1\\
807	1\\
808	1\\
809	1\\
810	1\\
811	1\\
812	1\\
813	1\\
814	1\\
815	1\\
816	1\\
817	1\\
818	1\\
819	1\\
820	1\\
821	1\\
822	1\\
823	1\\
824	1\\
825	1\\
826	1\\
827	1\\
828	1\\
829	1\\
830	1\\
831	1\\
832	1\\
833	1\\
834	1\\
835	1\\
836	1\\
837	1\\
838	1\\
839	1\\
840	1\\
841	1\\
842	1\\
843	1\\
844	1\\
845	1\\
846	1\\
847	1\\
848	1\\
849	1\\
850	1\\
851	0.6\\
852	0.6\\
853	0.6\\
854	0.6\\
855	0.6\\
856	0.6\\
857	0.6\\
858	0.6\\
859	0.6\\
860	0.6\\
861	0.6\\
862	0.6\\
863	0.6\\
864	0.6\\
865	0.6\\
866	0.6\\
867	0.6\\
868	0.6\\
869	0.6\\
870	0.6\\
871	0.6\\
872	0.6\\
873	0.6\\
874	0.6\\
875	0.6\\
876	0.6\\
877	0.6\\
878	0.6\\
879	0.6\\
880	0.6\\
881	0.6\\
882	0.6\\
883	0.6\\
884	0.6\\
885	0.6\\
886	0.6\\
887	0.6\\
888	0.6\\
889	0.6\\
890	0.6\\
891	0.6\\
892	0.6\\
893	0.6\\
894	0.6\\
895	0.6\\
896	0.6\\
897	0.6\\
898	0.6\\
899	0.6\\
900	0.6\\
901	0.6\\
902	0.6\\
903	0.6\\
904	0.6\\
905	0.6\\
906	0.6\\
907	0.6\\
908	0.6\\
909	0.6\\
910	0.6\\
911	0.6\\
912	0.6\\
913	0.6\\
914	0.6\\
915	0.6\\
916	0.6\\
917	0.6\\
918	0.6\\
919	0.6\\
920	0.6\\
921	0.6\\
922	0.6\\
923	0.6\\
924	0.6\\
925	0.6\\
926	0.6\\
927	0.6\\
928	0.6\\
929	0.6\\
930	0.6\\
931	0.6\\
932	0.6\\
933	0.6\\
934	0.6\\
935	0.6\\
936	0.6\\
937	0.6\\
938	0.6\\
939	0.6\\
940	0.6\\
941	0.6\\
942	0.6\\
943	0.6\\
944	0.6\\
945	0.6\\
946	0.6\\
947	0.6\\
948	0.6\\
949	0.6\\
950	0.6\\
951	0.6\\
952	0.6\\
953	0.6\\
954	0.6\\
955	0.6\\
956	0.6\\
957	0.6\\
958	0.6\\
959	0.6\\
960	0.6\\
961	0.6\\
962	0.6\\
963	0.6\\
964	0.6\\
965	0.6\\
966	0.6\\
967	0.6\\
968	0.6\\
969	0.6\\
970	0.6\\
971	0.6\\
972	0.6\\
973	0.6\\
974	0.6\\
975	0.6\\
976	0.6\\
977	0.6\\
978	0.6\\
979	0.6\\
980	0.6\\
981	0.6\\
982	0.6\\
983	0.6\\
984	0.6\\
985	0.6\\
986	0.6\\
987	0.6\\
988	0.6\\
989	0.6\\
990	0.6\\
991	0.6\\
992	0.6\\
993	0.6\\
994	0.6\\
995	0.6\\
996	0.6\\
997	0.6\\
998	0.6\\
999	0.6\\
1000	0.6\\
};
\addlegendentry{Yzad}

\end{axis}
\end{tikzpicture}%
\caption{Śledzenie wartości zadanej dla parametrów $K = 1,212$, $T_i = 15$, $T_d = 4$}
\end{figure}

Wskaźnik jakości regulacji:

\begin{equation}
E = 25,6710
\end{equation}

Wartości parametrów dla których wartość wyjścia najlepiej śledzi wartość zadaną (na oko) to:

\begin{equation}
K = 1,3; T_i = 10; T_d = 3;
\end{equation}

\begin{figure}[H]
\centering
% This file was created by matlab2tikz.
%
%The latest updates can be retrieved from
%  http://www.mathworks.com/matlabcentral/fileexchange/22022-matlab2tikz-matlab2tikz
%where you can also make suggestions and rate matlab2tikz.
%
\definecolor{mycolor1}{rgb}{0.00000,0.44700,0.74100}%
%
\begin{tikzpicture}

\begin{axis}[%
width=4.272in,
height=2.477in,
at={(0.717in,0.437in)},
scale only axis,
xmin=0,
xmax=1000,
xlabel style={font=\color{white!15!black}},
xlabel={k},
ymin=2.7,
ymax=3.3,
ylabel style={font=\color{white!15!black}},
ylabel={U(k)},
axis background/.style={fill=white}
]
\addplot[const plot, color=mycolor1, forget plot] table[row sep=crcr] {%
1	3\\
2	3\\
3	3\\
4	3\\
5	3\\
6	3\\
7	3\\
8	3\\
9	3\\
10	3\\
11	3\\
12	3.075\\
13	3\\
14	3.02925\\
15	3.0585\\
16	3.08775\\
17	3.117\\
18	3.14625\\
19	3.1755\\
20	3.20475\\
21	3.234\\
22	3.25962592719375\\
23	3.28547758778953\\
24	3.3\\
25	3.3\\
26	3.3\\
27	3.3\\
28	3.3\\
29	3.3\\
30	3.3\\
31	3.3\\
32	3.3\\
33	3.3\\
34	3.3\\
35	3.3\\
36	3.3\\
37	3.3\\
38	3.3\\
39	3.3\\
40	3.3\\
41	3.3\\
42	3.3\\
43	3.3\\
44	3.3\\
45	3.3\\
46	3.3\\
47	3.3\\
48	3.3\\
49	3.3\\
50	3.3\\
51	3.3\\
52	3.3\\
53	3.29999906788716\\
54	3.29989615394184\\
55	3.29969413415641\\
56	3.29939607445979\\
57	3.29900518175031\\
58	3.29852476159182\\
59	3.29795818181419\\
60	3.29730884133915\\
61	3.29658014362351\\
62	3.2957754741763\\
63	3.29489822670512\\
64	3.2939566673918\\
65	3.29296343597744\\
66	3.29193075207795\\
67	3.29087041086821\\
68	3.28979378207084\\
69	3.28871181170071\\
70	3.28763502609109\\
71	3.28657353779293\\
72	3.28553705299544\\
73	3.28453487799008\\
74	3.28357568874757\\
75	3.28266710362615\\
76	3.28181551294273\\
77	3.28102612746637\\
78	3.28030302497752\\
79	3.27964919493\\
80	3.27906658126531\\
81	3.27855612343809\\
82	3.27811779571788\\
83	3.27775064494117\\
84	3.27745283837678\\
85	3.27722174210043\\
86	3.27705402636426\\
87	3.27694577561768\\
88	3.27689259344666\\
89	3.27688970284671\\
90	3.27693204220213\\
91	3.27701435730552\\
92	3.27713128971614\\
93	3.27727746171895\\
94	3.27744755755259\\
95	3.27763639904149\\
96	3.2778390130042\\
97	3.27805068905315\\
98	3.2782670278833\\
99	3.27848398055016\\
100	3.27869787917931\\
101	3.27890545949888\\
102	3.27910387554193\\
103	3.27929070682806\\
104	3.27946395832833\\
105	3.27962205361544\\
106	3.27976382180672\\
107	3.27988847908774\\
108	3.27999560565695\\
109	3.28008511889793\\
110	3.28015724353904\\
111	3.2802124795202\\
112	3.28025156825236\\
113	3.28027545792607\\
114	3.2802852684994\\
115	3.28028225696565\\
116	3.28026778345733\\
117	3.28024327868175\\
118	3.28021021311212\\
119	3.28017006828628\\
120	3.28012431049755\\
121	3.28007436709875\\
122	3.28002160558078\\
123	3.27996731552982\\
124	3.27991269351272\\
125	3.2798588308879\\
126	3.27980670448984\\
127	3.27975717009078\\
128	3.27971095850442\\
129	3.27966867416406\\
130	3.27963079598116\\
131	3.2795976802692\\
132	3.27956956550261\\
133	3.27954657866953\\
134	3.27952874297145\\
135	3.27951598662136\\
136	3.27950815249504\\
137	3.27950500839691\\
138	3.27950625771235\\
139	3.27951155023166\\
140	3.27952049294683\\
141	3.27953266063978\\
142	3.2795476061003\\
143	3.2795648698318\\
144	3.27958398912433\\
145	3.27960450639507\\
146	3.27962597671786\\
147	3.27964797448368\\
148	3.27967009915378\\
149	3.27969198008599\\
150	3.279713280432\\
151	3.204713280432\\
152	3.279713280432\\
153	3.24398119531552\\
154	3.20824756674577\\
155	3.17251225450489\\
156	3.13677515772019\\
157	3.10103621332434\\
158	3.06529539399956\\
159	3.02955270568525\\
160	2.99380818472878\\
161	2.96168695426391\\
162	2.92934016572859\\
163	2.8952734647236\\
164	2.8645656615514\\
165	2.83719207628773\\
166	2.81313444750751\\
167	2.79238005668261\\
168	2.77492095208089\\
169	2.7607532618056\\
170	2.74987658664479\\
171	2.74211829793672\\
172	2.73732996063767\\
173	2.73561413989766\\
174	2.73689662253244\\
175	2.74087402965821\\
176	2.74724811725953\\
177	2.75572439904582\\
178	2.76601096397938\\
179	2.77781746497988\\
180	2.7908542579651\\
181	2.80484013696361\\
182	2.81950860756262\\
183	2.83459424736055\\
184	2.84982955641432\\
185	2.86496395010145\\
186	2.87977370408541\\
187	2.89406124465062\\
188	2.90765463892406\\
189	2.92040725560769\\
190	2.93219757060098\\
191	2.9429286861945\\
192	2.95252723709803\\
193	2.96094271043784\\
194	2.9681475722768\\
195	2.97413663949405\\
196	2.97892512506239\\
197	2.98254629097426\\
198	2.98504918259913\\
199	2.98649642002157\\
200	2.9869620256497\\
201	2.98652929038304\\
202	2.98528871680607\\
203	2.98333604350024\\
204	2.98077028927413\\
205	2.97769181467166\\
206	2.97420049750389\\
207	2.97039409972362\\
208	2.96636683726382\\
209	2.96220814448397\\
210	2.9580016275838\\
211	2.95382420267113\\
212	2.94974541261903\\
213	2.94582691525464\\
214	2.9421221385113\\
215	2.93867610133961\\
216	2.93552539479219\\
217	2.93269830971696\\
218	2.93021509360259\\
219	2.92808831917948\\
220	2.92632334817118\\
221	2.92491887414612\\
222	2.92386752894845\\
223	2.92315653782996\\
224	2.92276840894212\\
225	2.92268164306659\\
226	2.92287144973165\\
227	2.92331045670227\\
228	2.92396940120878\\
229	2.92481779281356\\
230	2.92582453929555\\
231	2.92695852833971\\
232	2.92818915916978\\
233	2.92948681955584\\
234	2.93082330486692\\
235	2.93217217704446\\
236	2.93350906256003\\
237	2.93481188956012\\
238	2.93606106544315\\
239	2.93723959702664\\
240	2.93833315624342\\
241	2.93933009496195\\
242	2.94022141306311\\
243	2.94100068433087\\
244	2.94166394503144\\
245	2.94220955026974\\
246	2.94263800332421\\
247	2.94295176317534\\
248	2.94315503536418\\
249	2.94325355115678\\
250	2.94325433975836\\
251	2.94316549802989\\
252	2.94299596181925\\
253	2.94275528263926\\
254	2.94245341301512\\
255	2.94210050339325\\
256	2.94170671306028\\
257	2.94128203707407\\
258	2.94083615076609\\
259	2.94037827294323\\
260	2.93991704850421\\
261	2.93946045079626\\
262	2.93901570367611\\
263	2.93858922290926\\
264	2.93818657624609\\
265	2.93781246125401\\
266	2.93747069976339\\
267	2.93716424760175\\
268	2.9368952181457\\
269	2.93666491811189\\
270	2.93647389393676\\
271	2.93632198705672\\
272	2.93620839639428\\
273	2.93613174637888\\
274	2.93609015888021\\
275	2.93608132750444\\
276	2.93610259279623\\
277	2.9361510169985\\
278	2.9362234571447\\
279	2.93631663539095\\
280	2.93642720563573\\
281	2.93655181561836\\
282	2.93668716383358\\
283	2.93683005074301\\
284	2.93697742390559\\
285	2.9371264167836\\
286	2.9372743811087\\
287	2.93741891281093\\
288	2.93755787162226\\
289	2.9376893945632\\
290	2.93781190360694\\
291	2.93792410788819\\
292	2.93802500088495\\
293	2.9381138530497\\
294	2.93819020040247\\
295	2.93825382962286\\
296	2.93830476019126\\
297	2.93834322413233\\
298	2.93836964390695\\
299	2.93838460898374\\
300	2.93838885159786\\
301	3.01338885159786\\
302	2.93838885159786\\
303	2.95136638032129\\
304	2.96433705568201\\
305	2.97730196307516\\
306	2.99026219224518\\
307	3.0032188185203\\
308	3.01617288589561\\
309	3.02912539209232\\
310	3.04207727567605\\
311	3.0514050604503\\
312	3.0609586561571\\
313	3.07333183824591\\
314	3.08447715169318\\
315	3.0944052788983\\
316	3.10312434884923\\
317	3.11064026866244\\
318	3.11695701914318\\
319	3.12207691811333\\
320	3.12600085485576\\
321	3.12890363052781\\
322	3.13093724181877\\
323	3.13195077161946\\
324	3.13187146399642\\
325	3.13081121944737\\
326	3.12887977544918\\
327	3.12618524548746\\
328	3.12283458378199\\
329	3.11893398450637\\
330	3.11458922330713\\
331	3.10989748550455\\
332	3.10494072870328\\
333	3.09980158402992\\
334	3.09457347125021\\
335	3.08934806661633\\
336	3.08420704162948\\
337	3.07922236230775\\
338	3.07445651024709\\
339	3.06996263676065\\
340	3.06578465995624\\
341	3.06195772227656\\
342	3.05850932315462\\
343	3.0554599653383\\
344	3.05282257250837\\
345	3.05060204921352\\
346	3.04879581038139\\
347	3.04739466422155\\
348	3.04638366046292\\
349	3.04574291352426\\
350	3.04544840877026\\
351	3.04547277899342\\
352	3.04578600380786\\
353	3.04635602648835\\
354	3.047149371439\\
355	3.04813181218646\\
356	3.04926903667085\\
357	3.0505272490004\\
358	3.05187369411603\\
359	3.05327710897594\\
360	3.05470810279951\\
361	3.05613946896147\\
362	3.05754643383201\\
363	3.05890685011858\\
364	3.06020133833962\\
365	3.06141337375284\\
366	3.06252931632448\\
367	3.06353838667122\\
368	3.06443259427652\\
369	3.06520662465066\\
370	3.06585769175482\\
371	3.06638536172648\\
372	3.06679135354417\\
373	3.06707932165211\\
374	3.06725462504869\\
375	3.06732408732115\\
376	3.06729575236777\\
377	3.06717864053892\\
378	3.06698250951579\\
379	3.06671762366658\\
380	3.06639453504785\\
381	3.06602387868025\\
382	3.06561618422433\\
383	3.06518170572878\\
384	3.06473027072234\\
385	3.06427114954431\\
386	3.06381294541834\\
387	3.06336350537096\\
388	3.06292985171175\\
389	3.0625181334511\\
390	3.0621335967425\\
391	3.06178057319517\\
392	3.06146248470769\\
393	3.06118186331825\\
394	3.06094038444894\\
395	3.06073891183549\\
396	3.06057755237999\\
397	3.06045571914435\\
398	3.0603722007154\\
399	3.06032523521703\\
400	3.06031258731528\\
401	3.06033162665511\\
402	3.06037940627856\\
403	3.06045273969878\\
404	3.0605482754408\\
405	3.06066256800394\\
406	3.06079214435047\\
407	3.06093356517743\\
408	3.06108348038058\\
409	3.0612386782685\\
410	3.06139612822864\\
411	3.06155301668362\\
412	3.06170677630313\\
413	3.0618551085539\\
414	3.06199599977561\\
415	3.06212773106406\\
416	3.06224888232354\\
417	3.06235833091794\\
418	3.06245524540496\\
419	3.06253907487921\\
420	3.06260953447964\\
421	3.06266658763368\\
422	3.0627104256171\\
423	3.06274144500405\\
424	3.06276022356874\\
425	3.06276749517806\\
426	3.06276412418559\\
427	3.06275107980227\\
428	3.0627294108788\\
429	3.0627002214905\\
430	3.06266464766891\\
431	3.06262383557499\\
432	3.06257892135989\\
433	3.06253101290915\\
434	3.06248117361819\\
435	3.06243040829959\\
436	3.0623796512783\\
437	3.0623297566892\\
438	3.06228149095275\\
439	3.06223552737017\\
440	3.06219244274833\\
441	3.06215271593842\\
442	3.0621167281495\\
443	3.0620847648806\\
444	3.06205701930032\\
445	3.06203359689365\\
446	3.06201452118949\\
447	3.0619997403799\\
448	3.06198913464348\\
449	3.06198252398919\\
450	3.06197967644393\\
451	3.06198031641643\\
452	3.06198413308134\\
453	3.06199078864049\\
454	3.0619999263323\\
455	3.06201117807602\\
456	3.06202417165285\\
457	3.06203853734257\\
458	3.0620539139502\\
459	3.06206995417309\\
460	3.06208632927452\\
461	3.06210273304393\\
462	3.06211888503829\\
463	3.06213453311111\\
464	3.0621494552475\\
465	3.06216346073362\\
466	3.06217639069763\\
467	3.06218811806688\\
468	3.06219854699187\\
469	3.06220761179245\\
470	3.06221527548468\\
471	3.06222152794934\\
472	3.06222638380343\\
473	3.06222988003619\\
474	3.06223207346946\\
475	3.06223303810049\\
476	3.06223286238197\\
477	3.06223164649066\\
478	3.0622294996316\\
479	3.06222653742049\\
480	3.0622228793817\\
481	3.06221864659421\\
482	3.06221395951263\\
483	3.06220893598521\\
484	3.06220368948522\\
485	3.06219832756761\\
486	3.06219295055746\\
487	3.06218765047282\\
488	3.06218251017986\\
489	3.06217760277486\\
490	3.06217299118408\\
491	3.06216872796956\\
492	3.06216485532659\\
493	3.06216140525652\\
494	3.06215839989688\\
495	3.06215585199002\\
496	3.06215376547022\\
497	3.06215213614951\\
498	3.06215095248195\\
499	3.06215019638696\\
500	3.06214984411257\\
501	3.13714984411257\\
502	3.06214984411257\\
503	3.06540051736922\\
504	3.06865146041689\\
505	3.07190263430655\\
506	3.07515399952075\\
507	3.0784055166891\\
508	3.08165714724024\\
509	3.08490885398453\\
510	3.08816060162372\\
511	3.08778828549059\\
512	3.08764169911338\\
513	3.09078471061612\\
514	3.09361032516374\\
515	3.09612240179603\\
516	3.09832403485554\\
517	3.10021765274573\\
518	3.10180510565805\\
519	3.10308774340715\\
520	3.10406648440206\\
521	3.10491699534205\\
522	3.10579281026965\\
523	3.10652236827626\\
524	3.10697011961753\\
525	3.10716619676496\\
526	3.10713990529602\\
527	3.10691989240234\\
528	3.10653429277736\\
529	3.10601085456495\\
530	3.1053770477505\\
531	3.10465169520409\\
532	3.10383817140328\\
533	3.10294126719137\\
534	3.10198028240997\\
535	3.10097925332008\\
536	3.09995940607793\\
537	3.09893927546674\\
538	3.0979347985322\\
539	3.09695938663515\\
540	3.09602397899335\\
541	3.09513748928303\\
542	3.09430795465323\\
543	3.09354316795145\\
544	3.09284990086502\\
545	3.09223300768589\\
546	3.09169534633163\\
547	3.09123803654083\\
548	3.09086070378409\\
549	3.09056171204323\\
550	3.09033838815033\\
551	3.16533838815033\\
552	3.09033838815033\\
553	3.09356816522335\\
554	3.09685542048747\\
555	3.10019410920517\\
556	3.10357785382701\\
557	3.10700010683685\\
558	3.1104542901916\\
559	3.11393391290791\\
560	3.11743266798853\\
561	3.11731313314355\\
562	3.11741582879676\\
563	3.12080651624004\\
564	3.12387714679692\\
565	3.12662715725574\\
566	3.1290557446455\\
567	3.13116198487819\\
568	3.13294492740047\\
569	3.13440366907786\\
570	3.13553741026537\\
571	3.13652096743284\\
572	3.13750801759787\\
573	3.13832741079033\\
574	3.13884391449792\\
575	3.13908819082967\\
576	3.13909042124781\\
577	3.13888042870239\\
578	3.13848777622729\\
579	3.13794184569313\\
580	3.13727189992887\\
581	3.13649865202802\\
582	3.13562738114572\\
583	3.13466474380551\\
584	3.13363186350349\\
585	3.13255455620304\\
586	3.1314557608509\\
587	3.13035563091936\\
588	3.12927160494744\\
589	3.12821845960524\\
590	3.1272083482865\\
591	3.12625123749617\\
592	3.12535605210898\\
593	3.12453130982701\\
594	3.1237843487569\\
595	3.12312043549272\\
596	3.12254269135678\\
597	3.12205235792109\\
598	3.12164905037271\\
599	3.12133100151569\\
600	3.12109529874635\\
601	3.19609529874635\\
602	3.12109529874635\\
603	3.12433042007437\\
604	3.12762793565554\\
605	3.13098122862226\\
606	3.13438332315585\\
607	3.13782705968147\\
608	3.14130524590918\\
609	3.1448107851086\\
610	3.1483367826758\\
611	3.1482449637451\\
612	3.14837533477248\\
613	3.15179344759509\\
614	3.15489081502782\\
615	3.15766647916134\\
616	3.16011929353074\\
617	3.16224804323727\\
618	3.1640515400654\\
619	3.16552869589147\\
620	3.16667857742365\\
621	3.16767593082786\\
622	3.1686744219035\\
623	3.16950291847267\\
624	3.17002623238654\\
625	3.17027510545468\\
626	3.17027983003388\\
627	3.17007036664434\\
628	3.16967643798352\\
629	3.16912760311536\\
630	3.16845331512251\\
631	3.1676744854239\\
632	3.16679659423842\\
633	3.16582649820046\\
634	3.16478551774033\\
635	3.16369966000525\\
636	3.16259204656768\\
637	3.1614830025364\\
638	3.16039012506099\\
639	3.15932833475409\\
640	3.15830991303027\\
641	3.15734493764873\\
642	3.15644242757638\\
643	3.15561097743626\\
644	3.15485798531715\\
645	3.15418876128341\\
646	3.15360645426751\\
647	3.15311231846352\\
648	3.15270596775142\\
649	3.15238562091048\\
650	3.15214833992511\\
651	3.22714833992511\\
652	3.15214833992511\\
653	3.15538379747363\\
654	3.15868218233104\\
655	3.16203681697278\\
656	3.16544066214849\\
657	3.16888649338445\\
658	3.17236705324685\\
659	3.17587518073879\\
660	3.17940391887866\\
661	3.17931489001346\\
662	3.17944803500039\\
663	3.18286889565918\\
664	3.18596894914234\\
665	3.18874719531262\\
666	3.19120245087521\\
667	3.1933334696562\\
668	3.19513903776959\\
669	3.19661804697723\\
670	3.19776954929066\\
671	3.19876828346635\\
672	3.19976791522642\\
673	3.20059731526782\\
674	3.20112130001588\\
675	3.20137061909885\\
676	3.20137557601588\\
677	3.20116614527683\\
678	3.20077206593299\\
679	3.2002229152854\\
680	3.19954816606593\\
681	3.1987687501998\\
682	3.19789016862674\\
683	3.19691929848372\\
684	3.19587748031072\\
685	3.1947907407751\\
686	3.19368222010451\\
687	3.19257226093875\\
688	3.19147847661663\\
689	3.19041580242229\\
690	3.18939653278924\\
691	3.18843075675263\\
692	3.18752750277473\\
693	3.18669537320201\\
694	3.18594177210808\\
695	3.1852720138509\\
696	3.18468925003201\\
697	3.18419473597922\\
698	3.18378808528006\\
699	3.18346751512074\\
700	3.1832300847304\\
701	3.1082300847304\\
702	3.1832300847304\\
703	3.17996558869726\\
704	3.1767640742409\\
705	3.17361885752295\\
706	3.17052289270769\\
707	3.16746894859863\\
708	3.1644497610271\\
709	3.16145816236429\\
710	3.15848718920314\\
711	3.15914659954517\\
712	3.15957671565273\\
713	3.15671528250431\\
714	3.15416768080096\\
715	3.15192514647494\\
716	3.14998026899486\\
717	3.14832691415707\\
718	3.14696014370133\\
719	3.14587613278901\\
720	3.14507208635069\\
721	3.14437140586862\\
722	3.14362068933386\\
723	3.14299193337607\\
724	3.14262106440393\\
725	3.14247858323136\\
726	3.14253620181586\\
727	3.1427666227557\\
728	3.143143340424\\
729	3.14364046217316\\
730	3.14423254815107\\
731	3.14490291242287\\
732	3.14565033490823\\
733	3.14647213872654\\
734	3.14735109170671\\
735	3.1482651737507\\
736	3.14919509693093\\
737	3.15012415661846\\
738	3.15103811229211\\
739	3.1519250945249\\
740	3.15277553500038\\
741	3.15358170870591\\
742	3.15433658071479\\
743	3.15503317620849\\
744	3.15566536208399\\
745	3.15622874806967\\
746	3.15672077176915\\
747	3.15714044890677\\
748	3.15748814021129\\
749	3.15776533138371\\
750	3.1579744230602\\
751	3.0829744230602\\
752	3.1579744230602\\
753	3.15150049586754\\
754	3.14497466059113\\
755	3.13840231395042\\
756	3.1317891560784\\
757	3.12514104164032\\
758	3.11846385350115\\
759	3.11176339720695\\
760	3.1050453149398\\
761	3.10194605404686\\
762	3.09862450344938\\
763	3.09217171873033\\
764	3.0863425234196\\
765	3.08113493980514\\
766	3.07654785200255\\
767	3.07258081160866\\
768	3.06923387503733\\
769	3.06650746847083\\
770	3.06440227673618\\
771	3.06274369754099\\
772	3.06137855107439\\
773	3.06047110750045\\
774	3.06013566373521\\
775	3.06031431015968\\
776	3.06095008507163\\
777	3.06198672665302\\
778	3.06336846558485\\
779	3.06503985262636\\
780	3.06694561618765\\
781	3.06903902372658\\
782	3.07128870639486\\
783	3.07366216563695\\
784	3.07611367584403\\
785	3.07859497856845\\
786	3.08106308877988\\
787	3.08348014136496\\
788	3.08581327654943\\
789	3.08803455813448\\
790	3.09012091928189\\
791	3.09205372164367\\
792	3.09381761510268\\
793	3.09539989852443\\
794	3.09679122740368\\
795	3.09798635377328\\
796	3.09898399724277\\
797	3.09978637148878\\
798	3.10039872983849\\
799	3.10082892498516\\
800	3.10108697865892\\
801	3.02608697865892\\
802	3.10108697865892\\
803	3.0976548334746\\
804	3.09410533449312\\
805	3.09045446389088\\
806	3.08671847788706\\
807	3.08291358598153\\
808	3.07905567221248\\
809	3.0751600563492\\
810	3.07124129350288\\
811	3.07094180386846\\
812	3.07042166382942\\
813	3.06662402987709\\
814	3.06316828103195\\
815	3.06005935698585\\
816	3.05730160581801\\
817	3.05489867838528\\
818	3.0528534592576\\
819	3.05116802937548\\
820	3.04984365592842\\
821	3.0487054584146\\
822	3.04759951866133\\
823	3.04669635789476\\
824	3.04612968844683\\
825	3.04586665957387\\
826	3.04587458541131\\
827	3.04612089225813\\
828	3.04657308764095\\
829	3.04719874646948\\
830	3.04796551021583\\
831	3.04884956854556\\
832	3.04984256965903\\
833	3.05093482979791\\
834	3.05210229911914\\
835	3.05331640430451\\
836	3.05455165921757\\
837	3.05578559561464\\
838	3.05699870909246\\
839	3.0581744169485\\
840	3.05929902522706\\
841	3.06036129332085\\
842	3.06135128668567\\
843	3.06225973867763\\
844	3.06307881609907\\
845	3.06380299882773\\
846	3.06442913657515\\
847	3.06495616376155\\
848	3.06538482503511\\
849	3.06571740911794\\
850	3.06595748909644\\
851	2.99095748909644\\
852	3.06595748909644\\
853	3.03995128084709\\
854	3.01387609690061\\
855	2.98773957148429\\
856	2.9615497058229\\
857	2.93531467480522\\
858	2.90904265959399\\
859	2.88274170488916\\
860	2.85641959984382\\
861	2.83371520809073\\
862	2.81078864314956\\
863	2.78567460471992\\
864	2.76301362468652\\
865	2.74279250422169\\
866	2.72500209426488\\
867	2.70963663421356\\
868	2.7\\
869	2.7\\
870	2.7\\
871	2.7\\
872	2.7\\
873	2.7\\
874	2.70068386755232\\
875	2.70335620397188\\
876	2.7077981234437\\
877	2.71379322022177\\
878	2.72096696791065\\
879	2.72860712638253\\
880	2.73629260774248\\
881	2.7440431613736\\
882	2.75187506891676\\
883	2.75980157583426\\
884	2.76780023058451\\
885	2.7757522452233\\
886	2.78348873019314\\
887	2.79086059732757\\
888	2.79774585727672\\
889	2.80407292974679\\
890	2.80982165011378\\
891	2.81498839972649\\
892	2.81956622859607\\
893	2.82354542412813\\
894	2.8269156028247\\
895	2.82967218092949\\
896	2.83182331019682\\
897	2.83339126245219\\
898	2.83441026386935\\
899	2.83492289490932\\
900	2.83497540750078\\
901	2.83461471281005\\
902	2.83388793229955\\
903	2.83284281429495\\
904	2.8315280024085\\
905	2.82999287839742\\
906	2.82828676231396\\
907	2.826457723048\\
908	2.82455142726765\\
909	2.82261025705467\\
910	2.8206728112853\\
911	2.81877375644418\\
912	2.81694382468237\\
913	2.81520980039928\\
914	2.81359446459179\\
915	2.81211652600254\\
916	2.81079059082235\\
917	2.80962721984457\\
918	2.80863308930971\\
919	2.80781124093091\\
920	2.80716139070742\\
921	2.80668026281518\\
922	2.80636192644084\\
923	2.80619813041557\\
924	2.80617863915685\\
925	2.80629157327614\\
926	2.80652375425086\\
927	2.80686104774727\\
928	2.80728869712376\\
929	2.80779163858679\\
930	2.80835479157952\\
931	2.8089633209553\\
932	2.80960287002673\\
933	2.81025976475574\\
934	2.81092118933036\\
935	2.81157533297777\\
936	2.81221150767801\\
937	2.8128202366808\\
938	2.81339331433901\\
939	2.81392383852709\\
940	2.81440621757063\\
941	2.81483615403277\\
942	2.81521060787016\\
943	2.81552774147164\\
944	2.815786849048\\
945	2.81598827282341\\
946	2.81613330849416\\
947	2.81622410243658\\
948	2.81626354312633\\
949	2.81625514915062\\
950	2.81620295605151\\
951	2.81611140404854\\
952	2.81598522847459\\
953	2.8158293545394\\
954	2.81564879782055\\
955	2.81544857167042\\
956	2.815233602518\\
957	2.8150086538336\\
958	2.81477825931359\\
959	2.8145466656365\\
960	2.81431778494556\\
961	2.81409515703163\\
962	2.81388192102733\\
963	2.81368079627882\\
964	2.81349407193622\\
965	2.81332360469699\\
966	2.81317082404683\\
967	2.81303674427226\\
968	2.81292198246521\\
969	2.81282678170511\\
970	2.81275103858508\\
971	2.812694334247\\
972	2.81265596810207\\
973	2.81263499343857\\
974	2.81263025415515\\
975	2.81264042190356\\
976	2.81266403297908\\
977	2.81269952435732\\
978	2.81274526834184\\
979	2.81279960535613\\
980	2.81286087448454\\
981	2.81292744143873\\
982	2.81299772369706\\
983	2.81307021263383\\
984	2.81314349252157\\
985	2.81321625635234\\
986	2.81328731848236\\
987	2.81335562415738\\
988	2.81342025602411\\
989	2.81348043777487\\
990	2.81353553510831\\
991	2.81358505421899\\
992	2.81362863805194\\
993	2.81366606057598\\
994	2.8136972193413\\
995	2.81372212659315\\
996	2.81374089921451\\
997	2.81375374776737\\
998	2.81376096489404\\
999	2.81376291332867\\
1000	2.81376001375439\\
};
\end{axis}
\end{tikzpicture}%
\caption{Sterowanie PID dla parametrów $K = 1,3$, $T_i = 10$, $T_d = 3$}
\end{figure}

\begin{figure}[H]
\centering
% This file was created by matlab2tikz.
%
%The latest updates can be retrieved from
%  http://www.mathworks.com/matlabcentral/fileexchange/22022-matlab2tikz-matlab2tikz
%where you can also make suggestions and rate matlab2tikz.
%
\definecolor{mycolor1}{rgb}{0.00000,0.44700,0.74100}%
\definecolor{mycolor2}{rgb}{0.85000,0.32500,0.09800}%
%
\begin{tikzpicture}

\begin{axis}[%
width=4.272in,
height=2.477in,
at={(0.717in,0.437in)},
scale only axis,
xmin=0,
xmax=1000,
xlabel style={font=\color{white!15!black}},
xlabel={k},
ymin=0.5,
ymax=1.4,
ylabel style={font=\color{white!15!black}},
ylabel={Y(k)},
axis background/.style={fill=white},
legend style={legend cell align=left, align=left, draw=white!15!black}
]
\addplot[const plot, color=mycolor1] table[row sep=crcr] {%
1	0.9\\
2	0.9\\
3	0.9\\
4	0.9\\
5	0.9\\
6	0.9\\
7	0.9\\
8	0.9\\
9	0.9\\
10	0.9\\
11	0.9\\
12	0.9\\
13	0.9\\
14	0.9\\
15	0.9\\
16	0.9\\
17	0.9\\
18	0.9\\
19	0.9\\
20	0.9\\
21	0.9\\
22	0.9003968325\\
23	0.90110505465075\\
24	0.901851548804444\\
25	0.902926732649044\\
26	0.9045740507257\\
27	0.906995673734868\\
28	0.910357581452216\\
29	0.91479409195594\\
30	0.92041189372332\\
31	0.927293631597811\\
32	0.935481917271842\\
33	0.944987479442384\\
34	0.955752143856375\\
35	0.967585143897724\\
36	0.980245690511738\\
37	0.993524900254613\\
38	1.00724220273058\\
39	1.02124212187654\\
40	1.03539139393444\\
41	1.04957638853821\\
42	1.06370080258905\\
43	1.07768359953247\\
44	1.09145716931113\\
45	1.10496568667539\\
46	1.11816364771215\\
47	1.13101456642337\\
48	1.1434898149686\\
49	1.15556759279779\\
50	1.16723201135847\\
51	1.17847228237882\\
52	1.18928199891931\\
53	1.19965849946121\\
54	1.20960230627283\\
55	1.21911663017192\\
56	1.22820693459549\\
57	1.23688055260421\\
58	1.24514635109392\\
59	1.25301443706996\\
60	1.26049590136546\\
61	1.26760259565856\\
62	1.27434693907117\\
63	1.28074174608457\\
64	1.28679954411603\\
65	1.29253204733993\\
66	1.29795023797434\\
67	1.3030644413896\\
68	1.3078843950437\\
69	1.31241931135295\\
70	1.31667793468543\\
71	1.32066859272397\\
72	1.32439924248811\\
73	1.32787751157148\\
74	1.33111076093811\\
75	1.33410618724198\\
76	1.33687093004876\\
77	1.33941215850932\\
78	1.34173714007823\\
79	1.34385329361559\\
80	1.34576822897515\\
81	1.34748977496456\\
82	1.34902599736423\\
83	1.3503852084965\\
84	1.35157596840274\\
85	1.35260707550363\\
86	1.35348754538761\\
87	1.35422657910142\\
88	1.3548335232566\\
89	1.35531782394082\\
90	1.35568897613632\\
91	1.35595647009444\\
92	1.35612973589315\\
93	1.35621808720922\\
94	1.35623066522872\\
95	1.35617638368734\\
96	1.35606387616748\\
97	1.35590144675004\\
98	1.35569702492415\\
99	1.35545812544111\\
100	1.35519181361515\\
101	1.35490467641866\\
102	1.35460279958943\\
103	1.35429175085935\\
104	1.35397656932081\\
105	1.35366176085724\\
106	1.35335129946519\\
107	1.35304863419235\\
108	1.35275670132403\\
109	1.35247794137879\\
110	1.35221432042028\\
111	1.35196735515504\\
112	1.35173814126113\\
113	1.35152738437876\\
114	1.35133543318901\\
115	1.35116231400986\\
116	1.35100776635043\\
117	1.35087127888471\\
118	1.35075212533525\\
119	1.35064939979385\\
120	1.35056205104808\\
121	1.35048891552808\\
122	1.35042874853686\\
123	1.35038025347687\\
124	1.35034210883664\\
125	1.35031299275196\\
126	1.35029160500585\\
127	1.3502766863805\\
128	1.35026703532044\\
129	1.35026152190965\\
130	1.35025909920554\\
131	1.35025881200833\\
132	1.35025980317699\\
133	1.35026131763001\\
134	1.3502627041929\\
135	1.35026341547329\\
136	1.35026300595865\\
137	1.35026112854246\\
138	1.35025752969034\\
139	1.35025204346031\\
140	1.35024458458987\\
141	1.35023514085807\\
142	1.35022376492329\\
143	1.35021056582744\\
144	1.35019570034494\\
145	1.35017936434093\\
146	1.35016178428769\\
147	1.35014320907172\\
148	1.35012390220676\\
149	1.35010413455032\\
150	1.35008417760403\\
151	1.35006429746031\\
152	1.35004474944139\\
153	1.35002577346057\\
154	1.35000759012025\\
155	1.34999039754741\\
156	1.34997436895457\\
157	1.34995965090253\\
158	1.34994636223132\\
159	1.34993459361745\\
160	1.34992440770813\\
161	1.34951889923522\\
162	1.34880333328902\\
163	1.34801637133565\\
164	1.34680623564434\\
165	1.34487818982496\\
166	1.34198761101855\\
167	1.33793381210464\\
168	1.33255453756716\\
169	1.32572106414159\\
170	1.31733384413307\\
171	1.30733781596696\\
172	1.29571365605078\\
173	1.2824452175893\\
174	1.26753853904809\\
175	1.25104365760431\\
176	1.23304727100621\\
177	1.21366635998876\\
178	1.19304265516873\\
179	1.1713378458413\\
180	1.14872944006864\\
181	1.12540626923593\\
182	1.10156388099674\\
183	1.07740205013679\\
184	1.05312296740804\\
185	1.02892759572966\\
186	1.00501147782133\\
187	0.98156127181985\\
188	0.958751911430767\\
189	0.936744301032577\\
190	0.915683468219405\\
191	0.895697151551403\\
192	0.87689483720541\\
193	0.859367140555324\\
194	0.843185398442461\\
195	0.828401526929037\\
196	0.815048252072926\\
197	0.803139700616667\\
198	0.792672284955702\\
199	0.783625826910249\\
200	0.775964873487657\\
201	0.769640163035288\\
202	0.764590200971363\\
203	0.760742909007944\\
204	0.758017323223758\\
205	0.75632531988144\\
206	0.755573340172202\\
207	0.755664080816826\\
208	0.75649812124768\\
209	0.757975463734339\\
210	0.759996967501476\\
211	0.762465661879657\\
212	0.765287927215516\\
213	0.768374535670495\\
214	0.771641546661306\\
215	0.775011053627311\\
216	0.778411780963361\\
217	0.781779532677985\\
218	0.785057497075944\\
219	0.78819641406661\\
220	0.791154613532383\\
221	0.793897934635967\\
222	0.79639953705089\\
223	0.798639615895301\\
224	0.800605032685999\\
225	0.802288874976543\\
226	0.803689957519954\\
227	0.804812277764049\\
228	0.805664438217677\\
229	0.80625904773855\\
230	0.806612113128949\\
231	0.80674243162307\\
232	0.806670993941009\\
233	0.806420406597751\\
234	0.806014341115425\\
235	0.805477016712229\\
236	0.804832721946634\\
237	0.80410537969662\\
238	0.803318158769883\\
239	0.802493134391884\\
240	0.801650998821049\\
241	0.800810822407754\\
242	0.799989864555977\\
243	0.799203433271251\\
244	0.798464791291063\\
245	0.79778510619742\\
246	0.797173441408183\\
247	0.796636784534522\\
248	0.796180109276034\\
249	0.795806466800069\\
250	0.795517102413963\\
251	0.795311593282831\\
252	0.795188002965221\\
253	0.795143048627405\\
254	0.795172276946966\\
255	0.795270244919908\\
256	0.795430702034841\\
257	0.795646770564863\\
258	0.795911121044516\\
259	0.796216140337837\\
260	0.796554090056421\\
261	0.796917253446428\\
262	0.797298069223946\\
263	0.797689251192951\\
264	0.798083892823816\\
265	0.79847555629814\\
266	0.798858345833424\\
267	0.799226965385318\\
268	0.799576761083186\\
269	0.799903748984284\\
270	0.800204628931693\\
271	0.800476785470349\\
272	0.800718276913917\\
273	0.800927813763172\\
274	0.801104727754768\\
275	0.801248932869047\\
276	0.801360879648473\\
277	0.801441504176269\\
278	0.801492173040124\\
279	0.801514625560706\\
280	0.801510914501775\\
281	0.801483346400498\\
282	0.801434422565709\\
283	0.801366781691079\\
284	0.801283144921891\\
285	0.80118626410091\\
286	0.80107887380304\\
287	0.80096364765217\\
288	0.800843159298981\\
289	0.800719848327149\\
290	0.800595991249099\\
291	0.800473677652468\\
292	0.800354791466003\\
293	0.800240997229543\\
294	0.800133731177851\\
295	0.800034196882735\\
296	0.799943365142501\\
297	0.799861977762359\\
298	0.799790554833885\\
299	0.799729405095798\\
300	0.799678638941685\\
301	0.799638183632529\\
302	0.799607800272158\\
303	0.799587102111522\\
304	0.799575573762118\\
305	0.79957259091919\\
306	0.799577440220629\\
307	0.799589338896979\\
308	0.799607453900748\\
309	0.799630920238458\\
310	0.799658858265823\\
311	0.800087252030063\\
312	0.800829896899739\\
313	0.801526913554124\\
314	0.802314448958003\\
315	0.803306971242467\\
316	0.804599897685075\\
317	0.80627193853803\\
318	0.808387185421943\\
319	0.810996970220403\\
320	0.814141517893137\\
321	0.817832237576928\\
322	0.822058342893119\\
323	0.82682333598127\\
324	0.832134339197458\\
325	0.837981296502411\\
326	0.844339788339714\\
327	0.851173479010461\\
328	0.858436240883071\\
329	0.866073994644319\\
330	0.874026300243561\\
331	0.882228655782662\\
332	0.890615216678407\\
333	0.899119386465001\\
334	0.907673219149908\\
335	0.916208290634322\\
336	0.924657350987353\\
337	0.932955694514448\\
338	0.941042287608186\\
339	0.948860689023248\\
340	0.956359792560657\\
341	0.963494373316752\\
342	0.970225388345921\\
343	0.976520118682778\\
344	0.982352309401972\\
345	0.98770229142151\\
346	0.992556980678792\\
347	0.996909733881368\\
348	1.00076008658177\\
349	1.00411339535527\\
350	1.00698040247153\\
351	1.00937674072753\\
352	1.01132239913817\\
353	1.01284116803809\\
354	1.0139600704795\\
355	1.01470878044002\\
356	1.01511903420175\\
357	1.01522404704477\\
358	1.01505794707236\\
359	1.0146552357442\\
360	1.01405028282807\\
361	1.01327686182708\\
362	1.01236773018765\\
363	1.01135425681932\\
364	1.01026609835369\\
365	1.00913092530819\\
366	1.0079741990589\\
367	1.00681899971261\\
368	1.0056859039023\\
369	1.00459291058741\\
370	1.00355541218626\\
371	1.00258620777425\\
372	1.00169555464129\\
373	1.00089125422008\\
374	1.00017876824074\\
375	0.999561360866317\\
376	0.99904026247886\\
377	0.99861485074555\\
378	0.998282844636035\\
379	0.998040507192143\\
380	0.997882853054599\\
381	0.997803857012575\\
382	0.99779666014669\\
383	0.997853770470491\\
384	0.997967255326939\\
385	0.998128923157442\\
386	0.998330492628963\\
387	0.99856374747751\\
388	0.998820675798544\\
389	0.99909359287906\\
390	0.99937524701533\\
391	0.999658908088884\\
392	0.999938438976968\\
393	1.00020835014948\\
394	1.00046383805041\\
395	1.00070080807695\\
396	1.0009158831536\\
397	1.00110639905052\\
398	1.00127038771589\\
399	1.00140654998108\\
400	1.00151421905542\\
401	1.00159331625788\\
402	1.00164430043515\\
403	1.00166811249474\\
404	1.00166611643787\\
405	1.00164003821447\\
406	1.00159190364311\\
407	1.00152397654534\\
408	1.00143869813951\\
409	1.00133862862597\\
410	1.00122639177664\\
411	1.00110462321982\\
412	1.00097592298746\\
413	1.00084281277039\\
414	1.00070769820757\\
415	1.00057283642144\\
416	1.00044030890337\\
417	1.00031199975319\\
418	1.00018957918462\\
419	1.00007449212675\\
420	0.999967951678888\\
421	0.999870937114806\\
422	0.999784196080508\\
423	0.999708250589163\\
424	0.99964340638599\\
425	0.999589765235247\\
426	0.999547239670111\\
427	0.99951556974366\\
428	0.999494341324697\\
429	0.99948300549489\\
430	0.999480898622941\\
431	0.999487262716258\\
432	0.999501265680006\\
433	0.99952202114661\\
434	0.999548607574855\\
435	0.999580086355753\\
436	0.999615518701678\\
437	0.999653981134935\\
438	0.999694579431357\\
439	0.999736460913058\\
440	0.999778825021436\\
441	0.999820932136559\\
442	0.999862110641679\\
443	0.999901762261401\\
444	0.999939365728923\\
445	0.999974478861247\\
446	1.0000067391416\\
447	1.00003586292502\\
448	1.00006164339664\\
449	1.00008394742205\\
450	1.0001027114364\\
451	1.0001179365222\\
452	1.0001296828272\\
453	1.00013806347171\\
454	1.00014323809091\\
455	1.00014540615114\\
456	1.00014480017174\\
457	1.0001416789738\\
458	1.00013632106689\\
459	1.0001290182729\\
460	1.00012006967369\\
461	1.00010977595652\\
462	1.00009843421817\\
463	1.00008633327581\\
464	1.00007374952002\\
465	1.00006094333331\\
466	1.0000481560859\\
467	1.00003560770982\\
468	1.00002349484263\\
469	1.00001198952323\\
470	1.00000123841427\\
471	0.999991362519225\\
472	0.999982457356436\\
473	0.999974593547984\\
474	0.999967817777953\\
475	0.999962154072194\\
476	0.999957605350476\\
477	0.999954155201463\\
478	0.999951769831483\\
479	0.999950400139309\\
480	0.999949983871121\\
481	0.99995044781243\\
482	0.999951709976786\\
483	0.999953681754603\\
484	0.999956269989249\\
485	0.999959378951586\\
486	0.999962912188326\\
487	0.999966774223801\\
488	0.999970872098985\\
489	0.999975116735722\\
490	0.999979424118093\\
491	0.99998371628665\\
492	0.999987922144747\\
493	0.999991978079466\\
494	0.99999582840252\\
495	0.99999942561911\\
496	1.00000273053496\\
497	1.00000571221353\\
498	1.00000834779705\\
499	1.00001062220599\\
500	1.00001252773249\\
501	1.00001406354371\\
502	1.00001523511121\\
503	1.00001605358234\\
504	1.00001653510923\\
505	1.00001670015035\\
506	1.0000165727588\\
507	1.00001617987058\\
508	1.00001555060467\\
509	1.00001471558604\\
510	1.00001370630078\\
511	1.00040938687002\\
512	1.00111634385559\\
513	1.00172392016373\\
514	1.00227704645393\\
515	1.00281406265013\\
516	1.00336749042988\\
517	1.00396472359866\\
518	1.00462864462469\\
519	1.005378174803\\
520	1.00622876478752\\
521	1.0071736562442\\
522	1.00818959391501\\
523	1.00927499574596\\
524	1.01044285867689\\
525	1.01170039690845\\
526	1.01304992484452\\
527	1.01448962762909\\
528	1.01601423267138\\
529	1.01761559400932\\
530	1.01928319999231\\
531	1.0210055401198\\
532	1.02277202429034\\
533	1.0245727687922\\
534	1.02639697486112\\
535	1.02823262041945\\
536	1.0300670318369\\
537	1.03188736390347\\
538	1.03368100073895\\
539	1.03543588868484\\
540	1.03714081076041\\
541	1.03878556621624\\
542	1.04036099084052\\
543	1.04185889769697\\
544	1.0432721046191\\
545	1.0445945486784\\
546	1.04582138463271\\
547	1.04694903207134\\
548	1.04797517997097\\
549	1.04889875606294\\
550	1.04971986728836\\
551	1.05043971882005\\
552	1.05106052379489\\
553	1.05158541486351\\
554	1.05201835689525\\
555	1.05236405258994\\
556	1.05262783779425\\
557	1.05281556975452\\
558	1.05293351269471\\
559	1.05298822431716\\
560	1.05298644616155\\
561	1.05333263244468\\
562	1.05394921456856\\
563	1.05443231932367\\
564	1.05483302951003\\
565	1.05519562267473\\
566	1.05555827515993\\
567	1.05595369385869\\
568	1.05640968435715\\
569	1.05694966297499\\
570	1.05759311919599\\
571	1.05833682011125\\
572	1.05916046354588\\
573	1.06006488133588\\
574	1.06106501687024\\
575	1.06216959444918\\
576	1.06338201723138\\
577	1.06470115866325\\
578	1.06612205987656\\
579	1.06763654398343\\
580	1.06923375684288\\
581	1.07090157113358\\
582	1.07262854843728\\
583	1.07440376931444\\
584	1.07621525118013\\
585	1.07804967488953\\
586	1.07989298958076\\
587	1.08173092445078\\
588	1.08354941970057\\
589	1.08533498728995\\
590	1.0870750107602\\
591	1.0887579473229\\
592	1.09037336741601\\
593	1.09191191109257\\
594	1.09336532773929\\
595	1.09472659976837\\
596	1.09599004756228\\
597	1.09715138050907\\
598	1.09820770296188\\
599	1.09915748268227\\
600	1.1000004882354\\
601	1.10073770303673\\
602	1.10137122845058\\
603	1.10190418733924\\
604	1.10234062769767\\
605	1.10268541839338\\
606	1.10294413404599\\
607	1.10312293249677\\
608	1.1032284294665\\
609	1.10326757419677\\
610	1.10324752919333\\
611	1.10357322078997\\
612	1.10416755575576\\
613	1.10462710229684\\
614	1.10500337757068\\
615	1.1053410795647\\
616	1.10567878457563\\
617	1.10604957333897\\
618	1.10648159450064\\
619	1.10699857293944\\
620	1.10762026941351\\
621	1.10834368157685\\
622	1.10914869493528\\
623	1.11003628786775\\
624	1.11102151121821\\
625	1.11211315872286\\
626	1.11331466659475\\
627	1.1146249071052\\
628	1.11603888856487\\
629	1.1175483725503\\
630	1.11914241786794\\
631	1.12080878808583\\
632	1.12253591731215\\
633	1.12431274396181\\
634	1.12612713213262\\
635	1.12796560154174\\
636	1.12981393559359\\
637	1.1316576962092\\
638	1.13348265761338\\
639	1.13527516968724\\
640	1.13702246011925\\
641	1.1387128385308\\
642	1.14033573775641\\
643	1.14188167163494\\
644	1.1433422758195\\
645	1.1447104322657\\
646	1.145980374691\\
647	1.14714773985568\\
648	1.14820957351633\\
649	1.14916429863164\\
650	1.15001165231066\\
651	1.15075259922795\\
652	1.15138923393206\\
653	1.15192468347357\\
654	1.15236301001512\\
655	1.15270910546886\\
656	1.15296857522063\\
657	1.15314761441455\\
658	1.15325288141846\\
659	1.15329137228585\\
660	1.15327029935151\\
661	1.15359464542537\\
662	1.15418737495578\\
663	1.15464510835916\\
664	1.15501941276866\\
665	1.15535503460211\\
666	1.155690596331\\
667	1.15605922200227\\
668	1.15648910020565\\
669	1.1570039919949\\
670	1.1576236902334\\
671	1.15834522017804\\
672	1.1591484900928\\
673	1.16003449649167\\
674	1.16101830408993\\
675	1.16210871639956\\
676	1.16330917547781\\
677	1.16461855572892\\
678	1.1660318641527\\
679	1.16754085787656\\
680	1.16913458845456\\
681	1.17080080976581\\
682	1.17252794419407\\
683	1.17430491679859\\
684	1.17611957706646\\
685	1.17795842920068\\
686	1.17980724052501\\
687	1.18165155663356\\
688	1.183477135477\\
689	1.18527031098941\\
690	1.18701829548606\\
691	1.1887093840055\\
692	1.19033299577327\\
693	1.19187963214186\\
694	1.19334091751603\\
695	1.1947097239258\\
696	1.19598027654118\\
697	1.1971482049811\\
698	1.19821054926872\\
699	1.19916572801499\\
700	1.20001347532299\\
701	1.20075475413974\\
702	1.20139165848492\\
703	1.20192731598551\\
704	1.20236579038155\\
705	1.20271197605165\\
706	1.20297148161954\\
707	1.20315050611813\\
708	1.2032557123336\\
709	1.20329410114738\\
710	1.20327289001505\\
711	1.20280340246781\\
712	1.20197949443207\\
713	1.20121916410976\\
714	1.20048418239853\\
715	1.19974267734762\\
716	1.19896828523762\\
717	1.19813939477381\\
718	1.19723847653431\\
719	1.19625149024025\\
720	1.19516736283582\\
721	1.1939966656711\\
722	1.19276584118506\\
723	1.19147906815355\\
724	1.19012543003654\\
725	1.18869930913946\\
726	1.18719952083662\\
727	1.18562856671054\\
728	1.18399199220324\\
729	1.18229783590438\\
730	1.18055615898757\\
731	1.17877771994491\\
732	1.17697210277387\\
733	1.17514797964956\\
734	1.17331477603314\\
735	1.17148301795\\
736	1.16966379732088\\
737	1.16786832757869\\
738	1.16610757631066\\
739	1.16439196343536\\
740	1.16273111497917\\
741	1.16113370855843\\
742	1.15960747443339\\
743	1.15815927213312\\
744	1.1567950765124\\
745	1.15551987346826\\
746	1.15433756859892\\
747	1.15325094422691\\
748	1.15226165621367\\
749	1.15137026334976\\
750	1.15057628326406\\
751	1.14987826762436\\
752	1.14927388477541\\
753	1.14875999903333\\
754	1.14833274762912\\
755	1.14798762385052\\
756	1.14771956985092\\
757	1.14752307614174\\
758	1.14739228361596\\
759	1.14732108472827\\
760	1.14730322110229\\
761	1.14693478039478\\
762	1.14629387728537\\
763	1.14576769479887\\
764	1.1452749569674\\
765	1.14474622511151\\
766	1.14412257054753\\
767	1.1433543874333\\
768	1.14240033028543\\
769	1.14122636251588\\
770	1.13980490394141\\
771	1.13813327881775\\
772	1.13622775539906\\
773	1.13408590561064\\
774	1.13169482228744\\
775	1.12905159646821\\
776	1.12616177549122\\
777	1.12303801303747\\
778	1.11969888827224\\
779	1.1161678739883\\
780	1.1124724360704\\
781	1.10864232036899\\
782	1.10470732601222\\
783	1.10069720924071\\
784	1.09664292622671\\
785	1.0925765150098\\
786	1.08853013218155\\
787	1.08453524608933\\
788	1.08062196520243\\
789	1.07681848310651\\
790	1.07315062404299\\
791	1.06964151989627\\
792	1.06631147833079\\
793	1.0631779602663\\
794	1.06025550341444\\
795	1.05755559675842\\
796	1.05508660916733\\
797	1.05285380248214\\
798	1.05085941456476\\
799	1.04910279996318\\
800	1.04758061769505\\
801	1.04628705506404\\
802	1.0452140722705\\
803	1.044351653951\\
804	1.04368806551219\\
805	1.0432101193333\\
806	1.04290345062832\\
807	1.04275279655942\\
808	1.04274227153947\\
809	1.04285563294737\\
810	1.04307653255715\\
811	1.04299140055327\\
812	1.04266965616485\\
813	1.04250699534601\\
814	1.04244392825164\\
815	1.04242839530598\\
816	1.04241511838254\\
817	1.04236500994948\\
818	1.04224463192628\\
819	1.0420256973767\\
820	1.04168460935379\\
821	1.04122123254866\\
822	1.04065317567022\\
823	1.03997755132816\\
824	1.0391780176565\\
825	1.03824509623151\\
826	1.03717522005222\\
827	1.03596988412573\\
828	1.03463488734745\\
829	1.03317965587997\\
830	1.03161663952905\\
831	1.02995984594171\\
832	1.02822282091232\\
833	1.02641876520716\\
834	1.02456206402348\\
835	1.02266850710263\\
836	1.02075463337065\\
837	1.01883717345544\\
838	1.01693257809918\\
839	1.01505662198516\\
840	1.01322407378949\\
841	1.01144846921064\\
842	1.00974205151795\\
843	1.00811579985152\\
844	1.00657937856382\\
845	1.00514100699203\\
846	1.00380735230144\\
847	1.00258348023216\\
848	1.00147285448576\\
849	1.00047737674732\\
850	0.999597460423544\\
851	0.998832129947184\\
852	0.998179132813485\\
853	0.997635052547799\\
854	0.99719542259527\\
855	0.996854848745601\\
856	0.996607142692001\\
857	0.996445462924645\\
858	0.996362458044904\\
859	0.996350408427594\\
860	0.996401362874079\\
861	0.996109629816194\\
862	0.995551614064934\\
863	0.995009666148765\\
864	0.994216691386937\\
865	0.992947798135629\\
866	0.991015297256109\\
867	0.988264238824071\\
868	0.984568429788473\\
869	0.979826882167447\\
870	0.973960646650634\\
871	0.966929205462071\\
872	0.958722682741073\\
873	0.949327563251365\\
874	0.938742061889922\\
875	0.926997470478845\\
876	0.914152732667802\\
877	0.900289718232804\\
878	0.885526609339621\\
879	0.870048740470661\\
880	0.854065600402944\\
881	0.837757545633109\\
882	0.821279147033163\\
883	0.804762184475895\\
884	0.788321943064686\\
885	0.772069348052869\\
886	0.756116297553503\\
887	0.740572555930676\\
888	0.725542316593946\\
889	0.711118753508792\\
890	0.697379335889381\\
891	0.684385756488678\\
892	0.672186159185987\\
893	0.660817075710471\\
894	0.650304933391636\\
895	0.640666706305716\\
896	0.631909779237839\\
897	0.62403174813625\\
898	0.617020556497771\\
899	0.610855074757561\\
900	0.605506172482791\\
901	0.600938020907226\\
902	0.597109302249915\\
903	0.593974236039276\\
904	0.591483458059632\\
905	0.589584813766831\\
906	0.588224143381368\\
907	0.587346096947157\\
908	0.586894964978029\\
909	0.586815487083717\\
910	0.587053594013179\\
911	0.587557049286112\\
912	0.588275984373666\\
913	0.5891633404078\\
914	0.590175231538074\\
915	0.591271240227415\\
916	0.592414648094076\\
917	0.593572600485164\\
918	0.594716201925493\\
919	0.595820542082976\\
920	0.596864655766684\\
921	0.597831424121497\\
922	0.598707425980132\\
923	0.599482747943879\\
924	0.600150760422345\\
925	0.600707865682163\\
926	0.60115322333697\\
927	0.601488458671674\\
928	0.60171735945684\\
929	0.601845567099499\\
930	0.601880267866927\\
931	0.601829889460008\\
932	0.601703807504322\\
933	0.601512065760973\\
934	0.601265113179839\\
935	0.60097356035893\\
936	0.600647957498086\\
937	0.600298595488258\\
938	0.599935331317265\\
939	0.599567438488636\\
940	0.599203482659606\\
941	0.598851222237878\\
942	0.598517533261986\\
943	0.598208357540609\\
944	0.597928672739817\\
945	0.597682482873989\\
946	0.597472827466267\\
947	0.597301807492771\\
948	0.597170626110886\\
949	0.597079642097725\\
950	0.597028433891878\\
951	0.597015872139279\\
952	0.597040198688923\\
953	0.597099110061235\\
954	0.597189843515262\\
955	0.597309263965637\\
956	0.597453950142212\\
957	0.597620278541056\\
958	0.597804503881963\\
959	0.598002834961496\\
960	0.598211504968482\\
961	0.598426835507268\\
962	0.598645293749683\\
963	0.598863542306555\\
964	0.599078481571407\\
965	0.599287284440605\\
966	0.599487423454155\\
967	0.599676690528423\\
968	0.599853209565049\\
969	0.600015442318847\\
970	0.60016218799078\\
971	0.600292577080268\\
972	0.600406060084125\\
973	0.600502391667822\\
974	0.600581610959129\\
975	0.600644018625313\\
976	0.600690151394004\\
977	0.600720754665501\\
978	0.600736753842057\\
979	0.600739224968596\\
980	0.60072936524077\\
981	0.600708463891536\\
982	0.600677873917782\\
983	0.60063898505529\\
984	0.600593198354631\\
985	0.600541902653678\\
986	0.600486453185301\\
987	0.600428152502531\\
988	0.600368233848821\\
989	0.600307847048944\\
990	0.600248046947038\\
991	0.600189784373033\\
992	0.600133899577584\\
993	0.600081118039004\\
994	0.600032048513815\\
995	0.599987183175549\\
996	0.599946899664326\\
997	0.599911464852545\\
998	0.599881040119561\\
999	0.599855687920309\\
1000	0.599835379429238\\
};
\addlegendentry{Y}

\addplot[const plot, color=mycolor2] table[row sep=crcr] {%
1	0.9\\
2	0.9\\
3	0.9\\
4	0.9\\
5	0.9\\
6	0.9\\
7	0.9\\
8	0.9\\
9	0.9\\
10	0.9\\
11	0.9\\
12	1.35\\
13	1.35\\
14	1.35\\
15	1.35\\
16	1.35\\
17	1.35\\
18	1.35\\
19	1.35\\
20	1.35\\
21	1.35\\
22	1.35\\
23	1.35\\
24	1.35\\
25	1.35\\
26	1.35\\
27	1.35\\
28	1.35\\
29	1.35\\
30	1.35\\
31	1.35\\
32	1.35\\
33	1.35\\
34	1.35\\
35	1.35\\
36	1.35\\
37	1.35\\
38	1.35\\
39	1.35\\
40	1.35\\
41	1.35\\
42	1.35\\
43	1.35\\
44	1.35\\
45	1.35\\
46	1.35\\
47	1.35\\
48	1.35\\
49	1.35\\
50	1.35\\
51	1.35\\
52	1.35\\
53	1.35\\
54	1.35\\
55	1.35\\
56	1.35\\
57	1.35\\
58	1.35\\
59	1.35\\
60	1.35\\
61	1.35\\
62	1.35\\
63	1.35\\
64	1.35\\
65	1.35\\
66	1.35\\
67	1.35\\
68	1.35\\
69	1.35\\
70	1.35\\
71	1.35\\
72	1.35\\
73	1.35\\
74	1.35\\
75	1.35\\
76	1.35\\
77	1.35\\
78	1.35\\
79	1.35\\
80	1.35\\
81	1.35\\
82	1.35\\
83	1.35\\
84	1.35\\
85	1.35\\
86	1.35\\
87	1.35\\
88	1.35\\
89	1.35\\
90	1.35\\
91	1.35\\
92	1.35\\
93	1.35\\
94	1.35\\
95	1.35\\
96	1.35\\
97	1.35\\
98	1.35\\
99	1.35\\
100	1.35\\
101	1.35\\
102	1.35\\
103	1.35\\
104	1.35\\
105	1.35\\
106	1.35\\
107	1.35\\
108	1.35\\
109	1.35\\
110	1.35\\
111	1.35\\
112	1.35\\
113	1.35\\
114	1.35\\
115	1.35\\
116	1.35\\
117	1.35\\
118	1.35\\
119	1.35\\
120	1.35\\
121	1.35\\
122	1.35\\
123	1.35\\
124	1.35\\
125	1.35\\
126	1.35\\
127	1.35\\
128	1.35\\
129	1.35\\
130	1.35\\
131	1.35\\
132	1.35\\
133	1.35\\
134	1.35\\
135	1.35\\
136	1.35\\
137	1.35\\
138	1.35\\
139	1.35\\
140	1.35\\
141	1.35\\
142	1.35\\
143	1.35\\
144	1.35\\
145	1.35\\
146	1.35\\
147	1.35\\
148	1.35\\
149	1.35\\
150	1.35\\
151	0.8\\
152	0.8\\
153	0.8\\
154	0.8\\
155	0.8\\
156	0.8\\
157	0.8\\
158	0.8\\
159	0.8\\
160	0.8\\
161	0.8\\
162	0.8\\
163	0.8\\
164	0.8\\
165	0.8\\
166	0.8\\
167	0.8\\
168	0.8\\
169	0.8\\
170	0.8\\
171	0.8\\
172	0.8\\
173	0.8\\
174	0.8\\
175	0.8\\
176	0.8\\
177	0.8\\
178	0.8\\
179	0.8\\
180	0.8\\
181	0.8\\
182	0.8\\
183	0.8\\
184	0.8\\
185	0.8\\
186	0.8\\
187	0.8\\
188	0.8\\
189	0.8\\
190	0.8\\
191	0.8\\
192	0.8\\
193	0.8\\
194	0.8\\
195	0.8\\
196	0.8\\
197	0.8\\
198	0.8\\
199	0.8\\
200	0.8\\
201	0.8\\
202	0.8\\
203	0.8\\
204	0.8\\
205	0.8\\
206	0.8\\
207	0.8\\
208	0.8\\
209	0.8\\
210	0.8\\
211	0.8\\
212	0.8\\
213	0.8\\
214	0.8\\
215	0.8\\
216	0.8\\
217	0.8\\
218	0.8\\
219	0.8\\
220	0.8\\
221	0.8\\
222	0.8\\
223	0.8\\
224	0.8\\
225	0.8\\
226	0.8\\
227	0.8\\
228	0.8\\
229	0.8\\
230	0.8\\
231	0.8\\
232	0.8\\
233	0.8\\
234	0.8\\
235	0.8\\
236	0.8\\
237	0.8\\
238	0.8\\
239	0.8\\
240	0.8\\
241	0.8\\
242	0.8\\
243	0.8\\
244	0.8\\
245	0.8\\
246	0.8\\
247	0.8\\
248	0.8\\
249	0.8\\
250	0.8\\
251	0.8\\
252	0.8\\
253	0.8\\
254	0.8\\
255	0.8\\
256	0.8\\
257	0.8\\
258	0.8\\
259	0.8\\
260	0.8\\
261	0.8\\
262	0.8\\
263	0.8\\
264	0.8\\
265	0.8\\
266	0.8\\
267	0.8\\
268	0.8\\
269	0.8\\
270	0.8\\
271	0.8\\
272	0.8\\
273	0.8\\
274	0.8\\
275	0.8\\
276	0.8\\
277	0.8\\
278	0.8\\
279	0.8\\
280	0.8\\
281	0.8\\
282	0.8\\
283	0.8\\
284	0.8\\
285	0.8\\
286	0.8\\
287	0.8\\
288	0.8\\
289	0.8\\
290	0.8\\
291	0.8\\
292	0.8\\
293	0.8\\
294	0.8\\
295	0.8\\
296	0.8\\
297	0.8\\
298	0.8\\
299	0.8\\
300	0.8\\
301	1\\
302	1\\
303	1\\
304	1\\
305	1\\
306	1\\
307	1\\
308	1\\
309	1\\
310	1\\
311	1\\
312	1\\
313	1\\
314	1\\
315	1\\
316	1\\
317	1\\
318	1\\
319	1\\
320	1\\
321	1\\
322	1\\
323	1\\
324	1\\
325	1\\
326	1\\
327	1\\
328	1\\
329	1\\
330	1\\
331	1\\
332	1\\
333	1\\
334	1\\
335	1\\
336	1\\
337	1\\
338	1\\
339	1\\
340	1\\
341	1\\
342	1\\
343	1\\
344	1\\
345	1\\
346	1\\
347	1\\
348	1\\
349	1\\
350	1\\
351	1\\
352	1\\
353	1\\
354	1\\
355	1\\
356	1\\
357	1\\
358	1\\
359	1\\
360	1\\
361	1\\
362	1\\
363	1\\
364	1\\
365	1\\
366	1\\
367	1\\
368	1\\
369	1\\
370	1\\
371	1\\
372	1\\
373	1\\
374	1\\
375	1\\
376	1\\
377	1\\
378	1\\
379	1\\
380	1\\
381	1\\
382	1\\
383	1\\
384	1\\
385	1\\
386	1\\
387	1\\
388	1\\
389	1\\
390	1\\
391	1\\
392	1\\
393	1\\
394	1\\
395	1\\
396	1\\
397	1\\
398	1\\
399	1\\
400	1\\
401	1\\
402	1\\
403	1\\
404	1\\
405	1\\
406	1\\
407	1\\
408	1\\
409	1\\
410	1\\
411	1\\
412	1\\
413	1\\
414	1\\
415	1\\
416	1\\
417	1\\
418	1\\
419	1\\
420	1\\
421	1\\
422	1\\
423	1\\
424	1\\
425	1\\
426	1\\
427	1\\
428	1\\
429	1\\
430	1\\
431	1\\
432	1\\
433	1\\
434	1\\
435	1\\
436	1\\
437	1\\
438	1\\
439	1\\
440	1\\
441	1\\
442	1\\
443	1\\
444	1\\
445	1\\
446	1\\
447	1\\
448	1\\
449	1\\
450	1\\
451	1\\
452	1\\
453	1\\
454	1\\
455	1\\
456	1\\
457	1\\
458	1\\
459	1\\
460	1\\
461	1\\
462	1\\
463	1\\
464	1\\
465	1\\
466	1\\
467	1\\
468	1\\
469	1\\
470	1\\
471	1\\
472	1\\
473	1\\
474	1\\
475	1\\
476	1\\
477	1\\
478	1\\
479	1\\
480	1\\
481	1\\
482	1\\
483	1\\
484	1\\
485	1\\
486	1\\
487	1\\
488	1\\
489	1\\
490	1\\
491	1\\
492	1\\
493	1\\
494	1\\
495	1\\
496	1\\
497	1\\
498	1\\
499	1\\
500	1\\
501	1.05\\
502	1.05\\
503	1.05\\
504	1.05\\
505	1.05\\
506	1.05\\
507	1.05\\
508	1.05\\
509	1.05\\
510	1.05\\
511	1.05\\
512	1.05\\
513	1.05\\
514	1.05\\
515	1.05\\
516	1.05\\
517	1.05\\
518	1.05\\
519	1.05\\
520	1.05\\
521	1.05\\
522	1.05\\
523	1.05\\
524	1.05\\
525	1.05\\
526	1.05\\
527	1.05\\
528	1.05\\
529	1.05\\
530	1.05\\
531	1.05\\
532	1.05\\
533	1.05\\
534	1.05\\
535	1.05\\
536	1.05\\
537	1.05\\
538	1.05\\
539	1.05\\
540	1.05\\
541	1.05\\
542	1.05\\
543	1.05\\
544	1.05\\
545	1.05\\
546	1.05\\
547	1.05\\
548	1.05\\
549	1.05\\
550	1.05\\
551	1.1\\
552	1.1\\
553	1.1\\
554	1.1\\
555	1.1\\
556	1.1\\
557	1.1\\
558	1.1\\
559	1.1\\
560	1.1\\
561	1.1\\
562	1.1\\
563	1.1\\
564	1.1\\
565	1.1\\
566	1.1\\
567	1.1\\
568	1.1\\
569	1.1\\
570	1.1\\
571	1.1\\
572	1.1\\
573	1.1\\
574	1.1\\
575	1.1\\
576	1.1\\
577	1.1\\
578	1.1\\
579	1.1\\
580	1.1\\
581	1.1\\
582	1.1\\
583	1.1\\
584	1.1\\
585	1.1\\
586	1.1\\
587	1.1\\
588	1.1\\
589	1.1\\
590	1.1\\
591	1.1\\
592	1.1\\
593	1.1\\
594	1.1\\
595	1.1\\
596	1.1\\
597	1.1\\
598	1.1\\
599	1.1\\
600	1.1\\
601	1.15\\
602	1.15\\
603	1.15\\
604	1.15\\
605	1.15\\
606	1.15\\
607	1.15\\
608	1.15\\
609	1.15\\
610	1.15\\
611	1.15\\
612	1.15\\
613	1.15\\
614	1.15\\
615	1.15\\
616	1.15\\
617	1.15\\
618	1.15\\
619	1.15\\
620	1.15\\
621	1.15\\
622	1.15\\
623	1.15\\
624	1.15\\
625	1.15\\
626	1.15\\
627	1.15\\
628	1.15\\
629	1.15\\
630	1.15\\
631	1.15\\
632	1.15\\
633	1.15\\
634	1.15\\
635	1.15\\
636	1.15\\
637	1.15\\
638	1.15\\
639	1.15\\
640	1.15\\
641	1.15\\
642	1.15\\
643	1.15\\
644	1.15\\
645	1.15\\
646	1.15\\
647	1.15\\
648	1.15\\
649	1.15\\
650	1.15\\
651	1.2\\
652	1.2\\
653	1.2\\
654	1.2\\
655	1.2\\
656	1.2\\
657	1.2\\
658	1.2\\
659	1.2\\
660	1.2\\
661	1.2\\
662	1.2\\
663	1.2\\
664	1.2\\
665	1.2\\
666	1.2\\
667	1.2\\
668	1.2\\
669	1.2\\
670	1.2\\
671	1.2\\
672	1.2\\
673	1.2\\
674	1.2\\
675	1.2\\
676	1.2\\
677	1.2\\
678	1.2\\
679	1.2\\
680	1.2\\
681	1.2\\
682	1.2\\
683	1.2\\
684	1.2\\
685	1.2\\
686	1.2\\
687	1.2\\
688	1.2\\
689	1.2\\
690	1.2\\
691	1.2\\
692	1.2\\
693	1.2\\
694	1.2\\
695	1.2\\
696	1.2\\
697	1.2\\
698	1.2\\
699	1.2\\
700	1.2\\
701	1.15\\
702	1.15\\
703	1.15\\
704	1.15\\
705	1.15\\
706	1.15\\
707	1.15\\
708	1.15\\
709	1.15\\
710	1.15\\
711	1.15\\
712	1.15\\
713	1.15\\
714	1.15\\
715	1.15\\
716	1.15\\
717	1.15\\
718	1.15\\
719	1.15\\
720	1.15\\
721	1.15\\
722	1.15\\
723	1.15\\
724	1.15\\
725	1.15\\
726	1.15\\
727	1.15\\
728	1.15\\
729	1.15\\
730	1.15\\
731	1.15\\
732	1.15\\
733	1.15\\
734	1.15\\
735	1.15\\
736	1.15\\
737	1.15\\
738	1.15\\
739	1.15\\
740	1.15\\
741	1.15\\
742	1.15\\
743	1.15\\
744	1.15\\
745	1.15\\
746	1.15\\
747	1.15\\
748	1.15\\
749	1.15\\
750	1.15\\
751	1.05\\
752	1.05\\
753	1.05\\
754	1.05\\
755	1.05\\
756	1.05\\
757	1.05\\
758	1.05\\
759	1.05\\
760	1.05\\
761	1.05\\
762	1.05\\
763	1.05\\
764	1.05\\
765	1.05\\
766	1.05\\
767	1.05\\
768	1.05\\
769	1.05\\
770	1.05\\
771	1.05\\
772	1.05\\
773	1.05\\
774	1.05\\
775	1.05\\
776	1.05\\
777	1.05\\
778	1.05\\
779	1.05\\
780	1.05\\
781	1.05\\
782	1.05\\
783	1.05\\
784	1.05\\
785	1.05\\
786	1.05\\
787	1.05\\
788	1.05\\
789	1.05\\
790	1.05\\
791	1.05\\
792	1.05\\
793	1.05\\
794	1.05\\
795	1.05\\
796	1.05\\
797	1.05\\
798	1.05\\
799	1.05\\
800	1.05\\
801	1\\
802	1\\
803	1\\
804	1\\
805	1\\
806	1\\
807	1\\
808	1\\
809	1\\
810	1\\
811	1\\
812	1\\
813	1\\
814	1\\
815	1\\
816	1\\
817	1\\
818	1\\
819	1\\
820	1\\
821	1\\
822	1\\
823	1\\
824	1\\
825	1\\
826	1\\
827	1\\
828	1\\
829	1\\
830	1\\
831	1\\
832	1\\
833	1\\
834	1\\
835	1\\
836	1\\
837	1\\
838	1\\
839	1\\
840	1\\
841	1\\
842	1\\
843	1\\
844	1\\
845	1\\
846	1\\
847	1\\
848	1\\
849	1\\
850	1\\
851	0.6\\
852	0.6\\
853	0.6\\
854	0.6\\
855	0.6\\
856	0.6\\
857	0.6\\
858	0.6\\
859	0.6\\
860	0.6\\
861	0.6\\
862	0.6\\
863	0.6\\
864	0.6\\
865	0.6\\
866	0.6\\
867	0.6\\
868	0.6\\
869	0.6\\
870	0.6\\
871	0.6\\
872	0.6\\
873	0.6\\
874	0.6\\
875	0.6\\
876	0.6\\
877	0.6\\
878	0.6\\
879	0.6\\
880	0.6\\
881	0.6\\
882	0.6\\
883	0.6\\
884	0.6\\
885	0.6\\
886	0.6\\
887	0.6\\
888	0.6\\
889	0.6\\
890	0.6\\
891	0.6\\
892	0.6\\
893	0.6\\
894	0.6\\
895	0.6\\
896	0.6\\
897	0.6\\
898	0.6\\
899	0.6\\
900	0.6\\
901	0.6\\
902	0.6\\
903	0.6\\
904	0.6\\
905	0.6\\
906	0.6\\
907	0.6\\
908	0.6\\
909	0.6\\
910	0.6\\
911	0.6\\
912	0.6\\
913	0.6\\
914	0.6\\
915	0.6\\
916	0.6\\
917	0.6\\
918	0.6\\
919	0.6\\
920	0.6\\
921	0.6\\
922	0.6\\
923	0.6\\
924	0.6\\
925	0.6\\
926	0.6\\
927	0.6\\
928	0.6\\
929	0.6\\
930	0.6\\
931	0.6\\
932	0.6\\
933	0.6\\
934	0.6\\
935	0.6\\
936	0.6\\
937	0.6\\
938	0.6\\
939	0.6\\
940	0.6\\
941	0.6\\
942	0.6\\
943	0.6\\
944	0.6\\
945	0.6\\
946	0.6\\
947	0.6\\
948	0.6\\
949	0.6\\
950	0.6\\
951	0.6\\
952	0.6\\
953	0.6\\
954	0.6\\
955	0.6\\
956	0.6\\
957	0.6\\
958	0.6\\
959	0.6\\
960	0.6\\
961	0.6\\
962	0.6\\
963	0.6\\
964	0.6\\
965	0.6\\
966	0.6\\
967	0.6\\
968	0.6\\
969	0.6\\
970	0.6\\
971	0.6\\
972	0.6\\
973	0.6\\
974	0.6\\
975	0.6\\
976	0.6\\
977	0.6\\
978	0.6\\
979	0.6\\
980	0.6\\
981	0.6\\
982	0.6\\
983	0.6\\
984	0.6\\
985	0.6\\
986	0.6\\
987	0.6\\
988	0.6\\
989	0.6\\
990	0.6\\
991	0.6\\
992	0.6\\
993	0.6\\
994	0.6\\
995	0.6\\
996	0.6\\
997	0.6\\
998	0.6\\
999	0.6\\
1000	0.6\\
};
\addlegendentry{Yzad}

\end{axis}
\end{tikzpicture}%
\caption{Śledzenie wartości zadanej dla parametrów $K = 1,3$, $T_i = 10$, $T_d = 3$}
\end{figure}

Wskaźnik jakości regulacji:

\begin{equation}
E = 20,5988
\end{equation}

%! TEX encoding = utf8
\chapter{Regulacja z skokowym zakłóceniem}

Parametr $D^Z$ został dobrany analogicznie do parametru D: przyjęto go jako wartość k, dla której odpowiedź skokowa toru zakłócenie-wyjście stabilizuje się:
$D^Z=68$

W doświadczeniach skok wartości zakłócenia następowował w chwili k=60 i wynosił 1. Użyto regulatora o najlepszych parametrach z poprzedniego podpunktu ($N_U=50$ $N=2$ $\lambda=0,4$)

\section{Bez pomiaru zakłócenia}

\begin{figure}[H]
\centering
% This file was created by matlab2tikz.
%
%The latest updates can be retrieved from
%  http://www.mathworks.com/matlabcentral/fileexchange/22022-matlab2tikz-matlab2tikz
%where you can also make suggestions and rate matlab2tikz.
%
\definecolor{mycolor1}{rgb}{0.00000,0.44700,0.74100}%
%
\begin{tikzpicture}

\begin{axis}[%
width=4.272in,
height=1.075in,
at={(0.717in,1.839in)},
scale only axis,
xmin=0,
xmax=150,
xlabel style={font=\color{white!15!black}},
xlabel={k},
ymin=-1,
ymax=2,
ylabel style={font=\color{white!15!black}},
ylabel={U(k)},
axis background/.style={fill=white}
]
\addplot[const plot, color=mycolor1, forget plot] table[row sep=crcr] {%
1	0\\
2	0\\
3	0\\
4	0\\
5	0\\
6	0\\
7	0\\
8	0\\
9	0\\
10	0\\
11	0\\
12	0\\
13	0\\
14	0\\
15	0\\
16	0\\
17	0\\
18	0\\
19	0\\
20	0\\
21	0\\
22	0\\
23	0\\
24	0\\
25	0\\
26	0\\
27	0\\
28	0\\
29	0\\
30	1.20577393725138\\
31	1.538576433506\\
32	1.46712127199355\\
33	1.24868968296282\\
34	1.01114447937986\\
35	0.808413088923491\\
36	0.655593901391529\\
37	0.54975127059157\\
38	0.481347710796871\\
39	0.4399381217831\\
40	0.416581469792359\\
41	0.404525444762055\\
42	0.399086664830369\\
43	0.397234397851693\\
44	0.397131373389285\\
45	0.397739526004153\\
46	0.39852169972768\\
47	0.39923446155087\\
48	0.39979385972787\\
49	0.400193942132608\\
50	0.400460651022531\\
51	0.400627917253023\\
52	0.40072678887092\\
53	0.400781648510957\\
54	0.400809906317153\\
55	0.40082312096247\\
56	0.400828485130777\\
57	0.40083019020383\\
58	0.400830499953435\\
59	0.400830517872924\\
60	0.400830696699621\\
61	0.400831155277304\\
62	0.400831862848653\\
63	0.157242287350464\\
64	-0.126237559941116\\
65	-0.363459520974733\\
66	-0.529921027129546\\
67	-0.629701331604633\\
68	-0.677317062855556\\
69	-0.6887063241844\\
70	-0.677425709267158\\
71	-0.653583721587496\\
72	-0.624044297714149\\
73	-0.593094737885873\\
74	-0.563177017411293\\
75	-0.535512839880241\\
76	-0.510574454643001\\
77	-0.488409874901524\\
78	-0.468851951370847\\
79	-0.451643720698643\\
80	-0.436507830034587\\
81	-0.423181049970157\\
82	-0.411428452429936\\
83	-0.401046682468229\\
84	-0.391862032768448\\
85	-0.383726537435592\\
86	-0.376513739545895\\
87	-0.370114871271103\\
88	-0.36443568884242\\
89	-0.359393962541464\\
90	-0.354917523842451\\
91	-0.35094274842261\\
92	-0.34741336425909\\
93	-0.344279496267529\\
94	-0.34149688203281\\
95	-0.339026212664641\\
96	-0.336832567538444\\
97	-0.33488492203586\\
98	-0.333155714294796\\
99	-0.331620461381576\\
100	-0.330257418021999\\
101	-0.329047272684659\\
102	-0.327972876821419\\
103	-0.327019003710337\\
104	-0.326172133782785\\
105	-0.325420263645123\\
106	-0.32475273627751\\
107	-0.324160090133692\\
108	-0.323633925086959\\
109	-0.32316678337286\\
110	-0.322752043869628\\
111	-0.322383828232466\\
112	-0.322056917557802\\
113	-0.321766678398499\\
114	-0.321508997081482\\
115	-0.321280221396033\\
116	-0.321077108825327\\
117	-0.32089678058657\\
118	-0.320736680827671\\
119	-0.320594540401633\\
120	-0.320468344704919\\
121	-0.320356305123746\\
122	-0.320256833683494\\
123	-0.320168520541867\\
124	-0.320090114006785\\
125	-0.320020502795781\\
126	-0.319958700285458\\
127	-0.319903830527761\\
128	-0.319855115834888\\
129	-0.319811865756863\\
130	-0.31977346729556\\
131	-0.319739376216471\\
132	-0.319709109335089\\
133	-0.319682237668568\\
134	-0.3196583803556\\
135	-0.319637199258342\\
136	-0.319618970545337\\
137	-0.319603693852693\\
138	-0.319591029401118\\
139	-0.319580465741697\\
140	-0.319571482995506\\
141	-0.319563647754877\\
142	-0.319556644418344\\
143	-0.31955026668201\\
144	-0.319544391412479\\
145	-0.319538949671546\\
146	-0.319533902432941\\
147	-0.319529223467581\\
148	-0.319524889033341\\
149	-0.319520872804399\\
150	-0.319517144261884\\
};
\end{axis}

\begin{axis}[%
width=4.272in,
height=1.075in,
at={(0.717in,0.346in)},
scale only axis,
xmin=0,
xmax=150,
xlabel style={font=\color{white!15!black}},
xlabel={k},
ymin=0,
ymax=1,
ylabel style={font=\color{white!15!black}},
ylabel={Z(k)},
axis background/.style={fill=white}
]
\addplot[const plot, color=mycolor1, forget plot] table[row sep=crcr] {%
1	0\\
2	0\\
3	0\\
4	0\\
5	0\\
6	0\\
7	0\\
8	0\\
9	0\\
10	0\\
11	0\\
12	0\\
13	0\\
14	0\\
15	0\\
16	0\\
17	0\\
18	0\\
19	0\\
20	0\\
21	0\\
22	0\\
23	0\\
24	0\\
25	0\\
26	0\\
27	0\\
28	0\\
29	0\\
30	0\\
31	0\\
32	0\\
33	0\\
34	0\\
35	0\\
36	0\\
37	0\\
38	0\\
39	0\\
40	0\\
41	0\\
42	0\\
43	0\\
44	0\\
45	0\\
46	0\\
47	0\\
48	0\\
49	0\\
50	0\\
51	0\\
52	0\\
53	0\\
54	0\\
55	0\\
56	0\\
57	0\\
58	0\\
59	0\\
60	1\\
61	1\\
62	1\\
63	1\\
64	1\\
65	1\\
66	1\\
67	1\\
68	1\\
69	1\\
70	1\\
71	1\\
72	1\\
73	1\\
74	1\\
75	1\\
76	1\\
77	1\\
78	1\\
79	1\\
80	1\\
81	1\\
82	1\\
83	1\\
84	1\\
85	1\\
86	1\\
87	1\\
88	1\\
89	1\\
90	1\\
91	1\\
92	1\\
93	1\\
94	1\\
95	1\\
96	1\\
97	1\\
98	1\\
99	1\\
100	1\\
101	1\\
102	1\\
103	1\\
104	1\\
105	1\\
106	1\\
107	1\\
108	1\\
109	1\\
110	1\\
111	1\\
112	1\\
113	1\\
114	1\\
115	1\\
116	1\\
117	1\\
118	1\\
119	1\\
120	1\\
121	1\\
122	1\\
123	1\\
124	1\\
125	1\\
126	1\\
127	1\\
128	1\\
129	1\\
130	1\\
131	1\\
132	1\\
133	1\\
134	1\\
135	1\\
136	1\\
137	1\\
138	1\\
139	1\\
140	1\\
141	1\\
142	1\\
143	1\\
144	1\\
145	1\\
146	1\\
147	1\\
148	1\\
149	1\\
150	1\\
};
\end{axis}
\end{tikzpicture}%
\caption{Zakłócenie i sygnał sterujący}
\end{figure}

\begin{figure}[H]
\centering
% This file was created by matlab2tikz.
%
%The latest updates can be retrieved from
%  http://www.mathworks.com/matlabcentral/fileexchange/22022-matlab2tikz-matlab2tikz
%where you can also make suggestions and rate matlab2tikz.
%
\definecolor{mycolor1}{rgb}{0.00000,0.44700,0.74100}%
\definecolor{mycolor2}{rgb}{0.85000,0.32500,0.09800}%
%
\begin{tikzpicture}

\begin{axis}[%
width=4.272in,
height=3.472in,
at={(0.717in,0.441in)},
scale only axis,
xmin=0,
xmax=150,
xlabel style={font=\color{white!15!black}},
xlabel={k},
ymin=0,
ymax=2.5,
ylabel style={font=\color{white!15!black}},
ylabel={Y(k)},
axis background/.style={fill=white},
legend style={legend cell align=left, align=left, draw=white!15!black}
]
\addplot[const plot, color=mycolor1] table[row sep=crcr] {%
1	0\\
2	0\\
3	0\\
4	0\\
5	0\\
6	0\\
7	0\\
8	0\\
9	0\\
10	0\\
11	0\\
12	0\\
13	0\\
14	0\\
15	0\\
16	0\\
17	0\\
18	0\\
19	0\\
20	0\\
21	0\\
22	0\\
23	0\\
24	0\\
25	0\\
26	0\\
27	0\\
28	0\\
29	0\\
30	0\\
31	0\\
32	0\\
33	0\\
34	0\\
35	0\\
36	0\\
37	0.178309849840735\\
38	0.3952760141839\\
39	0.588826611204609\\
40	0.738611928918155\\
41	0.844395923345594\\
42	0.913931740693088\\
43	0.956746557749988\\
44	0.981369563669113\\
45	0.994414676921337\\
46	1.00055965866249\\
47	1.00288312467568\\
48	1.00328286848648\\
49	1.00285171378131\\
50	1.00216956244591\\
51	1.0015102513556\\
52	1.00097786168913\\
53	1.00059084702661\\
54	1.00033054211392\\
55	1.00016694785008\\
56	1.0000709398569\\
57	1.00001893584117\\
58	0.999993751094316\\
59	0.999983795371811\\
60	0.999981757686305\\
61	0.999983323334048\\
62	0.999986132468706\\
63	1.20200901984749\\
64	1.38135527326315\\
65	1.54056803752264\\
66	1.68190550206501\\
67	1.8073727524717\\
68	1.91875009364266\\
69	2.01761822323469\\
70	2.0693584414719\\
71	2.07145087484841\\
72	2.03419876679132\\
73	1.97086040075434\\
74	1.89325390759107\\
75	1.81030478626227\\
76	1.72801207180565\\
77	1.649978341194\\
78	1.57806557235301\\
79	1.51298163218367\\
80	1.45473286931154\\
81	1.40294140734393\\
82	1.35705090023108\\
83	1.31645019607381\\
84	1.2805413319722\\
85	1.24877235928081\\
86	1.22064948792968\\
87	1.19573806672392\\
88	1.17365824335375\\
89	1.15407864407076\\
90	1.13670981890269\\
91	1.12129824892082\\
92	1.10762118766433\\
93	1.09548234664775\\
94	1.08470832378486\\
95	1.07514564261862\\
96	1.0666582770586\\
97	1.0591255573484\\
98	1.05244037632262\\
99	1.04650763550207\\
100	1.04124288664115\\
101	1.03657113610885\\
102	1.03242578773487\\
103	1.02874770539022\\
104	1.02548438039246\\
105	1.02258919144265\\
106	1.02002074664989\\
107	1.0177422985692\\
108	1.01572122425182\\
109	1.01392856319337\\
110	1.01233860682827\\
111	1.01092853389224\\
112	1.00967808657854\\
113	1.00856928295727\\
114	1.00758616161785\\
115	1.00671455493613\\
116	1.00594188776438\\
117	1.00525699869713\\
118	1.004649981383\\
119	1.00411204363537\\
120	1.00363538234608\\
121	1.00321307243013\\
122	1.00283896822795\\
123	1.00250761596836\\
124	1.00221417605189\\
125	1.00195435405332\\
126	1.00172433946584\\
127	1.00152075131856\\
128	1.00134058989688\\
129	1.0011811938812\\
130	1.00104020229636\\
131	1.00091552073247\\
132	1.00080529135788\\
133	1.00070786629906\\
134	1.00062178400983\\
135	1.0005457482944\\
136	1.00047860968678\\
137	1.00041934892189\\
138	1.00036706226403\\
139	1.00032094848399\\
140	1.00028029729996\\
141	1.00024447911798\\
142	1.00021293592591\\
143	1.00018508797721\\
144	1.00016037753897\\
145	1.00013831917287\\
146	1.00011852207629\\
147	1.00010068682141\\
148	1.0000845881299\\
149	1.0000700539239\\
150	1.0000569467785\\
};
\addlegendentry{Y}

\addplot[const plot, color=mycolor2] table[row sep=crcr] {%
1	0\\
2	0\\
3	0\\
4	0\\
5	0\\
6	0\\
7	0\\
8	0\\
9	0\\
10	0\\
11	0\\
12	0\\
13	0\\
14	0\\
15	0\\
16	0\\
17	0\\
18	0\\
19	0\\
20	0\\
21	0\\
22	0\\
23	0\\
24	0\\
25	0\\
26	0\\
27	0\\
28	0\\
29	0\\
30	1\\
31	1\\
32	1\\
33	1\\
34	1\\
35	1\\
36	1\\
37	1\\
38	1\\
39	1\\
40	1\\
41	1\\
42	1\\
43	1\\
44	1\\
45	1\\
46	1\\
47	1\\
48	1\\
49	1\\
50	1\\
51	1\\
52	1\\
53	1\\
54	1\\
55	1\\
56	1\\
57	1\\
58	1\\
59	1\\
60	1\\
61	1\\
62	1\\
63	1\\
64	1\\
65	1\\
66	1\\
67	1\\
68	1\\
69	1\\
70	1\\
71	1\\
72	1\\
73	1\\
74	1\\
75	1\\
76	1\\
77	1\\
78	1\\
79	1\\
80	1\\
81	1\\
82	1\\
83	1\\
84	1\\
85	1\\
86	1\\
87	1\\
88	1\\
89	1\\
90	1\\
91	1\\
92	1\\
93	1\\
94	1\\
95	1\\
96	1\\
97	1\\
98	1\\
99	1\\
100	1\\
101	1\\
102	1\\
103	1\\
104	1\\
105	1\\
106	1\\
107	1\\
108	1\\
109	1\\
110	1\\
111	1\\
112	1\\
113	1\\
114	1\\
115	1\\
116	1\\
117	1\\
118	1\\
119	1\\
120	1\\
121	1\\
122	1\\
123	1\\
124	1\\
125	1\\
126	1\\
127	1\\
128	1\\
129	1\\
130	1\\
131	1\\
132	1\\
133	1\\
134	1\\
135	1\\
136	1\\
137	1\\
138	1\\
139	1\\
140	1\\
141	1\\
142	1\\
143	1\\
144	1\\
145	1\\
146	1\\
147	1\\
148	1\\
149	1\\
150	1\\
151	1\\
152	1\\
153	1\\
154	1\\
155	1\\
156	1\\
157	1\\
158	1\\
159	1\\
160	1\\
161	1\\
162	1\\
163	1\\
164	1\\
165	1\\
166	1\\
167	1\\
168	1\\
169	1\\
170	1\\
171	1\\
172	1\\
173	1\\
174	1\\
175	1\\
176	1\\
177	1\\
178	1\\
179	1\\
180	1\\
181	1\\
182	1\\
183	1\\
184	1\\
185	1\\
186	1\\
187	1\\
188	1\\
189	1\\
190	1\\
191	1\\
192	1\\
193	1\\
194	1\\
195	1\\
196	1\\
197	1\\
198	1\\
199	1\\
200	1\\
};
\addlegendentry{Yzad}

\end{axis}
\end{tikzpicture}%
\caption{Wyjście obiektu bez pomiaru zakłócenia błąd $E=20,06103$}
\end{figure}

\section{Z pomiaru zakłócenia}

\begin{figure}[H]
\centering
% This file was created by matlab2tikz.
%
%The latest updates can be retrieved from
%  http://www.mathworks.com/matlabcentral/fileexchange/22022-matlab2tikz-matlab2tikz
%where you can also make suggestions and rate matlab2tikz.
%
\definecolor{mycolor1}{rgb}{0.00000,0.44700,0.74100}%
%
\begin{tikzpicture}

\begin{axis}[%
width=4.272in,
height=1.075in,
at={(0.717in,1.839in)},
scale only axis,
xmin=0,
xmax=150,
xlabel style={font=\color{white!15!black}},
xlabel={k},
ymin=-2,
ymax=2,
ylabel style={font=\color{white!15!black}},
ylabel={U(k)},
axis background/.style={fill=white}
]
\addplot[const plot, color=mycolor1, forget plot] table[row sep=crcr] {%
1	0\\
2	0\\
3	0\\
4	0\\
5	0\\
6	0\\
7	0\\
8	0\\
9	0\\
10	0\\
11	0\\
12	0\\
13	0\\
14	0\\
15	0\\
16	0\\
17	0\\
18	0\\
19	0\\
20	0\\
21	0\\
22	0\\
23	0\\
24	0\\
25	0\\
26	0\\
27	0\\
28	0\\
29	0\\
30	1.20577393725138\\
31	1.538576433506\\
32	1.46712127199355\\
33	1.24868968296282\\
34	1.01114447937986\\
35	0.808413088923491\\
36	0.655593901391529\\
37	0.54975127059157\\
38	0.481347710796871\\
39	0.4399381217831\\
40	0.416581469792359\\
41	0.404525444762055\\
42	0.399086664830369\\
43	0.397234397851693\\
44	0.397131373389285\\
45	0.397739526004153\\
46	0.39852169972768\\
47	0.39923446155087\\
48	0.39979385972787\\
49	0.400193942132608\\
50	0.400460651022531\\
51	0.400627917253023\\
52	0.40072678887092\\
53	0.400781648510957\\
54	0.400809906317153\\
55	0.40082312096247\\
56	0.400828485130777\\
57	0.40083019020383\\
58	0.400830499953435\\
59	0.400830517872924\\
60	0.400830696699621\\
61	-0.956483506635671\\
62	-1.42231705273603\\
63	-1.69160005144651\\
64	-1.57440118893383\\
65	-1.32303372579229\\
66	-1.06137980039552\\
67	-0.84045625110547\\
68	-0.673366200972875\\
69	-0.555872660126145\\
70	-0.477622141501927\\
71	-0.427672269010304\\
72	-0.396782870956662\\
73	-0.378019506667135\\
74	-0.36658239054199\\
75	-0.359359587761817\\
76	-0.354450780966218\\
77	-0.350764189059999\\
78	-0.347714140184759\\
79	-0.345012230373121\\
80	-0.34253454725985\\
81	-0.340236507617432\\
82	-0.338107839188851\\
83	-0.336149233661171\\
84	-0.334361662771399\\
85	-0.332742547278446\\
86	-0.331285267318234\\
87	-0.329980036413333\\
88	-0.328815119465039\\
89	-0.327777934873299\\
90	-0.326855883632153\\
91	-0.326036895437746\\
92	-0.325309741392093\\
93	-0.324664177228084\\
94	-0.324090974943965\\
95	-0.323581887756868\\
96	-0.32312957990848\\
97	-0.322727541734464\\
98	-0.322370002185988\\
99	-0.322051845401083\\
100	-0.321768534388962\\
101	-0.321516042815174\\
102	-0.321290794761743\\
103	-0.321089611814396\\
104	-0.320909666645529\\
105	-0.320748442254937\\
106	-0.32060369610405\\
107	-0.320473428479832\\
108	-0.320355854524882\\
109	-0.320249379459184\\
110	-0.320152576593155\\
111	-0.32006416779192\\
112	-0.31998300609928\\
113	-0.319908060269044\\
114	-0.319838400983534\\
115	-0.319773188565772\\
116	-0.319711662014469\\
117	-0.31965312921043\\
118	-0.319596958159853\\
119	-0.319542569154915\\
120	-0.319489427745129\\
121	-0.319437038424593\\
122	-0.319384938950585\\
123	-0.319332695218109\\
124	-0.319291037900742\\
125	-0.319262170092156\\
126	-0.319244549398616\\
127	-0.319235443425775\\
128	-0.319232150017157\\
129	-0.319232472512029\\
130	-0.319314879354428\\
131	-0.319410536080885\\
132	-0.319490386170859\\
133	-0.319546214751958\\
134	-0.319579456452548\\
135	-0.319595058163842\\
136	-0.319599009208302\\
137	-0.319596172496547\\
138	-0.319589868275447\\
139	-0.319582116847833\\
140	-0.319574032655541\\
141	-0.319566170441952\\
142	-0.319558771107165\\
143	-0.319551914946694\\
144	-0.319545607575466\\
145	-0.319539823385151\\
146	-0.319534525122145\\
147	-0.319529671522524\\
148	-0.319525219785899\\
149	-0.319521126280475\\
150	-0.319517346909615\\
};
\end{axis}

\begin{axis}[%
width=4.272in,
height=1.075in,
at={(0.717in,0.346in)},
scale only axis,
xmin=0,
xmax=150,
xlabel style={font=\color{white!15!black}},
xlabel={k},
ymin=0,
ymax=1,
ylabel style={font=\color{white!15!black}},
ylabel={Z(k)},
axis background/.style={fill=white}
]
\addplot[const plot, color=mycolor1, forget plot] table[row sep=crcr] {%
1	0\\
2	0\\
3	0\\
4	0\\
5	0\\
6	0\\
7	0\\
8	0\\
9	0\\
10	0\\
11	0\\
12	0\\
13	0\\
14	0\\
15	0\\
16	0\\
17	0\\
18	0\\
19	0\\
20	0\\
21	0\\
22	0\\
23	0\\
24	0\\
25	0\\
26	0\\
27	0\\
28	0\\
29	0\\
30	0\\
31	0\\
32	0\\
33	0\\
34	0\\
35	0\\
36	0\\
37	0\\
38	0\\
39	0\\
40	0\\
41	0\\
42	0\\
43	0\\
44	0\\
45	0\\
46	0\\
47	0\\
48	0\\
49	0\\
50	0\\
51	0\\
52	0\\
53	0\\
54	0\\
55	0\\
56	0\\
57	0\\
58	0\\
59	0\\
60	1\\
61	1\\
62	1\\
63	1\\
64	1\\
65	1\\
66	1\\
67	1\\
68	1\\
69	1\\
70	1\\
71	1\\
72	1\\
73	1\\
74	1\\
75	1\\
76	1\\
77	1\\
78	1\\
79	1\\
80	1\\
81	1\\
82	1\\
83	1\\
84	1\\
85	1\\
86	1\\
87	1\\
88	1\\
89	1\\
90	1\\
91	1\\
92	1\\
93	1\\
94	1\\
95	1\\
96	1\\
97	1\\
98	1\\
99	1\\
100	1\\
101	1\\
102	1\\
103	1\\
104	1\\
105	1\\
106	1\\
107	1\\
108	1\\
109	1\\
110	1\\
111	1\\
112	1\\
113	1\\
114	1\\
115	1\\
116	1\\
117	1\\
118	1\\
119	1\\
120	1\\
121	1\\
122	1\\
123	1\\
124	1\\
125	1\\
126	1\\
127	1\\
128	1\\
129	1\\
130	1\\
131	1\\
132	1\\
133	1\\
134	1\\
135	1\\
136	1\\
137	1\\
138	1\\
139	1\\
140	1\\
141	1\\
142	1\\
143	1\\
144	1\\
145	1\\
146	1\\
147	1\\
148	1\\
149	1\\
150	1\\
};
\end{axis}
\end{tikzpicture}%
\caption{Zakłócenie i sygnał sterujący}
\end{figure}

\begin{figure}[H]
\centering
% This file was created by matlab2tikz.
%
%The latest updates can be retrieved from
%  http://www.mathworks.com/matlabcentral/fileexchange/22022-matlab2tikz-matlab2tikz
%where you can also make suggestions and rate matlab2tikz.
%
\definecolor{mycolor1}{rgb}{0.00000,0.44700,0.74100}%
\definecolor{mycolor2}{rgb}{0.85000,0.32500,0.09800}%
%
\begin{tikzpicture}

\begin{axis}[%
width=4.272in,
height=3.472in,
at={(0.717in,0.441in)},
scale only axis,
xmin=0,
xmax=150,
xlabel style={font=\color{white!15!black}},
xlabel={k},
ymin=0,
ymax=2,
ylabel style={font=\color{white!15!black}},
ylabel={Y(k)},
axis background/.style={fill=white},
legend style={legend cell align=left, align=left, draw=white!15!black}
]
\addplot[const plot, color=mycolor1] table[row sep=crcr] {%
1	0\\
2	0\\
3	0\\
4	0\\
5	0\\
6	0\\
7	0\\
8	0\\
9	0\\
10	0\\
11	0\\
12	0\\
13	0\\
14	0\\
15	0\\
16	0\\
17	0\\
18	0\\
19	0\\
20	0\\
21	0\\
22	0\\
23	0\\
24	0\\
25	0\\
26	0\\
27	0\\
28	0\\
29	0\\
30	0\\
31	0\\
32	0\\
33	0\\
34	0\\
35	0\\
36	0\\
37	0.178309849840735\\
38	0.3952760141839\\
39	0.588826611204609\\
40	0.738611928918155\\
41	0.844395923345594\\
42	0.913931740693088\\
43	0.956746557749988\\
44	0.981369563669113\\
45	0.994414676921337\\
46	1.00055965866249\\
47	1.00288312467568\\
48	1.00328286848648\\
49	1.00285171378131\\
50	1.00216956244591\\
51	1.0015102513556\\
52	1.00097786168913\\
53	1.00059084702661\\
54	1.00033054211392\\
55	1.00016694785008\\
56	1.0000709398569\\
57	1.00001893584117\\
58	0.999993751094316\\
59	0.999983795371811\\
60	0.999981757686305\\
61	0.999983323334048\\
62	0.999986132468706\\
63	1.20200901984749\\
64	1.38135527326315\\
65	1.54056803752264\\
66	1.68190550206501\\
67	1.8073727524717\\
68	1.71803040143897\\
69	1.55917677439691\\
70	1.36465794710609\\
71	1.19432884419806\\
72	1.06712207890516\\
73	0.982470519346334\\
74	0.93224375885162\\
75	0.906808543559815\\
76	0.897687302718937\\
77	0.898398925006169\\
78	0.904431790486833\\
79	0.912879111214112\\
80	0.922006188243962\\
81	0.9308682152983\\
82	0.939016455979743\\
83	0.946291953364617\\
84	0.952690805820321\\
85	0.958281916806407\\
86	0.963160298348053\\
87	0.967422597423599\\
88	0.971156508416861\\
89	0.974437330618857\\
90	0.977328058647748\\
91	0.979880937840613\\
92	0.982139396763533\\
93	0.984139851291592\\
94	0.985913195886471\\
95	0.987485958320915\\
96	0.988881162546951\\
97	0.990118964143804\\
98	0.99121711968075\\
99	0.992191339692066\\
100	0.993055561966445\\
101	0.993822170587604\\
102	0.994502177542927\\
103	0.995105377638433\\
104	0.995640483431401\\
105	0.996115244361213\\
106	0.996536552741922\\
107	0.996910538408444\\
108	0.99724265332441\\
109	0.9975377471952\\
110	0.997800134981086\\
111	0.998033657113194\\
112	0.998241733147466\\
113	0.998427409533445\\
114	0.99859340211934\\
115	0.998742133960592\\
116	0.998875768946308\\
117	0.998996241707235\\
118	0.999105284221292\\
119	0.999204449488526\\
120	0.999295132607073\\
121	0.99937858954528\\
122	0.999455953872464\\
123	0.999528251681592\\
124	0.999596414911225\\
125	0.999661293250975\\
126	0.99972366479431\\
127	0.999784245584398\\
128	0.999843698182612\\
129	0.999902639375107\\
130	0.999961647120263\\
131	1.00001961925648\\
132	1.00007514066796\\
133	1.00012706158779\\
134	1.00017466309325\\
135	1.00021763176396\\
136	1.00025596503561\\
137	1.00027802769044\\
138	1.00028302707354\\
139	1.00027449165336\\
140	1.000256935637\\
141	1.00023437586694\\
142	1.00020984385587\\
143	1.00018529173658\\
144	1.00016182431907\\
145	1.00013997879337\\
146	1.00011995182548\\
147	1.00010175479276\\
148	1.0000853085142\\
149	1.00007049590441\\
150	1.00005718875066\\
};
\addlegendentry{Y}

\addplot[const plot, color=mycolor2] table[row sep=crcr] {%
1	0\\
2	0\\
3	0\\
4	0\\
5	0\\
6	0\\
7	0\\
8	0\\
9	0\\
10	0\\
11	0\\
12	0\\
13	0\\
14	0\\
15	0\\
16	0\\
17	0\\
18	0\\
19	0\\
20	0\\
21	0\\
22	0\\
23	0\\
24	0\\
25	0\\
26	0\\
27	0\\
28	0\\
29	0\\
30	1\\
31	1\\
32	1\\
33	1\\
34	1\\
35	1\\
36	1\\
37	1\\
38	1\\
39	1\\
40	1\\
41	1\\
42	1\\
43	1\\
44	1\\
45	1\\
46	1\\
47	1\\
48	1\\
49	1\\
50	1\\
51	1\\
52	1\\
53	1\\
54	1\\
55	1\\
56	1\\
57	1\\
58	1\\
59	1\\
60	1\\
61	1\\
62	1\\
63	1\\
64	1\\
65	1\\
66	1\\
67	1\\
68	1\\
69	1\\
70	1\\
71	1\\
72	1\\
73	1\\
74	1\\
75	1\\
76	1\\
77	1\\
78	1\\
79	1\\
80	1\\
81	1\\
82	1\\
83	1\\
84	1\\
85	1\\
86	1\\
87	1\\
88	1\\
89	1\\
90	1\\
91	1\\
92	1\\
93	1\\
94	1\\
95	1\\
96	1\\
97	1\\
98	1\\
99	1\\
100	1\\
101	1\\
102	1\\
103	1\\
104	1\\
105	1\\
106	1\\
107	1\\
108	1\\
109	1\\
110	1\\
111	1\\
112	1\\
113	1\\
114	1\\
115	1\\
116	1\\
117	1\\
118	1\\
119	1\\
120	1\\
121	1\\
122	1\\
123	1\\
124	1\\
125	1\\
126	1\\
127	1\\
128	1\\
129	1\\
130	1\\
131	1\\
132	1\\
133	1\\
134	1\\
135	1\\
136	1\\
137	1\\
138	1\\
139	1\\
140	1\\
141	1\\
142	1\\
143	1\\
144	1\\
145	1\\
146	1\\
147	1\\
148	1\\
149	1\\
150	1\\
151	1\\
152	1\\
153	1\\
154	1\\
155	1\\
156	1\\
157	1\\
158	1\\
159	1\\
160	1\\
161	1\\
162	1\\
163	1\\
164	1\\
165	1\\
166	1\\
167	1\\
168	1\\
169	1\\
170	1\\
171	1\\
172	1\\
173	1\\
174	1\\
175	1\\
176	1\\
177	1\\
178	1\\
179	1\\
180	1\\
181	1\\
182	1\\
183	1\\
184	1\\
185	1\\
186	1\\
187	1\\
188	1\\
189	1\\
190	1\\
191	1\\
192	1\\
193	1\\
194	1\\
195	1\\
196	1\\
197	1\\
198	1\\
199	1\\
200	1\\
};
\addlegendentry{Yzad}

\end{axis}
\end{tikzpicture}%
\caption{Wyjście obiektu z pomiarem zakłócenia błąd $E=10,98983$}
\end{figure}

\section{Wnioski}
Pomiar zakłócen znacznie polepszył regulację. Bez niego błąd wynosił E=20,06103, natomiast z pomiarem błąd zmalał prawie dwukrotnie (E=10,98983).
\smallskip


%! TEX encoding = utf8
\chapter{Regulacja z skokowym zakłóceniem sinusoidalnym}

W tym podpunkcie zakłócenie pojawia się w chwili k=60 i posiada ono charakter sinusoidalny.

\section{Bez pomiaru zakłócenia}

\begin{figure}[H]
\centering
% This file was created by matlab2tikz.
%
%The latest updates can be retrieved from
%  http://www.mathworks.com/matlabcentral/fileexchange/22022-matlab2tikz-matlab2tikz
%where you can also make suggestions and rate matlab2tikz.
%
\definecolor{mycolor1}{rgb}{0.00000,0.44700,0.74100}%
%
\begin{tikzpicture}

\begin{axis}[%
width=4.272in,
height=1.075in,
at={(0.717in,1.839in)},
scale only axis,
xmin=0,
xmax=150,
xlabel style={font=\color{white!15!black}},
xlabel={k},
ymin=-0.166609326859868,
ymax=1.538576433506,
ylabel style={font=\color{white!15!black}},
ylabel={U(k)},
axis background/.style={fill=white}
]
\addplot[const plot, color=mycolor1, forget plot] table[row sep=crcr] {%
1	0\\
2	0\\
3	0\\
4	0\\
5	0\\
6	0\\
7	0\\
8	0\\
9	0\\
10	0\\
11	0\\
12	0\\
13	0\\
14	0\\
15	0\\
16	0\\
17	0\\
18	0\\
19	0\\
20	0\\
21	0\\
22	0\\
23	0\\
24	0\\
25	0\\
26	0\\
27	0\\
28	0\\
29	0\\
30	1.20577393725138\\
31	1.538576433506\\
32	1.46712127199355\\
33	1.24868968296282\\
34	1.01114447937986\\
35	0.808413088923491\\
36	0.655593901391529\\
37	0.54975127059157\\
38	0.481347710796871\\
39	0.4399381217831\\
40	0.416581469792359\\
41	0.404525444762055\\
42	0.399086664830369\\
43	0.397234397851693\\
44	0.397131373389285\\
45	0.397739526004153\\
46	0.39852169972768\\
47	0.39923446155087\\
48	0.39979385972787\\
49	0.400193942132608\\
50	0.400460651022531\\
51	0.400627917253023\\
52	0.40072678887092\\
53	0.400781648510957\\
54	0.400809906317153\\
55	0.40082312096247\\
56	0.400828485130777\\
57	0.40083019020383\\
58	0.400830499953435\\
59	0.400830517872924\\
60	0.400830696699621\\
61	0.400831155277304\\
62	0.400831862848653\\
63	0.400832738153988\\
64	0.367435097102787\\
65	0.297730044122255\\
66	0.203403988540624\\
67	0.101390485220279\\
68	0.00865316048981549\\
69	-0.0605458871185013\\
70	-0.0959344027760143\\
71	-0.0916909796189358\\
72	-0.0464185959935534\\
73	0.0371931415874952\\
74	0.152927480323017\\
75	0.291770144407255\\
76	0.442767731529408\\
77	0.593970220173893\\
78	0.733405292823735\\
79	0.850025702457492\\
80	0.934568531419711\\
81	0.980268134743467\\
82	0.983372896403964\\
83	0.943428969198735\\
84	0.863310740495651\\
85	0.748996400777699\\
86	0.609106092740255\\
87	0.454238093762833\\
88	0.296153867174819\\
89	0.146874371085384\\
90	0.0177568170157902\\
91	-0.0813774171802304\\
92	-0.142996920768938\\
93	-0.162438658422041\\
94	-0.138266341734569\\
95	-0.0723813126568258\\
96	0.0301225449520763\\
97	0.161348729481161\\
98	0.311202977648383\\
99	0.46816667274684\\
100	0.620179504455711\\
101	0.755563704716845\\
102	0.863918997436818\\
103	0.936919645592213\\
104	0.968952485566794\\
105	0.957547029789689\\
106	0.903564660339399\\
107	0.811132405964747\\
108	0.687326376957459\\
109	0.541629125229492\\
110	0.385202529366831\\
111	0.230031947551548\\
112	0.088007250776386\\
113	-0.0299888119306022\\
114	-0.114915026505379\\
115	-0.160264935416994\\
116	-0.162565722553433\\
117	-0.121644515000728\\
118	-0.0406416863198654\\
119	0.0742298578755635\\
120	0.214160956892216\\
121	0.368421519852338\\
122	0.525183123813124\\
123	0.672425819673637\\
124	0.798859624639607\\
125	0.894790054106067\\
126	0.952861336109481\\
127	0.968620329077551\\
128	0.940857909528836\\
129	0.871701656806462\\
130	0.766452728922126\\
131	0.633179435355932\\
132	0.482098665656075\\
133	0.324792596866525\\
134	0.173320731277157\\
135	0.0392953405700984\\
136	-0.067009378456163\\
137	-0.137444144389846\\
138	-0.166609326859868\\
139	-0.152269079853141\\
140	-0.0955229009945729\\
141	-0.000721413095445669\\
142	0.124867163036136\\
143	0.271614248074047\\
144	0.428269113032137\\
145	0.582821449292206\\
146	0.723422164836795\\
147	0.839291812497467\\
148	0.921547006537011\\
149	0.963881472054508\\
150	0.963049516192931\\
};
\end{axis}

\begin{axis}[%
width=4.272in,
height=1.075in,
at={(0.717in,0.346in)},
scale only axis,
xmin=0,
xmax=150,
xlabel style={font=\color{white!15!black}},
xlabel={k},
ymin=-0.5,
ymax=0.5,
ylabel style={font=\color{white!15!black}},
ylabel={Z(k)},
axis background/.style={fill=white}
]
\addplot[const plot, color=mycolor1, forget plot] table[row sep=crcr] {%
1	0\\
2	0\\
3	0\\
4	0\\
5	0\\
6	0\\
7	0\\
8	0\\
9	0\\
10	0\\
11	0\\
12	0\\
13	0\\
14	0\\
15	0\\
16	0\\
17	0\\
18	0\\
19	0\\
20	0\\
21	0\\
22	0\\
23	0\\
24	0\\
25	0\\
26	0\\
27	0\\
28	0\\
29	0\\
30	0\\
31	0\\
32	0\\
33	0\\
34	0\\
35	0\\
36	0\\
37	0\\
38	0\\
39	0\\
40	0\\
41	0\\
42	0\\
43	0\\
44	0\\
45	0\\
46	0\\
47	0\\
48	0\\
49	0\\
50	0\\
51	0\\
52	0\\
53	0\\
54	0\\
55	0\\
56	0\\
57	0\\
58	0\\
59	0\\
60	0\\
61	0.137109644605363\\
62	0.263707692885933\\
63	0.370088426598019\\
64	0.448096100514978\\
65	0.491750207716154\\
66	0.497703978875882\\
67	0.465500964418924\\
68	0.397610028511525\\
69	0.299236072051978\\
70	0.177920995700533\\
71	0.0429654953665704\\
72	-0.0952839814377427\\
73	-0.226228452441949\\
74	-0.339828978196373\\
75	-0.42737630361942\\
76	-0.482158558464389\\
77	-0.499975827033521\\
78	-0.479462137331569\\
79	-0.422190184159834\\
80	-0.332550757489411\\
81	-0.217416119812539\\
82	-0.0856131398571169\\
83	0.0527534248035702\\
84	0.18707561528561\\
85	0.307055537171193\\
86	0.403494854999882\\
87	0.468999988387369\\
88	0.498548945471937\\
89	0.489876337049306\\
90	0.443647054047348\\
91	0.363405293248861\\
92	0.255302839237141\\
93	0.127627433998259\\
94	-0.00983260798168206\\
95	-0.146538826944862\\
96	-0.272010555444685\\
97	-0.376628425245164\\
98	-0.452371842179728\\
99	-0.493433889999937\\
100	-0.496666521227455\\
101	-0.461821904175253\\
102	-0.391571423118295\\
103	-0.291300874953792\\
104	-0.168697563726657\\
105	-0.0331609486756003\\
106	0.10491797116538\\
107	0.234953289381683\\
108	0.346975767288528\\
109	0.432397131825702\\
110	0.484668500862322\\
111	0.499782457325439\\
112	0.476580280326615\\
113	0.416840779220727\\
114	0.325143920078558\\
115	0.2085196997346\\
116	0.0759091866999565\\
117	-0.0625209514254677\\
118	-0.196157881858479\\
119	-0.314756246069852\\
120	-0.409223626578972\\
121	-0.472317622804679\\
122	-0.499201094217933\\
123	-0.487813002734079\\
124	-0.439026423481658\\
125	-0.356581609953743\\
126	-0.246799245143661\\
127	-0.118095862445442\\
128	0.019661413122442\\
129	0.155911334203973\\
130	0.280208215713932\\
131	0.383022759792753\\
132	0.456472625363814\\
133	0.494926732722869\\
134	0.49543697377182\\
135	0.45796423056276\\
136	0.385381374299828\\
137	0.283253014878502\\
138	0.159408886601137\\
139	0.0233435767186514\\
140	-0.114511383016329\\
141	-0.243587256230255\\
142	-0.353988360687203\\
143	-0.437250726932631\\
144	-0.486990993784778\\
145	-0.499395792696579\\
146	-0.473514102031107\\
147	-0.411330157727866\\
148	-0.317611330618286\\
149	-0.19954263297079\\
150	-0.0661758750488865\\
};
\end{axis}
\end{tikzpicture}%
\caption{Zakłócenie i sygnał sterujący}
\end{figure}

\begin{figure}[H]
\centering
% This file was created by matlab2tikz.
%
%The latest updates can be retrieved from
%  http://www.mathworks.com/matlabcentral/fileexchange/22022-matlab2tikz-matlab2tikz
%where you can also make suggestions and rate matlab2tikz.
%
\definecolor{mycolor1}{rgb}{0.00000,0.44700,0.74100}%
\definecolor{mycolor2}{rgb}{0.85000,0.32500,0.09800}%
%
\begin{tikzpicture}

\begin{axis}[%
width=4.272in,
height=2.472in,
at={(0.717in,0.441in)},
scale only axis,
xmin=0,
xmax=150,
xlabel style={font=\color{white!15!black}},
xlabel={k},
ymin=0,
ymax=2,
ylabel style={font=\color{white!15!black}},
ylabel={Y(k)},
axis background/.style={fill=white},
legend style={legend cell align=left, align=left, draw=white!15!black}
]
\addplot[const plot, color=mycolor1] table[row sep=crcr] {%
1	0\\
2	0\\
3	0\\
4	0\\
5	0\\
6	0\\
7	0\\
8	0\\
9	0\\
10	0\\
11	0\\
12	0\\
13	0\\
14	0\\
15	0\\
16	0\\
17	0\\
18	0\\
19	0\\
20	0\\
21	0\\
22	0\\
23	0\\
24	0\\
25	0\\
26	0\\
27	0\\
28	0\\
29	0\\
30	0\\
31	0\\
32	0\\
33	0\\
34	0\\
35	0\\
36	0\\
37	0.178309849840735\\
38	0.3952760141839\\
39	0.588826611204609\\
40	0.738611928918155\\
41	0.844395923345594\\
42	0.913931740693088\\
43	0.956746557749988\\
44	0.981369563669113\\
45	0.994414676921337\\
46	1.00055965866249\\
47	1.00288312467568\\
48	1.00328286848648\\
49	1.00285171378131\\
50	1.00216956244591\\
51	1.0015102513556\\
52	1.00097786168913\\
53	1.00059084702661\\
54	1.00033054211392\\
55	1.00016694785008\\
56	1.0000709398569\\
57	1.00001893584117\\
58	0.999993751094316\\
59	0.999983795371811\\
60	0.999981757686305\\
61	0.999983323334048\\
62	0.999986132468706\\
63	0.999989019847488\\
64	1.02769039166633\\
65	1.07785743800136\\
66	1.14388383488865\\
67	1.21825702710571\\
68	1.2930994268854\\
69	1.3607415194462\\
70	1.41428248576507\\
71	1.44315598097269\\
72	1.43834127875431\\
73	1.39481303289178\\
74	1.31233171988639\\
75	1.19526454819569\\
76	1.05185403895729\\
77	0.893184863574524\\
78	0.732005454196305\\
79	0.581511985029353\\
80	0.454178440083178\\
81	0.360704102732015\\
82	0.309140607678266\\
83	0.304250187548151\\
84	0.34713305902558\\
85	0.43514488245956\\
86	0.562105800870481\\
87	0.718782196695702\\
88	0.893602701398449\\
89	1.07355283515382\\
90	1.24517941697963\\
91	1.39562770943658\\
92	1.51363186161568\\
93	1.59038283185553\\
94	1.62020736608566\\
95	1.60100607935948\\
96	1.53441713853094\\
97	1.42569306380866\\
98	1.28330014654361\\
99	1.11827123530822\\
100	0.943361542209704\\
101	0.772072216871197\\
102	0.617616568715968\\
103	0.491908211752498\\
104	0.404648722787739\\
105	0.362584772692679\\
106	0.36899169590165\\
107	0.423423101866583\\
108	0.521745734670998\\
109	0.656456917183607\\
110	0.817260250636173\\
111	0.991855440326366\\
112	1.16688170125506\\
113	1.32894242251041\\
114	1.46563253883733\\
115	1.56648984972922\\
116	1.62379735649565\\
117	1.63317510912079\\
118	1.59391619169969\\
119	1.50904109066826\\
120	1.38506628009304\\
121	1.23150476774062\\
122	1.06013689479701\\
123	0.884107295539178\\
124	0.716917250624331\\
125	0.571389687206596\\
126	0.458686175987246\\
127	0.387451288793609\\
128	0.363149915988334\\
129	0.387648349519463\\
130	0.459071248883174\\
131	0.571945456452394\\
132	0.717619637065152\\
133	0.884927570462139\\
134	1.06104424530716\\
135	1.23246912221889\\
136	1.38606118370493\\
137	1.51004641857314\\
138	1.59492050168703\\
139	1.63417746481401\\
140	1.62480849478803\\
141	1.56753261851536\\
142	1.46674158939203\\
143	1.3301631153739\\
144	1.16826846252443\\
145	0.993469724082691\\
146	0.819168266989711\\
147	0.658727311975844\\
148	0.524447426927005\\
149	0.426623487371497\\
150	0.372755407745784\\
};
\addlegendentry{Y}

\addplot[const plot, color=mycolor2] table[row sep=crcr] {%
1	0\\
2	0\\
3	0\\
4	0\\
5	0\\
6	0\\
7	0\\
8	0\\
9	0\\
10	0\\
11	0\\
12	0\\
13	0\\
14	0\\
15	0\\
16	0\\
17	0\\
18	0\\
19	0\\
20	0\\
21	0\\
22	0\\
23	0\\
24	0\\
25	0\\
26	0\\
27	0\\
28	0\\
29	0\\
30	1\\
31	1\\
32	1\\
33	1\\
34	1\\
35	1\\
36	1\\
37	1\\
38	1\\
39	1\\
40	1\\
41	1\\
42	1\\
43	1\\
44	1\\
45	1\\
46	1\\
47	1\\
48	1\\
49	1\\
50	1\\
51	1\\
52	1\\
53	1\\
54	1\\
55	1\\
56	1\\
57	1\\
58	1\\
59	1\\
60	1\\
61	1\\
62	1\\
63	1\\
64	1\\
65	1\\
66	1\\
67	1\\
68	1\\
69	1\\
70	1\\
71	1\\
72	1\\
73	1\\
74	1\\
75	1\\
76	1\\
77	1\\
78	1\\
79	1\\
80	1\\
81	1\\
82	1\\
83	1\\
84	1\\
85	1\\
86	1\\
87	1\\
88	1\\
89	1\\
90	1\\
91	1\\
92	1\\
93	1\\
94	1\\
95	1\\
96	1\\
97	1\\
98	1\\
99	1\\
100	1\\
101	1\\
102	1\\
103	1\\
104	1\\
105	1\\
106	1\\
107	1\\
108	1\\
109	1\\
110	1\\
111	1\\
112	1\\
113	1\\
114	1\\
115	1\\
116	1\\
117	1\\
118	1\\
119	1\\
120	1\\
121	1\\
122	1\\
123	1\\
124	1\\
125	1\\
126	1\\
127	1\\
128	1\\
129	1\\
130	1\\
131	1\\
132	1\\
133	1\\
134	1\\
135	1\\
136	1\\
137	1\\
138	1\\
139	1\\
140	1\\
141	1\\
142	1\\
143	1\\
144	1\\
145	1\\
146	1\\
147	1\\
148	1\\
149	1\\
150	1\\
151	1\\
152	1\\
153	1\\
154	1\\
155	1\\
156	1\\
157	1\\
158	1\\
159	1\\
160	1\\
161	1\\
162	1\\
163	1\\
164	1\\
165	1\\
166	1\\
167	1\\
168	1\\
169	1\\
170	1\\
171	1\\
172	1\\
173	1\\
174	1\\
175	1\\
176	1\\
177	1\\
178	1\\
179	1\\
180	1\\
181	1\\
182	1\\
183	1\\
184	1\\
185	1\\
186	1\\
187	1\\
188	1\\
189	1\\
190	1\\
191	1\\
192	1\\
193	1\\
194	1\\
195	1\\
196	1\\
197	1\\
198	1\\
199	1\\
200	1\\
};
\addlegendentry{Yzad}

\end{axis}
\end{tikzpicture}%
\caption{Wyjście obiektu bez pomiaru zakłócenia błąd $E=24,84831$}
\end{figure}

\section{Z pomiaru zakłócenia}

\begin{figure}[H]
\centering
% This file was created by matlab2tikz.
%
%The latest updates can be retrieved from
%  http://www.mathworks.com/matlabcentral/fileexchange/22022-matlab2tikz-matlab2tikz
%where you can also make suggestions and rate matlab2tikz.
%
\definecolor{mycolor1}{rgb}{0.00000,0.44700,0.74100}%
%
\begin{tikzpicture}

\begin{axis}[%
width=4.272in,
height=1.075in,
at={(0.717in,1.839in)},
scale only axis,
xmin=0,
xmax=150,
xlabel style={font=\color{white!15!black}},
xlabel={k},
ymin=-1,
ymax=2,
ylabel style={font=\color{white!15!black}},
ylabel={U(k)},
axis background/.style={fill=white}
]
\addplot[const plot, color=mycolor1, forget plot] table[row sep=crcr] {%
1	0\\
2	0\\
3	0\\
4	0\\
5	0\\
6	0\\
7	0\\
8	0\\
9	0\\
10	0\\
11	0\\
12	0\\
13	0\\
14	0\\
15	0\\
16	0\\
17	0\\
18	0\\
19	0\\
20	0\\
21	0\\
22	0\\
23	0\\
24	0\\
25	0\\
26	0\\
27	0\\
28	0\\
29	0\\
30	1.20577393725138\\
31	1.538576433506\\
32	1.46712127199355\\
33	1.24868968296282\\
34	1.01114447937986\\
35	0.808413088923491\\
36	0.655593901391529\\
37	0.54975127059157\\
38	0.481347710796871\\
39	0.4399381217831\\
40	0.416581469792359\\
41	0.404525444762055\\
42	0.399086664830369\\
43	0.397234397851693\\
44	0.397131373389285\\
45	0.397739526004153\\
46	0.39852169972768\\
47	0.39923446155087\\
48	0.39979385972787\\
49	0.400193942132608\\
50	0.400460651022531\\
51	0.400627917253023\\
52	0.40072678887092\\
53	0.400781648510957\\
54	0.400809906317153\\
55	0.40082312096247\\
56	0.400828485130777\\
57	0.40083019020383\\
58	0.400830499953435\\
59	0.400830517872924\\
60	0.400830696699621\\
61	0.400831155277304\\
62	0.214730931936116\\
63	-0.0209719488252652\\
64	-0.261258242951346\\
65	-0.434715866905037\\
66	-0.509650749060827\\
67	-0.478907538584464\\
68	-0.350427648576571\\
69	-0.141442094127147\\
70	0.125227025584843\\
71	0.423754825116161\\
72	0.727374254174172\\
73	1.01019482072664\\
74	1.24887136357004\\
75	1.42410114207103\\
76	1.52187227258951\\
77	1.5343718343118\\
78	1.4604739940638\\
79	1.30575692701852\\
80	1.08203440454642\\
81	0.806427531060781\\
82	0.500041267418505\\
83	0.186341701868472\\
84	-0.110644495236152\\
85	-0.36817212926802\\
86	-0.566520768776604\\
87	-0.690506070473708\\
88	-0.730643462048976\\
89	-0.683875015508064\\
90	-0.553803768946816\\
91	-0.350417494900499\\
92	-0.0893230539378438\\
93	0.209449999744718\\
94	0.522984861644475\\
95	0.827234268771864\\
96	1.09886396357205\\
97	1.31704150445857\\
98	1.46503328470617\\
99	1.53148732360446\\
100	1.51130348333472\\
101	1.40602439487602\\
102	1.2237171212523\\
103	0.978354629388285\\
104	0.688744489415335\\
105	0.377086932472301\\
106	0.0672728136227735\\
107	-0.216948033048525\\
108	-0.45378762354228\\
109	-0.625090242758656\\
110	-0.717724389757285\\
111	-0.72458953549515\\
112	-0.64516050966913\\
113	-0.485527781202747\\
114	-0.257930544509929\\
115	0.0201815920005123\\
116	0.327486378739589\\
117	0.640423597704431\\
118	0.935001311829598\\
119	1.18863521354944\\
120	1.38188005764232\\
121	1.49992043686788\\
122	1.53370660905627\\
123	1.48064829667128\\
124	1.34481326819556\\
125	1.13661394932857\\
126	0.872010317098401\\
127	0.571286831823425\\
128	0.257497479486403\\
129	-0.0453016806086217\\
130	-0.313896857917448\\
131	-0.527707290333239\\
132	-0.67034288448544\\
133	-0.730866226398202\\
134	-0.704633962737328\\
135	-0.593654109425586\\
136	-0.406433163791048\\
137	-0.157323170154385\\
138	0.134578739199714\\
139	0.446894435253014\\
140	0.755680433401633\\
141	1.03726358344798\\
142	1.27005605898663\\
143	1.43621047755091\\
144	1.52298824614022\\
145	1.52373622091257\\
146	1.43839680475392\\
147	1.27351237892131\\
148	1.04172373316157\\
149	0.760800951344707\\
150	0.45228105338619\\
};
\end{axis}

\begin{axis}[%
width=4.272in,
height=1.075in,
at={(0.717in,0.346in)},
scale only axis,
xmin=0,
xmax=150,
xlabel style={font=\color{white!15!black}},
xlabel={k},
ymin=-0.5,
ymax=0.5,
ylabel style={font=\color{white!15!black}},
ylabel={Z(k)},
axis background/.style={fill=white}
]
\addplot[const plot, color=mycolor1, forget plot] table[row sep=crcr] {%
1	0\\
2	0\\
3	0\\
4	0\\
5	0\\
6	0\\
7	0\\
8	0\\
9	0\\
10	0\\
11	0\\
12	0\\
13	0\\
14	0\\
15	0\\
16	0\\
17	0\\
18	0\\
19	0\\
20	0\\
21	0\\
22	0\\
23	0\\
24	0\\
25	0\\
26	0\\
27	0\\
28	0\\
29	0\\
30	0\\
31	0\\
32	0\\
33	0\\
34	0\\
35	0\\
36	0\\
37	0\\
38	0\\
39	0\\
40	0\\
41	0\\
42	0\\
43	0\\
44	0\\
45	0\\
46	0\\
47	0\\
48	0\\
49	0\\
50	0\\
51	0\\
52	0\\
53	0\\
54	0\\
55	0\\
56	0\\
57	0\\
58	0\\
59	0\\
60	0\\
61	0.137109644605363\\
62	0.263707692885933\\
63	0.370088426598019\\
64	0.448096100514978\\
65	0.491750207716154\\
66	0.497703978875882\\
67	0.465500964418924\\
68	0.397610028511525\\
69	0.299236072051978\\
70	0.177920995700533\\
71	0.0429654953665704\\
72	-0.0952839814377427\\
73	-0.226228452441949\\
74	-0.339828978196373\\
75	-0.42737630361942\\
76	-0.482158558464389\\
77	-0.499975827033521\\
78	-0.479462137331569\\
79	-0.422190184159834\\
80	-0.332550757489411\\
81	-0.217416119812539\\
82	-0.0856131398571169\\
83	0.0527534248035702\\
84	0.18707561528561\\
85	0.307055537171193\\
86	0.403494854999882\\
87	0.468999988387369\\
88	0.498548945471937\\
89	0.489876337049306\\
90	0.443647054047348\\
91	0.363405293248861\\
92	0.255302839237141\\
93	0.127627433998259\\
94	-0.00983260798168206\\
95	-0.146538826944862\\
96	-0.272010555444685\\
97	-0.376628425245164\\
98	-0.452371842179728\\
99	-0.493433889999937\\
100	-0.496666521227455\\
101	-0.461821904175253\\
102	-0.391571423118295\\
103	-0.291300874953792\\
104	-0.168697563726657\\
105	-0.0331609486756003\\
106	0.10491797116538\\
107	0.234953289381683\\
108	0.346975767288528\\
109	0.432397131825702\\
110	0.484668500862322\\
111	0.499782457325439\\
112	0.476580280326615\\
113	0.416840779220727\\
114	0.325143920078558\\
115	0.2085196997346\\
116	0.0759091866999565\\
117	-0.0625209514254677\\
118	-0.196157881858479\\
119	-0.314756246069852\\
120	-0.409223626578972\\
121	-0.472317622804679\\
122	-0.499201094217933\\
123	-0.487813002734079\\
124	-0.439026423481658\\
125	-0.356581609953743\\
126	-0.246799245143661\\
127	-0.118095862445442\\
128	0.019661413122442\\
129	0.155911334203973\\
130	0.280208215713932\\
131	0.383022759792753\\
132	0.456472625363814\\
133	0.494926732722869\\
134	0.49543697377182\\
135	0.45796423056276\\
136	0.385381374299828\\
137	0.283253014878502\\
138	0.159408886601137\\
139	0.0233435767186514\\
140	-0.114511383016329\\
141	-0.243587256230255\\
142	-0.353988360687203\\
143	-0.437250726932631\\
144	-0.486990993784778\\
145	-0.499395792696579\\
146	-0.473514102031107\\
147	-0.411330157727866\\
148	-0.317611330618286\\
149	-0.19954263297079\\
150	-0.0661758750488865\\
};
\end{axis}
\end{tikzpicture}%
\caption{Zakłócenie i sygnał sterujący}
\end{figure}

\begin{figure}[H]
\centering
% This file was created by matlab2tikz.
%
%The latest updates can be retrieved from
%  http://www.mathworks.com/matlabcentral/fileexchange/22022-matlab2tikz-matlab2tikz
%where you can also make suggestions and rate matlab2tikz.
%
\definecolor{mycolor1}{rgb}{0.00000,0.44700,0.74100}%
\definecolor{mycolor2}{rgb}{0.85000,0.32500,0.09800}%
%
\begin{tikzpicture}

\begin{axis}[%
width=4.272in,
height=2.472in,
at={(0.717in,0.441in)},
scale only axis,
xmin=0,
xmax=150,
xlabel style={font=\color{white!15!black}},
xlabel={k},
ymin=0,
ymax=1.6,
ylabel style={font=\color{white!15!black}},
ylabel={Y(k)},
axis background/.style={fill=white},
legend style={legend cell align=left, align=left, draw=white!15!black}
]
\addplot[const plot, color=mycolor1] table[row sep=crcr] {%
1	0\\
2	0\\
3	0\\
4	0\\
5	0\\
6	0\\
7	0\\
8	0\\
9	0\\
10	0\\
11	0\\
12	0\\
13	0\\
14	0\\
15	0\\
16	0\\
17	0\\
18	0\\
19	0\\
20	0\\
21	0\\
22	0\\
23	0\\
24	0\\
25	0\\
26	0\\
27	0\\
28	0\\
29	0\\
30	0\\
31	0\\
32	0\\
33	0\\
34	0\\
35	0\\
36	0\\
37	0.178309849840735\\
38	0.3952760141839\\
39	0.588826611204609\\
40	0.738611928918155\\
41	0.844395923345594\\
42	0.913931740693088\\
43	0.956746557749988\\
44	0.981369563669113\\
45	0.994414676921337\\
46	1.00055965866249\\
47	1.00288312467568\\
48	1.00328286848648\\
49	1.00285171378131\\
50	1.00216956244591\\
51	1.0015102513556\\
52	1.00097786168913\\
53	1.00059084702661\\
54	1.00033054211392\\
55	1.00016694785008\\
56	1.0000709398569\\
57	1.00001893584117\\
58	0.999993751094316\\
59	0.999983795371811\\
60	0.999981757686305\\
61	0.999983323334048\\
62	0.999986132468706\\
63	0.999989019847488\\
64	1.02769039166633\\
65	1.07785743800136\\
66	1.14388383488865\\
67	1.21825702710571\\
68	1.2930994268854\\
69	1.33322091378285\\
70	1.32601502035809\\
71	1.2671442458361\\
72	1.16443866765771\\
73	1.03168480907564\\
74	0.884895090474091\\
75	0.74004323409537\\
76	0.611633577039953\\
77	0.51174754455895\\
78	0.449391155135608\\
79	0.430074605453853\\
80	0.455609866094442\\
81	0.524132470627041\\
82	0.630352758138219\\
83	0.766029424443592\\
84	0.92064107844384\\
85	1.08221413077009\\
86	1.23825073881429\\
87	1.37669054141374\\
88	1.48683549817689\\
89	1.56016883398025\\
90	1.59100628340117\\
91	1.57693019142299\\
92	1.51897339067351\\
93	1.42153880984012\\
94	1.29206095892398\\
95	1.14043520088899\\
96	0.978258518805964\\
97	0.817939945535744\\
98	0.671748824232798\\
99	0.550873842598998\\
100	0.464564964655175\\
101	0.41942403344034\\
102	0.418898423586764\\
103	0.463016558079608\\
104	0.548385562313704\\
105	0.66845123256106\\
106	0.814000386038117\\
107	0.973867077790593\\
108	1.13578854028287\\
109	1.28734522317379\\
110	1.41691286331053\\
111	1.51455359279093\\
112	1.57277776673893\\
113	1.58711810389253\\
114	1.55647212235143\\
115	1.48318661671283\\
116	1.37287769945925\\
117	1.23400020266406\\
118	1.07719945159143\\
119	0.914495106418787\\
120	0.758359642962815\\
121	0.620762120901881\\
122	0.512250549284042\\
123	0.44114320006078\\
124	0.41289086785032\\
125	0.429658968463653\\
126	0.490161514675904\\
127	0.589759697422233\\
128	0.720817514477712\\
129	0.873287181992868\\
130	1.03547944782663\\
131	1.19495975002102\\
132	1.33950128988026\\
133	1.45802236420153\\
134	1.54143598733721\\
135	1.58334660968404\\
136	1.58054049207643\\
137	1.53323213172339\\
138	1.44504622269753\\
139	1.32274125312588\\
140	1.17569195395478\\
141	1.01517071451575\\
142	0.853483309581122\\
143	0.703025249141952\\
144	0.575331376991314\\
145	0.480191480748581\\
146	0.424899691341431\\
147	0.413695218482439\\
148	0.447437304091153\\
149	0.523539316581199\\
150	0.636167038402987\\
};
\addlegendentry{Y}

\addplot[const plot, color=mycolor2] table[row sep=crcr] {%
1	0\\
2	0\\
3	0\\
4	0\\
5	0\\
6	0\\
7	0\\
8	0\\
9	0\\
10	0\\
11	0\\
12	0\\
13	0\\
14	0\\
15	0\\
16	0\\
17	0\\
18	0\\
19	0\\
20	0\\
21	0\\
22	0\\
23	0\\
24	0\\
25	0\\
26	0\\
27	0\\
28	0\\
29	0\\
30	1\\
31	1\\
32	1\\
33	1\\
34	1\\
35	1\\
36	1\\
37	1\\
38	1\\
39	1\\
40	1\\
41	1\\
42	1\\
43	1\\
44	1\\
45	1\\
46	1\\
47	1\\
48	1\\
49	1\\
50	1\\
51	1\\
52	1\\
53	1\\
54	1\\
55	1\\
56	1\\
57	1\\
58	1\\
59	1\\
60	1\\
61	1\\
62	1\\
63	1\\
64	1\\
65	1\\
66	1\\
67	1\\
68	1\\
69	1\\
70	1\\
71	1\\
72	1\\
73	1\\
74	1\\
75	1\\
76	1\\
77	1\\
78	1\\
79	1\\
80	1\\
81	1\\
82	1\\
83	1\\
84	1\\
85	1\\
86	1\\
87	1\\
88	1\\
89	1\\
90	1\\
91	1\\
92	1\\
93	1\\
94	1\\
95	1\\
96	1\\
97	1\\
98	1\\
99	1\\
100	1\\
101	1\\
102	1\\
103	1\\
104	1\\
105	1\\
106	1\\
107	1\\
108	1\\
109	1\\
110	1\\
111	1\\
112	1\\
113	1\\
114	1\\
115	1\\
116	1\\
117	1\\
118	1\\
119	1\\
120	1\\
121	1\\
122	1\\
123	1\\
124	1\\
125	1\\
126	1\\
127	1\\
128	1\\
129	1\\
130	1\\
131	1\\
132	1\\
133	1\\
134	1\\
135	1\\
136	1\\
137	1\\
138	1\\
139	1\\
140	1\\
141	1\\
142	1\\
143	1\\
144	1\\
145	1\\
146	1\\
147	1\\
148	1\\
149	1\\
150	1\\
151	1\\
152	1\\
153	1\\
154	1\\
155	1\\
156	1\\
157	1\\
158	1\\
159	1\\
160	1\\
161	1\\
162	1\\
163	1\\
164	1\\
165	1\\
166	1\\
167	1\\
168	1\\
169	1\\
170	1\\
171	1\\
172	1\\
173	1\\
174	1\\
175	1\\
176	1\\
177	1\\
178	1\\
179	1\\
180	1\\
181	1\\
182	1\\
183	1\\
184	1\\
185	1\\
186	1\\
187	1\\
188	1\\
189	1\\
190	1\\
191	1\\
192	1\\
193	1\\
194	1\\
195	1\\
196	1\\
197	1\\
198	1\\
199	1\\
200	1\\
};
\addlegendentry{Yzad}

\end{axis}
\end{tikzpicture}%
\caption{Wyjście obiektu z pomiarem zakłócenia błąd $E=22,30000$}
\end{figure}

\section{Wnioski}
Z powyższych przebiegów można stwierdzić, że w przypadku pomiaru zakłócenia sinusoidalnego regulator działa lepiej niż bez (współczynnik jakości regulacji z pomiarem zakłócenia wynosi E=22,30000, natomiast bez E=24,84831).

\smallskip


%! TEX encoding = utf8
\chapter{Regulacja z skokowym zakłóceniem sinusoidalnym}

Szum pomiarowy generowano poleceniem wgn() i dodawano go do ustawionej wartości zakłócenia

\section{Szum nałożony na zakłócenie skokowe}

\begin{figure}[H]
\centering
% This file was created by matlab2tikz.
%
%The latest updates can be retrieved from
%  http://www.mathworks.com/matlabcentral/fileexchange/22022-matlab2tikz-matlab2tikz
%where you can also make suggestions and rate matlab2tikz.
%
\definecolor{mycolor1}{rgb}{0.00000,0.44700,0.74100}%
%
\begin{tikzpicture}

\begin{axis}[%
width=4.272in,
height=1.075in,
at={(0.717in,1.839in)},
scale only axis,
xmin=0,
xmax=150,
xlabel style={font=\color{white!15!black}},
xlabel={k},
ymin=-2,
ymax=2,
ylabel style={font=\color{white!15!black}},
ylabel={U(k)},
axis background/.style={fill=white}
]
\addplot[const plot, color=mycolor1, forget plot] table[row sep=crcr] {%
1	0\\
2	0\\
3	0\\
4	0\\
5	0\\
6	0\\
7	0\\
8	0\\
9	0\\
10	0\\
11	0\\
12	0.000434587682167119\\
13	0.000243916940038817\\
14	0.0148660254779791\\
15	0.0227060712444275\\
16	0.0369278289165441\\
17	0.00808439480099233\\
18	0.00335351157592429\\
19	0.0101185113717708\\
20	0.00193286246509215\\
21	0.0209268804832885\\
22	-0.00613364717555564\\
23	-0.00572037393958827\\
24	-0.0353851146579039\\
25	-0.00225970448500858\\
26	-0.0231040343288302\\
27	-0.0107831986147378\\
28	-0.00978575951776835\\
29	-0.0146939963465967\\
30	1.18382673086061\\
31	1.54451008043006\\
32	1.45111273299934\\
33	1.26433272959979\\
34	1.03058592102796\\
35	0.828400315680257\\
36	0.653515734063443\\
37	0.561219052519699\\
38	0.475207620449167\\
39	0.430620238522634\\
40	0.406739146591162\\
41	0.390277638993814\\
42	0.408720936804318\\
43	0.385985687119288\\
44	0.394208863828704\\
45	0.384541142444709\\
46	0.407898987098295\\
47	0.40199698129465\\
48	0.392368027238047\\
49	0.384000594897505\\
50	0.394862045955653\\
51	0.424205975647106\\
52	0.427375185054801\\
53	0.412139824077039\\
54	0.393710183592335\\
55	0.387412663490683\\
56	0.399741477509157\\
57	0.368000346350359\\
58	0.38959394145227\\
59	0.382276269341309\\
60	0.40425880600975\\
61	-0.93662012576268\\
62	-1.41874098350105\\
63	-1.69942920985893\\
64	-1.60227974852856\\
65	-1.32576279787853\\
66	-1.03948608815925\\
67	-0.847936190459632\\
68	-0.672939397481384\\
69	-0.542015874811265\\
70	-0.463104603015313\\
71	-0.407704409676296\\
72	-0.405125706900011\\
73	-0.389840198710596\\
74	-0.378822379465069\\
75	-0.36837116325726\\
76	-0.345601422830382\\
77	-0.359572769519619\\
78	-0.352238823084512\\
79	-0.343869127180721\\
80	-0.338174786375953\\
81	-0.33463588419947\\
82	-0.322387254471842\\
83	-0.340027290482996\\
84	-0.353526091383658\\
85	-0.375668264346881\\
86	-0.354850116483786\\
87	-0.326820011745535\\
88	-0.322847993895688\\
89	-0.302340092877798\\
90	-0.304406086231478\\
91	-0.304102886471742\\
92	-0.305093608237757\\
93	-0.323739440407527\\
94	-0.311044130360401\\
95	-0.333433211840762\\
96	-0.339528038477274\\
97	-0.347952726502402\\
98	-0.334200288409619\\
99	-0.335543132543132\\
100	-0.347884142734273\\
101	-0.328805795829051\\
102	-0.31946396740704\\
103	-0.337540332787026\\
104	-0.285940633259678\\
105	-0.299877223645811\\
106	-0.336756666178557\\
107	-0.318519178062558\\
108	-0.316552033384519\\
109	-0.298917030961772\\
110	-0.320830679331427\\
111	-0.289762337646182\\
112	-0.30207640753067\\
113	-0.320164710897341\\
114	-0.324062605384603\\
115	-0.336771063141997\\
116	-0.331612279969093\\
117	-0.336685711605539\\
118	-0.30877605830235\\
119	-0.319177347975582\\
120	-0.29462663518455\\
121	-0.322402394937241\\
122	-0.315026198286825\\
123	-0.331404773800938\\
124	-0.317901915842669\\
125	-0.326034851407832\\
126	-0.335483731717474\\
127	-0.299147292710288\\
128	-0.288182937179179\\
129	-0.279196441777081\\
130	-0.316801759205649\\
131	-0.328139078111459\\
132	-0.323645209806376\\
133	-0.348693902449791\\
134	-0.334842453386796\\
135	-0.308946621215226\\
136	-0.30208343499389\\
137	-0.291650848659179\\
138	-0.323616583966211\\
139	-0.315446678963986\\
140	-0.310137121204141\\
141	-0.311952411615087\\
142	-0.30834712668762\\
143	-0.332162707675377\\
144	-0.335283884408796\\
145	-0.336461998161527\\
146	-0.334984548557458\\
147	-0.326844855944146\\
148	-0.309189610517883\\
149	-0.321714222026204\\
150	-0.306349786025846\\
};
\end{axis}

\begin{axis}[%
width=4.272in,
height=1.075in,
at={(0.717in,0.346in)},
scale only axis,
xmin=0,
xmax=150,
xlabel style={font=\color{white!15!black}},
xlabel={k},
ymin=-0.0185353157634621,
ymax=1.02473900289376,
ylabel style={font=\color{white!15!black}},
ylabel={Z(k)},
axis background/.style={fill=white}
]
\addplot[const plot, color=mycolor1, forget plot] table[row sep=crcr] {%
1	-0.0127902689177965\\
2	-0.000562272315001003\\
3	-0.00209680283172198\\
4	0.00192101558519392\\
5	0.00992332288990974\\
6	0.00576244193778696\\
7	0.0130578377155815\\
8	-0.00729305831865504\\
9	-0.00864574687578243\\
10	-0.000949209914065763\\
11	0.0138321391556811\\
12	0.0130368137352944\\
13	-0.00125812180556037\\
14	-0.00596873784956023\\
15	-0.0152064632561286\\
16	0.006718416006635\\
17	0.000228442900561173\\
18	-0.0117778858625395\\
19	-0.00334578164099747\\
20	-0.018294552086079\\
21	0.00429989596952203\\
22	-0.00173993462313665\\
23	0.0163124241928948\\
24	-0.0135942802143051\\
25	0.00970506681969236\\
26	0.00143639810487295\\
27	0.000826035911582753\\
28	0.00716664014557957\\
29	0.011934838620052\\
30	-0.0107106191874608\\
31	0.0131890220931254\\
32	-0.0121000314119464\\
33	-0.0107411490177333\\
34	-0.00672560081842942\\
35	0.00736462181575454\\
36	-0.0106996069280028\\
37	0.00334715119526138\\
38	0.0041188259525515\\
39	0.00154120187534393\\
40	0.00554570697670329\\
41	-0.0117284568845615\\
42	0.010075866859622\\
43	0.00113520280110208\\
44	0.00730050867444302\\
45	-0.0098351178051884\\
46	0.000520315567180542\\
47	0.00877599372594727\\
48	0.0101414104094213\\
49	-0.000843484107867966\\
50	-0.0185353157634621\\
51	-0.0109682070985933\\
52	0.00218627627525169\\
53	0.00794245850720322\\
54	0.00463124194931491\\
55	-0.00612630024220403\\
56	0.0224439981063933\\
57	0.000723476083593602\\
58	0.00865514396302755\\
59	-0.00415696466595698\\
60	0.988850618940738\\
61	1.00685252427933\\
62	1.01037673215279\\
63	1.01822211528431\\
64	0.994710115069232\\
65	0.983720332670044\\
66	1.01617302431708\\
67	1.00264136741915\\
68	0.988728538143408\\
69	0.994408241409648\\
70	0.991911548434626\\
71	1.01161004285597\\
72	1.00592105269147\\
73	1.00242747664485\\
74	1.00240477484679\\
75	0.99178499601153\\
76	1.00993112126213\\
77	1.00346394724518\\
78	0.997388664726334\\
79	0.99815287094912\\
80	0.998982650004212\\
81	0.99112956856172\\
82	1.00741377097305\\
83	1.01392207870888\\
84	1.02473900289376\\
85	1.00503919242691\\
86	0.991775194461694\\
87	1.00200981870646\\
88	0.989929474724909\\
89	0.993868284397659\\
90	0.993410366091505\\
91	0.99166770672488\\
92	1.00378179296887\\
93	0.988846574039918\\
94	1.00667241351747\\
95	1.00795332803666\\
96	1.01037491854348\\
97	0.999795688073973\\
98	1.00618953085869\\
99	1.01803063510716\\
100	1.00052993311293\\
101	0.998221099131658\\
102	1.01772482561955\\
103	0.974911963869191\\
104	0.995434108146408\\
105	1.02430377536937\\
106	0.995285459141013\\
107	0.994396715088517\\
108	0.987735326262364\\
109	1.00792950363598\\
110	0.978900595681557\\
111	0.99200604873994\\
112	1.00566975092899\\
113	0.999985571011374\\
114	1.00623924666255\\
115	1.00126440831035\\
116	1.00680867794188\\
117	0.987125840861393\\
118	1.00218084852525\\
119	0.984334329263042\\
120	1.00783281930837\\
121	0.996894300912784\\
122	1.00655324411573\\
123	0.994625326403292\\
124	1.00328643403253\\
125	1.01054071971429\\
126	0.980203006808458\\
127	0.981326118154128\\
128	0.981676443289746\\
129	1.0084859211514\\
130	1.00405187260062\\
131	0.9929754396344\\
132	1.01499046021285\\
133	1.0013779486573\\
134	0.984131786000966\\
135	0.989808544851624\\
136	0.986147662589741\\
137	1.00954886430307\\
138	0.99398879680629\\
139	0.988281091958547\\
140	0.994228896929033\\
141	0.991635694755469\\
142	1.00852969530087\\
143	1.00477331178677\\
144	1.00302320074947\\
145	1.00415776190618\\
146	1.00042974829305\\
147	0.990511467699428\\
148	1.00541608366255\\
149	0.991788717412811\\
150	0.989280949515509\\
};
\end{axis}
\end{tikzpicture}%
\caption{Zakłócenie i sygnał sterujący- szum mały}
\end{figure}

\begin{figure}[H]
\centering
% This file was created by matlab2tikz.
%
%The latest updates can be retrieved from
%  http://www.mathworks.com/matlabcentral/fileexchange/22022-matlab2tikz-matlab2tikz
%where you can also make suggestions and rate matlab2tikz.
%
\definecolor{mycolor1}{rgb}{0.00000,0.44700,0.74100}%
\definecolor{mycolor2}{rgb}{0.85000,0.32500,0.09800}%
%
\begin{tikzpicture}

\begin{axis}[%
width=4.272in,
height=2.472in,
at={(0.717in,0.441in)},
scale only axis,
xmin=0,
xmax=150,
xlabel style={font=\color{white!15!black}},
xlabel={k},
ymin=-0.00186808682706362,
ymax=2,
ylabel style={font=\color{white!15!black}},
ylabel={Y(k)},
axis background/.style={fill=white},
legend style={legend cell align=left, align=left, draw=white!15!black}
]
\addplot[const plot, color=mycolor1] table[row sep=crcr] {%
1	0\\
2	0\\
3	0\\
4	0\\
5	0\\
6	0\\
7	0\\
8	0\\
9	0\\
10	0\\
11	0\\
12	-0.000360422189218803\\
13	0.000792469706634898\\
14	0.00472493781969475\\
15	0.0079827333961109\\
16	0.00791864902040163\\
17	0.00684579032081949\\
18	0.00396662888242288\\
19	0.00584724490812563\\
20	0.0061273393896463\\
21	0.00606411684696123\\
22	0.00893970513055188\\
23	0.0107019657558752\\
24	0.0127959020143814\\
25	0.0127255038048354\\
26	0.0172642673361325\\
27	0.0140510155859638\\
28	0.0186590918200626\\
29	0.0171749097457456\\
30	0.0156742507449203\\
31	0.0111256686824404\\
32	0.0126129392277616\\
33	0.00621914384540173\\
34	0.0069762918338123\\
35	0.00257962960399696\\
36	-0.00186808682706362\\
37	0.172105716857767\\
38	0.392016508414502\\
39	0.581340581517473\\
40	0.734770397820644\\
41	0.84464112286607\\
42	0.917520483958294\\
43	0.961000095882878\\
44	0.984695328481809\\
45	0.998794555505403\\
46	1.00353327712258\\
47	1.00568933270872\\
48	1.00174132059024\\
49	1.00295873958958\\
50	1.00239853468134\\
51	1.00326493380062\\
52	1.00032916573188\\
53	0.997474388654733\\
54	0.995658197263311\\
55	0.995291695241968\\
56	0.99482788426863\\
57	0.995227622384729\\
58	0.997715486290281\\
59	1.00636346884125\\
60	1.00761175425614\\
61	1.00767679201389\\
62	1.00414556559246\\
63	1.20332661812959\\
64	1.37908181269963\\
65	1.53875267963916\\
66	1.68092521831718\\
67	1.80550727937128\\
68	1.71564192965489\\
69	1.56064986788526\\
70	1.36517894115472\\
71	1.18817723596471\\
72	1.05970740840429\\
73	0.977074922255067\\
74	0.928472747481985\\
75	0.904454151797008\\
76	0.897889889916646\\
77	0.901087900051859\\
78	0.908109808048019\\
79	0.917071159813784\\
80	0.924759169541135\\
81	0.930956862996999\\
82	0.937276665379171\\
83	0.94567386088255\\
84	0.948950640053255\\
85	0.955630012767176\\
86	0.963601490688777\\
87	0.973290611872396\\
88	0.97808995629401\\
89	0.981184654038629\\
90	0.983206921667194\\
91	0.980253985490951\\
92	0.974754849705425\\
93	0.972310045332846\\
94	0.97357949096496\\
95	0.977628646617417\\
96	0.981164113057873\\
97	0.987688921463476\\
98	0.993853882259114\\
99	0.999736714609548\\
100	1.00012001669331\\
101	1.00353515937302\\
102	1.00565913376847\\
103	1.00294256446791\\
104	0.998615270942214\\
105	1.00049016724531\\
106	0.99317270953045\\
107	0.988860636854901\\
108	0.99346065069468\\
109	0.993001030232689\\
110	0.989754752422339\\
111	0.99302950619745\\
112	0.998239464841674\\
113	0.991704432839535\\
114	0.991106805277306\\
115	0.993638258190465\\
116	0.997370661240286\\
117	0.998869207254477\\
118	1.0037697351366\\
119	1.00764483350568\\
120	1.00454936232726\\
121	1.00423323002199\\
122	0.998406211496392\\
123	0.998585602078173\\
124	0.995676939717159\\
125	0.99903207800771\\
126	0.998148789752944\\
127	1.00274423060469\\
128	1.00437377189278\\
129	1.00074503373856\\
130	0.995351144229704\\
131	0.992523230141652\\
132	0.994231328338882\\
133	0.993395800689149\\
134	0.995660360526058\\
135	1.0039027790031\\
136	1.01003840998304\\
137	1.00673589372884\\
138	1.00325756309494\\
139	0.999989991253909\\
140	0.998049319071127\\
141	0.994975529197151\\
142	0.994791093884091\\
143	0.99692272165602\\
144	0.999961990949339\\
145	1.00154818541282\\
146	1.0033451905905\\
147	1.00537929428935\\
148	1.00719421840387\\
149	1.00861670318917\\
150	1.00441165002825\\
};
\addlegendentry{Y}

\addplot[const plot, color=mycolor2] table[row sep=crcr] {%
1	0\\
2	0\\
3	0\\
4	0\\
5	0\\
6	0\\
7	0\\
8	0\\
9	0\\
10	0\\
11	0\\
12	0\\
13	0\\
14	0\\
15	0\\
16	0\\
17	0\\
18	0\\
19	0\\
20	0\\
21	0\\
22	0\\
23	0\\
24	0\\
25	0\\
26	0\\
27	0\\
28	0\\
29	0\\
30	1\\
31	1\\
32	1\\
33	1\\
34	1\\
35	1\\
36	1\\
37	1\\
38	1\\
39	1\\
40	1\\
41	1\\
42	1\\
43	1\\
44	1\\
45	1\\
46	1\\
47	1\\
48	1\\
49	1\\
50	1\\
51	1\\
52	1\\
53	1\\
54	1\\
55	1\\
56	1\\
57	1\\
58	1\\
59	1\\
60	1\\
61	1\\
62	1\\
63	1\\
64	1\\
65	1\\
66	1\\
67	1\\
68	1\\
69	1\\
70	1\\
71	1\\
72	1\\
73	1\\
74	1\\
75	1\\
76	1\\
77	1\\
78	1\\
79	1\\
80	1\\
81	1\\
82	1\\
83	1\\
84	1\\
85	1\\
86	1\\
87	1\\
88	1\\
89	1\\
90	1\\
91	1\\
92	1\\
93	1\\
94	1\\
95	1\\
96	1\\
97	1\\
98	1\\
99	1\\
100	1\\
101	1\\
102	1\\
103	1\\
104	1\\
105	1\\
106	1\\
107	1\\
108	1\\
109	1\\
110	1\\
111	1\\
112	1\\
113	1\\
114	1\\
115	1\\
116	1\\
117	1\\
118	1\\
119	1\\
120	1\\
121	1\\
122	1\\
123	1\\
124	1\\
125	1\\
126	1\\
127	1\\
128	1\\
129	1\\
130	1\\
131	1\\
132	1\\
133	1\\
134	1\\
135	1\\
136	1\\
137	1\\
138	1\\
139	1\\
140	1\\
141	1\\
142	1\\
143	1\\
144	1\\
145	1\\
146	1\\
147	1\\
148	1\\
149	1\\
150	1\\
};
\addlegendentry{Yzad}

\end{axis}
\end{tikzpicture}%
\caption{Wyjście dla pomiaru z szumem małym błąd ($E=10,8966$)}
\end{figure}

\begin{figure}[H]
\centering
% This file was created by matlab2tikz.
%
%The latest updates can be retrieved from
%  http://www.mathworks.com/matlabcentral/fileexchange/22022-matlab2tikz-matlab2tikz
%where you can also make suggestions and rate matlab2tikz.
%
\definecolor{mycolor1}{rgb}{0.00000,0.44700,0.74100}%
%
\begin{tikzpicture}

\begin{axis}[%
width=4.272in,
height=1.075in,
at={(0.717in,1.839in)},
scale only axis,
xmin=0,
xmax=150,
xlabel style={font=\color{white!15!black}},
xlabel={k},
ymin=-2,
ymax=2,
ylabel style={font=\color{white!15!black}},
ylabel={U(k)},
axis background/.style={fill=white}
]
\addplot[const plot, color=mycolor1, forget plot] table[row sep=crcr] {%
1	0\\
2	0\\
3	0\\
4	0\\
5	0\\
6	0\\
7	0\\
8	0\\
9	0\\
10	0\\
11	0\\
12	-0.0207382081833764\\
13	-0.150432916805117\\
14	-0.114070412292392\\
15	-0.138032119929149\\
16	-0.110391106356429\\
17	-0.0431938125308523\\
18	0.0797813120368421\\
19	-0.00944898217595301\\
20	0.00456197375392563\\
21	-0.00695559106065184\\
22	-0.0451516513854095\\
23	-0.00646889274841674\\
24	-0.0539562735266854\\
25	-0.0368217345473017\\
26	0.0130420142416154\\
27	-0.0391470699739274\\
28	-0.0337711227747506\\
29	-0.0767037352864819\\
30	1.21373363251892\\
31	1.57748369278364\\
32	1.48095697505767\\
33	1.23239760371956\\
34	0.965264569524801\\
35	0.78700451885889\\
36	0.657011962703161\\
37	0.5296538716866\\
38	0.398211352879038\\
39	0.375439470154172\\
40	0.390933765790895\\
41	0.436108684798621\\
42	0.41142157748765\\
43	0.444477099931547\\
44	0.412992779877918\\
45	0.381775732295022\\
46	0.356663107299105\\
47	0.360179084609673\\
48	0.331236901537313\\
49	0.307724410358705\\
50	0.460525703942413\\
51	0.513032615018699\\
52	0.516554998067525\\
53	0.546637263055875\\
54	0.465435514295844\\
55	0.43040341765487\\
56	0.37411819002623\\
57	0.353172247727878\\
58	0.388731800972962\\
59	0.413840222492077\\
60	0.364286385302505\\
61	-0.959572713123259\\
62	-1.37994755207964\\
63	-1.68144536778805\\
64	-1.58006818029631\\
65	-1.32015769827486\\
66	-1.10120471553292\\
67	-0.870263517923897\\
68	-0.684850688473676\\
69	-0.560558119141631\\
70	-0.496730856878388\\
71	-0.414437351644308\\
72	-0.348298439356271\\
73	-0.350090895267608\\
74	-0.358836400578496\\
75	-0.385575245419159\\
76	-0.371403605461378\\
77	-0.34544005808155\\
78	-0.372956572310239\\
79	-0.34744569964781\\
80	-0.382741222153163\\
81	-0.348384261431197\\
82	-0.342292096887487\\
83	-0.327011306062553\\
84	-0.286154383983979\\
85	-0.271788415741227\\
86	-0.2634446352335\\
87	-0.384748744347855\\
88	-0.347194507711677\\
89	-0.39789740540123\\
90	-0.368587140969201\\
91	-0.424601567733075\\
92	-0.393747590837207\\
93	-0.292693237433872\\
94	-0.281716595960061\\
95	-0.318092044014778\\
96	-0.319540562703438\\
97	-0.416262590092689\\
98	-0.345351935936652\\
99	-0.262828953110771\\
100	-0.298684929631751\\
101	-0.24942050446522\\
102	-0.294890367258289\\
103	-0.351209752120924\\
104	-0.348552890853989\\
105	-0.351467534436577\\
106	-0.320746898334297\\
107	-0.309220608450862\\
108	-0.303580928067648\\
109	-0.188390491314248\\
110	-0.288484794145148\\
111	-0.341019950878359\\
112	-0.291150489258594\\
113	-0.264257216888345\\
114	-0.329378070976033\\
115	-0.264607516046224\\
116	-0.250613530565349\\
117	-0.337919560525971\\
118	-0.311357889828788\\
119	-0.400307683501854\\
120	-0.307759341072966\\
121	-0.310371838779431\\
122	-0.326632813816994\\
123	-0.352409216174631\\
124	-0.363260027512877\\
125	-0.352774553668663\\
126	-0.32480438652277\\
127	-0.294235830839643\\
128	-0.325725175000938\\
129	-0.299975270910756\\
130	-0.334393017655264\\
131	-0.369462917243658\\
132	-0.243537046623792\\
133	-0.373960205529092\\
134	-0.295095181233656\\
135	-0.380638280158422\\
136	-0.401258399301341\\
137	-0.288741464720124\\
138	-0.313339344893201\\
139	-0.442352490034343\\
140	-0.345513088530733\\
141	-0.356197244853887\\
142	-0.319583721356274\\
143	-0.352253693563447\\
144	-0.356821286455329\\
145	-0.277038060540829\\
146	-0.267270797767871\\
147	-0.287436450591829\\
148	-0.289869847560188\\
149	-0.413062331101117\\
150	-0.365360087195853\\
};
\end{axis}

\begin{axis}[%
width=4.272in,
height=1.075in,
at={(0.717in,0.346in)},
scale only axis,
xmin=0,
xmax=150,
xlabel style={font=\color{white!15!black}},
xlabel={k},
ymin=-0.0904854053555337,
ymax=1.10961276291618,
ylabel style={font=\color{white!15!black}},
ylabel={Z(k)},
axis background/.style={fill=white}
]
\addplot[const plot, color=mycolor1, forget plot] table[row sep=crcr] {%
1	0.00320109444637361\\
2	-0.0114697652482808\\
3	-0.0233637874729122\\
4	-0.0467967259443988\\
5	0.00485004057057498\\
6	-0.0100249726914904\\
7	0.010085080649941\\
8	-0.0460600473605885\\
9	0.0418000771575006\\
10	0.0154067412766389\\
11	-0.0495410718950959\\
12	0.0334271605009296\\
13	-0.0185808413973367\\
14	0.00955708670807528\\
15	0.00694001008150868\\
16	-0.0277711572444773\\
17	-0.0904854053555337\\
18	0.0169357455017639\\
19	-0.0108161694404724\\
20	-0.0189059570026563\\
21	0.0132471829936653\\
22	-0.0218096193595153\\
23	0.0212249243557949\\
24	0.00477179668375428\\
25	-0.0313482556361717\\
26	0.0264628864912442\\
27	0.015005291161245\\
28	0.0395969610936456\\
29	-0.0282272068295588\\
30	-0.0283110569525606\\
31	0.00990823580967339\\
32	0.0210934406447205\\
33	0.0261873447336513\\
34	0.000205841451220666\\
35	-0.00745956625213493\\
36	0.0206440156465603\\
37	0.0621239727398428\\
38	0.0279730077032093\\
39	0.00267329330820531\\
40	-0.0181677298137033\\
41	0.0158035025525493\\
42	-0.0153087279055083\\
43	0.0075403130677599\\
44	0.0246103686996763\\
45	0.0292280764443272\\
46	0.0185985538746858\\
47	0.0435741078198438\\
48	0.0585402400337738\\
49	-0.0600094290057894\\
50	-0.0562469324408757\\
51	-0.029176477708716\\
52	-0.063180001832008\\
53	-0.011292495349813\\
54	-0.0106379260786009\\
55	0.00791916560463933\\
56	0.00904923387261777\\
57	-0.0216985553405167\\
58	-0.025719262352857\\
59	0.0250895918737771\\
60	0.987921830560224\\
61	0.956638716851639\\
62	1.00032595032863\\
63	1.00645522207993\\
64	0.987002232235905\\
65	1.02098364693596\\
66	1.00713985815599\\
67	0.993269309782233\\
68	0.998607504598126\\
69	1.01448898142715\\
70	0.98604115684184\\
71	0.966639655474745\\
72	0.995079993718909\\
73	1.0040840140282\\
74	1.01610953819819\\
75	0.999047410356347\\
76	0.985536850370828\\
77	1.01885778827324\\
78	0.996409915324052\\
79	1.0255203528016\\
80	0.997160188027999\\
81	0.99980102553007\\
82	0.997095110508513\\
83	0.970867887605604\\
84	0.970687218744004\\
85	0.969603533400639\\
86	1.05644127526651\\
87	0.993669468410765\\
88	1.02973842765711\\
89	1.01104129284893\\
90	1.05879560558852\\
91	1.02931658990097\\
92	0.961199996468672\\
93	0.989651789764824\\
94	1.02819633493728\\
95	1.00911322815705\\
96	1.07163176915501\\
97	0.998485574778364\\
98	0.950925749244625\\
99	1.01404314736891\\
100	0.971165755357454\\
101	1.0015632608155\\
102	1.03409080386578\\
103	1.00974546808027\\
104	1.00947540269712\\
105	0.993763404838673\\
106	0.995369716113573\\
107	0.99674061291084\\
108	0.911488419214795\\
109	1.01243645303507\\
110	1.03131436448876\\
111	0.958965244354826\\
112	0.951871134492461\\
113	1.01963810184937\\
114	0.952327950068639\\
115	0.946892641354316\\
116	1.02494961813974\\
117	0.97930977507839\\
118	1.03936728978602\\
119	0.959135011003551\\
120	0.980572468971941\\
121	1.00764307367438\\
122	1.01737194538327\\
123	1.01478789555557\\
124	1.00605440203498\\
125	0.992732929221832\\
126	0.981683555267178\\
127	1.01519414892686\\
128	0.987767277784009\\
129	1.01333265407652\\
130	1.03439627926951\\
131	0.928869879872464\\
132	1.05706267233577\\
133	0.980012495822449\\
134	1.04163006379163\\
135	1.0490656490426\\
136	0.953549047194403\\
137	1.00559528210681\\
138	1.10961276291618\\
139	0.993212907040124\\
140	1.01537760486832\\
141	1.01046350622815\\
142	1.04009458425805\\
143	1.0344850871029\\
144	0.970070603784824\\
145	0.986133439541569\\
146	1.01085359064413\\
147	0.998153122899923\\
148	1.08016293824117\\
149	1.01387118668564\\
150	1.01383552505555\\
};
\end{axis}
\end{tikzpicture}%
\caption{Zakłócenie i sygnał sterujący- szum średni}
\end{figure}

\begin{figure}[H]
\centering
% This file was created by matlab2tikz.
%
%The latest updates can be retrieved from
%  http://www.mathworks.com/matlabcentral/fileexchange/22022-matlab2tikz-matlab2tikz
%where you can also make suggestions and rate matlab2tikz.
%
\definecolor{mycolor1}{rgb}{0.00000,0.44700,0.74100}%
\definecolor{mycolor2}{rgb}{0.85000,0.32500,0.09800}%
%
\begin{tikzpicture}

\begin{axis}[%
width=4.272in,
height=2.472in,
at={(0.717in,0.441in)},
scale only axis,
xmin=0,
xmax=150,
xlabel style={font=\color{white!15!black}},
xlabel={k},
ymin=-0.5,
ymax=2,
ylabel style={font=\color{white!15!black}},
ylabel={Y(k)},
axis background/.style={fill=white},
legend style={legend cell align=left, align=left, draw=white!15!black}
]
\addplot[const plot, color=mycolor1] table[row sep=crcr] {%
1	0\\
2	0\\
3	0\\
4	0\\
5	0\\
6	0\\
7	0\\
8	0\\
9	0\\
10	0\\
11	0\\
12	0.0171990847891853\\
13	0.0266177756528847\\
14	0.02137108233\\
15	0.0330158905340646\\
16	0.0324154589389564\\
17	0.0371607688283554\\
18	0.0404626751668299\\
19	0.0329553325173019\\
20	-0.00605902650916875\\
21	-0.0151039519800049\\
22	-0.0333982298619926\\
23	-0.0484223791249575\\
24	-0.0462738924633196\\
25	-0.0336265428977369\\
26	-0.0262778180342901\\
27	-0.0211200895598107\\
28	-0.0255391325954317\\
29	-0.0235197344379676\\
30	-0.0187037173571748\\
31	-0.0165391897810835\\
32	-0.0262112161834222\\
33	-0.0277067475770525\\
34	-0.0288902506664775\\
35	-0.0271630677753391\\
36	-0.0311678044101274\\
37	0.150316023674103\\
38	0.373281883540571\\
39	0.574516220712918\\
40	0.735215261270069\\
41	0.83933500007451\\
42	0.90559530621634\\
43	0.94457176626247\\
44	0.969557330223427\\
45	0.967231763384215\\
46	0.966529086741118\\
47	0.971573194276677\\
48	0.983718344775683\\
49	0.989107067752872\\
50	1.00401198537534\\
51	1.0160467805734\\
52	0.998327457122411\\
53	0.979551505913913\\
54	0.968588481249788\\
55	0.947467851627694\\
56	0.935271690561167\\
57	0.946563984315869\\
58	0.968727846062358\\
59	0.990153202918692\\
60	1.00837889616589\\
61	1.01288522341163\\
62	1.02241153478458\\
63	1.21719747237203\\
64	1.38039304961991\\
65	1.53889774363332\\
66	1.68440934385087\\
67	1.80238142053909\\
68	1.71786259087622\\
69	1.56716414204696\\
70	1.37262598736424\\
71	1.20105450105281\\
72	1.07712637127835\\
73	0.983305975194058\\
74	0.922149803219423\\
75	0.895214852857951\\
76	0.887493167135894\\
77	0.889709116282479\\
78	0.898267838737148\\
79	0.911557350450668\\
80	0.929060880476776\\
81	0.938041587074719\\
82	0.947185498513306\\
83	0.950748823501142\\
84	0.957530080831976\\
85	0.958427531594741\\
86	0.95700564557366\\
87	0.950035674239705\\
88	0.948008085076292\\
89	0.964261224693255\\
90	0.967948732692378\\
91	0.9843696700401\\
92	0.997444516844014\\
93	1.0201864547535\\
94	1.01678567705152\\
95	1.0048935106127\\
96	0.992254306595325\\
97	0.992445312152983\\
98	0.980038591091436\\
99	0.985366065517575\\
100	0.989703804899045\\
101	0.985836683011232\\
102	0.990112515399646\\
103	0.985064027097679\\
104	0.972438504946846\\
105	0.977546687893186\\
106	0.989225862678535\\
107	0.994751727026012\\
108	1.00397123970342\\
109	1.00633343913586\\
110	1.00056522369379\\
111	0.978348292018517\\
112	0.978357882144549\\
113	0.986484264937203\\
114	0.980803259913713\\
115	0.975266367425932\\
116	1.00121925362521\\
117	0.996897628487367\\
118	0.984386038085821\\
119	0.996191177563293\\
120	1.00160152632388\\
121	1.00928134177887\\
122	1.00930998833365\\
123	1.01609405446809\\
124	1.01512157062821\\
125	1.01988736080289\\
126	1.01040197821745\\
127	1.01316745865777\\
128	1.01257884513284\\
129	1.00743077854999\\
130	1.00570171711401\\
131	0.996709308825152\\
132	0.995062927201976\\
133	1.0017189157352\\
134	0.99079146494102\\
135	1.00253503477304\\
136	1.00114733093277\\
137	1.00742346582271\\
138	1.00918214072249\\
139	1.0096726112476\\
140	1.00194998903748\\
141	1.02732573648675\\
142	1.01388814450321\\
143	1.00290455384572\\
144	1.00818433807832\\
145	1.01552320587371\\
146	1.0019189019121\\
147	0.990224212086093\\
148	0.981396554017562\\
149	0.983780320622568\\
150	0.978610045034473\\
};
\addlegendentry{Y}

\addplot[const plot, color=mycolor2] table[row sep=crcr] {%
1	0\\
2	0\\
3	0\\
4	0\\
5	0\\
6	0\\
7	0\\
8	0\\
9	0\\
10	0\\
11	0\\
12	0\\
13	0\\
14	0\\
15	0\\
16	0\\
17	0\\
18	0\\
19	0\\
20	0\\
21	0\\
22	0\\
23	0\\
24	0\\
25	0\\
26	0\\
27	0\\
28	0\\
29	0\\
30	1\\
31	1\\
32	1\\
33	1\\
34	1\\
35	1\\
36	1\\
37	1\\
38	1\\
39	1\\
40	1\\
41	1\\
42	1\\
43	1\\
44	1\\
45	1\\
46	1\\
47	1\\
48	1\\
49	1\\
50	1\\
51	1\\
52	1\\
53	1\\
54	1\\
55	1\\
56	1\\
57	1\\
58	1\\
59	1\\
60	1\\
61	1\\
62	1\\
63	1\\
64	1\\
65	1\\
66	1\\
67	1\\
68	1\\
69	1\\
70	1\\
71	1\\
72	1\\
73	1\\
74	1\\
75	1\\
76	1\\
77	1\\
78	1\\
79	1\\
80	1\\
81	1\\
82	1\\
83	1\\
84	1\\
85	1\\
86	1\\
87	1\\
88	1\\
89	1\\
90	1\\
91	1\\
92	1\\
93	1\\
94	1\\
95	1\\
96	1\\
97	1\\
98	1\\
99	1\\
100	1\\
101	1\\
102	1\\
103	1\\
104	1\\
105	1\\
106	1\\
107	1\\
108	1\\
109	1\\
110	1\\
111	1\\
112	1\\
113	1\\
114	1\\
115	1\\
116	1\\
117	1\\
118	1\\
119	1\\
120	1\\
121	1\\
122	1\\
123	1\\
124	1\\
125	1\\
126	1\\
127	1\\
128	1\\
129	1\\
130	1\\
131	1\\
132	1\\
133	1\\
134	1\\
135	1\\
136	1\\
137	1\\
138	1\\
139	1\\
140	1\\
141	1\\
142	1\\
143	1\\
144	1\\
145	1\\
146	1\\
147	1\\
148	1\\
149	1\\
150	1\\
};
\addlegendentry{Yzad}

\end{axis}
\end{tikzpicture}%
\caption{Wyjście dla pomiaru z szumem średnim ($E=11,5103$)}
\end{figure}

\begin{figure}[H]
\centering
% This file was created by matlab2tikz.
%
%The latest updates can be retrieved from
%  http://www.mathworks.com/matlabcentral/fileexchange/22022-matlab2tikz-matlab2tikz
%where you can also make suggestions and rate matlab2tikz.
%
\definecolor{mycolor1}{rgb}{0.00000,0.44700,0.74100}%
%
\begin{tikzpicture}

\begin{axis}[%
width=4.272in,
height=1.075in,
at={(0.717in,1.839in)},
scale only axis,
xmin=0,
xmax=150,
xlabel style={font=\color{white!15!black}},
xlabel={k},
ymin=-2,
ymax=2,
ylabel style={font=\color{white!15!black}},
ylabel={U(k)},
axis background/.style={fill=white}
]
\addplot[const plot, color=mycolor1, forget plot] table[row sep=crcr] {%
1	0\\
2	0\\
3	0\\
4	0\\
5	0\\
6	0\\
7	0\\
8	0\\
9	0\\
10	0\\
11	0\\
12	0.0112425860890454\\
13	-0.134335630286902\\
14	-0.185045568518065\\
15	-0.148969857682414\\
16	-0.00778489127349916\\
17	-0.0798120707501585\\
18	0.168225965498215\\
19	0.0916188978007313\\
20	-0.0722220242536749\\
21	0.0488008242735829\\
22	0.157855233346119\\
23	-0.14343327770835\\
24	0.0485039224786061\\
25	-0.036433576777133\\
26	0.159209827827496\\
27	0.0614319283106867\\
28	0.0173049608605574\\
29	0.0274095848003943\\
30	1.18800891525148\\
31	1.39617777178007\\
32	1.62922510713003\\
33	1.28986655541941\\
34	1.35891255938908\\
35	0.935995406690737\\
36	0.469266698185795\\
37	0.322647217669518\\
38	0.346646968394453\\
39	0.439481590846806\\
40	0.383268976338592\\
41	0.29246655266858\\
42	0.378859856495124\\
43	0.188892888667748\\
44	0.481935715401568\\
45	0.720550524607832\\
46	0.685009550176658\\
47	0.58625293701548\\
48	0.507770320014285\\
49	0.514059524170595\\
50	0.218170282437539\\
51	0.52232717212534\\
52	0.441860520138361\\
53	0.549217217809028\\
54	0.484277650711196\\
55	0.487961036977969\\
56	0.315620041316605\\
57	0.463722316937533\\
58	0.469563196132282\\
59	0.385209462770834\\
60	0.211458975380797\\
61	-0.825004355989058\\
62	-1.67574931700302\\
63	-1.64744814659328\\
64	-1.70431257042004\\
65	-1.49705972092496\\
66	-1.31556071724992\\
67	-0.862469776720799\\
68	-0.668809946507051\\
69	-0.469803098242145\\
70	-0.325456185604305\\
71	-0.38051458905945\\
72	-0.405580506298859\\
73	-0.339171685398677\\
74	-0.377086900847224\\
75	-0.716884446783785\\
76	-0.47952262596985\\
77	-0.519355939344809\\
78	-0.286475788211918\\
79	-0.203560385957308\\
80	-0.180104773190205\\
81	-0.314871862531214\\
82	-0.200695103851073\\
83	-0.0446848487364954\\
84	-0.226872081824324\\
85	-0.227909739842205\\
86	-0.385460509166599\\
87	-0.404143824506832\\
88	-0.504956316605368\\
89	-0.543708896569871\\
90	-0.248730616515689\\
91	-0.0986080827279952\\
92	-0.273307648067015\\
93	-0.331234728112738\\
94	-0.240730553960417\\
95	-0.493556243582254\\
96	-0.3905477114633\\
97	-0.404702650281956\\
98	-0.386926224472736\\
99	-0.440253112053931\\
100	-0.560964585240496\\
101	-0.360407559509751\\
102	-0.0869393599737121\\
103	-0.317085738279582\\
104	-0.127496509147312\\
105	-0.434716433423199\\
106	-0.259673126767248\\
107	-0.527315558228875\\
108	-0.263381428304811\\
109	-0.164467451986712\\
110	-0.356687009522812\\
111	-0.121607283387638\\
112	-0.369739278801802\\
113	-0.38016696499471\\
114	-0.35377025773657\\
115	-0.300767382387272\\
116	-0.588148930610815\\
117	-0.354436484801845\\
118	-0.332427800988533\\
119	-0.153982462540731\\
120	-0.118618852659075\\
121	-0.139898322557044\\
122	-0.486342661443302\\
123	-0.345015415514119\\
124	-0.293435759928378\\
125	-0.245055120585917\\
126	-0.382851551399373\\
127	-0.169932737664176\\
128	-0.382457292254044\\
129	-0.150301432642932\\
130	-0.278191793922966\\
131	-0.0558911232996129\\
132	-0.336204839271164\\
133	-0.434759615018006\\
134	-0.338575738115096\\
135	-0.425213763737322\\
136	-0.391166853208026\\
137	-0.528851296081234\\
138	-0.422015803888237\\
139	-0.511183118717043\\
140	-0.346263757709117\\
141	-0.223641003158337\\
142	-0.234058590141052\\
143	-0.218144794915123\\
144	-0.389034907091786\\
145	-0.328177551380401\\
146	-0.221278387779689\\
147	-0.250964604463829\\
148	-0.435785062092553\\
149	-0.308901906226218\\
150	-0.252937633039795\\
};
\end{axis}

\begin{axis}[%
width=4.272in,
height=1.075in,
at={(0.717in,0.346in)},
scale only axis,
xmin=0,
xmax=150,
xlabel style={font=\color{white!15!black}},
xlabel={k},
ymin=-0.5,
ymax=1.5,
ylabel style={font=\color{white!15!black}},
ylabel={Z(k)},
axis background/.style={fill=white}
]
\addplot[const plot, color=mycolor1, forget plot] table[row sep=crcr] {%
1	-0.107409163637826\\
2	0.0869552807313267\\
3	0.0981051414525262\\
4	-0.175882536913901\\
5	-0.0148095957567039\\
6	0.0251941746175391\\
7	-0.0610384625892835\\
8	0.0760979810218383\\
9	0.025442950225039\\
10	-0.0152833676937382\\
11	-0.0557410875209181\\
12	0.0677017656847446\\
13	0.0849297440422057\\
14	0.0469307182589951\\
15	-0.0239781704489258\\
16	0.0920985453809901\\
17	-0.0906881378407088\\
18	0.014444060627547\\
19	0.13964090281818\\
20	-0.0189698759604886\\
21	-0.0843835890826002\\
22	0.194343804984731\\
23	-0.0246921540574881\\
24	0.0441619155865907\\
25	-0.0803611484220362\\
26	0.0287609432704476\\
27	0.0467018093753898\\
28	0.000456160956191707\\
29	0.0326090358603265\\
30	0.120458671653349\\
31	-0.140411583207086\\
32	0.0205563555454331\\
33	-0.199119578942207\\
34	0.00284089339182698\\
35	0.193051540875233\\
36	0.0941136788326655\\
37	-0.000692445130187209\\
38	-0.0338241647272311\\
39	0.0528265783681366\\
40	0.105689959434016\\
41	0.00147406166024418\\
42	0.161476126691229\\
43	-0.0815854961856177\\
44	-0.199775912436652\\
45	-0.0748191408314749\\
46	-0.0199304020638472\\
47	-0.0372760077970442\\
48	-0.0847371160164792\\
49	0.133723804312817\\
50	-0.161631725794785\\
51	-0.0491817594931569\\
52	-0.0969301567007431\\
53	-0.0474107920526903\\
54	-0.066616382519409\\
55	0.0478374683115003\\
56	-0.104105813301215\\
57	-0.0820659090657572\\
58	0.010112960601311\\
59	0.108398253686458\\
60	0.814547276261703\\
61	1.17997527857881\\
62	0.912162569442147\\
63	1.06477327264332\\
64	1.10495650322525\\
65	1.13230933768457\\
66	0.948264103436202\\
67	1.00243149497111\\
68	0.983763312644646\\
69	0.939433891393262\\
70	1.0271709764009\\
71	1.03586434121791\\
72	0.964768487721931\\
73	1.01323599281997\\
74	1.26735817771994\\
75	0.999243392455546\\
76	1.07103029366614\\
77	0.95374606371871\\
78	0.947212209939989\\
79	0.96542129330955\\
80	1.05645577737344\\
81	0.919846085599793\\
82	0.814705321483177\\
83	1.00532160174532\\
84	0.956358478294511\\
85	1.02800074722662\\
86	1.00846058871809\\
87	1.07291190595681\\
88	1.09007530956458\\
89	0.867436170546814\\
90	0.847542117151499\\
91	1.05024333330629\\
92	1.03063914256495\\
93	0.906083953468134\\
94	1.12374021264301\\
95	1.0033011914555\\
96	1.01235683182212\\
97	1.02739393795508\\
98	1.07541658082824\\
99	1.15244975057338\\
100	0.97380242377443\\
101	0.826931649222705\\
102	1.10134642999901\\
103	0.90722639015537\\
104	1.12436802862236\\
105	0.941051130241411\\
106	1.14914817347865\\
107	0.919402190528573\\
108	0.88874009472468\\
109	1.09569395680994\\
110	0.863694841106271\\
111	1.06805849627241\\
112	1.04378236225657\\
113	0.985673245332796\\
114	0.965846749539944\\
115	1.20578106340642\\
116	0.96254649645742\\
117	0.980979680254716\\
118	0.90612675893883\\
119	0.918653763561683\\
120	0.94163812484714\\
121	1.17095253705239\\
122	0.960958018222896\\
123	0.93099346367202\\
124	0.955177929807091\\
125	1.07143763500996\\
126	0.868426875171689\\
127	1.06271220952632\\
128	0.866870515673932\\
129	0.986915184824033\\
130	0.817310382928203\\
131	1.05359224651102\\
132	1.06928647154366\\
133	0.931245024491926\\
134	1.03175757564404\\
135	1.01628423401243\\
136	1.11591159058306\\
137	1.01283880586636\\
138	1.10394581333017\\
139	0.988322885869547\\
140	0.935170308989575\\
141	0.996201879511236\\
142	0.980067518122789\\
143	1.08816698720157\\
144	0.994385592909307\\
145	0.91587611306126\\
146	0.984515616191396\\
147	1.11885946704643\\
148	0.958509272076415\\
149	0.93413832374886\\
150	1.09839906680987\\
};
\end{axis}
\end{tikzpicture}%
\caption{Zakłócenie i sygnał sterujący- szum duży}
\end{figure}s

\begin{figure}[H]
\centering
% This file was created by matlab2tikz.
%
%The latest updates can be retrieved from
%  http://www.mathworks.com/matlabcentral/fileexchange/22022-matlab2tikz-matlab2tikz
%where you can also make suggestions and rate matlab2tikz.
%
\definecolor{mycolor1}{rgb}{0.00000,0.44700,0.74100}%
\definecolor{mycolor2}{rgb}{0.85000,0.32500,0.09800}%
%
\begin{tikzpicture}

\begin{axis}[%
width=4.272in,
height=2.472in,
at={(0.717in,0.441in)},
scale only axis,
xmin=0,
xmax=150,
xlabel style={font=\color{white!15!black}},
xlabel={k},
ymin=-0.5,
ymax=2,
ylabel style={font=\color{white!15!black}},
ylabel={Y(k)},
axis background/.style={fill=white},
legend style={legend cell align=left, align=left, draw=white!15!black}
]
\addplot[const plot, color=mycolor1] table[row sep=crcr] {%
1	0\\
2	0\\
3	0\\
4	0\\
5	0\\
6	0\\
7	0\\
8	0\\
9	0\\
10	0\\
11	0\\
12	-0.00932395844835842\\
13	-0.0249732779094304\\
14	-0.0462342044946201\\
15	-0.0394131167562775\\
16	-0.0291642737086906\\
17	-0.0270716378804508\\
18	-0.0389079290265876\\
19	-0.0237096208524286\\
20	-0.0680251452328963\\
21	-0.0941598500406067\\
22	-0.0876328565079645\\
23	-0.0935652445106375\\
24	-0.122120161703767\\
25	-0.0545026504073886\\
26	-0.0481277018515096\\
27	-0.0515769869832808\\
28	-0.0620447344680745\\
29	-0.0323542221421181\\
30	-0.0453076857955384\\
31	-0.0386061143866624\\
32	-0.0380067986111862\\
33	0.00921090013668197\\
34	-0.0144914019676052\\
35	-0.00885789766551821\\
36	-0.0464455309252497\\
37	0.13297299082672\\
38	0.370901422819155\\
39	0.607109819388092\\
40	0.759182029522761\\
41	0.906059647050559\\
42	0.999818235608636\\
43	1.02930036567269\\
44	1.0133928124681\\
45	1.0346099991438\\
46	1.01775708565466\\
47	0.971039108631479\\
48	0.941347523001663\\
49	0.93811006122794\\
50	0.90361818161633\\
51	0.905235143127714\\
52	0.98696843462076\\
53	0.997301235423661\\
54	1.01684547763088\\
55	1.01429385710134\\
56	1.02361710569762\\
57	0.984922969422167\\
58	1.01697802135786\\
59	1.00363117492976\\
60	1.01222388560274\\
61	1.02981919645092\\
62	1.06622269235514\\
63	1.21609940324224\\
64	1.44389197096283\\
65	1.59310791507708\\
66	1.74419294506313\\
67	1.86032609303669\\
68	1.81392109315252\\
69	1.59995403299167\\
70	1.40925750005798\\
71	1.21307409297356\\
72	1.04631950723299\\
73	0.931034447578793\\
74	0.887760085140817\\
75	0.858057405998818\\
76	0.867196126012618\\
77	0.946060309614897\\
78	0.953079344392279\\
79	0.969027878998427\\
80	0.968101864843424\\
81	0.959734070119015\\
82	0.904855290162109\\
83	0.906136025538982\\
84	0.872360881053454\\
85	0.853922817034791\\
86	0.888596763219179\\
87	0.913862620448574\\
88	0.931873100899753\\
89	0.960717239557309\\
90	1.02323336062518\\
91	1.05725092730242\\
92	1.0427649655828\\
93	1.0029984142548\\
94	1.00503361227126\\
95	0.987027109732246\\
96	0.938468989841496\\
97	0.981069879603072\\
98	1.01729944192638\\
99	1.02714353245808\\
100	1.03057033442649\\
101	1.05644674305273\\
102	1.05806652873783\\
103	1.03710955752334\\
104	0.986102154962403\\
105	0.998174510744007\\
106	0.96125995860391\\
107	0.953588397585833\\
108	0.937593852028579\\
109	1.00574605035219\\
110	0.987824629558798\\
111	0.993855231443089\\
112	0.997165568239598\\
113	0.978197916007305\\
114	0.963569927369103\\
115	0.983085608300769\\
116	1.00384045785411\\
117	0.991119910303568\\
118	1.06276367671565\\
119	1.04209636634061\\
120	1.02545433606755\\
121	0.998932304082016\\
122	0.985464813917983\\
123	0.935796250662267\\
124	0.970470217009051\\
125	0.961922353460384\\
126	0.974689495219506\\
127	0.997557684545838\\
128	1.03981104821058\\
129	0.986426770771076\\
130	0.997736087793388\\
131	0.97558651548887\\
132	0.987495391306278\\
133	0.943962709221018\\
134	0.983960132587423\\
135	0.992338305008809\\
136	1.00561420603782\\
137	1.02004356413493\\
138	1.06277459564718\\
139	1.08128952777417\\
140	1.06192401049158\\
141	1.07620703859273\\
142	1.05238516694692\\
143	1.02454272298809\\
144	0.991128762126001\\
145	0.972299083403783\\
146	0.963468562227888\\
147	0.959651428865378\\
148	0.958492846141326\\
149	0.970711677462253\\
150	1.01183763005872\\
};
\addlegendentry{Y}

\addplot[const plot, color=mycolor2] table[row sep=crcr] {%
1	0\\
2	0\\
3	0\\
4	0\\
5	0\\
6	0\\
7	0\\
8	0\\
9	0\\
10	0\\
11	0\\
12	0\\
13	0\\
14	0\\
15	0\\
16	0\\
17	0\\
18	0\\
19	0\\
20	0\\
21	0\\
22	0\\
23	0\\
24	0\\
25	0\\
26	0\\
27	0\\
28	0\\
29	0\\
30	1\\
31	1\\
32	1\\
33	1\\
34	1\\
35	1\\
36	1\\
37	1\\
38	1\\
39	1\\
40	1\\
41	1\\
42	1\\
43	1\\
44	1\\
45	1\\
46	1\\
47	1\\
48	1\\
49	1\\
50	1\\
51	1\\
52	1\\
53	1\\
54	1\\
55	1\\
56	1\\
57	1\\
58	1\\
59	1\\
60	1\\
61	1\\
62	1\\
63	1\\
64	1\\
65	1\\
66	1\\
67	1\\
68	1\\
69	1\\
70	1\\
71	1\\
72	1\\
73	1\\
74	1\\
75	1\\
76	1\\
77	1\\
78	1\\
79	1\\
80	1\\
81	1\\
82	1\\
83	1\\
84	1\\
85	1\\
86	1\\
87	1\\
88	1\\
89	1\\
90	1\\
91	1\\
92	1\\
93	1\\
94	1\\
95	1\\
96	1\\
97	1\\
98	1\\
99	1\\
100	1\\
101	1\\
102	1\\
103	1\\
104	1\\
105	1\\
106	1\\
107	1\\
108	1\\
109	1\\
110	1\\
111	1\\
112	1\\
113	1\\
114	1\\
115	1\\
116	1\\
117	1\\
118	1\\
119	1\\
120	1\\
121	1\\
122	1\\
123	1\\
124	1\\
125	1\\
126	1\\
127	1\\
128	1\\
129	1\\
130	1\\
131	1\\
132	1\\
133	1\\
134	1\\
135	1\\
136	1\\
137	1\\
138	1\\
139	1\\
140	1\\
141	1\\
142	1\\
143	1\\
144	1\\
145	1\\
146	1\\
147	1\\
148	1\\
149	1\\
150	1\\
};
\addlegendentry{Yzad}

\end{axis}
\end{tikzpicture}%
\caption{Wyjście dla pomiaru z szumem dużym (błąd $E=12,1956$)}
\end{figure}


\section{Szum nałożony na zakłócenie sinusoidalne}

\begin{figure}[H]
\centering
% This file was created by matlab2tikz.
%
%The latest updates can be retrieved from
%  http://www.mathworks.com/matlabcentral/fileexchange/22022-matlab2tikz-matlab2tikz
%where you can also make suggestions and rate matlab2tikz.
%
\definecolor{mycolor1}{rgb}{0.00000,0.44700,0.74100}%
%
\begin{tikzpicture}

\begin{axis}[%
width=4.272in,
height=1.075in,
at={(0.717in,1.839in)},
scale only axis,
xmin=0,
xmax=150,
xlabel style={font=\color{white!15!black}},
xlabel={k},
ymin=-1,
ymax=2,
ylabel style={font=\color{white!15!black}},
ylabel={U(k)},
axis background/.style={fill=white}
]
\addplot[const plot, color=mycolor1, forget plot] table[row sep=crcr] {%
1	0\\
2	0\\
3	0\\
4	0\\
5	0\\
6	0\\
7	0\\
8	0\\
9	0\\
10	0\\
11	0\\
12	-0.00797246580289122\\
13	0.00200459182950845\\
14	-0.00584885268942529\\
15	-0.0157559776580183\\
16	-0.0347492438765453\\
17	-0.0115374049795036\\
18	-0.0343135249996197\\
19	-0.0555167164252526\\
20	-0.0217394886715013\\
21	-0.021343417739215\\
22	0.00432499817083925\\
23	0.025816691729009\\
24	-0.00870213383146115\\
25	-0.0111600020788221\\
26	-0.00366722366294168\\
27	-0.00613127223140198\\
28	0.000928962893574016\\
29	0.0185912614161591\\
30	1.2502298007929\\
31	1.55430033464826\\
32	1.47712806020729\\
33	1.24925210004155\\
34	1.0131299552185\\
35	0.825082093125621\\
36	0.660838025056799\\
37	0.543017532612562\\
38	0.483642965024297\\
39	0.424656355569902\\
40	0.408815716152663\\
41	0.395180587210482\\
42	0.410574665742057\\
43	0.407672108850289\\
44	0.393630970114338\\
45	0.402878167903892\\
46	0.391447492015039\\
47	0.366447212893125\\
48	0.372760487377774\\
49	0.371155425858008\\
50	0.359203626582762\\
51	0.373464579473232\\
52	0.389544308474927\\
53	0.379477473645659\\
54	0.416883011405546\\
55	0.423777709470757\\
56	0.394244831686506\\
57	0.400892230757417\\
58	0.402943324033469\\
59	0.400888403350649\\
60	0.410462288522805\\
61	0.412248301278818\\
62	0.228512683332551\\
63	-0.00568899751792373\\
64	-0.241637292312909\\
65	-0.42689046270651\\
66	-0.49994434119288\\
67	-0.498871278315594\\
68	-0.363442773007015\\
69	-0.165407751768776\\
70	0.11839697737902\\
71	0.422555759136269\\
72	0.713587617930029\\
73	0.988753170997633\\
74	1.24201266208231\\
75	1.40145022049695\\
76	1.514438969774\\
77	1.53104474717188\\
78	1.43733378043826\\
79	1.28452967058873\\
80	1.06707697560209\\
81	0.804065638122236\\
82	0.517137386855906\\
83	0.204826663278515\\
84	-0.118502933261323\\
85	-0.33696842549729\\
86	-0.545294090115191\\
87	-0.672309867708786\\
88	-0.724753832968382\\
89	-0.659060185358547\\
90	-0.555063151425206\\
91	-0.360696971441429\\
92	-0.0857929398018611\\
93	0.209809965388429\\
94	0.539580317574218\\
95	0.80903741130492\\
96	1.08657859595116\\
97	1.31006669996487\\
98	1.47429873108308\\
99	1.53478789107935\\
100	1.49153837524231\\
101	1.39426446275156\\
102	1.21190750361711\\
103	0.992838589036933\\
104	0.694452552684081\\
105	0.373565810521012\\
106	0.0851975302354054\\
107	-0.226853945720978\\
108	-0.487677495041706\\
109	-0.618303657120748\\
110	-0.689450919631019\\
111	-0.706218479164931\\
112	-0.644196012035963\\
113	-0.486690972025556\\
114	-0.260568392218909\\
115	0.017616694746722\\
116	0.340096727655241\\
117	0.666038056995758\\
118	0.92277851620047\\
119	1.17956396076923\\
120	1.38716626655265\\
121	1.48461125610128\\
122	1.54693324011095\\
123	1.48338375622814\\
124	1.3372832031161\\
125	1.14346252400019\\
126	0.875969101477247\\
127	0.572754738665823\\
128	0.230197660071874\\
129	-0.0631949528601308\\
130	-0.318641214400011\\
131	-0.524122136430428\\
132	-0.66080582574596\\
133	-0.742214194684139\\
134	-0.702760761453407\\
135	-0.583360364331544\\
136	-0.403216196489846\\
137	-0.152159303455243\\
138	0.125082254581204\\
139	0.430534913824439\\
140	0.734011932429442\\
141	1.02124659816192\\
142	1.25814231554976\\
143	1.41104022906453\\
144	1.53561971038067\\
145	1.53718598606356\\
146	1.47647125162648\\
147	1.27986450303286\\
148	1.0482320861725\\
149	0.76396944063423\\
150	0.475732004287599\\
};
\end{axis}

\begin{axis}[%
width=4.272in,
height=1.075in,
at={(0.717in,0.346in)},
scale only axis,
xmin=0,
xmax=150,
xlabel style={font=\color{white!15!black}},
xlabel={k},
ymin=-0.516346439270734,
ymax=0.512122531870199,
ylabel style={font=\color{white!15!black}},
ylabel={Z(k)},
axis background/.style={fill=white}
]
\addplot[const plot, color=mycolor1, forget plot] table[row sep=crcr] {%
1	-0.00837675995187059\\
2	-0.0130750261602475\\
3	0.00794142715421655\\
4	-0.00197263789445607\\
5	0.00649152816254485\\
6	-0.0083147427215967\\
7	0.00895954060001157\\
8	-0.0181348642042825\\
9	0.0156668345394797\\
10	0.00846500664558696\\
11	0.0011015483625554\\
12	-0.0116110152841623\\
13	-0.0039753570256193\\
14	0.00254294311040492\\
15	0.0120778943607426\\
16	-0.0103354161269991\\
17	0.0129514675428588\\
18	0.0276812303334221\\
19	-0.00495347149392365\\
20	0.00468765524421863\\
21	-0.00657296887156008\\
22	-0.0171695724227432\\
23	0.014705191041064\\
24	0.0069413726560036\\
25	-0.00510696986674642\\
26	0.00113387383320868\\
27	-0.00229807147664015\\
28	-0.0146173284001932\\
29	-0.0288233613993966\\
30	-0.000474676075293698\\
31	-0.00462465950361891\\
32	-0.00576645894751125\\
33	-0.00845968359502791\\
34	-0.0181723551410033\\
35	-0.00521730614435447\\
36	0.00161435878047334\\
37	-0.0106181087486791\\
38	0.00450493739221112\\
39	-0.00272798073223805\\
40	-0.00101454350734858\\
41	-0.0142910594792867\\
42	-0.00764409438691379\\
43	0.00410137543311993\\
44	-0.00789932749060492\\
45	0.00161637129369329\\
46	0.0197790399736349\\
47	0.00795266427309762\\
48	0.0103736673316936\\
49	0.0236032996494344\\
50	0.0117539509969254\\
51	0.00397740426136556\\
52	0.0195124871677449\\
53	-0.00904656364205178\\
54	-0.00682152191566547\\
55	0.020182915767099\\
56	0.00454343178361193\\
57	-0.000421041798190231\\
58	0.00458167304241593\\
59	-0.00264144136031506\\
60	-0.00291985147351653\\
61	0.13324166316945\\
62	0.258167763556006\\
63	0.360651147313723\\
64	0.447497618688489\\
65	0.486003360015093\\
66	0.512122531870199\\
67	0.467856231716641\\
68	0.407823794015632\\
69	0.297837766722554\\
70	0.17710548466994\\
71	0.055547962886154\\
72	-0.0805986414177825\\
73	-0.225899043881557\\
74	-0.322904196743418\\
75	-0.422619018035537\\
76	-0.478473256280212\\
77	-0.47805581979119\\
78	-0.464105286282962\\
79	-0.413117582006605\\
80	-0.328528161665858\\
81	-0.222799939822209\\
82	-0.0867652947401764\\
83	0.0709734253566001\\
84	0.166860548705007\\
85	0.300685762563961\\
86	0.401606822758281\\
87	0.470482738806816\\
88	0.480710145032296\\
89	0.495167012528826\\
90	0.450944372662087\\
91	0.35400394401487\\
92	0.252709969741182\\
93	0.115491985662538\\
94	0.00641288868664602\\
95	-0.142803702861963\\
96	-0.274519473391611\\
97	-0.385882984260254\\
98	-0.451385238884499\\
99	-0.476591091599417\\
100	-0.493913927229994\\
101	-0.458289010381283\\
102	-0.403575178749728\\
103	-0.289639588194874\\
104	-0.160935946087456\\
105	-0.0469753027060979\\
106	0.116200761368226\\
107	0.259607685232014\\
108	0.331396595104371\\
109	0.411730653623517\\
110	0.483942514207537\\
111	0.507373571727501\\
112	0.475763638842218\\
113	0.41525604631429\\
114	0.325208900443571\\
115	0.197873621718251\\
116	0.0594700359654598\\
117	-0.0467054531454365\\
118	-0.193231740454931\\
119	-0.326632331853076\\
120	-0.398464259213315\\
121	-0.485291338817356\\
122	-0.50041879651624\\
123	-0.480405287345292\\
124	-0.447433599737882\\
125	-0.360076317991005\\
126	-0.246502925576433\\
127	-0.0985721278408086\\
128	0.0242516484418504\\
129	0.151918214899291\\
130	0.277954369280465\\
131	0.380219930364021\\
132	0.469539870989595\\
133	0.492082981374197\\
134	0.487519868873454\\
135	0.460370419445229\\
136	0.384434914461479\\
137	0.29118536165658\\
138	0.169003281885769\\
139	0.034214604903766\\
140	-0.108039695869876\\
141	-0.237261754528763\\
142	-0.334955597460584\\
143	-0.44885098230494\\
144	-0.490464086413987\\
145	-0.516346439270734\\
146	-0.464320241440745\\
147	-0.411114845559313\\
148	-0.320604733955713\\
149	-0.216752791405048\\
150	-0.0643415563376665\\
};
\end{axis}
\end{tikzpicture}%
\caption{Zakłócenie i sygnał sterujący- szum mały}
\end{figure}

\begin{figure}[H]
\centering
% This file was created by matlab2tikz.
%
%The latest updates can be retrieved from
%  http://www.mathworks.com/matlabcentral/fileexchange/22022-matlab2tikz-matlab2tikz
%where you can also make suggestions and rate matlab2tikz.
%
\definecolor{mycolor1}{rgb}{0.00000,0.44700,0.74100}%
\definecolor{mycolor2}{rgb}{0.85000,0.32500,0.09800}%
%
\begin{tikzpicture}

\begin{axis}[%
width=4.272in,
height=2.472in,
at={(0.717in,0.441in)},
scale only axis,
xmin=0,
xmax=150,
xlabel style={font=\color{white!15!black}},
xlabel={k},
ymin=-0.0129563240453394,
ymax=1.5858560065226,
ylabel style={font=\color{white!15!black}},
ylabel={Y(k)},
axis background/.style={fill=white},
legend style={legend cell align=left, align=left, draw=white!15!black}
]
\addplot[const plot, color=mycolor1] table[row sep=crcr] {%
1	0\\
2	0\\
3	0\\
4	0\\
5	0\\
6	0\\
7	0\\
8	0\\
9	0\\
10	0\\
11	0\\
12	0.00661190755297367\\
13	0.0108228395529902\\
14	0.0128816396100534\\
15	0.0119605608493312\\
16	0.012515496133535\\
17	0.0141650992574017\\
18	0.0174053713783122\\
19	0.0144334885575827\\
20	0.0177792928239368\\
21	0.0224580090592082\\
22	0.0183923911673259\\
23	0.0136926649834132\\
24	0.0103133854426419\\
25	0.00164344718840924\\
26	-0.00307967366550126\\
27	-0.0043229459713338\\
28	-0.00798578547487149\\
29	-0.00635056169795089\\
30	-0.00237233823913904\\
31	-0.00622629429181626\\
32	-0.0129563240453394\\
33	-0.0121859719639629\\
34	-0.0127307734485949\\
35	-0.0124447404859615\\
36	-0.0101087561574802\\
37	0.172289487086579\\
38	0.391599176963107\\
39	0.587864853652845\\
40	0.736241804857874\\
41	0.844013266294662\\
42	0.916008639231995\\
43	0.959764565384368\\
44	0.980775031204806\\
45	0.993202260362089\\
46	0.998559478353058\\
47	0.998720838762625\\
48	0.998807975834839\\
49	1.00475974427386\\
50	1.00727870526329\\
51	1.00795526610523\\
52	1.01251189007763\\
53	1.01246624065266\\
54	1.007061725829\\
55	1.00604953806813\\
56	0.998924293577819\\
57	0.991061519497201\\
58	0.991345595533938\\
59	0.990645766820627\\
60	0.987498668138342\\
61	0.991135867035167\\
62	0.994113998428275\\
63	0.992565965182881\\
64	1.01966840284155\\
65	1.06930884547395\\
66	1.13382986739663\\
67	1.21010211244773\\
68	1.28596153741797\\
69	1.3315241892872\\
70	1.32706022555337\\
71	1.2729834639463\\
72	1.17060062436622\\
73	1.03858436882012\\
74	0.890835173231531\\
75	0.746413957726962\\
76	0.613761646923735\\
77	0.515810743562292\\
78	0.45350722323742\\
79	0.432166405698319\\
80	0.458364056926952\\
81	0.528158472518383\\
82	0.63187909834801\\
83	0.766420590061585\\
84	0.918713182806534\\
85	1.0761674425447\\
86	1.23260282611281\\
87	1.36444136323672\\
88	1.47332401000347\\
89	1.54936040263116\\
90	1.58367486643334\\
91	1.56507708740422\\
92	1.51352795178903\\
93	1.4209924580182\\
94	1.29222794471273\\
95	1.14094197330234\\
96	0.979983319073778\\
97	0.822814957018095\\
98	0.675534634355928\\
99	0.554379560764932\\
100	0.46601045004448\\
101	0.423504300688355\\
102	0.423497548358389\\
103	0.465945548018534\\
104	0.550672135840102\\
105	0.669375422521058\\
106	0.81566929370831\\
107	0.97404295822494\\
108	1.13130607592655\\
109	1.28370428349912\\
110	1.42052755314304\\
111	1.51531365343192\\
112	1.56866442013437\\
113	1.5858560065226\\
114	1.55545324665712\\
115	1.47705875949682\\
116	1.3678113884962\\
117	1.23345864662743\\
118	1.07728241487313\\
119	0.911532304496885\\
120	0.758893563076887\\
121	0.621560845259923\\
122	0.510276787343201\\
123	0.443499303233827\\
124	0.416314093240437\\
125	0.431000227842964\\
126	0.49174612188551\\
127	0.590403310409094\\
128	0.718604885067394\\
129	0.87339358828471\\
130	1.04007797985964\\
131	1.19902350843508\\
132	1.34341358811327\\
133	1.46177206433012\\
134	1.54458442814641\\
135	1.58491467148182\\
136	1.57865806486079\\
137	1.52906891226481\\
138	1.44214961845524\\
139	1.32121294891427\\
140	1.17416881389236\\
141	1.01585945903577\\
142	0.857663815377715\\
143	0.708459861422542\\
144	0.582166422789301\\
145	0.48871119210174\\
146	0.427636370058959\\
147	0.412030150287836\\
148	0.439818776295348\\
149	0.516419043592907\\
150	0.625648281869585\\
};
\addlegendentry{Y}

\addplot[const plot, color=mycolor2] table[row sep=crcr] {%
1	0\\
2	0\\
3	0\\
4	0\\
5	0\\
6	0\\
7	0\\
8	0\\
9	0\\
10	0\\
11	0\\
12	0\\
13	0\\
14	0\\
15	0\\
16	0\\
17	0\\
18	0\\
19	0\\
20	0\\
21	0\\
22	0\\
23	0\\
24	0\\
25	0\\
26	0\\
27	0\\
28	0\\
29	0\\
30	1\\
31	1\\
32	1\\
33	1\\
34	1\\
35	1\\
36	1\\
37	1\\
38	1\\
39	1\\
40	1\\
41	1\\
42	1\\
43	1\\
44	1\\
45	1\\
46	1\\
47	1\\
48	1\\
49	1\\
50	1\\
51	1\\
52	1\\
53	1\\
54	1\\
55	1\\
56	1\\
57	1\\
58	1\\
59	1\\
60	1\\
61	1\\
62	1\\
63	1\\
64	1\\
65	1\\
66	1\\
67	1\\
68	1\\
69	1\\
70	1\\
71	1\\
72	1\\
73	1\\
74	1\\
75	1\\
76	1\\
77	1\\
78	1\\
79	1\\
80	1\\
81	1\\
82	1\\
83	1\\
84	1\\
85	1\\
86	1\\
87	1\\
88	1\\
89	1\\
90	1\\
91	1\\
92	1\\
93	1\\
94	1\\
95	1\\
96	1\\
97	1\\
98	1\\
99	1\\
100	1\\
101	1\\
102	1\\
103	1\\
104	1\\
105	1\\
106	1\\
107	1\\
108	1\\
109	1\\
110	1\\
111	1\\
112	1\\
113	1\\
114	1\\
115	1\\
116	1\\
117	1\\
118	1\\
119	1\\
120	1\\
121	1\\
122	1\\
123	1\\
124	1\\
125	1\\
126	1\\
127	1\\
128	1\\
129	1\\
130	1\\
131	1\\
132	1\\
133	1\\
134	1\\
135	1\\
136	1\\
137	1\\
138	1\\
139	1\\
140	1\\
141	1\\
142	1\\
143	1\\
144	1\\
145	1\\
146	1\\
147	1\\
148	1\\
149	1\\
150	1\\
};
\addlegendentry{Yzad}

\end{axis}
\end{tikzpicture}%
\caption{Wyjście dla pomiaru z szumem małym (błąd $E=22,3174$)}
\end{figure}


\begin{figure}[H]
\centering
% This file was created by matlab2tikz.
%
%The latest updates can be retrieved from
%  http://www.mathworks.com/matlabcentral/fileexchange/22022-matlab2tikz-matlab2tikz
%where you can also make suggestions and rate matlab2tikz.
%
\definecolor{mycolor1}{rgb}{0.00000,0.44700,0.74100}%
%
\begin{tikzpicture}

\begin{axis}[%
width=4.272in,
height=1.075in,
at={(0.717in,1.839in)},
scale only axis,
xmin=0,
xmax=150,
xlabel style={font=\color{white!15!black}},
xlabel={k},
ymin=-1,
ymax=2,
ylabel style={font=\color{white!15!black}},
ylabel={U(k)},
axis background/.style={fill=white}
]
\addplot[const plot, color=mycolor1, forget plot] table[row sep=crcr] {%
1	0\\
2	0\\
3	0\\
4	0\\
5	0\\
6	0\\
7	0\\
8	0\\
9	0\\
10	0\\
11	0\\
12	0.00736362754671591\\
13	-0.0160624325883666\\
14	0.0404043961983865\\
15	0.0549591080243713\\
16	0.109260568931026\\
17	0.100853982351326\\
18	0.0253072294760826\\
19	0.015321021791237\\
20	-0.0400803097390891\\
21	-0.0482448109032252\\
22	-0.0398254236240949\\
23	-0.0462963815948263\\
24	-0.0739029482253168\\
25	-0.0413273707423866\\
26	-0.0223355735545961\\
27	0.0157530128350385\\
28	0.0818714169883389\\
29	0.072926109735794\\
30	1.23883031546048\\
31	1.60397239568955\\
32	1.46558799300815\\
33	1.23604379508342\\
34	0.991731295418022\\
35	0.782178656233809\\
36	0.643699030789394\\
37	0.54620810536267\\
38	0.483754958610594\\
39	0.404575445933957\\
40	0.411368879671991\\
41	0.515170112776115\\
42	0.466076221805596\\
43	0.427757276036346\\
44	0.355992657441825\\
45	0.306805804352034\\
46	0.39056964707961\\
47	0.354951394799636\\
48	0.385871484814013\\
49	0.405217264696846\\
50	0.407706211608612\\
51	0.413210489262927\\
52	0.404282009974102\\
53	0.395682485967972\\
54	0.429701607049728\\
55	0.457080663009863\\
56	0.384315727997654\\
57	0.40657297323802\\
58	0.329487163658426\\
59	0.4140135920628\\
60	0.423426789294152\\
61	0.35847152081368\\
62	0.144098121029789\\
63	-0.00884547564501501\\
64	-0.264975926272828\\
65	-0.474390983025054\\
66	-0.573242522856034\\
67	-0.443828630035694\\
68	-0.399175072406509\\
69	-0.16146390089326\\
70	0.112617832903949\\
71	0.443013866359424\\
72	0.792773350825318\\
73	1.04570204698166\\
74	1.28650014437329\\
75	1.33859564754644\\
76	1.48425612224512\\
77	1.51749893247877\\
78	1.46659614476398\\
79	1.34876142988268\\
80	1.01894311843574\\
81	0.767706926636894\\
82	0.477993640330601\\
83	0.141788731344672\\
84	-0.107670325207461\\
85	-0.351590934938685\\
86	-0.517536999453197\\
87	-0.631147217614242\\
88	-0.766062432094818\\
89	-0.722968183286035\\
90	-0.579328167290488\\
91	-0.428399671152597\\
92	-0.105349901793887\\
93	0.197280955692214\\
94	0.600425796376072\\
95	0.850956850653366\\
96	1.19047410746618\\
97	1.39227003721158\\
98	1.45755594118747\\
99	1.50887132359919\\
100	1.52206594363862\\
101	1.41756927648137\\
102	1.21708993624895\\
103	0.961766004060622\\
104	0.747649846815367\\
105	0.443307316722845\\
106	0.170176989141394\\
107	-0.186278535622269\\
108	-0.382663142790877\\
109	-0.659605029322408\\
110	-0.706260592488187\\
111	-0.706992244223328\\
112	-0.733621563539725\\
113	-0.626021854641011\\
114	-0.300660767501889\\
115	-0.00736184472051388\\
116	0.249028179127083\\
117	0.646141317057361\\
118	0.863782206593481\\
119	1.14912951455021\\
120	1.37296955031968\\
121	1.52993693667958\\
122	1.4943389512448\\
123	1.45718334439727\\
124	1.37012963083526\\
125	1.1448380507425\\
126	0.9353160775521\\
127	0.597841788266556\\
128	0.214696744806207\\
129	-0.0814218816790386\\
130	-0.298612758188831\\
131	-0.529530367132049\\
132	-0.625260953691092\\
133	-0.796790729892282\\
134	-0.64167334421189\\
135	-0.550791979740045\\
136	-0.406708334077891\\
137	-0.120044361237163\\
138	0.137324630232317\\
139	0.420650983690489\\
140	0.686391147783828\\
141	1.0454813526345\\
142	1.27586559743302\\
143	1.47133032574543\\
144	1.54441221814875\\
145	1.51799917345082\\
146	1.46036517029778\\
147	1.24123531599499\\
148	1.01316256035993\\
149	0.76394230803024\\
150	0.441177836701096\\
};
\end{axis}

\begin{axis}[%
width=4.272in,
height=1.075in,
at={(0.717in,0.346in)},
scale only axis,
xmin=0,
xmax=150,
xlabel style={font=\color{white!15!black}},
xlabel={k},
ymin=-0.521017855307123,
ymax=0.556404933143915,
ylabel style={font=\color{white!15!black}},
ylabel={Z(k)},
axis background/.style={fill=white}
]
\addplot[const plot, color=mycolor1, forget plot] table[row sep=crcr] {%
1	-0.00821345639507819\\
2	-0.0428421881969322\\
3	-0.0366015298115334\\
4	-0.024096124429006\\
5	-0.00791516792870989\\
6	-0.0521384667174699\\
7	-0.00575504863351627\\
8	0.0194697397219508\\
9	-0.0119114863192416\\
10	-0.0220738133902148\\
11	0.00961719566440613\\
12	0.0348192043017277\\
13	-0.0141229840966605\\
14	-0.0147587371800649\\
15	-0.0454046355889962\\
16	-0.0309180980805826\\
17	0.0191600695646258\\
18	-0.00359407164580768\\
19	0.0241775591251869\\
20	0.0211840175792573\\
21	0.0118833236407034\\
22	0.0242436906736496\\
23	0.048176409800976\\
24	0.0182407826190953\\
25	0.0134704495271324\\
26	-0.00135759649092143\\
27	-0.0392838056850159\\
28	-0.015967973901705\\
29	0.0103043232072752\\
30	-0.0351462930236096\\
31	0.0148052882897143\\
32	0.0102019149126247\\
33	0.00316295557727557\\
34	0.00953219840932371\\
35	0.000753369569096519\\
36	-0.000697431932500447\\
37	-0.000276950024419697\\
38	0.0293919519736422\\
39	-0.00285972080105493\\
40	-0.0835203632103234\\
41	-0.0153734126014004\\
42	0.00619710357954776\\
43	0.0303490809595354\\
44	0.0436496894980746\\
45	-0.0297716943016982\\
46	0.0240607620886922\\
47	0.00752370517330569\\
48	-0.00658934735762414\\
49	0.000782322170900923\\
50	-0.000888318693824446\\
51	0.00443854535571105\\
52	0.00751313477833919\\
53	-0.0212362353253655\\
54	-0.0330458021302289\\
55	0.0305402227651014\\
56	-0.00694931135743456\\
57	0.0447315514229789\\
58	-0.0292253481586334\\
59	-0.0187883824178735\\
60	0.0449446760906472\\
61	0.181622106760085\\
62	0.231167831872316\\
63	0.376618916757131\\
64	0.490505763739525\\
65	0.533894406213104\\
66	0.45707204069725\\
67	0.516678503675773\\
68	0.418533041580582\\
69	0.306425397807076\\
70	0.170787652112084\\
71	0.0124181006999414\\
72	-0.0922792528457918\\
73	-0.234347046964146\\
74	-0.266776208840582\\
75	-0.421366087487956\\
76	-0.487642074229287\\
77	-0.500419457371631\\
78	-0.498838624138249\\
79	-0.356674117362049\\
80	-0.31240628714901\\
81	-0.215047765706894\\
82	-0.0500923123428492\\
83	0.0517058274264049\\
84	0.183984561433969\\
85	0.2894536706192\\
86	0.384031067431999\\
87	0.519740416079992\\
88	0.522850904905192\\
89	0.492625414854033\\
90	0.497543622352004\\
91	0.364152661376455\\
92	0.26447479853034\\
93	0.0827262786007183\\
94	0.00519686153422911\\
95	-0.192307073544353\\
96	-0.303262712888671\\
97	-0.34660640869076\\
98	-0.44126516489549\\
99	-0.521017855307123\\
100	-0.509256407831634\\
101	-0.453712689526292\\
102	-0.384613607379762\\
103	-0.345421378162904\\
104	-0.206826892351429\\
105	-0.0892254009792449\\
106	0.102625584159464\\
107	0.180526859057515\\
108	0.369171200607572\\
109	0.40060144204747\\
110	0.449680212650729\\
111	0.556404933143915\\
112	0.544960871418072\\
113	0.390932949304279\\
114	0.323972309179303\\
115	0.270595336454232\\
116	0.0588244639483252\\
117	-0.00820871267489531\\
118	-0.170233267715982\\
119	-0.313000523147682\\
120	-0.420390430166868\\
121	-0.418775541147117\\
122	-0.476715706108705\\
123	-0.5078227553448\\
124	-0.426607324668593\\
125	-0.38434519242426\\
126	-0.242089177109922\\
127	-0.0696536236517038\\
128	0.0364799160609692\\
129	0.131888550709027\\
130	0.291212440456367\\
131	0.360956300897808\\
132	0.520328965329816\\
133	0.438204973066564\\
134	0.474581513774828\\
135	0.482359727127619\\
136	0.35943991107949\\
137	0.283818688856025\\
138	0.180107107751742\\
139	0.0631155038422491\\
140	-0.143828226449222\\
141	-0.248959259718153\\
142	-0.368903954391837\\
143	-0.440863107043008\\
144	-0.475306792333295\\
145	-0.51895812545223\\
146	-0.448212240211377\\
147	-0.397836843788076\\
148	-0.332547740572066\\
149	-0.190816271464969\\
150	-0.0350518813672616\\
};
\end{axis}
\end{tikzpicture}%
\caption{Zakłócenie i sygnał sterujący- szum średni}
\end{figure}

\begin{figure}[H]
\centering
% This file was created by matlab2tikz.
%
%The latest updates can be retrieved from
%  http://www.mathworks.com/matlabcentral/fileexchange/22022-matlab2tikz-matlab2tikz
%where you can also make suggestions and rate matlab2tikz.
%
\definecolor{mycolor1}{rgb}{0.00000,0.44700,0.74100}%
\definecolor{mycolor2}{rgb}{0.85000,0.32500,0.09800}%
%
\begin{tikzpicture}

\begin{axis}[%
width=4.272in,
height=3.472in,
at={(0.717in,0.441in)},
scale only axis,
xmin=0,
xmax=150,
xlabel style={font=\color{white!15!black}},
xlabel={k},
ymin=-0.0313379684250232,
ymax=1.59173510806796,
ylabel style={font=\color{white!15!black}},
ylabel={Y(k)},
axis background/.style={fill=white},
legend style={legend cell align=left, align=left, draw=white!15!black}
]
\addplot[const plot, color=mycolor1] table[row sep=crcr] {%
1	0\\
2	0\\
3	0\\
4	0\\
5	0\\
6	0\\
7	0\\
8	0\\
9	0\\
10	0\\
11	0\\
12	-0.00610697189516439\\
13	-0.0133625443838905\\
14	-0.0131954648057447\\
15	-0.00776192647403886\\
16	-0.0126431581802659\\
17	-0.0169333344588578\\
18	-0.0267715488272743\\
19	-0.0313379684250232\\
20	-0.0285384359401706\\
21	-0.0222943357248857\\
22	-0.00854055827755179\\
23	0.0116301172573173\\
24	0.0273440233538207\\
25	0.033427586483142\\
26	0.0423542757030205\\
27	0.0361181764791491\\
28	0.0280512646191601\\
29	0.0187369231716221\\
30	0.00153681383638179\\
31	-0.0134494904829362\\
32	-0.0171694233929866\\
33	-0.0270982316316649\\
34	-0.0202778522273564\\
35	-0.00516040081805009\\
36	0.00623608422662193\\
37	0.19066179825924\\
38	0.416325291473013\\
39	0.607903689414484\\
40	0.754322707018864\\
41	0.861970381610019\\
42	0.925450927575138\\
43	0.948487545177236\\
44	0.970452425344302\\
45	0.986345067399863\\
46	0.99432986826537\\
47	1.00515305964797\\
48	1.01537262430309\\
49	1.02896241639698\\
50	1.03248299886268\\
51	1.02228028424092\\
52	1.00700295129217\\
53	1.00469474055983\\
54	0.998363428485713\\
55	0.997571358999585\\
56	0.993820229498338\\
57	0.988529272248833\\
58	0.997564850589736\\
59	0.996804149620801\\
60	1.00533289110514\\
61	1.00295994795789\\
62	1.00724506090503\\
63	1.01359934275883\\
64	1.04998218382462\\
65	1.08091377797073\\
66	1.1496737593946\\
67	1.23522872936044\\
68	1.31052022973486\\
69	1.32979838393785\\
70	1.33433430027157\\
71	1.27757324567918\\
72	1.16865694718949\\
73	1.02368497417707\\
74	0.875464357322714\\
75	0.724078937188886\\
76	0.591546280842658\\
77	0.505415331046481\\
78	0.446422515573049\\
79	0.434827689739059\\
80	0.46439810307279\\
81	0.53330531044027\\
82	0.639119447584405\\
83	0.771678865456005\\
84	0.922742323993681\\
85	1.09118412421368\\
86	1.2514908528901\\
87	1.37800788855228\\
88	1.47777443336862\\
89	1.54373619414653\\
90	1.57878201556712\\
91	1.56985668619985\\
92	1.51424694223384\\
93	1.43423661377559\\
94	1.31148169991747\\
95	1.15402201300669\\
96	0.974937002739612\\
97	0.81344644644403\\
98	0.646022649673574\\
99	0.517837994359871\\
100	0.437953151835934\\
101	0.407941990241035\\
102	0.405787192314739\\
103	0.461763982855803\\
104	0.560174742101307\\
105	0.679936936058839\\
106	0.810537581290622\\
107	0.965086163529416\\
108	1.11883942264606\\
109	1.27138363210191\\
110	1.38973969213241\\
111	1.50391424883981\\
112	1.56743598840642\\
113	1.59173510806796\\
114	1.57849070911962\\
115	1.52913675466952\\
116	1.40583811866535\\
117	1.26680609756248\\
118	1.12351853456431\\
119	0.941150916850047\\
120	0.773471683772734\\
121	0.633172417919594\\
122	0.519285813613129\\
123	0.433066335160665\\
124	0.41633066722255\\
125	0.425778380992544\\
126	0.475384465618684\\
127	0.576160823621116\\
128	0.705932547526483\\
129	0.853892987335142\\
130	1.02303753322325\\
131	1.18942222099247\\
132	1.32961009276209\\
133	1.45963153979494\\
134	1.54170624286637\\
135	1.58977539179643\\
136	1.56875218081519\\
137	1.51987670418738\\
138	1.43708680433343\\
139	1.31637030280166\\
140	1.16006738748106\\
141	1.01396074983543\\
142	0.86649415326897\\
143	0.708677400821659\\
144	0.584835512018856\\
145	0.486369318375361\\
146	0.426122000658813\\
147	0.407016345085725\\
148	0.438342817723465\\
149	0.521097566154491\\
150	0.641645806565739\\
};
\addlegendentry{Y}

\addplot[const plot, color=mycolor2] table[row sep=crcr] {%
1	0\\
2	0\\
3	0\\
4	0\\
5	0\\
6	0\\
7	0\\
8	0\\
9	0\\
10	0\\
11	0\\
12	0\\
13	0\\
14	0\\
15	0\\
16	0\\
17	0\\
18	0\\
19	0\\
20	0\\
21	0\\
22	0\\
23	0\\
24	0\\
25	0\\
26	0\\
27	0\\
28	0\\
29	0\\
30	1\\
31	1\\
32	1\\
33	1\\
34	1\\
35	1\\
36	1\\
37	1\\
38	1\\
39	1\\
40	1\\
41	1\\
42	1\\
43	1\\
44	1\\
45	1\\
46	1\\
47	1\\
48	1\\
49	1\\
50	1\\
51	1\\
52	1\\
53	1\\
54	1\\
55	1\\
56	1\\
57	1\\
58	1\\
59	1\\
60	1\\
61	1\\
62	1\\
63	1\\
64	1\\
65	1\\
66	1\\
67	1\\
68	1\\
69	1\\
70	1\\
71	1\\
72	1\\
73	1\\
74	1\\
75	1\\
76	1\\
77	1\\
78	1\\
79	1\\
80	1\\
81	1\\
82	1\\
83	1\\
84	1\\
85	1\\
86	1\\
87	1\\
88	1\\
89	1\\
90	1\\
91	1\\
92	1\\
93	1\\
94	1\\
95	1\\
96	1\\
97	1\\
98	1\\
99	1\\
100	1\\
101	1\\
102	1\\
103	1\\
104	1\\
105	1\\
106	1\\
107	1\\
108	1\\
109	1\\
110	1\\
111	1\\
112	1\\
113	1\\
114	1\\
115	1\\
116	1\\
117	1\\
118	1\\
119	1\\
120	1\\
121	1\\
122	1\\
123	1\\
124	1\\
125	1\\
126	1\\
127	1\\
128	1\\
129	1\\
130	1\\
131	1\\
132	1\\
133	1\\
134	1\\
135	1\\
136	1\\
137	1\\
138	1\\
139	1\\
140	1\\
141	1\\
142	1\\
143	1\\
144	1\\
145	1\\
146	1\\
147	1\\
148	1\\
149	1\\
150	1\\
};
\addlegendentry{Yzad}

\end{axis}
\end{tikzpicture}%
\caption{Wyjście dla pomiaru z szumem średnim (błąd $E=22,5508$)}
\end{figure}

\begin{figure}[H]
\centering
% This file was created by matlab2tikz.
%
%The latest updates can be retrieved from
%  http://www.mathworks.com/matlabcentral/fileexchange/22022-matlab2tikz-matlab2tikz
%where you can also make suggestions and rate matlab2tikz.
%
\definecolor{mycolor1}{rgb}{0.00000,0.44700,0.74100}%
%
\begin{tikzpicture}

\begin{axis}[%
width=4.272in,
height=1.075in,
at={(0.717in,1.839in)},
scale only axis,
xmin=0,
xmax=150,
xlabel style={font=\color{white!15!black}},
xlabel={k},
ymin=-1,
ymax=2,
ylabel style={font=\color{white!15!black}},
ylabel={U(k)},
axis background/.style={fill=white}
]
\addplot[const plot, color=mycolor1, forget plot] table[row sep=crcr] {%
1	0\\
2	0\\
3	0\\
4	0\\
5	0\\
6	0\\
7	0\\
8	0\\
9	0\\
10	0\\
11	0\\
12	0.00324368789232185\\
13	-0.0832186160608578\\
14	0.0234590446708595\\
15	0.0554324983454778\\
16	0.0906861067889487\\
17	-0.0213083822744724\\
18	0.106163502367857\\
19	0.147634824270168\\
20	0.0381869235112201\\
21	0.0953510015872512\\
22	-0.0835174493495838\\
23	0.185316380993027\\
24	-0.125175706368705\\
25	0.0883554516046265\\
26	-0.0323516464756629\\
27	0.0687938238100707\\
28	-0.0809514579431054\\
29	0.168227410936538\\
30	1.38962504325928\\
31	1.39970319554812\\
32	1.53973126296579\\
33	1.30369829628368\\
34	1.02325407972807\\
35	0.908184424344121\\
36	0.637427156179882\\
37	0.366128547172871\\
38	0.370376016681494\\
39	0.126427289825198\\
40	0.50890999532876\\
41	0.46786679386732\\
42	0.604046072921393\\
43	0.437206138875684\\
44	0.593075171389021\\
45	0.677853131460056\\
46	0.546105181408531\\
47	0.676719792142314\\
48	0.548823134579967\\
49	0.471907639350824\\
50	0.56392350544659\\
51	0.285048558211537\\
52	0.24523412527991\\
53	0.320329412689597\\
54	0.484880448674022\\
55	0.42892009799224\\
56	0.407829572342065\\
57	0.642510064558163\\
58	0.252608357582889\\
59	0.267225272383404\\
60	0.49751340295468\\
61	0.117078838277651\\
62	0.138878886101115\\
63	0.0868815855763187\\
64	-0.094919212182722\\
65	-0.389736897360707\\
66	-0.310880634445116\\
67	-0.392476434792676\\
68	-0.325860155795656\\
69	-0.127051290165661\\
70	0.044059376817385\\
71	0.44658603894924\\
72	0.719935471358573\\
73	0.957922733495194\\
74	1.49880048543865\\
75	1.73404530047423\\
76	1.77481017946485\\
77	1.56953210645736\\
78	1.35958000781249\\
79	1.26118666791061\\
80	0.798528978559958\\
81	0.502295231172377\\
82	0.348810936321043\\
83	-0.343906099476875\\
84	-0.219161471631099\\
85	-0.551917023676755\\
86	-0.527014041048137\\
87	-0.375758644016504\\
88	-0.795488712024809\\
89	-0.47985092493841\\
90	-0.505479969327501\\
91	-0.549819042462364\\
92	-0.239593849081199\\
93	0.0874172223289448\\
94	0.532479384081418\\
95	0.844947389137553\\
96	0.902990267312811\\
97	1.34941088450374\\
98	1.51906160108137\\
99	1.68086973673354\\
100	1.66782129851089\\
101	1.51783240546951\\
102	1.19201541742265\\
103	0.950794993831345\\
104	0.762447737946286\\
105	0.177461847960062\\
106	0.129928585033505\\
107	-0.179069137144045\\
108	-0.418583329198181\\
109	-0.373700709523821\\
110	-0.498560778147848\\
111	-0.458929634624885\\
112	-0.21993337184856\\
113	-0.674223583469616\\
114	-0.27411153998649\\
115	-0.246734745755761\\
116	0.178156844410317\\
117	0.661411957305089\\
118	0.927899716492082\\
119	1.13570598954763\\
120	1.49801298190921\\
121	1.35269601983521\\
122	1.3255971678798\\
123	1.38193893471208\\
124	1.21405715543073\\
125	1.28803620667721\\
126	0.86866737490331\\
127	0.637741599810967\\
128	0.060864054108107\\
129	0.0276625408989819\\
130	-0.575687587801954\\
131	-0.574514061677072\\
132	-0.754560816466395\\
133	-0.906686818870102\\
134	-0.872680982720572\\
135	-0.501476886918432\\
136	-0.19547867652258\\
137	0.129101103413968\\
138	0.145409658017746\\
139	0.491413422889821\\
140	0.800720871837492\\
141	1.04195024135594\\
142	1.1346096195571\\
143	1.40067655852227\\
144	1.45004382692768\\
145	1.83093845713989\\
146	1.64593954380315\\
147	1.53392856852834\\
148	1.192777214245\\
149	0.877142577937881\\
150	0.488166702960397\\
};
\end{axis}

\begin{axis}[%
width=4.272in,
height=1.075in,
at={(0.717in,0.346in)},
scale only axis,
xmin=0,
xmax=150,
xlabel style={font=\color{white!15!black}},
xlabel={k},
ymin=-1,
ymax=1,
ylabel style={font=\color{white!15!black}},
ylabel={Z(k)},
axis background/.style={fill=white}
]
\addplot[const plot, color=mycolor1, forget plot] table[row sep=crcr] {%
1	-0.121417529661296\\
2	-0.0606400402810642\\
3	-0.0273702157088698\\
4	0.117574314851635\\
5	-0.0594572028552305\\
6	0.0372033911049199\\
7	0.0905823580174678\\
8	0.0231662966956191\\
9	0.0084797972206992\\
10	-0.061054743367174\\
11	0.0841319599238164\\
12	0.162861729235296\\
13	0.0475538005392531\\
14	0.0307314205467728\\
15	0.0235369783552629\\
16	0.103403848881578\\
17	-0.0330897452682976\\
18	-0.0512192994488433\\
19	0.0531863818361121\\
20	-0.030648126656273\\
21	0.0891823691199314\\
22	-0.146386882738798\\
23	0.13263660386309\\
24	-0.0730619104668757\\
25	0.0238549415338402\\
26	-0.046787928599774\\
27	0.0713570614848185\\
28	-0.141749625288212\\
29	-0.106729373133174\\
30	0.163237969705383\\
31	-0.0975389008770207\\
32	-0.0698361575607913\\
33	0.00623171020060246\\
34	-0.0798704973000339\\
35	0.0146706668329981\\
36	0.117592606433444\\
37	0.00686839767411369\\
38	0.172618638117959\\
39	-0.144974016310585\\
40	-0.0284642940549723\\
41	-0.0874041553991488\\
42	0.0368500196188389\\
43	-0.126166848520152\\
44	-0.175959450516116\\
45	-0.0393533130365602\\
46	-0.181503754424699\\
47	-0.0898105103831431\\
48	-0.0547302197513294\\
49	-0.167039209868734\\
50	0.056391937485226\\
51	0.0305612360273877\\
52	-0.0627148707686647\\
53	-0.139233918383177\\
54	-0.0275295627881658\\
55	-0.0167850126274036\\
56	-0.217294717941297\\
57	0.130741754247125\\
58	0.04355296971779\\
59	-0.175354588812438\\
60	0.197807947328765\\
61	0.121754987971753\\
62	0.116368620826331\\
63	0.277483959150128\\
64	0.474366820911612\\
65	0.34777074542478\\
66	0.450349534139364\\
67	0.455350379625331\\
68	0.355113095284557\\
69	0.319602758377346\\
70	0.10671964988907\\
71	0.0179138712569102\\
72	-0.0762112446470521\\
73	-0.448237146507341\\
74	-0.531649681693246\\
75	-0.534084142515634\\
76	-0.474173174909664\\
77	-0.479696359912881\\
78	-0.552574290935876\\
79	-0.290946832965375\\
80	-0.219250328909321\\
81	-0.235126766014637\\
82	0.246940442708463\\
83	0.016222426634184\\
84	0.285414868534048\\
85	0.300753379987339\\
86	0.231009668909553\\
87	0.656983425981\\
88	0.392193792434199\\
89	0.497249450025052\\
90	0.635728545074285\\
91	0.42767840834596\\
92	0.292402506884106\\
93	0.113085450549022\\
94	0.0146875818797266\\
95	0.0424779720417796\\
96	-0.318106559282738\\
97	-0.398782328820436\\
98	-0.497929657518012\\
99	-0.534638832394489\\
100	-0.51391468100712\\
101	-0.399934491290318\\
102	-0.382750476386639\\
103	-0.367372890731005\\
104	-0.00350141736477\\
105	-0.120117919513325\\
106	0.0718743089561674\\
107	0.242963030005894\\
108	0.178755249078182\\
109	0.331488961790831\\
110	0.355302315093584\\
111	0.233352761053212\\
112	0.6898802812288\\
113	0.343805436164683\\
114	0.40914986784909\\
115	0.210239107887603\\
116	-0.0235046463679003\\
117	-0.0665599315824533\\
118	-0.155683424778113\\
119	-0.429037602326312\\
120	-0.289413181836716\\
121	-0.352633750338359\\
122	-0.508938742015331\\
123	-0.426378262200406\\
124	-0.560886529252951\\
125	-0.299394230793753\\
126	-0.259630247239521\\
127	0.0427559329067452\\
128	-0.0760459004112\\
129	0.357062255767803\\
130	0.282216703070868\\
131	0.42757353697973\\
132	0.601425555645584\\
133	0.605221767029687\\
134	0.407850617010756\\
135	0.366440287794889\\
136	0.293899172666902\\
137	0.39072804451053\\
138	0.14583907783331\\
139	-0.0211275106994021\\
140	-0.106259680735257\\
141	-0.140010442111225\\
142	-0.369877328249327\\
143	-0.406322508705426\\
144	-0.718531488052545\\
145	-0.566154179241892\\
146	-0.58017262195686\\
147	-0.475245760611751\\
148	-0.393808096228376\\
149	-0.241753210889353\\
150	-0.115308686109502\\
};
\end{axis}
\end{tikzpicture}%
\caption{Zakłócenie i sygnał sterujący- szum duży}
\end{figure}

\begin{figure}[H]
\centering
% This file was created by matlab2tikz.
%
%The latest updates can be retrieved from
%  http://www.mathworks.com/matlabcentral/fileexchange/22022-matlab2tikz-matlab2tikz
%where you can also make suggestions and rate matlab2tikz.
%
\definecolor{mycolor1}{rgb}{0.00000,0.44700,0.74100}%
\definecolor{mycolor2}{rgb}{0.85000,0.32500,0.09800}%
%
\begin{tikzpicture}

\begin{axis}[%
width=4.272in,
height=3.472in,
at={(0.717in,0.441in)},
scale only axis,
xmin=0,
xmax=150,
xlabel style={font=\color{white!15!black}},
xlabel={k},
ymin=-0.0188652048941365,
ymax=2,
ylabel style={font=\color{white!15!black}},
ylabel={Y(k)},
axis background/.style={fill=white},
legend style={legend cell align=left, align=left, draw=white!15!black}
]
\addplot[const plot, color=mycolor1] table[row sep=crcr] {%
1	0\\
2	0\\
3	0\\
4	0\\
5	0\\
6	0\\
7	0\\
8	0\\
9	0\\
10	0\\
11	0\\
12	-0.00269012937841067\\
13	-0.0188652048941365\\
14	-0.00364896949049241\\
15	0.0259950216949079\\
16	0.0292339883474295\\
17	0.028914739931801\\
18	0.0273696239361059\\
19	0.0427927064421145\\
20	0.0163192719849326\\
21	0.00443575333291835\\
22	0.0200779316044247\\
23	0.0228418013820127\\
24	0.0337830747466491\\
25	0.0146701811668533\\
26	0.0611216132013507\\
27	0.0458076128020086\\
28	0.0605049708813649\\
29	0.0335231816795162\\
30	0.0724418405869809\\
31	0.0194266451040148\\
32	0.00989772187753173\\
33	0.0387519067714376\\
34	0.0262838931977876\\
35	-0.000876927418755225\\
36	0.026481103733563\\
37	0.215247075506651\\
38	0.414171450541569\\
39	0.642464825973987\\
40	0.798550790244159\\
41	0.93733443406347\\
42	0.984882207799394\\
43	1.01487906850937\\
44	0.991393099730144\\
45	0.99594321736274\\
46	0.930697771142147\\
47	0.917104547941327\\
48	0.927519867669372\\
49	0.928763853988976\\
50	0.925330348132385\\
51	0.952625912636967\\
52	0.968122964089486\\
53	1.00953299703183\\
54	1.06123807155597\\
55	1.07120249318332\\
56	1.05392467143132\\
57	1.07482369435565\\
58	1.05508912519212\\
59	0.989723129286152\\
60	1.01143170386858\\
61	1.03643432782836\\
62	1.00649098420924\\
63	1.05207536711386\\
64	1.11162345165898\\
65	1.10733608908083\\
66	1.13668273217508\\
67	1.23521406071321\\
68	1.24138700764157\\
69	1.26833269426577\\
70	1.28341125425861\\
71	1.24721910103792\\
72	1.16059305225085\\
73	1.04652883762254\\
74	0.910408145320121\\
75	0.775198069039614\\
76	0.605035223664573\\
77	0.459871853098285\\
78	0.388977564395576\\
79	0.380729093576964\\
80	0.411662832373514\\
81	0.510184448847906\\
82	0.694985722006711\\
83	0.890522765296725\\
84	1.04118328450198\\
85	1.24966688827712\\
86	1.37983884343122\\
87	1.48647042349522\\
88	1.54159936538084\\
89	1.55253437705247\\
90	1.54350705407061\\
91	1.49280264418731\\
92	1.41359092816365\\
93	1.36647299004057\\
94	1.29722183950333\\
95	1.14023209178952\\
96	1.00227179805944\\
97	0.85009763904081\\
98	0.708187979818548\\
99	0.549365415317837\\
100	0.437339631245194\\
101	0.383332638576244\\
102	0.377349490556697\\
103	0.390241509087679\\
104	0.496296748423716\\
105	0.627745985763463\\
106	0.781026581169674\\
107	0.998935606611188\\
108	1.15618928846502\\
109	1.29419665173674\\
110	1.42038699512824\\
111	1.49419495498133\\
112	1.50503934142805\\
113	1.50877554345088\\
114	1.43793339465472\\
115	1.42582637221255\\
116	1.34425520986939\\
117	1.25982895470856\\
118	1.14322898192475\\
119	1.02125695073277\\
120	0.83278041258066\\
121	0.699000474808178\\
122	0.525029970634317\\
123	0.458049222719562\\
124	0.457246607064673\\
125	0.468125858867085\\
126	0.53082101509477\\
127	0.619807924637339\\
128	0.739499161262456\\
129	0.857442804872731\\
130	1.03856219069873\\
131	1.15754631557984\\
132	1.36689911439647\\
133	1.48117001372397\\
134	1.57976436711296\\
135	1.61711243799991\\
136	1.64160745739361\\
137	1.52974521766093\\
138	1.41327046035357\\
139	1.26015437373077\\
140	1.11194304241634\\
141	0.92593868651241\\
142	0.772849126926958\\
143	0.659322264645004\\
144	0.596782194479661\\
145	0.497019662520364\\
146	0.452184520493078\\
147	0.397693863302896\\
148	0.420689648765706\\
149	0.458490167522489\\
150	0.559447505419244\\
};
\addlegendentry{Y}

\addplot[const plot, color=mycolor2] table[row sep=crcr] {%
1	0\\
2	0\\
3	0\\
4	0\\
5	0\\
6	0\\
7	0\\
8	0\\
9	0\\
10	0\\
11	0\\
12	0\\
13	0\\
14	0\\
15	0\\
16	0\\
17	0\\
18	0\\
19	0\\
20	0\\
21	0\\
22	0\\
23	0\\
24	0\\
25	0\\
26	0\\
27	0\\
28	0\\
29	0\\
30	1\\
31	1\\
32	1\\
33	1\\
34	1\\
35	1\\
36	1\\
37	1\\
38	1\\
39	1\\
40	1\\
41	1\\
42	1\\
43	1\\
44	1\\
45	1\\
46	1\\
47	1\\
48	1\\
49	1\\
50	1\\
51	1\\
52	1\\
53	1\\
54	1\\
55	1\\
56	1\\
57	1\\
58	1\\
59	1\\
60	1\\
61	1\\
62	1\\
63	1\\
64	1\\
65	1\\
66	1\\
67	1\\
68	1\\
69	1\\
70	1\\
71	1\\
72	1\\
73	1\\
74	1\\
75	1\\
76	1\\
77	1\\
78	1\\
79	1\\
80	1\\
81	1\\
82	1\\
83	1\\
84	1\\
85	1\\
86	1\\
87	1\\
88	1\\
89	1\\
90	1\\
91	1\\
92	1\\
93	1\\
94	1\\
95	1\\
96	1\\
97	1\\
98	1\\
99	1\\
100	1\\
101	1\\
102	1\\
103	1\\
104	1\\
105	1\\
106	1\\
107	1\\
108	1\\
109	1\\
110	1\\
111	1\\
112	1\\
113	1\\
114	1\\
115	1\\
116	1\\
117	1\\
118	1\\
119	1\\
120	1\\
121	1\\
122	1\\
123	1\\
124	1\\
125	1\\
126	1\\
127	1\\
128	1\\
129	1\\
130	1\\
131	1\\
132	1\\
133	1\\
134	1\\
135	1\\
136	1\\
137	1\\
138	1\\
139	1\\
140	1\\
141	1\\
142	1\\
143	1\\
144	1\\
145	1\\
146	1\\
147	1\\
148	1\\
149	1\\
150	1\\
};
\addlegendentry{Yzad}

\end{axis}
\end{tikzpicture}%
\caption{Wyjście dla pomiaru z szumem dużym (błąd $E=21,8056$)}
\end{figure}

\section{Wnioski}
Występowanie szumu pomiarowego pogarsza jakość regulacji.
Szum jest generowany w sposób losowy, dlatego też zwiększanie wartości błędów szumu pomiarowego, nie zawsze będzie skutkowało pogorszeniem regulacji. Widać to na przykładnie zakłócenia sinusoidalnego: nałożenie dużego szumu skutkowało mniejszym błędem niż w przypadku małego i średniego szumu.
\smallskip


%! TEX encoding = utf8
\chapter{Laboratorium}

\section{Określenie wartości pomiaru temperatury w punkcie pracy}

W celu określenia wartości pomiaru temperatury w punkcie pracy ustawiono moc wentylatora  $W1 = 50\%$,a moc grzałki $G1 = 25\%$.
Po czasie około 5 minut temperatura odczytywana przez czujnik temperatury zaczeła się stabilizować  na poziomie  $T1 = 28,4^{\circ} C$. 
Niestety z powodu ciągłego ruchu powietrza związanego z przemieszczaniem się osób w sali i dużej ilości tych osób wpływających na temperaturę sali oraz czułość stanowiska pomiarowego temperatura odczytywana przez czujnik zaczeła odbiegać i lekko oscylować od tej temperatury.

\begin{figure}[H]
\centering
% This file was created by matlab2tikz.
%
%The latest updates can be retrieved from
%  http://www.mathworks.com/matlabcentral/fileexchange/22022-matlab2tikz-matlab2tikz
%where you can also make suggestions and rate matlab2tikz.
%
\definecolor{mycolor1}{rgb}{0.00000,0.44700,0.74100}%
%
\begin{tikzpicture}

\begin{axis}[%
width=4.521in,
height=3.566in,
at={(0.758in,0.481in)},
scale only axis,
xmin=0,
xmax=400,
xlabel style={font=\color{white!15!black}},
xlabel={k},
ymin=21,
ymax=29,
ylabel style={font=\color{white!15!black}},
ylabel={$\text{T[}^\circ\text{C]}$},
axis background/.style={fill=white}
]
\addplot[const plot, color=mycolor1, forget plot] table[row sep=crcr] {%
1	21.12\\
2	21.12\\
3	21.12\\
4	21.06\\
5	21.06\\
6	21.06\\
7	21.06\\
8	21.12\\
9	21.12\\
10	21.12\\
11	21.12\\
12	21.18\\
13	21.18\\
14	21.18\\
15	21.25\\
16	21.31\\
17	21.31\\
18	21.37\\
19	21.43\\
20	21.56\\
21	21.56\\
22	21.62\\
23	21.68\\
24	21.75\\
25	21.87\\
26	21.93\\
27	22\\
28	22.06\\
29	22.12\\
30	22.18\\
31	22.31\\
32	22.37\\
33	22.43\\
34	22.5\\
35	22.62\\
36	22.68\\
37	22.75\\
38	22.81\\
39	22.93\\
40	23\\
41	23.06\\
42	23.12\\
43	23.18\\
44	23.25\\
45	23.31\\
46	23.43\\
47	23.5\\
48	23.56\\
49	23.62\\
50	23.68\\
51	23.75\\
52	23.81\\
53	23.87\\
54	23.93\\
55	24\\
56	24.06\\
57	24.12\\
58	24.18\\
59	24.25\\
60	24.31\\
61	24.37\\
62	24.43\\
63	24.43\\
64	24.5\\
65	24.56\\
66	24.62\\
67	24.68\\
68	24.75\\
69	24.75\\
70	24.81\\
71	24.87\\
72	24.93\\
73	25\\
74	25\\
75	25.06\\
76	25.06\\
77	25.12\\
78	25.18\\
79	25.25\\
80	25.25\\
81	25.31\\
82	25.31\\
83	25.37\\
84	25.43\\
85	25.43\\
86	25.5\\
87	25.56\\
88	25.56\\
89	25.62\\
90	25.68\\
91	25.68\\
92	25.75\\
93	25.75\\
94	25.81\\
95	25.81\\
96	25.81\\
97	25.87\\
98	25.87\\
99	25.93\\
100	25.93\\
101	26\\
102	26\\
103	26.06\\
104	26.06\\
105	26.12\\
106	26.12\\
107	26.18\\
108	26.18\\
109	26.18\\
110	26.25\\
111	26.25\\
112	26.31\\
113	26.31\\
114	26.31\\
115	26.37\\
116	26.37\\
117	26.37\\
118	26.37\\
119	26.43\\
120	26.43\\
121	26.43\\
122	26.5\\
123	26.5\\
124	26.5\\
125	26.5\\
126	26.5\\
127	26.56\\
128	26.56\\
129	26.62\\
130	26.62\\
131	26.68\\
132	26.68\\
133	26.68\\
134	26.75\\
135	26.75\\
136	26.81\\
137	26.81\\
138	26.81\\
139	26.81\\
140	26.87\\
141	26.93\\
142	26.93\\
143	26.93\\
144	26.93\\
145	26.93\\
146	27\\
147	27\\
148	27.06\\
149	27.06\\
150	27.06\\
151	27.12\\
152	27.12\\
153	27.18\\
154	27.18\\
155	27.18\\
156	27.18\\
157	27.25\\
158	27.25\\
159	27.25\\
160	27.31\\
161	27.31\\
162	27.31\\
163	27.37\\
164	27.37\\
165	27.37\\
166	27.37\\
167	27.37\\
168	27.43\\
169	27.43\\
170	27.43\\
171	27.5\\
172	27.5\\
173	27.5\\
174	27.5\\
175	27.5\\
176	27.56\\
177	27.56\\
178	27.56\\
179	27.56\\
180	27.56\\
181	27.56\\
182	27.62\\
183	27.62\\
184	27.62\\
185	27.62\\
186	27.62\\
187	27.62\\
188	27.62\\
189	27.62\\
190	27.62\\
191	27.68\\
192	27.68\\
193	27.68\\
194	27.68\\
195	27.68\\
196	27.68\\
197	27.75\\
198	27.68\\
199	27.68\\
200	27.68\\
201	27.75\\
202	27.75\\
203	27.75\\
204	27.75\\
205	27.75\\
206	27.75\\
207	27.75\\
208	27.75\\
209	27.81\\
210	27.81\\
211	27.81\\
212	27.81\\
213	27.81\\
214	27.81\\
215	27.87\\
216	27.87\\
217	27.87\\
218	27.87\\
219	27.87\\
220	27.87\\
221	27.87\\
222	27.87\\
223	27.93\\
224	27.93\\
225	27.93\\
226	28\\
227	28\\
228	28\\
229	28\\
230	28.06\\
231	28.06\\
232	28.06\\
233	28.12\\
234	28.12\\
235	28.12\\
236	28.12\\
237	28.12\\
238	28.12\\
239	28.12\\
240	28.12\\
241	28.06\\
242	28.12\\
243	28.12\\
244	28.06\\
245	28.12\\
246	28.06\\
247	28.06\\
248	28.06\\
249	28.06\\
250	28.06\\
251	28.06\\
252	28.06\\
253	28.06\\
254	28.06\\
255	28.06\\
256	28.06\\
257	28.06\\
258	28.06\\
259	28.12\\
260	28.12\\
261	28.12\\
262	28.18\\
263	28.18\\
264	28.18\\
265	28.18\\
266	28.25\\
267	28.25\\
268	28.25\\
269	28.25\\
270	28.25\\
271	28.25\\
272	28.25\\
273	28.25\\
274	28.25\\
275	28.25\\
276	28.25\\
277	28.25\\
278	28.25\\
279	28.25\\
280	28.25\\
281	28.25\\
282	28.25\\
283	28.25\\
284	28.25\\
285	28.25\\
286	28.25\\
287	28.25\\
288	28.25\\
289	28.25\\
290	28.25\\
291	28.25\\
292	28.31\\
293	28.31\\
294	28.31\\
295	28.31\\
296	28.31\\
297	28.31\\
298	28.37\\
299	28.37\\
300	28.37\\
301	28.31\\
302	28.37\\
303	28.31\\
304	28.31\\
305	28.31\\
306	28.31\\
307	28.37\\
308	28.37\\
309	28.37\\
310	28.37\\
311	28.43\\
312	28.43\\
313	28.43\\
314	28.43\\
315	28.43\\
316	28.43\\
317	28.43\\
318	28.43\\
319	28.43\\
320	28.43\\
321	28.43\\
322	28.43\\
323	28.43\\
324	28.43\\
325	28.43\\
326	28.43\\
327	28.43\\
328	28.43\\
329	28.43\\
330	28.43\\
331	28.43\\
332	28.43\\
333	28.37\\
334	28.37\\
335	28.43\\
336	28.43\\
337	28.37\\
338	28.43\\
339	28.43\\
340	28.43\\
341	28.43\\
342	28.43\\
343	28.43\\
344	28.5\\
345	28.43\\
346	28.43\\
347	28.43\\
348	28.5\\
349	28.5\\
350	28.5\\
351	28.5\\
352	28.5\\
353	28.5\\
354	28.5\\
355	28.5\\
356	28.5\\
357	28.5\\
358	28.5\\
359	28.5\\
360	28.5\\
361	28.5\\
362	28.5\\
363	28.56\\
364	28.5\\
365	28.56\\
366	28.56\\
367	28.56\\
368	28.5\\
369	28.5\\
370	28.5\\
371	28.5\\
372	28.5\\
373	28.5\\
374	28.5\\
375	28.5\\
376	28.5\\
377	28.5\\
378	28.56\\
379	28.56\\
380	28.56\\
381	28.62\\
382	28.62\\
383	28.62\\
384	28.68\\
385	28.68\\
386	28.62\\
387	28.68\\
388	28.68\\
389	28.68\\
390	28.68\\
391	28.68\\
392	28.68\\
393	28.68\\
394	28.68\\
395	28.68\\
396	28.68\\
397	28.68\\
398	28.68\\
399	28.68\\
400	28.68\\
};
\end{axis}
\end{tikzpicture}%
\caption{Pomiar temperatury w punkcie pracy}
\end{figure}

\section{Wyznaczenie odpowiedzi skokowych}

Rozpoczynając z punktu pracy wyznaczono odpowiedzi skokowe dla trzech różnych wartości sygnału sterującego  $G1 = 35\%$  $G1 = 45\%$ i $G1 = 55\%$.

\begin{figure}[H]
\centering
% This file was created by matlab2tikz.
%
%The latest updates can be retrieved from
%  http://www.mathworks.com/matlabcentral/fileexchange/22022-matlab2tikz-matlab2tikz
%where you can also make suggestions and rate matlab2tikz.
%
\definecolor{mycolor1}{rgb}{0.00000,0.44700,0.74100}%
\definecolor{mycolor2}{rgb}{0.85000,0.32500,0.09800}%
\definecolor{mycolor3}{rgb}{0.92900,0.69400,0.12500}%
%
\begin{tikzpicture}

\begin{axis}[%
width=4.521in,
height=3.566in,
at={(0.758in,0.481in)},
scale only axis,
xmin=0,
xmax=350,
xlabel style={font=\color{white!15!black}},
xlabel={k},
ymin=28,
ymax=40,
ylabel style={font=\color{white!15!black}},
ylabel={$\text{T[}^\circ\text{C]}$},
axis background/.style={fill=white}
]
\addplot[const plot, color=mycolor1, forget plot] table[row sep=crcr] {%
1	28.5\\
2	28.5\\
3	28.56\\
4	28.56\\
5	28.56\\
6	28.5\\
7	28.5\\
8	28.5\\
9	28.5\\
10	28.5\\
11	28.5\\
12	28.5\\
13	28.5\\
14	28.5\\
15	28.5\\
16	28.5\\
17	28.43\\
18	28.43\\
19	28.43\\
20	28.43\\
21	28.43\\
22	28.43\\
23	28.43\\
24	28.43\\
25	28.43\\
26	28.43\\
27	28.43\\
28	28.43\\
29	28.5\\
30	28.5\\
31	28.56\\
32	28.56\\
33	28.56\\
34	28.62\\
35	28.62\\
36	28.62\\
37	28.68\\
38	28.68\\
39	28.68\\
40	28.68\\
41	28.75\\
42	28.75\\
43	28.75\\
44	28.81\\
45	28.87\\
46	28.87\\
47	28.87\\
48	28.93\\
49	28.93\\
50	29\\
51	29\\
52	29.06\\
53	29.06\\
54	29.12\\
55	29.18\\
56	29.25\\
57	29.25\\
58	29.25\\
59	29.31\\
60	29.37\\
61	29.37\\
62	29.37\\
63	29.43\\
64	29.5\\
65	29.5\\
66	29.56\\
67	29.62\\
68	29.62\\
69	29.68\\
70	29.68\\
71	29.68\\
72	29.75\\
73	29.75\\
74	29.81\\
75	29.81\\
76	29.81\\
77	29.87\\
78	29.87\\
79	29.93\\
80	29.93\\
81	29.93\\
82	29.93\\
83	30\\
84	30\\
85	30\\
86	30.06\\
87	30.06\\
88	30.06\\
89	30.12\\
90	30.12\\
91	30.12\\
92	30.18\\
93	30.18\\
94	30.18\\
95	30.18\\
96	30.18\\
97	30.25\\
98	30.25\\
99	30.25\\
100	30.25\\
101	30.31\\
102	30.31\\
103	30.31\\
104	30.37\\
105	30.37\\
106	30.37\\
107	30.37\\
108	30.37\\
109	30.43\\
110	30.43\\
111	30.43\\
112	30.5\\
113	30.5\\
114	30.5\\
115	30.5\\
116	30.56\\
117	30.56\\
118	30.62\\
119	30.62\\
120	30.62\\
121	30.68\\
122	30.68\\
123	30.68\\
124	30.68\\
125	30.68\\
126	30.68\\
127	30.68\\
128	30.75\\
129	30.75\\
130	30.75\\
131	30.75\\
132	30.75\\
133	30.75\\
134	30.75\\
135	30.81\\
136	30.81\\
137	30.81\\
138	30.87\\
139	30.87\\
140	30.87\\
141	30.87\\
142	30.93\\
143	30.93\\
144	30.93\\
145	31\\
146	31\\
147	31\\
148	31\\
149	31.06\\
150	31.06\\
151	31.06\\
152	31.06\\
153	31.06\\
154	31.06\\
155	31.12\\
156	31.12\\
157	31.12\\
158	31.12\\
159	31.12\\
160	31.12\\
161	31.12\\
162	31.12\\
163	31.12\\
164	31.12\\
165	31.18\\
166	31.18\\
167	31.18\\
168	31.18\\
169	31.18\\
170	31.18\\
171	31.18\\
172	31.18\\
173	31.18\\
174	31.18\\
175	31.18\\
176	31.18\\
177	31.25\\
178	31.25\\
179	31.25\\
180	31.25\\
181	31.25\\
182	31.25\\
183	31.25\\
184	31.25\\
185	31.25\\
186	31.25\\
187	31.25\\
188	31.25\\
189	31.18\\
190	31.18\\
191	31.18\\
192	31.25\\
193	31.18\\
194	31.25\\
195	31.25\\
196	31.25\\
197	31.25\\
198	31.25\\
199	31.25\\
200	31.25\\
201	31.25\\
202	31.25\\
203	31.25\\
204	31.25\\
205	31.25\\
206	31.25\\
207	31.25\\
208	31.25\\
209	31.25\\
210	31.25\\
211	31.25\\
212	31.31\\
213	31.31\\
214	31.31\\
215	31.31\\
216	31.31\\
217	31.37\\
218	31.37\\
219	31.37\\
220	31.37\\
221	31.37\\
222	31.43\\
223	31.43\\
224	31.43\\
225	31.43\\
226	31.37\\
227	31.37\\
228	31.37\\
229	31.37\\
230	31.37\\
231	31.37\\
232	31.37\\
233	31.37\\
234	31.37\\
235	31.37\\
236	31.37\\
237	31.37\\
238	31.37\\
239	31.43\\
240	31.43\\
241	31.43\\
242	31.43\\
243	31.43\\
244	31.43\\
245	31.43\\
246	31.43\\
247	31.43\\
248	31.43\\
249	31.43\\
250	31.43\\
251	31.43\\
252	31.43\\
253	31.43\\
254	31.43\\
255	31.5\\
256	31.43\\
257	31.43\\
258	31.5\\
259	31.5\\
260	31.5\\
261	31.5\\
262	31.5\\
263	31.5\\
264	31.5\\
265	31.5\\
266	31.43\\
267	31.5\\
268	31.5\\
269	31.5\\
270	31.43\\
271	31.5\\
272	31.43\\
273	31.43\\
274	31.43\\
275	31.43\\
276	31.43\\
277	31.43\\
278	31.43\\
279	31.43\\
280	31.43\\
281	31.43\\
282	31.43\\
283	31.43\\
284	31.43\\
285	31.43\\
286	31.43\\
287	31.43\\
288	31.43\\
289	31.37\\
290	31.37\\
291	31.43\\
292	31.43\\
293	31.43\\
294	31.43\\
295	31.43\\
296	31.43\\
297	31.43\\
298	31.43\\
299	31.43\\
300	31.43\\
301	31.43\\
302	31.43\\
303	31.43\\
304	31.43\\
305	31.43\\
306	31.43\\
307	31.43\\
308	31.5\\
309	31.5\\
310	31.5\\
311	31.5\\
312	31.56\\
313	31.56\\
314	31.56\\
315	31.56\\
316	31.56\\
317	31.56\\
318	31.56\\
319	31.56\\
320	31.56\\
321	31.56\\
322	31.62\\
323	31.56\\
324	31.56\\
325	31.56\\
326	31.56\\
327	31.56\\
328	31.56\\
329	31.56\\
330	31.56\\
331	31.56\\
332	31.56\\
333	31.56\\
334	31.56\\
335	31.56\\
336	31.56\\
337	31.56\\
338	31.56\\
339	31.56\\
340	31.56\\
341	31.62\\
342	31.62\\
343	31.62\\
344	31.62\\
345	31.62\\
346	31.62\\
347	31.62\\
348	31.62\\
349	31.62\\
350	31.56\\
};
\addplot[const plot, color=mycolor2, forget plot] table[row sep=crcr] {%
1	28.56\\
2	28.56\\
3	28.59\\
4	28.59\\
5	28.59\\
6	28.56\\
7	28.59\\
8	28.59\\
9	28.59\\
10	28.59\\
11	28.59\\
12	28.59\\
13	28.59\\
14	28.59\\
15	28.59\\
16	28.59\\
17	28.555\\
18	28.555\\
19	28.555\\
20	28.555\\
21	28.555\\
22	28.59\\
23	28.59\\
24	28.59\\
25	28.59\\
26	28.62\\
27	28.62\\
28	28.62\\
29	28.655\\
30	28.685\\
31	28.715\\
32	28.745\\
33	28.78\\
34	28.81\\
35	28.84\\
36	28.87\\
37	28.93\\
38	28.965\\
39	28.995\\
40	29.055\\
41	29.125\\
42	29.185\\
43	29.215\\
44	29.31\\
45	29.37\\
46	29.435\\
47	29.465\\
48	29.555\\
49	29.59\\
50	29.685\\
51	29.715\\
52	29.81\\
53	29.84\\
54	29.935\\
55	29.995\\
56	30.09\\
57	30.125\\
58	30.185\\
59	30.245\\
60	30.34\\
61	30.37\\
62	30.435\\
63	30.525\\
64	30.59\\
65	30.625\\
66	30.715\\
67	30.775\\
68	30.84\\
69	30.9\\
70	30.965\\
71	30.995\\
72	31.06\\
73	31.125\\
74	31.185\\
75	31.215\\
76	31.245\\
77	31.31\\
78	31.34\\
79	31.4\\
80	31.43\\
81	31.465\\
82	31.495\\
83	31.59\\
84	31.625\\
85	31.655\\
86	31.715\\
87	31.745\\
88	31.78\\
89	31.84\\
90	31.87\\
91	31.9\\
92	31.965\\
93	31.995\\
94	32.025\\
95	32.055\\
96	32.09\\
97	32.155\\
98	32.185\\
99	32.215\\
100	32.25\\
101	32.28\\
102	32.34\\
103	32.34\\
104	32.4\\
105	32.435\\
106	32.465\\
107	32.525\\
108	32.525\\
109	32.59\\
110	32.62\\
111	32.62\\
112	32.715\\
113	32.715\\
114	32.75\\
115	32.75\\
116	32.81\\
117	32.84\\
118	32.87\\
119	32.87\\
120	32.9\\
121	32.965\\
122	32.965\\
123	32.965\\
124	32.965\\
125	32.995\\
126	32.995\\
127	33.025\\
128	33.06\\
129	33.09\\
130	33.125\\
131	33.125\\
132	33.155\\
133	33.185\\
134	33.185\\
135	33.245\\
136	33.28\\
137	33.28\\
138	33.34\\
139	33.34\\
140	33.37\\
141	33.37\\
142	33.4\\
143	33.43\\
144	33.43\\
145	33.465\\
146	33.465\\
147	33.5\\
148	33.5\\
149	33.53\\
150	33.53\\
151	33.56\\
152	33.56\\
153	33.56\\
154	33.59\\
155	33.65\\
156	33.65\\
157	33.65\\
158	33.65\\
159	33.685\\
160	33.685\\
161	33.685\\
162	33.715\\
163	33.715\\
164	33.715\\
165	33.775\\
166	33.775\\
167	33.805\\
168	33.805\\
169	33.84\\
170	33.87\\
171	33.87\\
172	33.9\\
173	33.93\\
174	33.93\\
175	33.93\\
176	33.965\\
177	34\\
178	34\\
179	34\\
180	34\\
181	34\\
182	34\\
183	34\\
184	34\\
185	34\\
186	34.03\\
187	34.06\\
188	34.06\\
189	34.055\\
190	34.09\\
191	34.09\\
192	34.125\\
193	34.12\\
194	34.155\\
195	34.125\\
196	34.125\\
197	34.125\\
198	34.125\\
199	34.125\\
200	34.125\\
201	34.125\\
202	34.155\\
203	34.155\\
204	34.155\\
205	34.185\\
206	34.215\\
207	34.215\\
208	34.215\\
209	34.25\\
210	34.25\\
211	34.28\\
212	34.31\\
213	34.31\\
214	34.34\\
215	34.34\\
216	34.37\\
217	34.4\\
218	34.435\\
219	34.435\\
220	34.465\\
221	34.465\\
222	34.495\\
223	34.495\\
224	34.495\\
225	34.495\\
226	34.465\\
227	34.465\\
228	34.495\\
229	34.495\\
230	34.495\\
231	34.525\\
232	34.525\\
233	34.525\\
234	34.525\\
235	34.525\\
236	34.525\\
237	34.525\\
238	34.525\\
239	34.555\\
240	34.555\\
241	34.555\\
242	34.555\\
243	34.555\\
244	34.555\\
245	34.555\\
246	34.525\\
247	34.555\\
248	34.555\\
249	34.525\\
250	34.555\\
251	34.555\\
252	34.555\\
253	34.59\\
254	34.62\\
255	34.655\\
256	34.65\\
257	34.68\\
258	34.715\\
259	34.75\\
260	34.75\\
261	34.78\\
262	34.78\\
263	34.78\\
264	34.78\\
265	34.78\\
266	34.745\\
267	34.78\\
268	34.78\\
269	34.78\\
270	34.775\\
271	34.81\\
272	34.775\\
273	34.775\\
274	34.775\\
275	34.775\\
276	34.775\\
277	34.775\\
278	34.775\\
279	34.805\\
280	34.805\\
281	34.84\\
282	34.84\\
283	34.84\\
284	34.84\\
285	34.84\\
286	34.84\\
287	34.84\\
288	34.805\\
289	34.775\\
290	34.775\\
291	34.775\\
292	34.775\\
293	34.775\\
294	34.745\\
295	34.745\\
296	34.745\\
297	34.745\\
298	34.745\\
299	34.745\\
300	34.775\\
301	34.745\\
302	34.745\\
303	34.745\\
304	34.745\\
305	34.775\\
306	34.775\\
307	34.805\\
308	34.84\\
309	34.84\\
310	34.84\\
311	34.84\\
312	34.87\\
313	34.87\\
314	34.87\\
315	34.905\\
316	34.905\\
317	34.905\\
318	34.935\\
319	34.905\\
320	34.935\\
321	34.935\\
322	34.965\\
323	34.935\\
324	34.965\\
325	34.965\\
326	34.995\\
327	34.995\\
328	34.995\\
329	35.03\\
330	35.03\\
331	35.03\\
332	35.06\\
333	35.06\\
334	35.06\\
335	35.03\\
336	35.03\\
337	35.06\\
338	35.03\\
339	35.06\\
340	35.06\\
341	35.09\\
342	35.06\\
343	35.06\\
344	35.06\\
345	35.025\\
346	35.025\\
347	34.995\\
348	34.995\\
349	34.995\\
350	34.935\\
};
\addplot[const plot, color=mycolor3, forget plot] table[row sep=crcr] {%
1	28.62\\
2	28.62\\
3	28.62\\
4	28.62\\
5	28.62\\
6	28.62\\
7	28.68\\
8	28.68\\
9	28.68\\
10	28.68\\
11	28.68\\
12	28.68\\
13	28.68\\
14	28.68\\
15	28.68\\
16	28.68\\
17	28.68\\
18	28.68\\
19	28.68\\
20	28.68\\
21	28.68\\
22	28.75\\
23	28.75\\
24	28.75\\
25	28.75\\
26	28.81\\
27	28.81\\
28	28.81\\
29	28.81\\
30	28.87\\
31	28.87\\
32	28.93\\
33	29\\
34	29\\
35	29.06\\
36	29.12\\
37	29.18\\
38	29.25\\
39	29.31\\
40	29.43\\
41	29.5\\
42	29.62\\
43	29.68\\
44	29.81\\
45	29.87\\
46	30\\
47	30.06\\
48	30.18\\
49	30.25\\
50	30.37\\
51	30.43\\
52	30.56\\
53	30.62\\
54	30.75\\
55	30.81\\
56	30.93\\
57	31\\
58	31.12\\
59	31.18\\
60	31.31\\
61	31.37\\
62	31.5\\
63	31.62\\
64	31.68\\
65	31.75\\
66	31.87\\
67	31.93\\
68	32.06\\
69	32.12\\
70	32.25\\
71	32.31\\
72	32.37\\
73	32.5\\
74	32.56\\
75	32.62\\
76	32.68\\
77	32.75\\
78	32.81\\
79	32.87\\
80	32.93\\
81	33\\
82	33.06\\
83	33.18\\
84	33.25\\
85	33.31\\
86	33.37\\
87	33.43\\
88	33.5\\
89	33.56\\
90	33.62\\
91	33.68\\
92	33.75\\
93	33.81\\
94	33.87\\
95	33.93\\
96	34\\
97	34.06\\
98	34.12\\
99	34.18\\
100	34.25\\
101	34.25\\
102	34.37\\
103	34.37\\
104	34.43\\
105	34.5\\
106	34.56\\
107	34.68\\
108	34.68\\
109	34.75\\
110	34.81\\
111	34.81\\
112	34.93\\
113	34.93\\
114	35\\
115	35\\
116	35.06\\
117	35.12\\
118	35.12\\
119	35.12\\
120	35.18\\
121	35.25\\
122	35.25\\
123	35.25\\
124	35.25\\
125	35.31\\
126	35.31\\
127	35.37\\
128	35.37\\
129	35.43\\
130	35.5\\
131	35.5\\
132	35.56\\
133	35.62\\
134	35.62\\
135	35.68\\
136	35.75\\
137	35.75\\
138	35.81\\
139	35.81\\
140	35.87\\
141	35.87\\
142	35.87\\
143	35.93\\
144	35.93\\
145	35.93\\
146	35.93\\
147	36\\
148	36\\
149	36\\
150	36\\
151	36.06\\
152	36.06\\
153	36.06\\
154	36.12\\
155	36.18\\
156	36.18\\
157	36.18\\
158	36.18\\
159	36.25\\
160	36.25\\
161	36.25\\
162	36.31\\
163	36.31\\
164	36.31\\
165	36.37\\
166	36.37\\
167	36.43\\
168	36.43\\
169	36.5\\
170	36.56\\
171	36.56\\
172	36.62\\
173	36.68\\
174	36.68\\
175	36.68\\
176	36.75\\
177	36.75\\
178	36.75\\
179	36.75\\
180	36.75\\
181	36.75\\
182	36.75\\
183	36.75\\
184	36.75\\
185	36.75\\
186	36.81\\
187	36.87\\
188	36.87\\
189	36.93\\
190	37\\
191	37\\
192	37\\
193	37.06\\
194	37.06\\
195	37\\
196	37\\
197	37\\
198	37\\
199	37\\
200	37\\
201	37\\
202	37.06\\
203	37.06\\
204	37.06\\
205	37.12\\
206	37.18\\
207	37.18\\
208	37.18\\
209	37.25\\
210	37.25\\
211	37.31\\
212	37.31\\
213	37.31\\
214	37.37\\
215	37.37\\
216	37.43\\
217	37.43\\
218	37.5\\
219	37.5\\
220	37.56\\
221	37.56\\
222	37.56\\
223	37.56\\
224	37.56\\
225	37.56\\
226	37.56\\
227	37.56\\
228	37.62\\
229	37.62\\
230	37.62\\
231	37.68\\
232	37.68\\
233	37.68\\
234	37.68\\
235	37.68\\
236	37.68\\
237	37.68\\
238	37.68\\
239	37.68\\
240	37.68\\
241	37.68\\
242	37.68\\
243	37.68\\
244	37.68\\
245	37.68\\
246	37.62\\
247	37.68\\
248	37.68\\
249	37.62\\
250	37.68\\
251	37.68\\
252	37.68\\
253	37.75\\
254	37.81\\
255	37.81\\
256	37.87\\
257	37.93\\
258	37.93\\
259	38\\
260	38\\
261	38.06\\
262	38.06\\
263	38.06\\
264	38.06\\
265	38.06\\
266	38.06\\
267	38.06\\
268	38.06\\
269	38.06\\
270	38.12\\
271	38.12\\
272	38.12\\
273	38.12\\
274	38.12\\
275	38.12\\
276	38.12\\
277	38.12\\
278	38.12\\
279	38.18\\
280	38.18\\
281	38.25\\
282	38.25\\
283	38.25\\
284	38.25\\
285	38.25\\
286	38.25\\
287	38.25\\
288	38.18\\
289	38.18\\
290	38.18\\
291	38.12\\
292	38.12\\
293	38.12\\
294	38.06\\
295	38.06\\
296	38.06\\
297	38.06\\
298	38.06\\
299	38.06\\
300	38.12\\
301	38.06\\
302	38.06\\
303	38.06\\
304	38.06\\
305	38.12\\
306	38.12\\
307	38.18\\
308	38.18\\
309	38.18\\
310	38.18\\
311	38.18\\
312	38.18\\
313	38.18\\
314	38.18\\
315	38.25\\
316	38.25\\
317	38.25\\
318	38.31\\
319	38.25\\
320	38.31\\
321	38.31\\
322	38.31\\
323	38.31\\
324	38.37\\
325	38.37\\
326	38.43\\
327	38.43\\
328	38.43\\
329	38.5\\
330	38.5\\
331	38.5\\
332	38.56\\
333	38.56\\
334	38.56\\
335	38.5\\
336	38.5\\
337	38.56\\
338	38.5\\
339	38.56\\
340	38.56\\
341	38.56\\
342	38.5\\
343	38.5\\
344	38.5\\
345	38.43\\
346	38.43\\
347	38.37\\
348	38.37\\
349	38.37\\
350	38.31\\
};
\end{axis}
\end{tikzpicture}%
\caption{Odpowiedzi skokowe dla trzech różnych wartości sygnału sterującego}
\end{figure}

Analizując otrzymane wykresy można wywnioskować że właściwości statyczne procesu są w przybliżeniu liniowe, zmiany wartości odpowiedzi skokowej dla tych samych chwil są w przybliżeniu proporcjonalne jak również sam kształt wykresów jest w przybliżeniu podobny. W konsekwencji postanowiono wyznaczyć wzmocnienie statyczne procesu.

\begin{equation}
K_s_t_a_t = 0.3303
\end{equation}


\begin{figure}[H]
\centering
% This file was created by matlab2tikz.
%
%The latest updates can be retrieved from
%  http://www.mathworks.com/matlabcentral/fileexchange/22022-matlab2tikz-matlab2tikz
%where you can also make suggestions and rate matlab2tikz.
%
\definecolor{mycolor1}{rgb}{0.00000,0.44700,0.74100}%
%
\begin{tikzpicture}

\begin{axis}[%
width=4.521in,
height=3.566in,
at={(0.758in,0.481in)},
scale only axis,
xmin=25,
xmax=55,
xlabel style={font=\color{white!15!black}},
xlabel={u},
ymin=28,
ymax=40,
ylabel style={font=\color{white!15!black}},
ylabel={y},
axis background/.style={fill=white}
]
\addplot [color=mycolor1, forget plot]
  table[row sep=crcr]{%
25	28.4\\
35	31.56\\
45	34.935\\
55	38.31\\
};
\end{axis}

\begin{axis}[%
width=5.833in,
height=4.375in,
at={(0in,0in)},
scale only axis,
xmin=0,
xmax=1,
ymin=0,
ymax=1,
axis line style={draw=none},
ticks=none,
axis x line*=bottom,
axis y line*=left
]
\end{axis}
\end{tikzpicture}%
\caption{Charakterystyka statyczna procesu}
\end{figure}

\section{Aproksymacja odpowiedzi skokowej}

Dokonano aproksymacji odpowiedzi skokowej dla wartości sygnału starującego $G1 = 35\%$.

\begin{figure}[H]
\centering
% This file was created by matlab2tikz.
%
%The latest updates can be retrieved from
%  http://www.mathworks.com/matlabcentral/fileexchange/22022-matlab2tikz-matlab2tikz
%where you can also make suggestions and rate matlab2tikz.
%
\definecolor{mycolor1}{rgb}{0.00000,0.44700,0.74100}%
\definecolor{mycolor2}{rgb}{0.85000,0.32500,0.09800}%
%
\begin{tikzpicture}

\begin{axis}[%
width=4.521in,
height=3.566in,
at={(0.758in,0.481in)},
scale only axis,
xmin=0,
xmax=350,
xlabel style={font=\color{white!15!black}},
xlabel={k},
ymin=-0.05,
ymax=0.35,
ylabel style={font=\color{white!15!black}},
ylabel={s},
axis background/.style={fill=white},
legend style={at={(0.03,0.97)}, anchor=north west, legend cell align=left, align=left, draw=white!15!black}
]
\addplot[const plot, color=mycolor1] table[row sep=crcr] {%
1	0\\
2	0\\
3	0.00599999999999987\\
4	0.00599999999999987\\
5	0.00599999999999987\\
6	0\\
7	0\\
8	0\\
9	0\\
10	0\\
11	0\\
12	0\\
13	0\\
14	0\\
15	0\\
16	0\\
17	-0.00700000000000003\\
18	-0.00700000000000003\\
19	-0.00700000000000003\\
20	-0.00700000000000003\\
21	-0.00700000000000003\\
22	-0.00700000000000003\\
23	-0.00700000000000003\\
24	-0.00700000000000003\\
25	-0.00700000000000003\\
26	-0.00700000000000003\\
27	-0.00700000000000003\\
28	-0.00700000000000003\\
29	0\\
30	0\\
31	0.00599999999999987\\
32	0.00599999999999987\\
33	0.00599999999999987\\
34	0.0120000000000001\\
35	0.0120000000000001\\
36	0.0120000000000001\\
37	0.018\\
38	0.018\\
39	0.018\\
40	0.018\\
41	0.025\\
42	0.025\\
43	0.025\\
44	0.0309999999999999\\
45	0.0370000000000001\\
46	0.0370000000000001\\
47	0.0370000000000001\\
48	0.043\\
49	0.043\\
50	0.05\\
51	0.05\\
52	0.0559999999999999\\
53	0.0559999999999999\\
54	0.0620000000000001\\
55	0.068\\
56	0.075\\
57	0.075\\
58	0.075\\
59	0.0809999999999999\\
60	0.0870000000000001\\
61	0.0870000000000001\\
62	0.0870000000000001\\
63	0.093\\
64	0.1\\
65	0.1\\
66	0.106\\
67	0.112\\
68	0.112\\
69	0.118\\
70	0.118\\
71	0.118\\
72	0.125\\
73	0.125\\
74	0.131\\
75	0.131\\
76	0.131\\
77	0.137\\
78	0.137\\
79	0.143\\
80	0.143\\
81	0.143\\
82	0.143\\
83	0.15\\
84	0.15\\
85	0.15\\
86	0.156\\
87	0.156\\
88	0.156\\
89	0.162\\
90	0.162\\
91	0.162\\
92	0.168\\
93	0.168\\
94	0.168\\
95	0.168\\
96	0.168\\
97	0.175\\
98	0.175\\
99	0.175\\
100	0.175\\
101	0.181\\
102	0.181\\
103	0.181\\
104	0.187\\
105	0.187\\
106	0.187\\
107	0.187\\
108	0.187\\
109	0.193\\
110	0.193\\
111	0.193\\
112	0.2\\
113	0.2\\
114	0.2\\
115	0.2\\
116	0.206\\
117	0.206\\
118	0.212\\
119	0.212\\
120	0.212\\
121	0.218\\
122	0.218\\
123	0.218\\
124	0.218\\
125	0.218\\
126	0.218\\
127	0.218\\
128	0.225\\
129	0.225\\
130	0.225\\
131	0.225\\
132	0.225\\
133	0.225\\
134	0.225\\
135	0.231\\
136	0.231\\
137	0.231\\
138	0.237\\
139	0.237\\
140	0.237\\
141	0.237\\
142	0.243\\
143	0.243\\
144	0.243\\
145	0.25\\
146	0.25\\
147	0.25\\
148	0.25\\
149	0.256\\
150	0.256\\
151	0.256\\
152	0.256\\
153	0.256\\
154	0.256\\
155	0.262\\
156	0.262\\
157	0.262\\
158	0.262\\
159	0.262\\
160	0.262\\
161	0.262\\
162	0.262\\
163	0.262\\
164	0.262\\
165	0.268\\
166	0.268\\
167	0.268\\
168	0.268\\
169	0.268\\
170	0.268\\
171	0.268\\
172	0.268\\
173	0.268\\
174	0.268\\
175	0.268\\
176	0.268\\
177	0.275\\
178	0.275\\
179	0.275\\
180	0.275\\
181	0.275\\
182	0.275\\
183	0.275\\
184	0.275\\
185	0.275\\
186	0.275\\
187	0.275\\
188	0.275\\
189	0.268\\
190	0.268\\
191	0.268\\
192	0.275\\
193	0.268\\
194	0.275\\
195	0.275\\
196	0.275\\
197	0.275\\
198	0.275\\
199	0.275\\
200	0.275\\
201	0.275\\
202	0.275\\
203	0.275\\
204	0.275\\
205	0.275\\
206	0.275\\
207	0.275\\
208	0.275\\
209	0.275\\
210	0.275\\
211	0.275\\
212	0.281\\
213	0.281\\
214	0.281\\
215	0.281\\
216	0.281\\
217	0.287\\
218	0.287\\
219	0.287\\
220	0.287\\
221	0.287\\
222	0.293\\
223	0.293\\
224	0.293\\
225	0.293\\
226	0.287\\
227	0.287\\
228	0.287\\
229	0.287\\
230	0.287\\
231	0.287\\
232	0.287\\
233	0.287\\
234	0.287\\
235	0.287\\
236	0.287\\
237	0.287\\
238	0.287\\
239	0.293\\
240	0.293\\
241	0.293\\
242	0.293\\
243	0.293\\
244	0.293\\
245	0.293\\
246	0.293\\
247	0.293\\
248	0.293\\
249	0.293\\
250	0.293\\
251	0.293\\
252	0.293\\
253	0.293\\
254	0.293\\
255	0.3\\
256	0.293\\
257	0.293\\
258	0.3\\
259	0.3\\
260	0.3\\
261	0.3\\
262	0.3\\
263	0.3\\
264	0.3\\
265	0.3\\
266	0.293\\
267	0.3\\
268	0.3\\
269	0.3\\
270	0.293\\
271	0.3\\
272	0.293\\
273	0.293\\
274	0.293\\
275	0.293\\
276	0.293\\
277	0.293\\
278	0.293\\
279	0.293\\
280	0.293\\
281	0.293\\
282	0.293\\
283	0.293\\
284	0.293\\
285	0.293\\
286	0.293\\
287	0.293\\
288	0.293\\
289	0.287\\
290	0.287\\
291	0.293\\
292	0.293\\
293	0.293\\
294	0.293\\
295	0.293\\
296	0.293\\
297	0.293\\
298	0.293\\
299	0.293\\
300	0.293\\
301	0.293\\
302	0.293\\
303	0.293\\
304	0.293\\
305	0.293\\
306	0.293\\
307	0.293\\
308	0.3\\
309	0.3\\
310	0.3\\
311	0.3\\
312	0.306\\
313	0.306\\
314	0.306\\
315	0.306\\
316	0.306\\
317	0.306\\
318	0.306\\
319	0.306\\
320	0.306\\
321	0.306\\
322	0.312\\
323	0.306\\
324	0.306\\
325	0.306\\
326	0.306\\
327	0.306\\
328	0.306\\
329	0.306\\
330	0.306\\
331	0.306\\
332	0.306\\
333	0.306\\
334	0.306\\
335	0.306\\
336	0.306\\
337	0.306\\
338	0.306\\
339	0.306\\
340	0.306\\
341	0.312\\
342	0.312\\
343	0.312\\
344	0.312\\
345	0.312\\
346	0.312\\
347	0.312\\
348	0.312\\
349	0.312\\
350	0.306\\
};
\addlegendentry{odpowiedź układu}

\addplot[const plot, color=mycolor2] table[row sep=crcr] {%
1	0\\
2	0\\
3	0\\
4	0\\
5	0\\
6	0\\
7	0\\
8	0\\
9	0\\
10	0\\
11	0\\
12	0\\
13	0.0001448414725457\\
14	0.000428209042231097\\
15	0.000843993922684382\\
16	0.0013862879075529\\
17	0.00204937752540018\\
18	0.00282773835422124\\
19	0.00371602949139211\\
20	0.00470908817497558\\
21	0.00580192455240992\\
22	0.00698971659270791\\
23	0.00826780513839316\\
24	0.00963168909349638\\
25	0.0110770207440287\\
26	0.0125996012074409\\
27	0.0141953760076655\\
28	0.0158604307724282\\
29	0.0175909870495974\\
30	0.0193833982394246\\
31	0.0212341456396099\\
32	0.0231398346002038\\
33	0.0250971907854343\\
34	0.0271030565396237\\
35	0.0291543873544299\\
36	0.0312482484347216\\
37	0.0333818113604628\\
38	0.0355523508420518\\
39	0.0377572415666248\\
40	0.0399939551328979\\
41	0.0422600570721855\\
42	0.0445532039532921\\
43	0.0468711405690354\\
44	0.0492116972022163\\
45	0.051572786968907\\
46	0.0539524032369846\\
47	0.0563486171178911\\
48	0.0587595750296514\\
49	0.0611834963292357\\
50	0.0636186710123969\\
51	0.0660634574791676\\
52	0.0685162803632449\\
53	0.0709756284235381\\
54	0.0734400524962003\\
55	0.0759081635055075\\
56	0.0783786305319914\\
57	0.0808501789362746\\
58	0.083321588537096\\
59	0.0857916918420557\\
60	0.0882593723296441\\
61	0.090723562781161\\
62	0.0931832436611634\\
63	0.0956374415451199\\
64	0.0980852275929806\\
65	0.100525716067408\\
66	0.102958062895448\\
67	0.105381464272446\\
68	0.107795155307051\\
69	0.11019840870619\\
70	0.112590533498884\\
71	0.114970873797879\\
72	0.117338807598004\\
73	0.11969374561028\\
74	0.122035130130772\\
75	0.124362433943218\\
76	0.126675159254525\\
77	0.128972836662187\\
78	0.131255024152766\\
79	0.133521306130557\\
80	0.135771292475596\\
81	0.138004617630206\\
82	0.140220939713266\\
83	0.142419939661443\\
84	0.144601320396624\\
85	0.146764806018813\\
86	0.148910141023783\\
87	0.151037089544785\\
88	0.153145434617632\\
89	0.155234977468506\\
90	0.157305536823838\\
91	0.159356948241649\\
92	0.161389063463727\\
93	0.163401749788069\\
94	0.165394889460991\\
95	0.167368379088361\\
96	0.169322129065412\\
97	0.171256063024585\\
98	0.173170117300912\\
99	0.175064240414413\\
100	0.176938392569044\\
101	0.178792545167692\\
102	0.180626680342775\\
103	0.182440790501996\\
104	0.184234877888801\\
105	0.186008954157126\\
106	0.18776303996002\\
107	0.189497164551724\\
108	0.191211365402843\\
109	0.192905687828204\\
110	0.194580184627038\\
111	0.196234915735131\\
112	0.197869947888584\\
113	0.199485354298849\\
114	0.201081214338703\\
115	0.202657613238836\\
116	0.204214641794754\\
117	0.205752396083662\\
118	0.207270977191071\\
119	0.208770490946799\\
120	0.210251047670113\\
121	0.211712761923732\\
122	0.213155752276424\\
123	0.21458014107393\\
124	0.21598605421799\\
125	0.217373620953196\\
126	0.218742973661461\\
127	0.220094247663851\\
128	0.221427581029581\\
129	0.222743114391934\\
130	0.224040990770902\\
131	0.225321355402351\\
132	0.226584355573489\\
133	0.227830140464467\\
134	0.229058860995904\\
135	0.230270669682169\\
136	0.231465720490227\\
137	0.232644168703889\\
138	0.233806170793283\\
139	0.234951884289402\\
140	0.236081467663548\\
141	0.237195080211537\\
142	0.2382928819425\\
143	0.239375033472146\\
144	0.240441695920339\\
145	0.241493030812849\\
146	0.242529199987153\\
147	0.243550365502146\\
148	0.244556689551646\\
149	0.245548334381558\\
150	0.246525462210594\\
151	0.247488235154418\\
152	0.248436815153113\\
153	0.249371363901865\\
154	0.250292042784744\\
155	0.251199012811495\\
156	0.252092434557233\\
157	0.252972468104936\\
158	0.253839272990665\\
159	0.254693008151399\\
160	0.255533831875402\\
161	0.256361901755043\\
162	0.257177374641981\\
163	0.257980406604631\\
164	0.25877115288784\\
165	0.259549767874693\\
166	0.260316405050369\\
167	0.261071216967991\\
168	0.261814355216388\\
169	0.262545970389702\\
170	0.263266212058781\\
171	0.263975228744293\\
172	0.264673167891497\\
173	0.265360175846608\\
174	0.266036397834712\\
175	0.266701977939158\\
176	0.267357059082385\\
177	0.268001783008128\\
178	0.268636290264946\\
179	0.269260720191031\\
180	0.269875210900252\\
181	0.270479899269371\\
182	0.271074920926406\\
183	0.271660410240089\\
184	0.27223650031037\\
185	0.272803322959942\\
186	0.273361008726729\\
187	0.273909686857315\\
188	0.274449485301264\\
189	0.274980530706308\\
190	0.275502948414351\\
191	0.276016862458274\\
192	0.276522395559497\\
193	0.277019669126263\\
194	0.277508803252629\\
195	0.277989916718118\\
196	0.278463126988012\\
197	0.278928550214255\\
198	0.279386301236948\\
199	0.279836493586387\\
200	0.280279239485653\\
201	0.280714649853699\\
202	0.281142834308926\\
203	0.281563901173225\\
204	0.281977957476455\\
205	0.282385108961349\\
206	0.282785460088811\\
207	0.2831791140436\\
208	0.283566172740372\\
209	0.283946736830064\\
210	0.284320905706601\\
211	0.284688777513917\\
212	0.285050449153262\\
213	0.285406016290783\\
214	0.285755573365366\\
215	0.286099213596724\\
216	0.286437028993713\\
217	0.286769110362866\\
218	0.287095547317127\\
219	0.287416428284779\\
220	0.287731840518547\\
221	0.288041870104865\\
222	0.288346601973302\\
223	0.288646119906125\\
224	0.288940506547999\\
225	0.289229843415809\\
226	0.289514210908587\\
227	0.289793688317552\\
228	0.290068353836234\\
229	0.290338284570685\\
230	0.290603556549775\\
231	0.290864244735536\\
232	0.291120423033589\\
233	0.291372164303605\\
234	0.291619540369822\\
235	0.291862622031595\\
236	0.292101479073979\\
237	0.29233618027834\\
238	0.292566793432977\\
239	0.292793385343771\\
240	0.293016021844824\\
241	0.293234767809117\\
242	0.293449687159153\\
243	0.293660842877598\\
244	0.293868297017907\\
245	0.294072110714936\\
246	0.29427234419553\\
247	0.294469056789082\\
248	0.294662306938074\\
249	0.29485215220857\\
250	0.295038649300684\\
251	0.295221854059002\\
252	0.295401821482962\\
253	0.295578605737191\\
254	0.295752260161784\\
255	0.295922837282541\\
256	0.296090388821145\\
257	0.29625496570528\\
258	0.296416618078693\\
259	0.296575395311195\\
260	0.296731346008596\\
261	0.296884518022577\\
262	0.297034958460488\\
263	0.297182713695085\\
264	0.297327829374191\\
265	0.297470350430287\\
266	0.297610321090026\\
267	0.297747784883673\\
268	0.297882784654471\\
269	0.298015362567928\\
270	0.298145560121022\\
271	0.298273418151331\\
272	0.298398976846081\\
273	0.29852227575111\\
274	0.298643353779753\\
275	0.298762249221644\\
276	0.298878999751433\\
277	0.29899364243742\\
278	0.299106213750104\\
279	0.299216749570647\\
280	0.299325285199255\\
281	0.29943185536347\\
282	0.299536494226378\\
283	0.299639235394734\\
284	0.299740111926992\\
285	0.29983915634126\\
286	0.299936400623162\\
287	0.300031876233613\\
288	0.300125614116514\\
289	0.30021764470635\\
290	0.300307997935716\\
291	0.300396703242744\\
292	0.300483789578454\\
293	0.300569285414011\\
294	0.300653218747906\\
295	0.30073561711304\\
296	0.300816507583739\\
297	0.300895916782667\\
298	0.300973870887671\\
299	0.301050395638531\\
300	0.301125516343629\\
301	0.301199257886543\\
302	0.301271644732547\\
303	0.301342700935036\\
304	0.301412450141872\\
305	0.301480915601637\\
306	0.30154812016982\\
307	0.301614086314919\\
308	0.301678836124456\\
309	0.301742391310924\\
310	0.301804773217652\\
311	0.301866002824588\\
312	0.301926100754014\\
313	0.301985087276175\\
314	0.302042982314837\\
315	0.302099805452772\\
316	0.302155575937164\\
317	0.302210312684943\\
318	0.302264034288045\\
319	0.302316759018602\\
320	0.302368504834059\\
321	0.302419289382214\\
322	0.302469130006199\\
323	0.30251804374938\\
324	0.302566047360194\\
325	0.302613157296917\\
326	0.30265938973236\\
327	0.302704760558506\\
328	0.302749285391071\\
329	0.302792979574004\\
330	0.302835858183927\\
331	0.302877936034497\\
332	0.302919227680723\\
333	0.3029597474232\\
334	0.3029995093123\\
335	0.303038527152282\\
336	0.303076814505363\\
337	0.303114384695706\\
338	0.303151250813368\\
339	0.303187425718177\\
340	0.303222922043556\\
341	0.303257752200289\\
342	0.303291928380227\\
343	0.303325462559947\\
344	0.303358366504342\\
345	0.303390651770172\\
346	0.303422329709545\\
347	0.303453411473358\\
348	0.303483908014679\\
349	0.303513830092078\\
350	0.303543188272903\\
};
\addlegendentry{odpowiedź aproksymowana}

\end{axis}

\begin{axis}[%
width=5.833in,
height=4.375in,
at={(0in,0in)},
scale only axis,
xmin=0,
xmax=1,
ymin=0,
ymax=1,
axis line style={draw=none},
ticks=none,
axis x line*=bottom,
axis y line*=left
]
\end{axis}
\end{tikzpicture}%
\caption{Aproksymacja odpowiedzi skokowej}
\end{figure}

W celu wyznaczenia optymalnych parametrów optymalizacji posłużono się algorytmem genetycznym o losowej populacji początkowej

\section{Dobranie nastaw regulatora PID i parametrów algorytmu DMC}
 
Wartości nastaw regulatora PID

\begin{equation}
K = 30  T_i = 35  T_d = 4.5  T_p = 1
\end{equation}

Wskaźnik jakości regulacji:

\begin{equation}
E = 20,5988
\end{equation}



\end{document}

