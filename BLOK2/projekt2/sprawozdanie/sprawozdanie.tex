%! TEX encoding = utf8
\documentclass[a4paper,titlepage,11pt,twosides,floatssmall]{mwrep}
\usepackage[left=2.5cm,right=2.5cm,top=2.5cm,bottom=2.5cm]{geometry}
\usepackage[OT1]{fontenc}
\usepackage{polski}
\usepackage{amsmath}
\usepackage{amsfonts}
\usepackage{amssymb}
\usepackage{graphicx}
\usepackage{float}
\usepackage{url}
\usepackage{tikz}
\usetikzlibrary{arrows,calc,decorations.markings,math,arrows.meta}
\usepackage{rotating}
\usepackage[percent]{overpic}
\usepackage[utf8]{inputenc}
\usepackage{xcolor}
\usepackage{colortbl}
\usepackage{pgfplots}
\usetikzlibrary{pgfplots.groupplots}
\usepackage{listings}
\usepackage{matlab-prettifier}
\usepackage{enumitem,amssymb}
\definecolor{szary}{rgb}{0.95,0.95,0.95}
\usepackage{siunitx}
\sisetup{detect-weight,exponent-product=\cdot,output-decimal-marker={,},per-mode=symbol,binary-units=true,range-phrase={-},range-units=single}
\SendSettingsToPgf
%konfiguracje pakietu listings
\lstset{
	backgroundcolor=\color{szary},
	frame=single,
	breaklines=true,
}
\lstdefinestyle{customlatex}{
	basicstyle=\footnotesize\ttfamily,
	%basicstyle=\small\ttfamily,
}
\lstdefinestyle{customc}{
	breaklines=true,
	frame=tb,
	language=C,
	xleftmargin=0pt,
	showstringspaces=false,
	basicstyle=\small\ttfamily,
	keywordstyle=\bfseries\color{green!40!black},
	commentstyle=\itshape\color{purple!40!black},
	identifierstyle=\color{blue},
	stringstyle=\color{orange},
}
\lstdefinestyle{custommatlab}{
	captionpos=t,
	breaklines=true,
	frame=tb,
	xleftmargin=0pt,
	language=matlab,
	showstringspaces=false,
	basicstyle=\footnotesize\ttfamily,
	%basicstyle=\scriptsize\ttfamily,
	keywordstyle=\bfseries\color{green!40!black},
	commentstyle=\itshape\color{purple!40!black},
	identifierstyle=\color{blue},
	stringstyle=\color{orange},
}

%wymiar tekstu (bez �ywej paginy)
\textwidth 160mm \textheight 247mm

%ustawienia pakietu pgfplots
\pgfplotsset{
tick label style={font=\scriptsize},
label style={font=\small},
legend style={font=\small},
title style={font=\small}
}

\def\figurename{Rys.}
\def\tablename{Tab.}

%konfiguracja liczby p�ywaj�cych element�w
\setcounter{topnumber}{0}%2
\setcounter{bottomnumber}{3}%1
\setcounter{totalnumber}{5}%3
\renewcommand{\textfraction}{0.01}%0.2
\renewcommand{\topfraction}{0.95}%0.7
\renewcommand{\bottomfraction}{0.95}%0.3
\renewcommand{\floatpagefraction}{0.35}%0.5

\begin{document}
\frenchspacing
\pagestyle{uheadings}

%strona tytu�owa
\title{\bf Sprawozdanie z projektu i ćwiczenia laboratoryjnego nr 1, zadanie nr 1\vskip 0.1cm}
\author{Stanislau Stankevich, Rafał Bednarz, Ostrysz Jakub}
\date{2021}

\makeatletter
\renewcommand{\maketitle}{\begin{titlepage}
\begin{center}{\LARGE {\bf
Wydział Elektroniki i Technik Informacyjnych}}\\
\vspace{0.4cm}
{\LARGE {\bf Politechnika Warszawska}}\\
\vspace{0.3cm}
\end{center}
\vspace{5cm}
\begin{center}
{\bf \LARGE Projektowanie układów sterowania\\ (projekt grupowy) \vskip 0.1cm}
\end{center}
\vspace{1cm}
\begin{center}
{\bf \LARGE \@title}
\end{center}
\vspace{2cm}
\begin{center}
{\bf \Large \@author \par}
\end{center}
\vspace*{\stretch{6}}
\begin{center}
\bf{\large{Warszawa, \@date\vskip 0.1cm}}
\end{center}
\end{titlepage}
}
\makeatother

\maketitle

\tableofcontents
%! TEX encoding = utf8
\chapter{Sprawdzenie poprawności podanych wartości}
Żeby sprawdzić poprawność podanych wartości podajemy na wejscie sterowanie $u = 0$ oraz zakłócenie $z = 0$ i patrzymy na jakiej wartości się ustali $y$.

\begin{figure}[H]
\centering
% This file was created by matlab2tikz.
%
%The latest updates can be retrieved from
%  http://www.mathworks.com/matlabcentral/fileexchange/22022-matlab2tikz-matlab2tikz
%where you can also make suggestions and rate matlab2tikz.
%
\definecolor{mycolor1}{rgb}{0.00000,0.44700,0.74100}%
%
\begin{tikzpicture}

\begin{axis}[%
width=5.521in,
height=4.566in,
at={(0.758in,0.481in)},
scale only axis,
xmin=0,
xmax=300,
xlabel style={font=\color{white!15!black}},
xlabel={k},
ymin=-0.1,
ymax=1,
ylabel style={font=\color{white!15!black}},
ylabel={Y(k)},
axis background/.style={fill=white}
]
\addplot[const plot, color=mycolor1, forget plot] table[row sep=crcr] {%
1	0\\
2	0\\
3	0\\
4	0\\
5	0\\
6	0\\
7	0\\
8	0\\
9	0\\
10	0\\
11	0\\
12	0\\
13	0\\
14	0\\
15	0\\
16	0\\
17	0\\
18	0\\
19	0\\
20	0\\
21	0\\
22	0\\
23	0\\
24	0\\
25	0\\
26	0\\
27	0\\
28	0\\
29	0\\
30	0\\
31	0\\
32	0\\
33	0\\
34	0\\
35	0\\
36	0\\
37	0\\
38	0\\
39	0\\
40	0\\
41	0\\
42	0\\
43	0\\
44	0\\
45	0\\
46	0\\
47	0\\
48	0\\
49	0\\
50	0\\
51	0\\
52	0\\
53	0\\
54	0\\
55	0\\
56	0\\
57	0\\
58	0\\
59	0\\
60	0\\
61	0\\
62	0\\
63	0\\
64	0\\
65	0\\
66	0\\
67	0\\
68	0\\
69	0\\
70	0\\
71	0\\
72	0\\
73	0\\
74	0\\
75	0\\
76	0\\
77	0\\
78	0\\
79	0\\
80	0\\
81	0\\
82	0\\
83	0\\
84	0\\
85	0\\
86	0\\
87	0\\
88	0\\
89	0\\
90	0\\
91	0\\
92	0\\
93	0\\
94	0\\
95	0\\
96	0\\
97	0\\
98	0\\
99	0\\
100	0\\
101	0\\
102	0\\
103	0\\
104	0\\
105	0\\
106	0\\
107	0\\
108	0\\
109	0\\
110	0\\
111	0\\
112	0\\
113	0\\
114	0\\
115	0\\
116	0\\
117	0\\
118	0\\
119	0\\
120	0\\
121	0\\
122	0\\
123	0\\
124	0\\
125	0\\
126	0\\
127	0\\
128	0\\
129	0\\
130	0\\
131	0\\
132	0\\
133	0\\
134	0\\
135	0\\
136	0\\
137	0\\
138	0\\
139	0\\
140	0\\
141	0\\
142	0\\
143	0\\
144	0\\
145	0\\
146	0\\
147	0\\
148	0\\
149	0\\
150	0\\
151	0\\
152	0\\
153	0\\
154	0\\
155	0\\
156	0\\
157	0\\
158	0\\
159	0\\
160	0\\
161	0\\
162	0\\
163	0\\
164	0\\
165	0\\
166	0\\
167	0\\
168	0\\
169	0\\
170	0\\
171	0\\
172	0\\
173	0\\
174	0\\
175	0\\
176	0\\
177	0\\
178	0\\
179	0\\
180	0\\
181	0\\
182	0\\
183	0\\
184	0\\
185	0\\
186	0\\
187	0\\
188	0\\
189	0\\
190	0\\
191	0\\
192	0\\
193	0\\
194	0\\
195	0\\
196	0\\
197	0\\
198	0\\
199	0\\
200	0\\
201	0\\
202	0\\
203	0\\
204	0\\
205	0\\
206	0\\
207	0\\
208	0\\
209	0\\
210	0\\
211	0\\
212	0\\
213	0\\
214	0\\
215	0\\
216	0\\
217	0\\
218	0\\
219	0\\
220	0\\
221	0\\
222	0\\
223	0\\
224	0\\
225	0\\
226	0\\
227	0\\
228	0\\
229	0\\
230	0\\
231	0\\
232	0\\
233	0\\
234	0\\
235	0\\
236	0\\
237	0\\
238	0\\
239	0\\
240	0\\
241	0\\
242	0\\
243	0\\
244	0\\
245	0\\
246	0\\
247	0\\
248	0\\
249	0\\
250	0\\
251	0\\
252	0\\
253	0\\
254	0\\
255	0\\
256	0\\
257	0\\
258	0\\
259	0\\
260	0\\
261	0\\
262	0\\
263	0\\
264	0\\
265	0\\
266	0\\
267	0\\
268	0\\
269	0\\
270	0\\
271	0\\
272	0\\
273	0\\
274	0\\
275	0\\
276	0\\
277	0\\
278	0\\
279	0\\
280	0\\
281	0\\
282	0\\
283	0\\
284	0\\
285	0\\
286	0\\
287	0\\
288	0\\
289	0\\
290	0\\
291	0\\
292	0\\
293	0\\
294	0\\
295	0\\
296	0\\
297	0\\
298	0\\
299	0\\
300	0\\
};
\end{axis}
\end{tikzpicture}%
\caption{Przebieg wyjścia obiektu przy stałym wejściu i zakłóceniu: $u = z = 0$}
\end{figure}

Jak możemy obersować wyjście się ustala na poprawnej wartości, czyli na 0.

%! TEX encoding = utf8
\chapter{Odpowiedzi skokowe}

Rozważamy punkt pracy oraz 6 różnych wartości skoku, z zera do: $-0,5$, $-0,25$, $0,25$, $0,5$, $0,75$, $1,0$.

\section{Opowiedzi skokowe}

\begin{figure}[H]
\centering
% This file was created by matlab2tikz.
%
%The latest updates can be retrieved from
%  http://www.mathworks.com/matlabcentral/fileexchange/22022-matlab2tikz-matlab2tikz
%where you can also make suggestions and rate matlab2tikz.
%
\definecolor{mycolor1}{rgb}{0.00000,0.44700,0.74100}%
\definecolor{mycolor2}{rgb}{0.85000,0.32500,0.09800}%
\definecolor{mycolor3}{rgb}{0.92900,0.69400,0.12500}%
\definecolor{mycolor4}{rgb}{0.49400,0.18400,0.55600}%
\definecolor{mycolor5}{rgb}{0.46600,0.67400,0.18800}%
\definecolor{mycolor6}{rgb}{0.30100,0.74500,0.93300}%
\definecolor{mycolor7}{rgb}{0.63500,0.07800,0.18400}%
%
\begin{tikzpicture}

\begin{axis}[%
width=6.102in,
height=6.417in,
at={(1.024in,0.866in)},
scale only axis,
xmin=0,
xmax=300,
xlabel style={font=\color{white!15!black}},
xlabel={k},
ymin=-0.8,
ymax=0.1,
ylabel style={font=\color{white!15!black}},
ylabel={Y(k)},
axis background/.style={fill=white},
legend style={at={(0.97,0.03)}, anchor=south east, legend cell align=left, align=left, draw=white!15!black}
]
\addplot[const plot, color=mycolor1] table[row sep=crcr] {%
1	0\\
2	0\\
3	0\\
4	0\\
5	0\\
6	0\\
7	0\\
8	0\\
9	0\\
10	0\\
11	0\\
12	0\\
13	0\\
14	0\\
15	-0.0335675877192982\\
16	-0.112965625142105\\
17	-0.212521981792259\\
18	-0.31513883462998\\
19	-0.41036001081684\\
20	-0.492623525565041\\
21	-0.5597813733745\\
22	-0.611911428472393\\
23	-0.650410015074312\\
24	-0.677333122976111\\
25	-0.694944862934428\\
26	-0.705429889631609\\
27	-0.71072929563746\\
28	-0.712464770401887\\
29	-0.711922131286509\\
30	-0.710071688301026\\
31	-0.707608712847838\\
32	-0.70500224248527\\
33	-0.702544465392307\\
34	-0.700396013802861\\
35	-0.698624752625891\\
36	-0.697237210244591\\
37	-0.696202804510307\\
38	-0.695471602257874\\
39	-0.69498663288036\\
40	-0.694691852874273\\
41	-0.694536805208319\\
42	-0.694478892004551\\
43	-0.694484021861918\\
44	-0.694526231178164\\
45	-0.694586728534931\\
46	-0.694652681377337\\
47	-0.694715958230594\\
48	-0.694771957427408\\
49	-0.694818592565868\\
50	-0.694855462432398\\
51	-0.694883205334342\\
52	-0.694903021243572\\
53	-0.694916336792617\\
54	-0.694924585428763\\
55	-0.694929075890161\\
56	-0.69493092508991\\
57	-0.69493103538224\\
58	-0.694930100299094\\
59	-0.69492862672469\\
60	-0.694926964866447\\
61	-0.694925340176121\\
62	-0.694923883564747\\
63	-0.694922657887373\\
64	-0.69492167982743\\
65	-0.694920937075836\\
66	-0.694920401165058\\
67	-0.694920036562932\\
68	-0.694919806722385\\
69	-0.694919677775033\\
70	-0.69491962048998\\
71	-0.694919611023644\\
72	-0.694919630882301\\
73	-0.694919666419212\\
74	-0.694919708099766\\
75	-0.694919749694496\\
76	-0.69491978750165\\
77	-0.694919819657184\\
78	-0.694919845558801\\
79	-0.694919865409643\\
80	-0.69491987987422\\
81	-0.694919889832076\\
82	-0.694919896211796\\
83	-0.694919899887748\\
84	-0.694919901623475\\
85	-0.694919902047935\\
86	-0.694919901653444\\
87	-0.694919900806729\\
88	-0.694919899766789\\
89	-0.694919898705208\\
90	-0.694919897726102\\
91	-0.694919896884061\\
92	-0.694919896199279\\
93	-0.694919895669655\\
94	-0.694919895279998\\
95	-0.694919895008691\\
96	-0.69491989483223\\
97	-0.694919894728098\\
98	-0.69491989467639\\
99	-0.694919894660544\\
100	-0.694919894667477\\
101	-0.69491989468735\\
102	-0.694919894713144\\
103	-0.694919894740146\\
104	-0.694919894765443\\
105	-0.694919894787452\\
106	-0.694919894805525\\
107	-0.694919894819632\\
108	-0.694919894830109\\
109	-0.694919894837484\\
110	-0.694919894842348\\
111	-0.694919894845282\\
112	-0.694919894846802\\
113	-0.694919894847343\\
114	-0.694919894847253\\
115	-0.694919894846796\\
116	-0.69491989484616\\
117	-0.694919894845476\\
118	-0.694919894844824\\
119	-0.69491989484425\\
120	-0.694919894843774\\
121	-0.694919894843398\\
122	-0.694919894843117\\
123	-0.694919894842917\\
124	-0.694919894842784\\
125	-0.694919894842702\\
126	-0.694919894842658\\
127	-0.69491989484264\\
128	-0.69491989484264\\
129	-0.69491989484265\\
130	-0.694919894842665\\
131	-0.694919894842683\\
132	-0.6949198948427\\
133	-0.694919894842715\\
134	-0.694919894842727\\
135	-0.694919894842737\\
136	-0.694919894842745\\
137	-0.69491989484275\\
138	-0.694919894842754\\
139	-0.694919894842756\\
140	-0.694919894842757\\
141	-0.694919894842757\\
142	-0.694919894842757\\
143	-0.694919894842757\\
144	-0.694919894842757\\
145	-0.694919894842756\\
146	-0.694919894842756\\
147	-0.694919894842756\\
148	-0.694919894842755\\
149	-0.694919894842755\\
150	-0.694919894842755\\
151	-0.694919894842755\\
152	-0.694919894842755\\
153	-0.694919894842755\\
154	-0.694919894842755\\
155	-0.694919894842755\\
156	-0.694919894842755\\
157	-0.694919894842755\\
158	-0.694919894842755\\
159	-0.694919894842755\\
160	-0.694919894842755\\
161	-0.694919894842755\\
162	-0.694919894842755\\
163	-0.694919894842755\\
164	-0.694919894842755\\
165	-0.694919894842755\\
166	-0.694919894842755\\
167	-0.694919894842755\\
168	-0.694919894842755\\
169	-0.694919894842755\\
170	-0.694919894842755\\
171	-0.694919894842755\\
172	-0.694919894842755\\
173	-0.694919894842755\\
174	-0.694919894842755\\
175	-0.694919894842755\\
176	-0.694919894842755\\
177	-0.694919894842755\\
178	-0.694919894842755\\
179	-0.694919894842755\\
180	-0.694919894842755\\
181	-0.694919894842755\\
182	-0.694919894842755\\
183	-0.694919894842755\\
184	-0.694919894842755\\
185	-0.694919894842755\\
186	-0.694919894842755\\
187	-0.694919894842755\\
188	-0.694919894842755\\
189	-0.694919894842755\\
190	-0.694919894842755\\
191	-0.694919894842755\\
192	-0.694919894842755\\
193	-0.694919894842755\\
194	-0.694919894842755\\
195	-0.694919894842755\\
196	-0.694919894842755\\
197	-0.694919894842755\\
198	-0.694919894842755\\
199	-0.694919894842755\\
200	-0.694919894842755\\
201	-0.694919894842755\\
202	-0.694919894842755\\
203	-0.694919894842755\\
204	-0.694919894842755\\
205	-0.694919894842755\\
206	-0.694919894842755\\
207	-0.694919894842755\\
208	-0.694919894842755\\
209	-0.694919894842755\\
210	-0.694919894842755\\
211	-0.694919894842755\\
212	-0.694919894842755\\
213	-0.694919894842755\\
214	-0.694919894842755\\
215	-0.694919894842755\\
216	-0.694919894842755\\
217	-0.694919894842755\\
218	-0.694919894842755\\
219	-0.694919894842755\\
220	-0.694919894842755\\
221	-0.694919894842755\\
222	-0.694919894842755\\
223	-0.694919894842755\\
224	-0.694919894842755\\
225	-0.694919894842755\\
226	-0.694919894842755\\
227	-0.694919894842755\\
228	-0.694919894842755\\
229	-0.694919894842755\\
230	-0.694919894842755\\
231	-0.694919894842755\\
232	-0.694919894842755\\
233	-0.694919894842755\\
234	-0.694919894842755\\
235	-0.694919894842755\\
236	-0.694919894842755\\
237	-0.694919894842755\\
238	-0.694919894842755\\
239	-0.694919894842755\\
240	-0.694919894842755\\
241	-0.694919894842755\\
242	-0.694919894842755\\
243	-0.694919894842755\\
244	-0.694919894842755\\
245	-0.694919894842755\\
246	-0.694919894842755\\
247	-0.694919894842755\\
248	-0.694919894842755\\
249	-0.694919894842755\\
250	-0.694919894842755\\
251	-0.694919894842755\\
252	-0.694919894842755\\
253	-0.694919894842755\\
254	-0.694919894842755\\
255	-0.694919894842755\\
256	-0.694919894842755\\
257	-0.694919894842755\\
258	-0.694919894842755\\
259	-0.694919894842755\\
260	-0.694919894842755\\
261	-0.694919894842755\\
262	-0.694919894842755\\
263	-0.694919894842755\\
264	-0.694919894842755\\
265	-0.694919894842755\\
266	-0.694919894842755\\
267	-0.694919894842755\\
268	-0.694919894842755\\
269	-0.694919894842755\\
270	-0.694919894842755\\
271	-0.694919894842755\\
272	-0.694919894842755\\
273	-0.694919894842755\\
274	-0.694919894842755\\
275	-0.694919894842755\\
276	-0.694919894842755\\
277	-0.694919894842755\\
278	-0.694919894842755\\
279	-0.694919894842755\\
280	-0.694919894842755\\
281	-0.694919894842755\\
282	-0.694919894842755\\
283	-0.694919894842755\\
284	-0.694919894842755\\
285	-0.694919894842755\\
286	-0.694919894842755\\
287	-0.694919894842755\\
288	-0.694919894842755\\
289	-0.694919894842755\\
290	-0.694919894842755\\
291	-0.694919894842755\\
292	-0.694919894842755\\
293	-0.694919894842755\\
294	-0.694919894842755\\
295	-0.694919894842755\\
296	-0.694919894842755\\
297	-0.694919894842755\\
298	-0.694919894842755\\
299	-0.694919894842755\\
300	-0.694919894842755\\
};
\addlegendentry{Skok U z 0.00 do -0.50}

\addplot[const plot, color=mycolor2] table[row sep=crcr] {%
1	0\\
2	0\\
3	0\\
4	0\\
5	0\\
6	0\\
7	0\\
8	0\\
9	0\\
10	0\\
11	0\\
12	0\\
13	0\\
14	0\\
15	-0.0150905587027915\\
16	-0.0484195449261625\\
17	-0.0864115139873757\\
18	-0.121395119761953\\
19	-0.149839723783237\\
20	-0.170855068765489\\
21	-0.185040572876755\\
22	-0.193679087044566\\
23	-0.198223356168033\\
24	-0.200006749973101\\
25	-0.200110976760491\\
26	-0.19933367609057\\
27	-0.198212181266038\\
28	-0.197072885924991\\
29	-0.196086789406331\\
30	-0.195320314175638\\
31	-0.194776439892249\\
32	-0.194424976657385\\
33	-0.194222917734573\\
34	-0.194126771051117\\
35	-0.194098993056308\\
36	-0.19411046158561\\
37	-0.194140551866487\\
38	-0.194175964595706\\
39	-0.194209076957524\\
40	-0.194236282449366\\
41	-0.19425656155894\\
42	-0.194270375315654\\
43	-0.194278883192304\\
44	-0.194283439888477\\
45	-0.194285307778275\\
46	-0.194285521654744\\
47	-0.194284851357772\\
48	-0.194283820249498\\
49	-0.194282749881076\\
50	-0.194281811811171\\
51	-0.1942810757299\\
52	-0.194280548816684\\
53	-0.194280204963872\\
54	-0.194280004604477\\
55	-0.194279906864281\\
56	-0.194279876020257\\
57	-0.194279884099752\\
58	-0.194279911117332\\
59	-0.194279944058589\\
60	-0.194279975362056\\
61	-0.194280001358456\\
62	-0.194280020910386\\
63	-0.19428003434934\\
64	-0.194280042717708\\
65	-0.194280047276223\\
66	-0.194280049218616\\
67	-0.194280049533891\\
68	-0.194280048964422\\
69	-0.194280048019476\\
70	-0.194280047015419\\
71	-0.194280046123964\\
72	-0.194280045417687\\
73	-0.194280044907653\\
74	-0.194280044571604\\
75	-0.194280044373251\\
76	-0.194280044274239\\
77	-0.194280044240613\\
78	-0.194280044245527\\
79	-0.194280044269632\\
80	-0.194280044300208\\
81	-0.194280044329764\\
82	-0.194280044354582\\
83	-0.194280044373416\\
84	-0.194280044386478\\
85	-0.194280044394699\\
86	-0.19428004439925\\
87	-0.194280044401257\\
88	-0.194280044401666\\
89	-0.194280044401192\\
90	-0.194280044400329\\
91	-0.194280044399388\\
92	-0.194280044398542\\
93	-0.194280044397865\\
94	-0.194280044397372\\
95	-0.194280044397044\\
96	-0.194280044396848\\
97	-0.194280044396748\\
98	-0.194280044396711\\
99	-0.194280044396713\\
100	-0.194280044396735\\
101	-0.194280044396763\\
102	-0.194280044396791\\
103	-0.194280044396815\\
104	-0.194280044396833\\
105	-0.194280044396845\\
106	-0.194280044396854\\
107	-0.194280044396858\\
108	-0.19428004439686\\
109	-0.194280044396861\\
110	-0.19428004439686\\
111	-0.194280044396859\\
112	-0.194280044396859\\
113	-0.194280044396858\\
114	-0.194280044396857\\
115	-0.194280044396857\\
116	-0.194280044396856\\
117	-0.194280044396856\\
118	-0.194280044396856\\
119	-0.194280044396856\\
120	-0.194280044396856\\
121	-0.194280044396856\\
122	-0.194280044396856\\
123	-0.194280044396856\\
124	-0.194280044396856\\
125	-0.194280044396856\\
126	-0.194280044396856\\
127	-0.194280044396856\\
128	-0.194280044396856\\
129	-0.194280044396856\\
130	-0.194280044396856\\
131	-0.194280044396856\\
132	-0.194280044396856\\
133	-0.194280044396856\\
134	-0.194280044396856\\
135	-0.194280044396856\\
136	-0.194280044396856\\
137	-0.194280044396856\\
138	-0.194280044396856\\
139	-0.194280044396856\\
140	-0.194280044396856\\
141	-0.194280044396856\\
142	-0.194280044396856\\
143	-0.194280044396856\\
144	-0.194280044396856\\
145	-0.194280044396856\\
146	-0.194280044396856\\
147	-0.194280044396856\\
148	-0.194280044396856\\
149	-0.194280044396856\\
150	-0.194280044396856\\
151	-0.194280044396856\\
152	-0.194280044396856\\
153	-0.194280044396856\\
154	-0.194280044396856\\
155	-0.194280044396856\\
156	-0.194280044396856\\
157	-0.194280044396856\\
158	-0.194280044396856\\
159	-0.194280044396856\\
160	-0.194280044396856\\
161	-0.194280044396856\\
162	-0.194280044396856\\
163	-0.194280044396856\\
164	-0.194280044396856\\
165	-0.194280044396856\\
166	-0.194280044396856\\
167	-0.194280044396856\\
168	-0.194280044396856\\
169	-0.194280044396856\\
170	-0.194280044396856\\
171	-0.194280044396856\\
172	-0.194280044396856\\
173	-0.194280044396856\\
174	-0.194280044396856\\
175	-0.194280044396856\\
176	-0.194280044396856\\
177	-0.194280044396856\\
178	-0.194280044396856\\
179	-0.194280044396856\\
180	-0.194280044396856\\
181	-0.194280044396856\\
182	-0.194280044396856\\
183	-0.194280044396856\\
184	-0.194280044396856\\
185	-0.194280044396856\\
186	-0.194280044396856\\
187	-0.194280044396856\\
188	-0.194280044396856\\
189	-0.194280044396856\\
190	-0.194280044396856\\
191	-0.194280044396856\\
192	-0.194280044396856\\
193	-0.194280044396856\\
194	-0.194280044396856\\
195	-0.194280044396856\\
196	-0.194280044396856\\
197	-0.194280044396856\\
198	-0.194280044396856\\
199	-0.194280044396856\\
200	-0.194280044396856\\
201	-0.194280044396856\\
202	-0.194280044396856\\
203	-0.194280044396856\\
204	-0.194280044396856\\
205	-0.194280044396856\\
206	-0.194280044396856\\
207	-0.194280044396856\\
208	-0.194280044396856\\
209	-0.194280044396856\\
210	-0.194280044396856\\
211	-0.194280044396856\\
212	-0.194280044396856\\
213	-0.194280044396856\\
214	-0.194280044396856\\
215	-0.194280044396856\\
216	-0.194280044396856\\
217	-0.194280044396856\\
218	-0.194280044396856\\
219	-0.194280044396856\\
220	-0.194280044396856\\
221	-0.194280044396856\\
222	-0.194280044396856\\
223	-0.194280044396856\\
224	-0.194280044396856\\
225	-0.194280044396856\\
226	-0.194280044396856\\
227	-0.194280044396856\\
228	-0.194280044396856\\
229	-0.194280044396856\\
230	-0.194280044396856\\
231	-0.194280044396856\\
232	-0.194280044396856\\
233	-0.194280044396856\\
234	-0.194280044396856\\
235	-0.194280044396856\\
236	-0.194280044396856\\
237	-0.194280044396856\\
238	-0.194280044396856\\
239	-0.194280044396856\\
240	-0.194280044396856\\
241	-0.194280044396856\\
242	-0.194280044396856\\
243	-0.194280044396856\\
244	-0.194280044396856\\
245	-0.194280044396856\\
246	-0.194280044396856\\
247	-0.194280044396856\\
248	-0.194280044396856\\
249	-0.194280044396856\\
250	-0.194280044396856\\
251	-0.194280044396856\\
252	-0.194280044396856\\
253	-0.194280044396856\\
254	-0.194280044396856\\
255	-0.194280044396856\\
256	-0.194280044396856\\
257	-0.194280044396856\\
258	-0.194280044396856\\
259	-0.194280044396856\\
260	-0.194280044396856\\
261	-0.194280044396856\\
262	-0.194280044396856\\
263	-0.194280044396856\\
264	-0.194280044396856\\
265	-0.194280044396856\\
266	-0.194280044396856\\
267	-0.194280044396856\\
268	-0.194280044396856\\
269	-0.194280044396856\\
270	-0.194280044396856\\
271	-0.194280044396856\\
272	-0.194280044396856\\
273	-0.194280044396856\\
274	-0.194280044396856\\
275	-0.194280044396856\\
276	-0.194280044396856\\
277	-0.194280044396856\\
278	-0.194280044396856\\
279	-0.194280044396856\\
280	-0.194280044396856\\
281	-0.194280044396856\\
282	-0.194280044396856\\
283	-0.194280044396856\\
284	-0.194280044396856\\
285	-0.194280044396856\\
286	-0.194280044396856\\
287	-0.194280044396856\\
288	-0.194280044396856\\
289	-0.194280044396856\\
290	-0.194280044396856\\
291	-0.194280044396856\\
292	-0.194280044396856\\
293	-0.194280044396856\\
294	-0.194280044396856\\
295	-0.194280044396856\\
296	-0.194280044396856\\
297	-0.194280044396856\\
298	-0.194280044396856\\
299	-0.194280044396856\\
300	-0.194280044396856\\
};
\addlegendentry{Skok U z 0.00 do -0.25}

\addplot[const plot, color=mycolor3] table[row sep=crcr] {%
1	0\\
2	0\\
3	0\\
4	0\\
5	0\\
6	0\\
7	0\\
8	0\\
9	0\\
10	0\\
11	0\\
12	0\\
13	0\\
14	0\\
15	0\\
16	0\\
17	0\\
18	0\\
19	0\\
20	0\\
21	0\\
22	0\\
23	0\\
24	0\\
25	0\\
26	0\\
27	0\\
28	0\\
29	0\\
30	0\\
31	0\\
32	0\\
33	0\\
34	0\\
35	0\\
36	0\\
37	0\\
38	0\\
39	0\\
40	0\\
41	0\\
42	0\\
43	0\\
44	0\\
45	0\\
46	0\\
47	0\\
48	0\\
49	0\\
50	0\\
51	0\\
52	0\\
53	0\\
54	0\\
55	0\\
56	0\\
57	0\\
58	0\\
59	0\\
60	0\\
61	0\\
62	0\\
63	0\\
64	0\\
65	0\\
66	0\\
67	0\\
68	0\\
69	0\\
70	0\\
71	0\\
72	0\\
73	0\\
74	0\\
75	0\\
76	0\\
77	0\\
78	0\\
79	0\\
80	0\\
81	0\\
82	0\\
83	0\\
84	0\\
85	0\\
86	0\\
87	0\\
88	0\\
89	0\\
90	0\\
91	0\\
92	0\\
93	0\\
94	0\\
95	0\\
96	0\\
97	0\\
98	0\\
99	0\\
100	0\\
101	0\\
102	0\\
103	0\\
104	0\\
105	0\\
106	0\\
107	0\\
108	0\\
109	0\\
110	0\\
111	0\\
112	0\\
113	0\\
114	0\\
115	0\\
116	0\\
117	0\\
118	0\\
119	0\\
120	0\\
121	0\\
122	0\\
123	0\\
124	0\\
125	0\\
126	0\\
127	0\\
128	0\\
129	0\\
130	0\\
131	0\\
132	0\\
133	0\\
134	0\\
135	0\\
136	0\\
137	0\\
138	0\\
139	0\\
140	0\\
141	0\\
142	0\\
143	0\\
144	0\\
145	0\\
146	0\\
147	0\\
148	0\\
149	0\\
150	0\\
151	0\\
152	0\\
153	0\\
154	0\\
155	0\\
156	0\\
157	0\\
158	0\\
159	0\\
160	0\\
161	0\\
162	0\\
163	0\\
164	0\\
165	0\\
166	0\\
167	0\\
168	0\\
169	0\\
170	0\\
171	0\\
172	0\\
173	0\\
174	0\\
175	0\\
176	0\\
177	0\\
178	0\\
179	0\\
180	0\\
181	0\\
182	0\\
183	0\\
184	0\\
185	0\\
186	0\\
187	0\\
188	0\\
189	0\\
190	0\\
191	0\\
192	0\\
193	0\\
194	0\\
195	0\\
196	0\\
197	0\\
198	0\\
199	0\\
200	0\\
201	0\\
202	0\\
203	0\\
204	0\\
205	0\\
206	0\\
207	0\\
208	0\\
209	0\\
210	0\\
211	0\\
212	0\\
213	0\\
214	0\\
215	0\\
216	0\\
217	0\\
218	0\\
219	0\\
220	0\\
221	0\\
222	0\\
223	0\\
224	0\\
225	0\\
226	0\\
227	0\\
228	0\\
229	0\\
230	0\\
231	0\\
232	0\\
233	0\\
234	0\\
235	0\\
236	0\\
237	0\\
238	0\\
239	0\\
240	0\\
241	0\\
242	0\\
243	0\\
244	0\\
245	0\\
246	0\\
247	0\\
248	0\\
249	0\\
250	0\\
251	0\\
252	0\\
253	0\\
254	0\\
255	0\\
256	0\\
257	0\\
258	0\\
259	0\\
260	0\\
261	0\\
262	0\\
263	0\\
264	0\\
265	0\\
266	0\\
267	0\\
268	0\\
269	0\\
270	0\\
271	0\\
272	0\\
273	0\\
274	0\\
275	0\\
276	0\\
277	0\\
278	0\\
279	0\\
280	0\\
281	0\\
282	0\\
283	0\\
284	0\\
285	0\\
286	0\\
287	0\\
288	0\\
289	0\\
290	0\\
291	0\\
292	0\\
293	0\\
294	0\\
295	0\\
296	0\\
297	0\\
298	0\\
299	0\\
300	0\\
};
\addlegendentry{Skok U z 0.00 do 0.00}

\addplot[const plot, color=mycolor4] table[row sep=crcr] {%
1	0\\
2	0\\
3	0\\
4	0\\
5	0\\
6	0\\
7	0\\
8	0\\
9	0\\
10	0\\
11	0\\
12	0\\
13	0\\
14	0\\
15	0.00916194129720854\\
16	0.0237109142051116\\
17	0.0347854477739337\\
18	0.0414098483215953\\
19	0.0448070287362194\\
20	0.0463239521732835\\
21	0.0468966103926382\\
22	0.0470569189157674\\
23	0.0470665175232664\\
24	0.047037056606585\\
25	0.047007921724263\\
26	0.0469883933986156\\
27	0.0469775416307898\\
28	0.046972307011315\\
29	0.0469701232168774\\
30	0.046969380787595\\
31	0.046969224712116\\
32	0.0469692594342373\\
33	0.0469693282748527\\
34	0.0469693827322709\\
35	0.0469694159507636\\
36	0.0469694332427358\\
37	0.0469694410832711\\
38	0.0469694441068266\\
39	0.0469694449934194\\
40	0.0469694450807196\\
41	0.046969444948964\\
42	0.0469694448080906\\
43	0.0469694447111181\\
44	0.0469694446563247\\
45	0.0469694446295051\\
46	0.0469694446181243\\
47	0.0469694446141454\\
48	0.046969444613232\\
49	0.046969444613348\\
50	0.0469694446136714\\
51	0.0469694446139384\\
52	0.0469694446141047\\
53	0.0469694446141926\\
54	0.046969444614233\\
55	0.046969444614249\\
56	0.0469694446142538\\
57	0.0469694446142545\\
58	0.0469694446142539\\
59	0.0469694446142532\\
60	0.0469694446142527\\
61	0.0469694446142525\\
62	0.0469694446142523\\
63	0.0469694446142523\\
64	0.0469694446142522\\
65	0.0469694446142522\\
66	0.0469694446142522\\
67	0.0469694446142522\\
68	0.0469694446142522\\
69	0.0469694446142522\\
70	0.0469694446142522\\
71	0.0469694446142522\\
72	0.0469694446142522\\
73	0.0469694446142522\\
74	0.0469694446142522\\
75	0.0469694446142522\\
76	0.0469694446142522\\
77	0.0469694446142522\\
78	0.0469694446142522\\
79	0.0469694446142522\\
80	0.0469694446142522\\
81	0.0469694446142522\\
82	0.0469694446142522\\
83	0.0469694446142522\\
84	0.0469694446142522\\
85	0.0469694446142522\\
86	0.0469694446142522\\
87	0.0469694446142522\\
88	0.0469694446142522\\
89	0.0469694446142522\\
90	0.0469694446142522\\
91	0.0469694446142522\\
92	0.0469694446142522\\
93	0.0469694446142522\\
94	0.0469694446142522\\
95	0.0469694446142522\\
96	0.0469694446142522\\
97	0.0469694446142522\\
98	0.0469694446142522\\
99	0.0469694446142522\\
100	0.0469694446142522\\
101	0.0469694446142522\\
102	0.0469694446142522\\
103	0.0469694446142522\\
104	0.0469694446142522\\
105	0.0469694446142522\\
106	0.0469694446142522\\
107	0.0469694446142522\\
108	0.0469694446142522\\
109	0.0469694446142522\\
110	0.0469694446142522\\
111	0.0469694446142522\\
112	0.0469694446142522\\
113	0.0469694446142522\\
114	0.0469694446142522\\
115	0.0469694446142522\\
116	0.0469694446142522\\
117	0.0469694446142522\\
118	0.0469694446142522\\
119	0.0469694446142522\\
120	0.0469694446142522\\
121	0.0469694446142522\\
122	0.0469694446142522\\
123	0.0469694446142522\\
124	0.0469694446142522\\
125	0.0469694446142522\\
126	0.0469694446142522\\
127	0.0469694446142522\\
128	0.0469694446142522\\
129	0.0469694446142522\\
130	0.0469694446142522\\
131	0.0469694446142522\\
132	0.0469694446142522\\
133	0.0469694446142522\\
134	0.0469694446142522\\
135	0.0469694446142522\\
136	0.0469694446142522\\
137	0.0469694446142522\\
138	0.0469694446142522\\
139	0.0469694446142522\\
140	0.0469694446142522\\
141	0.0469694446142522\\
142	0.0469694446142522\\
143	0.0469694446142522\\
144	0.0469694446142522\\
145	0.0469694446142522\\
146	0.0469694446142522\\
147	0.0469694446142522\\
148	0.0469694446142522\\
149	0.0469694446142522\\
150	0.0469694446142522\\
151	0.0469694446142522\\
152	0.0469694446142522\\
153	0.0469694446142522\\
154	0.0469694446142522\\
155	0.0469694446142522\\
156	0.0469694446142522\\
157	0.0469694446142522\\
158	0.0469694446142522\\
159	0.0469694446142522\\
160	0.0469694446142522\\
161	0.0469694446142522\\
162	0.0469694446142522\\
163	0.0469694446142522\\
164	0.0469694446142522\\
165	0.0469694446142522\\
166	0.0469694446142522\\
167	0.0469694446142522\\
168	0.0469694446142522\\
169	0.0469694446142522\\
170	0.0469694446142522\\
171	0.0469694446142522\\
172	0.0469694446142522\\
173	0.0469694446142522\\
174	0.0469694446142522\\
175	0.0469694446142522\\
176	0.0469694446142522\\
177	0.0469694446142522\\
178	0.0469694446142522\\
179	0.0469694446142522\\
180	0.0469694446142522\\
181	0.0469694446142522\\
182	0.0469694446142522\\
183	0.0469694446142522\\
184	0.0469694446142522\\
185	0.0469694446142522\\
186	0.0469694446142522\\
187	0.0469694446142522\\
188	0.0469694446142522\\
189	0.0469694446142522\\
190	0.0469694446142522\\
191	0.0469694446142522\\
192	0.0469694446142522\\
193	0.0469694446142522\\
194	0.0469694446142522\\
195	0.0469694446142522\\
196	0.0469694446142522\\
197	0.0469694446142522\\
198	0.0469694446142522\\
199	0.0469694446142522\\
200	0.0469694446142522\\
201	0.0469694446142522\\
202	0.0469694446142522\\
203	0.0469694446142522\\
204	0.0469694446142522\\
205	0.0469694446142522\\
206	0.0469694446142522\\
207	0.0469694446142522\\
208	0.0469694446142522\\
209	0.0469694446142522\\
210	0.0469694446142522\\
211	0.0469694446142522\\
212	0.0469694446142522\\
213	0.0469694446142522\\
214	0.0469694446142522\\
215	0.0469694446142522\\
216	0.0469694446142522\\
217	0.0469694446142522\\
218	0.0469694446142522\\
219	0.0469694446142522\\
220	0.0469694446142522\\
221	0.0469694446142522\\
222	0.0469694446142522\\
223	0.0469694446142522\\
224	0.0469694446142522\\
225	0.0469694446142522\\
226	0.0469694446142522\\
227	0.0469694446142522\\
228	0.0469694446142522\\
229	0.0469694446142522\\
230	0.0469694446142522\\
231	0.0469694446142522\\
232	0.0469694446142522\\
233	0.0469694446142522\\
234	0.0469694446142522\\
235	0.0469694446142522\\
236	0.0469694446142522\\
237	0.0469694446142522\\
238	0.0469694446142522\\
239	0.0469694446142522\\
240	0.0469694446142522\\
241	0.0469694446142522\\
242	0.0469694446142522\\
243	0.0469694446142522\\
244	0.0469694446142522\\
245	0.0469694446142522\\
246	0.0469694446142522\\
247	0.0469694446142522\\
248	0.0469694446142522\\
249	0.0469694446142522\\
250	0.0469694446142522\\
251	0.0469694446142522\\
252	0.0469694446142522\\
253	0.0469694446142522\\
254	0.0469694446142522\\
255	0.0469694446142522\\
256	0.0469694446142522\\
257	0.0469694446142522\\
258	0.0469694446142522\\
259	0.0469694446142522\\
260	0.0469694446142522\\
261	0.0469694446142522\\
262	0.0469694446142522\\
263	0.0469694446142522\\
264	0.0469694446142522\\
265	0.0469694446142522\\
266	0.0469694446142522\\
267	0.0469694446142522\\
268	0.0469694446142522\\
269	0.0469694446142522\\
270	0.0469694446142522\\
271	0.0469694446142522\\
272	0.0469694446142522\\
273	0.0469694446142522\\
274	0.0469694446142522\\
275	0.0469694446142522\\
276	0.0469694446142522\\
277	0.0469694446142522\\
278	0.0469694446142522\\
279	0.0469694446142522\\
280	0.0469694446142522\\
281	0.0469694446142522\\
282	0.0469694446142522\\
283	0.0469694446142522\\
284	0.0469694446142522\\
285	0.0469694446142522\\
286	0.0469694446142522\\
287	0.0469694446142522\\
288	0.0469694446142522\\
289	0.0469694446142522\\
290	0.0469694446142522\\
291	0.0469694446142522\\
292	0.0469694446142522\\
293	0.0469694446142522\\
294	0.0469694446142522\\
295	0.0469694446142522\\
296	0.0469694446142522\\
297	0.0469694446142522\\
298	0.0469694446142522\\
299	0.0469694446142522\\
300	0.0469694446142522\\
};
\addlegendentry{Skok U z 0.00 do 0.25}

\addplot[const plot, color=mycolor5] table[row sep=crcr] {%
1	0\\
2	0\\
3	0\\
4	0\\
5	0\\
6	0\\
7	0\\
8	0\\
9	0\\
10	0\\
11	0\\
12	0\\
13	0\\
14	0\\
15	0.0149374122807018\\
16	0.035320856777193\\
17	0.0488293794044997\\
18	0.0561147600020654\\
19	0.0596439430779413\\
20	0.0612356364241139\\
21	0.0619148017418959\\
22	0.062190952713034\\
23	0.0622981522140596\\
24	0.0623377779031105\\
25	0.0623516109313598\\
26	0.0623560896317763\\
27	0.0623573801404703\\
28	0.06235767363587\\
29	0.06235769714626\\
30	0.062357668800254\\
31	0.0623576429737223\\
32	0.0623576273557165\\
33	0.0623576192920019\\
34	0.062357615491721\\
35	0.0623576138125338\\
36	0.0623576131082947\\
37	0.062357612826517\\
38	0.0623576127189196\\
39	0.0623576126798786\\
40	0.0623576126665644\\
41	0.062357612662397\\
42	0.0623576126612666\\
43	0.0623576126610484\\
44	0.0623576126610581\\
45	0.0623576126611003\\
46	0.0623576126611317\\
47	0.0623576126611494\\
48	0.0623576126611583\\
49	0.0623576126611623\\
50	0.0623576126611641\\
51	0.0623576126611648\\
52	0.0623576126611651\\
53	0.0623576126611652\\
54	0.0623576126611652\\
55	0.0623576126611653\\
56	0.0623576126611652\\
57	0.0623576126611652\\
58	0.0623576126611652\\
59	0.0623576126611652\\
60	0.0623576126611652\\
61	0.0623576126611652\\
62	0.0623576126611652\\
63	0.0623576126611652\\
64	0.0623576126611652\\
65	0.0623576126611652\\
66	0.0623576126611652\\
67	0.0623576126611652\\
68	0.0623576126611652\\
69	0.0623576126611652\\
70	0.0623576126611652\\
71	0.0623576126611652\\
72	0.0623576126611652\\
73	0.0623576126611652\\
74	0.0623576126611652\\
75	0.0623576126611652\\
76	0.0623576126611652\\
77	0.0623576126611652\\
78	0.0623576126611652\\
79	0.0623576126611652\\
80	0.0623576126611652\\
81	0.0623576126611652\\
82	0.0623576126611652\\
83	0.0623576126611652\\
84	0.0623576126611652\\
85	0.0623576126611652\\
86	0.0623576126611652\\
87	0.0623576126611652\\
88	0.0623576126611652\\
89	0.0623576126611652\\
90	0.0623576126611652\\
91	0.0623576126611652\\
92	0.0623576126611652\\
93	0.0623576126611652\\
94	0.0623576126611652\\
95	0.0623576126611652\\
96	0.0623576126611652\\
97	0.0623576126611652\\
98	0.0623576126611652\\
99	0.0623576126611652\\
100	0.0623576126611652\\
101	0.0623576126611652\\
102	0.0623576126611652\\
103	0.0623576126611652\\
104	0.0623576126611652\\
105	0.0623576126611652\\
106	0.0623576126611652\\
107	0.0623576126611652\\
108	0.0623576126611652\\
109	0.0623576126611652\\
110	0.0623576126611652\\
111	0.0623576126611652\\
112	0.0623576126611652\\
113	0.0623576126611652\\
114	0.0623576126611652\\
115	0.0623576126611652\\
116	0.0623576126611652\\
117	0.0623576126611652\\
118	0.0623576126611652\\
119	0.0623576126611652\\
120	0.0623576126611652\\
121	0.0623576126611652\\
122	0.0623576126611652\\
123	0.0623576126611652\\
124	0.0623576126611652\\
125	0.0623576126611652\\
126	0.0623576126611652\\
127	0.0623576126611652\\
128	0.0623576126611652\\
129	0.0623576126611652\\
130	0.0623576126611652\\
131	0.0623576126611652\\
132	0.0623576126611652\\
133	0.0623576126611652\\
134	0.0623576126611652\\
135	0.0623576126611652\\
136	0.0623576126611652\\
137	0.0623576126611652\\
138	0.0623576126611652\\
139	0.0623576126611652\\
140	0.0623576126611652\\
141	0.0623576126611652\\
142	0.0623576126611652\\
143	0.0623576126611652\\
144	0.0623576126611652\\
145	0.0623576126611652\\
146	0.0623576126611652\\
147	0.0623576126611652\\
148	0.0623576126611652\\
149	0.0623576126611652\\
150	0.0623576126611652\\
151	0.0623576126611652\\
152	0.0623576126611652\\
153	0.0623576126611652\\
154	0.0623576126611652\\
155	0.0623576126611652\\
156	0.0623576126611652\\
157	0.0623576126611652\\
158	0.0623576126611652\\
159	0.0623576126611652\\
160	0.0623576126611652\\
161	0.0623576126611652\\
162	0.0623576126611652\\
163	0.0623576126611652\\
164	0.0623576126611652\\
165	0.0623576126611652\\
166	0.0623576126611652\\
167	0.0623576126611652\\
168	0.0623576126611652\\
169	0.0623576126611652\\
170	0.0623576126611652\\
171	0.0623576126611652\\
172	0.0623576126611652\\
173	0.0623576126611652\\
174	0.0623576126611652\\
175	0.0623576126611652\\
176	0.0623576126611652\\
177	0.0623576126611652\\
178	0.0623576126611652\\
179	0.0623576126611652\\
180	0.0623576126611652\\
181	0.0623576126611652\\
182	0.0623576126611652\\
183	0.0623576126611652\\
184	0.0623576126611652\\
185	0.0623576126611652\\
186	0.0623576126611652\\
187	0.0623576126611652\\
188	0.0623576126611652\\
189	0.0623576126611652\\
190	0.0623576126611652\\
191	0.0623576126611652\\
192	0.0623576126611652\\
193	0.0623576126611652\\
194	0.0623576126611652\\
195	0.0623576126611652\\
196	0.0623576126611652\\
197	0.0623576126611652\\
198	0.0623576126611652\\
199	0.0623576126611652\\
200	0.0623576126611652\\
201	0.0623576126611652\\
202	0.0623576126611652\\
203	0.0623576126611652\\
204	0.0623576126611652\\
205	0.0623576126611652\\
206	0.0623576126611652\\
207	0.0623576126611652\\
208	0.0623576126611652\\
209	0.0623576126611652\\
210	0.0623576126611652\\
211	0.0623576126611652\\
212	0.0623576126611652\\
213	0.0623576126611652\\
214	0.0623576126611652\\
215	0.0623576126611652\\
216	0.0623576126611652\\
217	0.0623576126611652\\
218	0.0623576126611652\\
219	0.0623576126611652\\
220	0.0623576126611652\\
221	0.0623576126611652\\
222	0.0623576126611652\\
223	0.0623576126611652\\
224	0.0623576126611652\\
225	0.0623576126611652\\
226	0.0623576126611652\\
227	0.0623576126611652\\
228	0.0623576126611652\\
229	0.0623576126611652\\
230	0.0623576126611652\\
231	0.0623576126611652\\
232	0.0623576126611652\\
233	0.0623576126611652\\
234	0.0623576126611652\\
235	0.0623576126611652\\
236	0.0623576126611652\\
237	0.0623576126611652\\
238	0.0623576126611652\\
239	0.0623576126611652\\
240	0.0623576126611652\\
241	0.0623576126611652\\
242	0.0623576126611652\\
243	0.0623576126611652\\
244	0.0623576126611652\\
245	0.0623576126611652\\
246	0.0623576126611652\\
247	0.0623576126611652\\
248	0.0623576126611652\\
249	0.0623576126611652\\
250	0.0623576126611652\\
251	0.0623576126611652\\
252	0.0623576126611652\\
253	0.0623576126611652\\
254	0.0623576126611652\\
255	0.0623576126611652\\
256	0.0623576126611652\\
257	0.0623576126611652\\
258	0.0623576126611652\\
259	0.0623576126611652\\
260	0.0623576126611652\\
261	0.0623576126611652\\
262	0.0623576126611652\\
263	0.0623576126611652\\
264	0.0623576126611652\\
265	0.0623576126611652\\
266	0.0623576126611652\\
267	0.0623576126611652\\
268	0.0623576126611652\\
269	0.0623576126611652\\
270	0.0623576126611652\\
271	0.0623576126611652\\
272	0.0623576126611652\\
273	0.0623576126611652\\
274	0.0623576126611652\\
275	0.0623576126611652\\
276	0.0623576126611652\\
277	0.0623576126611652\\
278	0.0623576126611652\\
279	0.0623576126611652\\
280	0.0623576126611652\\
281	0.0623576126611652\\
282	0.0623576126611652\\
283	0.0623576126611652\\
284	0.0623576126611652\\
285	0.0623576126611652\\
286	0.0623576126611652\\
287	0.0623576126611652\\
288	0.0623576126611652\\
289	0.0623576126611652\\
290	0.0623576126611652\\
291	0.0623576126611652\\
292	0.0623576126611652\\
293	0.0623576126611652\\
294	0.0623576126611652\\
295	0.0623576126611652\\
296	0.0623576126611652\\
297	0.0623576126611652\\
298	0.0623576126611652\\
299	0.0623576126611652\\
300	0.0623576126611652\\
};
\addlegendentry{Skok U z 0.00 do 0.50}

\addplot[const plot, color=mycolor6] table[row sep=crcr] {%
1	0\\
2	0\\
3	0\\
4	0\\
5	0\\
6	0\\
7	0\\
8	0\\
9	0\\
10	0\\
11	0\\
12	0\\
13	0\\
14	0\\
15	0.0199798481468771\\
16	0.0447848083375242\\
17	0.0601297465363197\\
18	0.0681153267352949\\
19	0.0719760077764885\\
20	0.0737738156815136\\
21	0.0745938026972925\\
22	0.0749633286294141\\
23	0.0751286671851614\\
24	0.0752023261210796\\
25	0.075235055052244\\
26	0.0752495740370943\\
27	0.0752560084486156\\
28	0.0752588582538279\\
29	0.0752601199569219\\
30	0.0752606784240591\\
31	0.075260925582425\\
32	0.075261034956452\\
33	0.0752610833546389\\
34	0.0752611047701843\\
35	0.0752611142460742\\
36	0.0752611184388843\\
37	0.0752611202940678\\
38	0.0752611211149227\\
39	0.0752611214781216\\
40	0.0752611216388238\\
41	0.0752611217099285\\
42	0.0752611217413897\\
43	0.0752611217553101\\
44	0.0752611217614694\\
45	0.0752611217641946\\
46	0.0752611217654004\\
47	0.075261121765934\\
48	0.07526112176617\\
49	0.0752611217662745\\
50	0.0752611217663207\\
51	0.0752611217663411\\
52	0.0752611217663502\\
53	0.0752611217663542\\
54	0.075261121766356\\
55	0.0752611217663567\\
56	0.0752611217663571\\
57	0.0752611217663572\\
58	0.0752611217663573\\
59	0.0752611217663573\\
60	0.0752611217663573\\
61	0.0752611217663573\\
62	0.0752611217663573\\
63	0.0752611217663573\\
64	0.0752611217663573\\
65	0.0752611217663573\\
66	0.0752611217663573\\
67	0.0752611217663573\\
68	0.0752611217663573\\
69	0.0752611217663573\\
70	0.0752611217663573\\
71	0.0752611217663573\\
72	0.0752611217663573\\
73	0.0752611217663573\\
74	0.0752611217663573\\
75	0.0752611217663573\\
76	0.0752611217663573\\
77	0.0752611217663573\\
78	0.0752611217663573\\
79	0.0752611217663573\\
80	0.0752611217663573\\
81	0.0752611217663573\\
82	0.0752611217663573\\
83	0.0752611217663573\\
84	0.0752611217663573\\
85	0.0752611217663573\\
86	0.0752611217663573\\
87	0.0752611217663573\\
88	0.0752611217663573\\
89	0.0752611217663573\\
90	0.0752611217663573\\
91	0.0752611217663573\\
92	0.0752611217663573\\
93	0.0752611217663573\\
94	0.0752611217663573\\
95	0.0752611217663573\\
96	0.0752611217663573\\
97	0.0752611217663573\\
98	0.0752611217663573\\
99	0.0752611217663573\\
100	0.0752611217663573\\
101	0.0752611217663573\\
102	0.0752611217663573\\
103	0.0752611217663573\\
104	0.0752611217663573\\
105	0.0752611217663573\\
106	0.0752611217663573\\
107	0.0752611217663573\\
108	0.0752611217663573\\
109	0.0752611217663573\\
110	0.0752611217663573\\
111	0.0752611217663573\\
112	0.0752611217663573\\
113	0.0752611217663573\\
114	0.0752611217663573\\
115	0.0752611217663573\\
116	0.0752611217663573\\
117	0.0752611217663573\\
118	0.0752611217663573\\
119	0.0752611217663573\\
120	0.0752611217663573\\
121	0.0752611217663573\\
122	0.0752611217663573\\
123	0.0752611217663573\\
124	0.0752611217663573\\
125	0.0752611217663573\\
126	0.0752611217663573\\
127	0.0752611217663573\\
128	0.0752611217663573\\
129	0.0752611217663573\\
130	0.0752611217663573\\
131	0.0752611217663573\\
132	0.0752611217663573\\
133	0.0752611217663573\\
134	0.0752611217663573\\
135	0.0752611217663573\\
136	0.0752611217663573\\
137	0.0752611217663573\\
138	0.0752611217663573\\
139	0.0752611217663573\\
140	0.0752611217663573\\
141	0.0752611217663573\\
142	0.0752611217663573\\
143	0.0752611217663573\\
144	0.0752611217663573\\
145	0.0752611217663573\\
146	0.0752611217663573\\
147	0.0752611217663573\\
148	0.0752611217663573\\
149	0.0752611217663573\\
150	0.0752611217663573\\
151	0.0752611217663573\\
152	0.0752611217663573\\
153	0.0752611217663573\\
154	0.0752611217663573\\
155	0.0752611217663573\\
156	0.0752611217663573\\
157	0.0752611217663573\\
158	0.0752611217663573\\
159	0.0752611217663573\\
160	0.0752611217663573\\
161	0.0752611217663573\\
162	0.0752611217663573\\
163	0.0752611217663573\\
164	0.0752611217663573\\
165	0.0752611217663573\\
166	0.0752611217663573\\
167	0.0752611217663573\\
168	0.0752611217663573\\
169	0.0752611217663573\\
170	0.0752611217663573\\
171	0.0752611217663573\\
172	0.0752611217663573\\
173	0.0752611217663573\\
174	0.0752611217663573\\
175	0.0752611217663573\\
176	0.0752611217663573\\
177	0.0752611217663573\\
178	0.0752611217663573\\
179	0.0752611217663573\\
180	0.0752611217663573\\
181	0.0752611217663573\\
182	0.0752611217663573\\
183	0.0752611217663573\\
184	0.0752611217663573\\
185	0.0752611217663573\\
186	0.0752611217663573\\
187	0.0752611217663573\\
188	0.0752611217663573\\
189	0.0752611217663573\\
190	0.0752611217663573\\
191	0.0752611217663573\\
192	0.0752611217663573\\
193	0.0752611217663573\\
194	0.0752611217663573\\
195	0.0752611217663573\\
196	0.0752611217663573\\
197	0.0752611217663573\\
198	0.0752611217663573\\
199	0.0752611217663573\\
200	0.0752611217663573\\
201	0.0752611217663573\\
202	0.0752611217663573\\
203	0.0752611217663573\\
204	0.0752611217663573\\
205	0.0752611217663573\\
206	0.0752611217663573\\
207	0.0752611217663573\\
208	0.0752611217663573\\
209	0.0752611217663573\\
210	0.0752611217663573\\
211	0.0752611217663573\\
212	0.0752611217663573\\
213	0.0752611217663573\\
214	0.0752611217663573\\
215	0.0752611217663573\\
216	0.0752611217663573\\
217	0.0752611217663573\\
218	0.0752611217663573\\
219	0.0752611217663573\\
220	0.0752611217663573\\
221	0.0752611217663573\\
222	0.0752611217663573\\
223	0.0752611217663573\\
224	0.0752611217663573\\
225	0.0752611217663573\\
226	0.0752611217663573\\
227	0.0752611217663573\\
228	0.0752611217663573\\
229	0.0752611217663573\\
230	0.0752611217663573\\
231	0.0752611217663573\\
232	0.0752611217663573\\
233	0.0752611217663573\\
234	0.0752611217663573\\
235	0.0752611217663573\\
236	0.0752611217663573\\
237	0.0752611217663573\\
238	0.0752611217663573\\
239	0.0752611217663573\\
240	0.0752611217663573\\
241	0.0752611217663573\\
242	0.0752611217663573\\
243	0.0752611217663573\\
244	0.0752611217663573\\
245	0.0752611217663573\\
246	0.0752611217663573\\
247	0.0752611217663573\\
248	0.0752611217663573\\
249	0.0752611217663573\\
250	0.0752611217663573\\
251	0.0752611217663573\\
252	0.0752611217663573\\
253	0.0752611217663573\\
254	0.0752611217663573\\
255	0.0752611217663573\\
256	0.0752611217663573\\
257	0.0752611217663573\\
258	0.0752611217663573\\
259	0.0752611217663573\\
260	0.0752611217663573\\
261	0.0752611217663573\\
262	0.0752611217663573\\
263	0.0752611217663573\\
264	0.0752611217663573\\
265	0.0752611217663573\\
266	0.0752611217663573\\
267	0.0752611217663573\\
268	0.0752611217663573\\
269	0.0752611217663573\\
270	0.0752611217663573\\
271	0.0752611217663573\\
272	0.0752611217663573\\
273	0.0752611217663573\\
274	0.0752611217663573\\
275	0.0752611217663573\\
276	0.0752611217663573\\
277	0.0752611217663573\\
278	0.0752611217663573\\
279	0.0752611217663573\\
280	0.0752611217663573\\
281	0.0752611217663573\\
282	0.0752611217663573\\
283	0.0752611217663573\\
284	0.0752611217663573\\
285	0.0752611217663573\\
286	0.0752611217663573\\
287	0.0752611217663573\\
288	0.0752611217663573\\
289	0.0752611217663573\\
290	0.0752611217663573\\
291	0.0752611217663573\\
292	0.0752611217663573\\
293	0.0752611217663573\\
294	0.0752611217663573\\
295	0.0752611217663573\\
296	0.0752611217663573\\
297	0.0752611217663573\\
298	0.0752611217663573\\
299	0.0752611217663573\\
300	0.0752611217663573\\
};
\addlegendentry{Skok U z 0.00 do 0.75}

\addplot[const plot, color=mycolor7] table[row sep=crcr] {%
1	0\\
2	0\\
3	0\\
4	0\\
5	0\\
6	0\\
7	0\\
8	0\\
9	0\\
10	0\\
11	0\\
12	0\\
13	0\\
14	0\\
15	0.0249067777777778\\
16	0.0540264030651852\\
17	0.0713864999938143\\
18	0.0802973721942573\\
19	0.0846280344990221\\
20	0.0866849533107856\\
21	0.0876520115004241\\
22	0.0881045698142786\\
23	0.0883159045205792\\
24	0.0884144960135448\\
25	0.0884604697743512\\
26	0.0884819030636113\\
27	0.0884918944337098\\
28	0.0884965518120525\\
29	0.0884987227571013\\
30	0.0884997346904105\\
31	0.0885002063764041\\
32	0.0885004262399126\\
33	0.0885005287231659\\
34	0.0885005764928595\\
35	0.0885005987593575\\
36	0.0885006091382576\\
37	0.0885006139760891\\
38	0.0885006162311077\\
39	0.0885006172822209\\
40	0.0885006177721677\\
41	0.0885006180005426\\
42	0.0885006181069931\\
43	0.0885006181566119\\
44	0.0885006181797404\\
45	0.088500618190521\\
46	0.0885006181955462\\
47	0.0885006181978885\\
48	0.0885006181989803\\
49	0.0885006181994892\\
50	0.0885006181997264\\
51	0.088500618199837\\
52	0.0885006181998885\\
53	0.0885006181999125\\
54	0.0885006181999237\\
55	0.088500618199929\\
56	0.0885006181999314\\
57	0.0885006181999325\\
58	0.0885006181999331\\
59	0.0885006181999333\\
60	0.0885006181999334\\
61	0.0885006181999334\\
62	0.0885006181999335\\
63	0.0885006181999335\\
64	0.0885006181999335\\
65	0.0885006181999335\\
66	0.0885006181999335\\
67	0.0885006181999335\\
68	0.0885006181999335\\
69	0.0885006181999335\\
70	0.0885006181999335\\
71	0.0885006181999335\\
72	0.0885006181999335\\
73	0.0885006181999335\\
74	0.0885006181999335\\
75	0.0885006181999335\\
76	0.0885006181999335\\
77	0.0885006181999335\\
78	0.0885006181999335\\
79	0.0885006181999335\\
80	0.0885006181999335\\
81	0.0885006181999335\\
82	0.0885006181999335\\
83	0.0885006181999335\\
84	0.0885006181999335\\
85	0.0885006181999335\\
86	0.0885006181999335\\
87	0.0885006181999335\\
88	0.0885006181999335\\
89	0.0885006181999335\\
90	0.0885006181999335\\
91	0.0885006181999335\\
92	0.0885006181999335\\
93	0.0885006181999335\\
94	0.0885006181999335\\
95	0.0885006181999335\\
96	0.0885006181999335\\
97	0.0885006181999335\\
98	0.0885006181999335\\
99	0.0885006181999335\\
100	0.0885006181999335\\
101	0.0885006181999335\\
102	0.0885006181999335\\
103	0.0885006181999335\\
104	0.0885006181999335\\
105	0.0885006181999335\\
106	0.0885006181999335\\
107	0.0885006181999335\\
108	0.0885006181999335\\
109	0.0885006181999335\\
110	0.0885006181999335\\
111	0.0885006181999335\\
112	0.0885006181999335\\
113	0.0885006181999335\\
114	0.0885006181999335\\
115	0.0885006181999335\\
116	0.0885006181999335\\
117	0.0885006181999335\\
118	0.0885006181999335\\
119	0.0885006181999335\\
120	0.0885006181999335\\
121	0.0885006181999335\\
122	0.0885006181999335\\
123	0.0885006181999335\\
124	0.0885006181999335\\
125	0.0885006181999335\\
126	0.0885006181999335\\
127	0.0885006181999335\\
128	0.0885006181999335\\
129	0.0885006181999335\\
130	0.0885006181999335\\
131	0.0885006181999335\\
132	0.0885006181999335\\
133	0.0885006181999335\\
134	0.0885006181999335\\
135	0.0885006181999335\\
136	0.0885006181999335\\
137	0.0885006181999335\\
138	0.0885006181999335\\
139	0.0885006181999335\\
140	0.0885006181999335\\
141	0.0885006181999335\\
142	0.0885006181999335\\
143	0.0885006181999335\\
144	0.0885006181999335\\
145	0.0885006181999335\\
146	0.0885006181999335\\
147	0.0885006181999335\\
148	0.0885006181999335\\
149	0.0885006181999335\\
150	0.0885006181999335\\
151	0.0885006181999335\\
152	0.0885006181999335\\
153	0.0885006181999335\\
154	0.0885006181999335\\
155	0.0885006181999335\\
156	0.0885006181999335\\
157	0.0885006181999335\\
158	0.0885006181999335\\
159	0.0885006181999335\\
160	0.0885006181999335\\
161	0.0885006181999335\\
162	0.0885006181999335\\
163	0.0885006181999335\\
164	0.0885006181999335\\
165	0.0885006181999335\\
166	0.0885006181999335\\
167	0.0885006181999335\\
168	0.0885006181999335\\
169	0.0885006181999335\\
170	0.0885006181999335\\
171	0.0885006181999335\\
172	0.0885006181999335\\
173	0.0885006181999335\\
174	0.0885006181999335\\
175	0.0885006181999335\\
176	0.0885006181999335\\
177	0.0885006181999335\\
178	0.0885006181999335\\
179	0.0885006181999335\\
180	0.0885006181999335\\
181	0.0885006181999335\\
182	0.0885006181999335\\
183	0.0885006181999335\\
184	0.0885006181999335\\
185	0.0885006181999335\\
186	0.0885006181999335\\
187	0.0885006181999335\\
188	0.0885006181999335\\
189	0.0885006181999335\\
190	0.0885006181999335\\
191	0.0885006181999335\\
192	0.0885006181999335\\
193	0.0885006181999335\\
194	0.0885006181999335\\
195	0.0885006181999335\\
196	0.0885006181999335\\
197	0.0885006181999335\\
198	0.0885006181999335\\
199	0.0885006181999335\\
200	0.0885006181999335\\
201	0.0885006181999335\\
202	0.0885006181999335\\
203	0.0885006181999335\\
204	0.0885006181999335\\
205	0.0885006181999335\\
206	0.0885006181999335\\
207	0.0885006181999335\\
208	0.0885006181999335\\
209	0.0885006181999335\\
210	0.0885006181999335\\
211	0.0885006181999335\\
212	0.0885006181999335\\
213	0.0885006181999335\\
214	0.0885006181999335\\
215	0.0885006181999335\\
216	0.0885006181999335\\
217	0.0885006181999335\\
218	0.0885006181999335\\
219	0.0885006181999335\\
220	0.0885006181999335\\
221	0.0885006181999335\\
222	0.0885006181999335\\
223	0.0885006181999335\\
224	0.0885006181999335\\
225	0.0885006181999335\\
226	0.0885006181999335\\
227	0.0885006181999335\\
228	0.0885006181999335\\
229	0.0885006181999335\\
230	0.0885006181999335\\
231	0.0885006181999335\\
232	0.0885006181999335\\
233	0.0885006181999335\\
234	0.0885006181999335\\
235	0.0885006181999335\\
236	0.0885006181999335\\
237	0.0885006181999335\\
238	0.0885006181999335\\
239	0.0885006181999335\\
240	0.0885006181999335\\
241	0.0885006181999335\\
242	0.0885006181999335\\
243	0.0885006181999335\\
244	0.0885006181999335\\
245	0.0885006181999335\\
246	0.0885006181999335\\
247	0.0885006181999335\\
248	0.0885006181999335\\
249	0.0885006181999335\\
250	0.0885006181999335\\
251	0.0885006181999335\\
252	0.0885006181999335\\
253	0.0885006181999335\\
254	0.0885006181999335\\
255	0.0885006181999335\\
256	0.0885006181999335\\
257	0.0885006181999335\\
258	0.0885006181999335\\
259	0.0885006181999335\\
260	0.0885006181999335\\
261	0.0885006181999335\\
262	0.0885006181999335\\
263	0.0885006181999335\\
264	0.0885006181999335\\
265	0.0885006181999335\\
266	0.0885006181999335\\
267	0.0885006181999335\\
268	0.0885006181999335\\
269	0.0885006181999335\\
270	0.0885006181999335\\
271	0.0885006181999335\\
272	0.0885006181999335\\
273	0.0885006181999335\\
274	0.0885006181999335\\
275	0.0885006181999335\\
276	0.0885006181999335\\
277	0.0885006181999335\\
278	0.0885006181999335\\
279	0.0885006181999335\\
280	0.0885006181999335\\
281	0.0885006181999335\\
282	0.0885006181999335\\
283	0.0885006181999335\\
284	0.0885006181999335\\
285	0.0885006181999335\\
286	0.0885006181999335\\
287	0.0885006181999335\\
288	0.0885006181999335\\
289	0.0885006181999335\\
290	0.0885006181999335\\
291	0.0885006181999335\\
292	0.0885006181999335\\
293	0.0885006181999335\\
294	0.0885006181999335\\
295	0.0885006181999335\\
296	0.0885006181999335\\
297	0.0885006181999335\\
298	0.0885006181999335\\
299	0.0885006181999335\\
300	0.0885006181999335\\
};
\addlegendentry{Skok U z 0.00 do 1.00}

\end{axis}
\end{tikzpicture}%
\caption{Wykresy odpowiedzi skokowych}
\end{figure}

Jak widać wartość skoku na wyjściu jest proporcjonalna wartości skoku wejścia.




\chapter{Odpowiedzi skokowe dla DMC}

\begin{figure}[H]
\centering
% This file was created by matlab2tikz.
%
%The latest updates can be retrieved from
%  http://www.mathworks.com/matlabcentral/fileexchange/22022-matlab2tikz-matlab2tikz
%where you can also make suggestions and rate matlab2tikz.
%
\definecolor{mycolor1}{rgb}{0.00000,0.44700,0.74100}%
%
\begin{tikzpicture}

\begin{axis}[%
width=4.521in,
height=3.566in,
at={(0.758in,0.481in)},
scale only axis,
xmin=0,
xmax=250,
xlabel style={font=\color{white!15!black}},
xlabel={k},
ymin=0,
ymax=25,
ylabel style={font=\color{white!15!black}},
ylabel={Y(k)},
axis background/.style={fill=white}
]
\addplot[const plot, color=mycolor1, forget plot] table[row sep=crcr] {%
1	0.412834169999999\\
2	0.778953825762\\
3	1.22552522529589\\
4	1.73632424400836\\
5	2.2973725681653\\
6	2.89666065333908\\
7	3.67178323313577\\
8	4.45732700941061\\
9	5.24635122393158\\
10	6.03303546347107\\
11	6.81253378871436\\
12	7.58084641517885\\
13	8.33470690777239\\
14	9.07148308327445\\
15	9.78909002135876\\
16	10.4859137677517\\
17	11.1607444753708\\
18	11.8127178731484\\
19	12.4412640797858\\
20	13.046062892749\\
21	13.6270047830398\\
22	14.184156915098\\
23	14.7177335899142\\
24	15.2280705791815\\
25	15.7156028801154\\
26	16.1808454753177\\
27	16.6243767305496\\
28	17.0468241062239\\
29	17.4488518964494\\
30	17.8311507431239\\
31	18.194428702367\\
32	18.5394036669541\\
33	18.8667969717402\\
34	19.1773280297007\\
35	19.4717098644647\\
36	19.7506454213526\\
37	20.0148245531913\\
38	20.2649215897808\\
39	20.5015934110202\\
40	20.7254779535271\\
41	20.937193089265\\
42	21.1373358223465\\
43	21.3264817569342\\
44	21.505184795115\\
45	21.673977028868\\
46	21.8333687948682\\
47	21.983848864934\\
48	22.1258847485047\\
49	22.2599230866795\\
50	22.3863901201136\\
51	22.5056922154907\\
52	22.6182164374214\\
53	22.7243311544802\\
54	22.8243866697262\\
55	22.9187158674818\\
56	23.0076348693884\\
57	23.0914436938494\\
58	23.1704269139164\\
59	23.2448543094999\\
60	23.3149815105027\\
61	23.3810506280955\\
62	23.4432908718881\\
63	23.5019191512178\\
64	23.5571406591723\\
65	23.6091494383078\\
66	23.658128927316\\
67	23.7042524881435\\
68	23.7476839132784\\
69	23.7885779130991\\
70	23.8270805833273\\
71	23.8633298527562\\
72	23.8974559115258\\
73	23.9295816203016\\
74	23.9598229007815\\
75	23.9882891080122\\
76	24.0150833850333\\
77	24.0403030004047\\
78	24.064039669191\\
79	24.0863798579946\\
80	24.1074050746372\\
81	24.1271921430916\\
82	24.1458134642665\\
83	24.1633372632405\\
84	24.179827823533\\
85	24.1953457089908\\
86	24.2099479738547\\
87	24.2236883615561\\
88	24.2366174927793\\
89	24.2487830433066\\
90	24.2602299121465\\
91	24.2710003804292\\
92	24.2811342615325\\
93	24.2906690428849\\
94	24.2996400198743\\
95	24.3080804222725\\
96	24.3160215335675\\
97	24.3234928035784\\
98	24.3305219547112\\
99	24.3371350821961\\
100	24.3433567486312\\
101	24.3492100731427\\
102	24.3547168154546\\
103	24.3598974551478\\
104	24.3647712663749\\
105	24.3693563882808\\
106	24.3736698913707\\
107	24.377727840049\\
108	24.3815453515461\\
109	24.3851366514363\\
110	24.3885151259376\\
111	24.3916933711773\\
112	24.3946832395954\\
113	24.3974958836484\\
114	24.4001417969673\\
115	24.4026308531171\\
116	24.4049723420931\\
117	24.4071750046853\\
118	24.4092470648343\\
119	24.4111962600926\\
120	24.4130298703024\\
121	24.4147547445927\\
122	24.4163773267922\\
123	24.4179036793511\\
124	24.4193395058582\\
125	24.4206901722344\\
126	24.4219607266797\\
127	24.4231559184477\\
128	24.4242802155143\\
129	24.4253378212059\\
130	24.4263326898489\\
131	24.4272685414954\\
132	24.4281488757823\\
133	24.4289769849714\\
134	24.4297559662215\\
135	24.4304887331354\\
136	24.4311780266248\\
137	24.4318264251348\\
138	24.4324363542631\\
139	24.4330100958117\\
140	24.433549796303\\
141	24.434057474993\\
142	24.4345350314101\\
143	24.4349842524485\\
144	24.4354068190422\\
145	24.4358043124429\\
146	24.436178220128\\
147	24.4365299413583\\
148	24.4368607924057\\
149	24.4371720114737\\
150	24.4374647633239\\
151	24.4377401436306\\
152	24.4379991830767\\
153	24.4382428512066\\
154	24.4384720600515\\
155	24.43868766754\\
156	24.4388904807055\\
157	24.4390812587054\\
158	24.4392607156593\\
159	24.439429523321\\
160	24.4395883135904\\
161	24.4397376808774\\
162	24.4398781843261\\
163	24.4400103499059\\
164	24.4401346723795\\
165	24.4402516171543\\
166	24.4403616220233\\
167	24.4404650988038\\
168	24.4405624348776\\
169	24.4406539946413\\
170	24.4407401208696\\
171	24.440821135998\\
172	24.4408973433296\\
173	24.4409690281702\\
174	24.4410364588959\\
175	24.441099887958\\
176	24.4411595528271\\
177	24.4412156768828\\
178	24.441268470249\\
179	24.441318130581\\
180	24.4413648438049\\
181	24.4414087848139\\
182	24.4414501181226\\
183	24.4414889984832\\
184	24.4415255714645\\
185	24.4415599739969\\
186	24.441592334885\\
187	24.4416227752898\\
188	24.4416514091822\\
189	24.4416783437698\\
190	24.441703679898\\
191	24.4417275124277\\
192	24.4417499305906\\
193	24.4417710183227\\
194	24.441790854579\\
195	24.4418095136292\\
196	24.4418270653353\\
197	24.4418435754133\\
198	24.4418591056794\\
199	24.4418737142815\\
200	24.4418874559164\\
201	24.4419003820351\\
202	24.4419125410352\\
203	24.4419239784421\\
204	24.4419347370796\\
205	24.4419448572301\\
206	24.4419543767851\\
207	24.4419633313879\\
208	24.4419717545661\\
209	24.4419796778578\\
210	24.4419871309295\\
211	24.4419941416869\\
212	24.4420007363798\\
213	24.4420069397001\\
214	24.4420127748745\\
215	24.4420182637509\\
216	24.442023426881\\
217	24.4420282835966\\
218	24.4420328520824\\
219	24.4420371494439\\
220	24.4420411917712\\
221	24.4420449941998\\
222	24.4420485709669\\
223	24.4420519354647\\
224	24.4420551002906\\
225	24.4420580772946\\
226	24.4420608776231\\
227	24.4420635117613\\
228	24.442065989572\\
229	24.4420683203328\\
230	24.4420705127705\\
231	24.4420725750941\\
232	24.4420745150256\\
233	24.4420763398285\\
234	24.4420780563352\\
235	24.4420796709728\\
236	24.4420811897869\\
237	24.4420826184643\\
238	24.4420839623543\\
239	24.4420852264887\\
240	24.4420864156009\\
241	24.4420875341431\\
242	24.4420885863034\\
243	24.4420895760214\\
244	24.4420905070028\\
245	24.4420913827335\\
246	24.4420922064923\\
247	24.4420929813637\\
248	24.442093710249\\
249	24.4420943958773\\
250	24.4420950408157\\
};
\end{axis}
\end{tikzpicture}%
\caption{}
\end{figure}

%\begin{figure}[H]
%\centering
%% This file was created by matlab2tikz.
%
%The latest updates can be retrieved from
%  http://www.mathworks.com/matlabcentral/fileexchange/22022-matlab2tikz-matlab2tikz
%where you can also make suggestions and rate matlab2tikz.
%
\definecolor{mycolor1}{rgb}{0.00000,0.44700,0.74100}%
\definecolor{mycolor2}{rgb}{0.85000,0.32500,0.09800}%
%
\begin{tikzpicture}

\begin{axis}[%
width=4.521in,
height=1.493in,
at={(0.758in,2.554in)},
scale only axis,
xmin=0,
xmax=300,
xlabel style={font=\color{white!15!black}},
xlabel={k},
ymin=0,
ymax=2,
axis background/.style={fill=white},
title style={font=\bfseries},
title={Warto�� sterowania},
legend style={legend cell align=left, align=left, draw=white!15!black}
]
\addplot[const plot, color=mycolor1] table[row sep=crcr] {%
1	0\\
2	0\\
3	0\\
4	0\\
5	0\\
6	0\\
7	0\\
8	0\\
9	0\\
10	0\\
11	0\\
12	0\\
13	0\\
14	0\\
15	0\\
16	0\\
17	0\\
18	0\\
19	0\\
20	0\\
21	0\\
22	0\\
23	0\\
24	0\\
25	0\\
26	0\\
27	0\\
28	0\\
29	0\\
30	0\\
31	0\\
32	0\\
33	0\\
34	0\\
35	0\\
36	0\\
37	0\\
38	0\\
39	0\\
40	0\\
41	0\\
42	0\\
43	0\\
44	0\\
45	0\\
46	0\\
47	0\\
48	0\\
49	0\\
50	0\\
51	0\\
52	0\\
53	0\\
54	0\\
55	0\\
56	0\\
57	0\\
58	0\\
59	0\\
60	0\\
61	0\\
62	0\\
63	0\\
64	0\\
65	0\\
66	0\\
67	0\\
68	0\\
69	0\\
70	0\\
71	0\\
72	0\\
73	0\\
74	0\\
75	0\\
76	0\\
77	0\\
78	0\\
79	0\\
80	0\\
81	0\\
82	0\\
83	0\\
84	0\\
85	0\\
86	0\\
87	0\\
88	0\\
89	0\\
90	0\\
91	0\\
92	0\\
93	0\\
94	0\\
95	0\\
96	0\\
97	0\\
98	0\\
99	0\\
100	0\\
101	0\\
102	0\\
103	0\\
104	0\\
105	0\\
106	0\\
107	0\\
108	0\\
109	0\\
110	0\\
111	0\\
112	0\\
113	0\\
114	0\\
115	0\\
116	0\\
117	0\\
118	0\\
119	0\\
120	0\\
121	0\\
122	0\\
123	0\\
124	0\\
125	0\\
126	0\\
127	0\\
128	0\\
129	0\\
130	0\\
131	0\\
132	0\\
133	0\\
134	0\\
135	0\\
136	0\\
137	0\\
138	0\\
139	0\\
140	0\\
141	0\\
142	0\\
143	0\\
144	0\\
145	0\\
146	0\\
147	0\\
148	0\\
149	0\\
150	0\\
151	0\\
152	0\\
153	0\\
154	0\\
155	0\\
156	0\\
157	0\\
158	0\\
159	0\\
160	0\\
161	0\\
162	0\\
163	0\\
164	0\\
165	0\\
166	0\\
167	0\\
168	0\\
169	0\\
170	0\\
171	0\\
172	0\\
173	0\\
174	0\\
175	0\\
176	0\\
177	0\\
178	0\\
179	0\\
180	0\\
181	0\\
182	0\\
183	0\\
184	0\\
185	0\\
186	0\\
187	0\\
188	0\\
189	0\\
190	0\\
191	0\\
192	0\\
193	0\\
194	0\\
195	0\\
196	0\\
197	0\\
198	0\\
199	0\\
200	0\\
201	0\\
202	0\\
203	0\\
204	0\\
205	0\\
206	0\\
207	0\\
208	0\\
209	0\\
210	0\\
211	0\\
212	0\\
213	0\\
214	0\\
215	0\\
216	0\\
217	0\\
218	0\\
219	0\\
220	0\\
221	0\\
222	0\\
223	0\\
224	0\\
225	0\\
226	0\\
227	0\\
228	0\\
229	0\\
230	0\\
231	0\\
232	0\\
233	0\\
234	0\\
235	0\\
236	0\\
237	0\\
238	0\\
239	0\\
240	0\\
241	0\\
242	0\\
243	0\\
244	0\\
245	0\\
246	0\\
247	0\\
248	0\\
249	0\\
250	0\\
251	0\\
252	0\\
253	0\\
254	0\\
255	0\\
256	0\\
257	0\\
258	0\\
259	0\\
260	0\\
261	0\\
262	0\\
263	0\\
264	0\\
265	0\\
266	0\\
267	0\\
268	0\\
269	0\\
270	0\\
271	0\\
272	0\\
273	0\\
274	0\\
275	0\\
276	0\\
277	0\\
278	0\\
279	0\\
280	0\\
281	0\\
282	0\\
283	0\\
284	0\\
285	0\\
286	0\\
287	0\\
288	0\\
289	0\\
290	0\\
291	0\\
292	0\\
293	0\\
294	0\\
295	0\\
296	0\\
297	0\\
298	0\\
299	0\\
300	0\\
};
\addlegendentry{U(k)}

\addplot[const plot, color=mycolor2] table[row sep=crcr] {%
1	0\\
2	0\\
3	0\\
4	0\\
5	0\\
6	0\\
7	0\\
8	0\\
9	0\\
10	1\\
11	1\\
12	1\\
13	1\\
14	1\\
15	1\\
16	1\\
17	1\\
18	1\\
19	1\\
20	1\\
21	1\\
22	1\\
23	1\\
24	1\\
25	1\\
26	1\\
27	1\\
28	1\\
29	1\\
30	1\\
31	1\\
32	1\\
33	1\\
34	1\\
35	1\\
36	1\\
37	1\\
38	1\\
39	1\\
40	1\\
41	1\\
42	1\\
43	1\\
44	1\\
45	1\\
46	1\\
47	1\\
48	1\\
49	1\\
50	1\\
51	1\\
52	1\\
53	1\\
54	1\\
55	1\\
56	1\\
57	1\\
58	1\\
59	1\\
60	1\\
61	1\\
62	1\\
63	1\\
64	1\\
65	1\\
66	1\\
67	1\\
68	1\\
69	1\\
70	1\\
71	1\\
72	1\\
73	1\\
74	1\\
75	1\\
76	1\\
77	1\\
78	1\\
79	1\\
80	1\\
81	1\\
82	1\\
83	1\\
84	1\\
85	1\\
86	1\\
87	1\\
88	1\\
89	1\\
90	1\\
91	1\\
92	1\\
93	1\\
94	1\\
95	1\\
96	1\\
97	1\\
98	1\\
99	1\\
100	1\\
101	1\\
102	1\\
103	1\\
104	1\\
105	1\\
106	1\\
107	1\\
108	1\\
109	1\\
110	1\\
111	1\\
112	1\\
113	1\\
114	1\\
115	1\\
116	1\\
117	1\\
118	1\\
119	1\\
120	1\\
121	1\\
122	1\\
123	1\\
124	1\\
125	1\\
126	1\\
127	1\\
128	1\\
129	1\\
130	1\\
131	1\\
132	1\\
133	1\\
134	1\\
135	1\\
136	1\\
137	1\\
138	1\\
139	1\\
140	1\\
141	1\\
142	1\\
143	1\\
144	1\\
145	1\\
146	1\\
147	1\\
148	1\\
149	1\\
150	1\\
151	1\\
152	1\\
153	1\\
154	1\\
155	1\\
156	1\\
157	1\\
158	1\\
159	1\\
160	1\\
161	1\\
162	1\\
163	1\\
164	1\\
165	1\\
166	1\\
167	1\\
168	1\\
169	1\\
170	1\\
171	1\\
172	1\\
173	1\\
174	1\\
175	1\\
176	1\\
177	1\\
178	1\\
179	1\\
180	1\\
181	1\\
182	1\\
183	1\\
184	1\\
185	1\\
186	1\\
187	1\\
188	1\\
189	1\\
190	1\\
191	1\\
192	1\\
193	1\\
194	1\\
195	1\\
196	1\\
197	1\\
198	1\\
199	1\\
200	1\\
201	1\\
202	1\\
203	1\\
204	1\\
205	1\\
206	1\\
207	1\\
208	1\\
209	1\\
210	1\\
211	1\\
212	1\\
213	1\\
214	1\\
215	1\\
216	1\\
217	1\\
218	1\\
219	1\\
220	1\\
221	1\\
222	1\\
223	1\\
224	1\\
225	1\\
226	1\\
227	1\\
228	1\\
229	1\\
230	1\\
231	1\\
232	1\\
233	1\\
234	1\\
235	1\\
236	1\\
237	1\\
238	1\\
239	1\\
240	1\\
241	1\\
242	1\\
243	1\\
244	1\\
245	1\\
246	1\\
247	1\\
248	1\\
249	1\\
250	1\\
251	1\\
252	1\\
253	1\\
254	1\\
255	1\\
256	1\\
257	1\\
258	1\\
259	1\\
260	1\\
261	1\\
262	1\\
263	1\\
264	1\\
265	1\\
266	1\\
267	1\\
268	1\\
269	1\\
270	1\\
271	1\\
272	1\\
273	1\\
274	1\\
275	1\\
276	1\\
277	1\\
278	1\\
279	1\\
280	1\\
281	1\\
282	1\\
283	1\\
284	1\\
285	1\\
286	1\\
287	1\\
288	1\\
289	1\\
290	1\\
291	1\\
292	1\\
293	1\\
294	1\\
295	1\\
296	1\\
297	1\\
298	1\\
299	1\\
300	1\\
};
\addlegendentry{Z(k)}

\end{axis}

\begin{axis}[%
width=4.521in,
height=1.493in,
at={(0.758in,0.481in)},
scale only axis,
xmin=0,
xmax=300,
xlabel style={font=\color{white!15!black}},
xlabel={k},
ymin=0,
ymax=2,
ylabel style={font=\color{white!15!black}},
ylabel={Y(k)},
axis background/.style={fill=white},
title style={font=\bfseries},
title={Odpowied� skokowa}
]
\addplot[const plot, color=mycolor1, forget plot] table[row sep=crcr] {%
1	0\\
2	0\\
3	0\\
4	0\\
5	0\\
6	0\\
7	0\\
8	0\\
9	0\\
10	0\\
11	0\\
12	0\\
13	0.20202\\
14	0.381363772\\
15	0.5405745884792\\
16	0.681910601930065\\
17	0.807376801662065\\
18	0.918753389257166\\
19	1.01762097400741\\
20	1.10538294469291\\
21	1.18328533412576\\
22	1.25243445742761\\
23	1.31381257352359\\
24	1.36829179137883\\
25	1.41664641767974\\
26	1.45956392062001\\
27	1.49765466487875\\
28	1.53146055549941\\
29	1.56146271294626\\
30	1.58808828791264\\
31	1.61171651228869\\
32	1.63268407189206\\
33	1.6512898769727\\
34	1.66779929798444\\
35	1.68244792655289\\
36	1.69544491485312\\
37	1.70697594064711\\
38	1.71720583993623\\
39	1.7262809444819\\
40	1.73433115727287\\
41	1.74147179531066\\
42	1.7478052257929\\
43	1.75342231885168\\
44	1.75840373740865\\
45	1.7628210824046\\
46	1.76673790961448\\
47	1.77021063244259\\
48	1.77328932347903\\
49	1.77601842616608\\
50	1.77843738665143\\
51	1.78058121477559\\
52	1.78248098213803\\
53	1.7841642642963\\
54	1.78565553336142\\
55	1.78697650655121\\
56	1.78814645563942\\
57	1.78918248168534\\
58	1.79009975893699\\
59	1.7909117513645\\
60	1.791630404893\\
61	1.79226631806014\\
62	1.7928288935179\\
63	1.7933264725271\\
64	1.79376645435222\\
65	1.79415540225022\\
66	1.79449913755705\\
67	1.79480282320733\\
68	1.7950710378724\\
69	1.79530784176955\\
70	1.79551683507687\\
71	1.79570120978355\\
72	1.79586379571225\\
73	1.79600710136771\\
74	1.79613335019233\\
75	1.79624451274432\\
76	1.79634233525612\\
77	1.79642836497965\\
78	1.79650397267911\\
79	1.79657037259177\\
80	1.79662864014108\\
81	1.7966797276547\\
82	1.79672447831156\\
83	1.79676363851692\\
84	1.79679786888231\\
85	1.79682775396694\\
86	1.79685381091999\\
87	1.79687649714742\\
88	1.79689621711284\\
89	1.79691332837016\\
90	1.79692814691417\\
91	1.79694095192608\\
92	1.79695198998197\\
93	1.79696147878477\\
94	1.79696961047339\\
95	1.79697655455666\\
96	1.79698246051441\\
97	1.79698746010321\\
98	1.79699166940006\\
99	1.79699519061374\\
100	1.79699811368989\\
101	1.79700051773321\\
102	1.79700247226746\\
103	1.79700403835162\\
104	1.79700526956837\\
105	1.79700621289954\\
106	1.79700690950111\\
107	1.79700739538939\\
108	1.79700770204824\\
109	1.79700785696642\\
110	1.797007884113\\
111	1.79700780435784\\
112	1.79700763584335\\
113	1.79700739431327\\
114	1.79700709340309\\
115	1.79700674489673\\
116	1.79700635895323\\
117	1.79700594430688\\
118	1.79700550844388\\
119	1.79700505775816\\
120	1.79700459768881\\
121	1.79700413284126\\
122	1.79700366709395\\
123	1.79700320369234\\
124	1.79700274533158\\
125	1.79700229422931\\
126	1.79700185218951\\
127	1.7970014206587\\
128	1.79700100077522\\
129	1.79700059341239\\
130	1.79700019921639\\
131	1.79699981863939\\
132	1.7969994519685\\
133	1.79699909935105\\
134	1.79699876081664\\
135	1.79699843629634\\
136	1.79699812563939\\
137	1.79699782862767\\
138	1.79699754498826\\
139	1.79699727440427\\
140	1.7969970165242\\
141	1.79699677096999\\
142	1.79699653734388\\
143	1.79699631523434\\
144	1.79699610422104\\
145	1.79699590387911\\
146	1.79699571378272\\
147	1.79699553350805\\
148	1.7969953626358\\
149	1.79699520075323\\
150	1.7969950474558\\
151	1.79699490234855\\
152	1.79699476504709\\
153	1.79699463517849\\
154	1.7969945123818\\
155	1.79699439630853\\
156	1.79699428662288\\
157	1.7969941830019\\
158	1.79699408513551\\
159	1.79699399272646\\
160	1.79699390549016\\
161	1.79699382315454\\
162	1.79699374545974\\
163	1.79699367215784\\
164	1.79699360301259\\
165	1.79699353779898\\
166	1.79699347630295\\
167	1.79699341832097\\
168	1.79699336365969\\
169	1.79699331213552\\
170	1.79699326357426\\
171	1.79699321781069\\
172	1.79699317468824\\
173	1.79699313405853\\
174	1.79699309578107\\
175	1.79699305972288\\
176	1.79699302575812\\
177	1.79699299376777\\
178	1.79699296363927\\
179	1.79699293526624\\
180	1.79699290854814\\
181	1.79699288339001\\
182	1.79699285970213\\
183	1.79699283739981\\
184	1.7969928164031\\
185	1.79699279663651\\
186	1.79699277802883\\
187	1.79699276051287\\
188	1.79699274402526\\
189	1.79699272850622\\
190	1.79699271389937\\
191	1.79699270015157\\
192	1.79699268721271\\
193	1.79699267503556\\
194	1.79699266357561\\
195	1.79699265279091\\
196	1.79699264264193\\
197	1.79699263309143\\
198	1.79699262410431\\
199	1.79699261564753\\
200	1.79699260768995\\
201	1.79699260020224\\
202	1.79699259315678\\
203	1.79699258652757\\
204	1.79699258029012\\
205	1.79699257442135\\
206	1.79699256889957\\
207	1.79699256370431\\
208	1.79699255881634\\
209	1.79699255421754\\
210	1.79699254989084\\
211	1.79699254582019\\
212	1.79699254199048\\
213	1.79699253838747\\
214	1.79699253499778\\
215	1.79699253180881\\
216	1.7969925288087\\
217	1.79699252598627\\
218	1.79699252333103\\
219	1.79699252083309\\
220	1.79699251848315\\
221	1.79699251627244\\
222	1.79699251419274\\
223	1.79699251223629\\
224	1.79699251039579\\
225	1.79699250866438\\
226	1.7969925070356\\
227	1.79699250550338\\
228	1.796992504062\\
229	1.79699250270607\\
230	1.79699250143053\\
231	1.79699250023063\\
232	1.79699249910187\\
233	1.79699249804005\\
234	1.7969924970412\\
235	1.79699249610159\\
236	1.7969924952177\\
237	1.79699249438624\\
238	1.79699249360409\\
239	1.79699249286833\\
240	1.79699249217621\\
241	1.79699249152515\\
242	1.79699249091271\\
243	1.7969924903366\\
244	1.79699248979466\\
245	1.79699248928488\\
246	1.79699248880533\\
247	1.79699248835424\\
248	1.79699248792991\\
249	1.79699248753075\\
250	1.79699248715528\\
251	1.79699248680208\\
252	1.79699248646984\\
253	1.79699248615731\\
254	1.79699248586332\\
255	1.79699248558678\\
256	1.79699248532664\\
257	1.79699248508194\\
258	1.79699248485176\\
259	1.79699248463524\\
260	1.79699248443157\\
261	1.79699248423998\\
262	1.79699248405976\\
263	1.79699248389023\\
264	1.79699248373076\\
265	1.79699248358076\\
266	1.79699248343965\\
267	1.79699248330692\\
268	1.79699248318207\\
269	1.79699248306463\\
270	1.79699248295415\\
271	1.79699248285023\\
272	1.79699248275248\\
273	1.79699248266053\\
274	1.79699248257403\\
275	1.79699248249267\\
276	1.79699248241613\\
277	1.79699248234414\\
278	1.79699248227642\\
279	1.79699248221272\\
280	1.7969924821528\\
281	1.79699248209644\\
282	1.79699248204342\\
283	1.79699248199354\\
284	1.79699248194663\\
285	1.7969924819025\\
286	1.79699248186099\\
287	1.79699248182194\\
288	1.79699248178521\\
289	1.79699248175066\\
290	1.79699248171816\\
291	1.79699248168759\\
292	1.79699248165883\\
293	1.79699248163178\\
294	1.79699248160634\\
295	1.7969924815824\\
296	1.79699248155989\\
297	1.79699248153871\\
298	1.79699248151878\\
299	1.79699248150005\\
300	1.79699248148242\\
};
\end{axis}
\end{tikzpicture}%
%\caption{}
%\end{figure}

\begin{figure}[H]
\centering
% This file was created by matlab2tikz.
%
%The latest updates can be retrieved from
%  http://www.mathworks.com/matlabcentral/fileexchange/22022-matlab2tikz-matlab2tikz
%where you can also make suggestions and rate matlab2tikz.
%
\definecolor{mycolor1}{rgb}{0.00000,0.44700,0.74100}%
%
\begin{tikzpicture}

\begin{axis}[%
width=4.521in,
height=3.566in,
at={(0.758in,0.481in)},
scale only axis,
xmin=0,
xmax=300,
xlabel style={font=\color{white!15!black}},
xlabel={k},
ymin=0,
ymax=2.5,
ylabel style={font=\color{white!15!black}},
ylabel={Y(k)},
axis background/.style={fill=white}
]
\addplot[const plot, color=mycolor1, forget plot] table[row sep=crcr] {%
1	0\\
2	0\\
3	0\\
4	0\\
5	0\\
6	0\\
7	0.14788\\
8	0.287003368\\
9	0.4178875887248\\
10	0.541019681620169\\
11	0.656857981328252\\
12	0.765833815583596\\
13	0.868353086271741\\
14	0.964797759090307\\
15	1.05552726696406\\
16	1.14087983209031\\
17	1.2211737112286\\
18	1.29670836859919\\
19	1.36776558051679\\
20	1.43461047566053\\
21	1.4974925146662\\
22	1.55664641252315\\
23	1.61229300706489\\
24	1.6646400766589\\
25	1.71388311002751\\
26	1.76020603096696\\
27	1.8037818805757\\
28	1.84477345945558\\
29	1.88333393220961\\
30	1.91960739642823\\
31	1.95372941823058\\
32	1.98582753630976\\
33	2.01602173631893\\
34	2.04442489733007\\
35	2.07114321199739\\
36	2.09627658196347\\
37	2.11991898995759\\
38	2.14215884995146\\
39	2.16307933665916\\
40	2.18275869559299\\
41	2.20127053481677\\
42	2.2186840994719\\
43	2.23506453008862\\
44	2.25047310563613\\
45	2.26496747220972\\
46	2.27860185820011\\
47	2.29142727674155\\
48	2.30349171618796\\
49	2.31484031932293\\
50	2.32551555196791\\
51	2.33555736161404\\
52	2.34500332666624\\
53	2.35388879685375\\
54	2.3622470253288\\
55	2.37010929294415\\
56	2.37750502517179\\
57	2.38446190209753\\
58	2.39100596190081\\
59	2.39716169820485\\
60	2.40295215165975\\
61	2.40839899609941\\
62	2.41352261959357\\
63	2.41834220069678\\
64	2.42287578017859\\
65	2.42714032850259\\
66	2.43115180930567\\
67	2.43492523911456\\
68	2.43847474352232\\
69	2.44181361003448\\
70	2.44495433778203\\
71	2.44790868428693\\
72	2.45068770945463\\
73	2.45330181695808\\
74	2.45576079316756\\
75	2.45807384377197\\
76	2.46024962822822\\
77	2.46229629216758\\
78	2.46422149788002\\
79	2.46603245299044\\
80	2.46773593743402\\
81	2.46933832883159\\
82	2.47084562635968\\
83	2.47226347320473\\
84	2.47359717768524\\
85	2.47485173312098\\
86	2.47603183652342\\
87	2.47714190617744\\
88	2.47818609817987\\
89	2.47916832199701\\
90	2.48009225509899\\
91	2.48096135672601\\
92	2.48177888083776\\
93	2.48254788829452\\
94	2.48327125831562\\
95	2.48395169925794\\
96	2.48459175875496\\
97	2.48519383325411\\
98	2.4857601769884\\
99	2.48629291041548\\
100	2.48679402815619\\
101	2.48726540646188\\
102	2.48770881023874\\
103	2.48812589965527\\
104	2.48851823635772\\
105	2.48888728931666\\
106	2.48923444032666\\
107	2.48956098917959\\
108	2.48986815853095\\
109	2.49015709847745\\
110	2.49042889086289\\
111	2.49068455332858\\
112	2.49092504312342\\
113	2.49115126068779\\
114	2.49136405302485\\
115	2.49156421687177\\
116	2.49175250168271\\
117	2.49192961243486\\
118	2.4920962122679\\
119	2.49225292496687\\
120	2.49240033729765\\
121	2.4925390012039\\
122	2.49266943587359\\
123	2.4927921296829\\
124	2.49290754202472\\
125	2.49301610502877\\
126	2.49311822517946\\
127	2.49321428483788\\
128	2.4933046436734\\
129	2.49338964001034\\
130	2.4934695920947\\
131	2.49354479928574\\
132	2.4936155431768\\
133	2.49368208864968\\
134	2.49374468486638\\
135	2.49380356620202\\
136	2.49385895312237\\
137	2.49391105300933\\
138	2.4939600609374\\
139	2.49400616040409\\
140	2.49404952401695\\
141	2.49409031413988\\
142	2.49412868350103\\
143	2.49416477576464\\
144	2.49419872606899\\
145	2.49423066153234\\
146	2.49426070172891\\
147	2.4942889591366\\
148	2.49431553955811\\
149	2.49434054251712\\
150	2.49436406163089\\
151	2.49438618496083\\
152	2.49440699534216\\
153	2.49442657069415\\
154	2.49444498431178\\
155	2.49446230514023\\
156	2.49447859803301\\
157	2.49449392399478\\
158	2.49450834040979\\
159	2.4945219012567\\
160	2.49453465731073\\
161	2.49454665633373\\
162	2.49455794325308\\
163	2.49456856032983\\
164	2.49457854731699\\
165	2.49458794160836\\
166	2.49459677837853\\
167	2.4946050907146\\
168	2.49461290974005\\
169	2.49462026473128\\
170	2.49462718322725\\
171	2.49463369113254\\
172	2.4946398128144\\
173	2.49464557119396\\
174	2.49465098783205\\
175	2.49465608300993\\
176	2.49466087580524\\
177	2.49466538416342\\
178	2.4946696249649\\
179	2.49467361408832\\
180	2.49467736646996\\
181	2.49468089615971\\
182	2.49468421637361\\
183	2.49468733954338\\
184	2.49469027736297\\
185	2.49469304083232\\
186	2.49469564029856\\
187	2.49469808549475\\
188	2.49470038557633\\
189	2.49470254915539\\
190	2.49470458433291\\
191	2.49470649872912\\
192	2.49470829951201\\
193	2.49470999342416\\
194	2.49471158680801\\
195	2.4947130856296\\
196	2.4947144955009\\
197	2.49471582170082\\
198	2.49471706919499\\
199	2.49471824265435\\
200	2.49471934647263\\
201	2.4947203847828\\
202	2.49472136147257\\
203	2.49472228019891\\
204	2.49472314440176\\
205	2.49472395731691\\
206	2.49472472198814\\
207	2.49472544127856\\
208	2.49472611788139\\
209	2.49472675432998\\
210	2.49472735300737\\
211	2.49472791615517\\
212	2.49472844588193\\
213	2.4947289441711\\
214	2.49472941288838\\
215	2.49472985378879\\
216	2.49473026852315\\
217	2.49473065864435\\
218	2.4947310256131\\
219	2.49473137080342\\
220	2.49473169550779\\
221	2.49473200094199\\
222	2.49473228824964\\
223	2.4947325585065\\
224	2.49473281272447\\
225	2.49473305185541\\
226	2.49473327679469\\
227	2.49473348838454\\
228	2.4947336874172\\
229	2.49473387463791\\
230	2.49473405074766\\
231	2.49473421640586\\
232	2.49473437223277\\
233	2.49473451881185\\
234	2.49473465669193\\
235	2.49473478638926\\
236	2.49473490838947\\
237	2.49473502314935\\
238	2.4947351310986\\
239	2.49473523264141\\
240	2.49473532815797\\
241	2.49473541800593\\
242	2.49473550252169\\
243	2.49473558202172\\
244	2.49473565680367\\
245	2.49473572714755\\
246	2.49473579331675\\
247	2.49473585555901\\
248	2.49473591410739\\
249	2.49473596918112\\
250	2.49473602098639\\
251	2.49473606971718\\
252	2.49473611555596\\
253	2.49473615867435\\
254	2.4947361992338\\
255	2.49473623738617\\
256	2.49473627327433\\
257	2.49473630703263\\
258	2.49473633878749\\
259	2.4947363686578\\
260	2.4947363967554\\
261	2.49473642318549\\
262	2.49473644804705\\
263	2.49473647143315\\
264	2.49473649343136\\
265	2.49473651412405\\
266	2.49473653358869\\
267	2.49473655189816\\
268	2.49473656912103\\
269	2.49473658532178\\
270	2.49473660056106\\
271	2.49473661489594\\
272	2.49473662838009\\
273	2.49473664106399\\
274	2.49473665299515\\
275	2.49473666421824\\
276	2.49473667477526\\
277	2.49473668470577\\
278	2.49473669404693\\
279	2.49473670283372\\
280	2.49473671109904\\
281	2.49473671887384\\
282	2.49473672618723\\
283	2.4947367330666\\
284	2.49473673953769\\
285	2.49473674562475\\
286	2.49473675135056\\
287	2.49473675673656\\
288	2.49473676180292\\
289	2.49473676656861\\
290	2.49473677105147\\
291	2.49473677105147\\
292	2.49473677105147\\
293	2.49473677105147\\
294	2.49473677105147\\
295	2.49473677105147\\
296	2.49473677105147\\
297	2.49473677105147\\
298	2.49473677105147\\
299	2.49473677105147\\
300	2.49473677105147\\
};
\end{axis}
\end{tikzpicture}%
\caption{}
\end{figure}

\begin{figure}[H]
\centering
% This file was created by matlab2tikz.
%
%The latest updates can be retrieved from
%  http://www.mathworks.com/matlabcentral/fileexchange/22022-matlab2tikz-matlab2tikz
%where you can also make suggestions and rate matlab2tikz.
%
\definecolor{mycolor1}{rgb}{0.00000,0.44700,0.74100}%
%
\begin{tikzpicture}

\begin{axis}[%
width=4.521in,
height=3.566in,
at={(0.758in,0.481in)},
scale only axis,
xmin=0,
xmax=300,
xlabel style={font=\color{white!15!black}},
xlabel={k},
ymin=0,
ymax=1.8,
ylabel style={font=\color{white!15!black}},
ylabel={Y(k)},
axis background/.style={fill=white}
]
\addplot[const plot, color=mycolor1, forget plot] table[row sep=crcr] {%
1	0\\
2	0\\
3	0.20202\\
4	0.381363772\\
5	0.5405745884792\\
6	0.681910601930065\\
7	0.807376801662065\\
8	0.918753389257166\\
9	1.01762097400741\\
10	1.10538294469291\\
11	1.18328533412576\\
12	1.25243445742761\\
13	1.31381257352359\\
14	1.36829179137883\\
15	1.41664641767974\\
16	1.45956392062001\\
17	1.49765466487875\\
18	1.53146055549941\\
19	1.56146271294626\\
20	1.58808828791264\\
21	1.61171651228869\\
22	1.63268407189206\\
23	1.6512898769727\\
24	1.66779929798444\\
25	1.68244792655289\\
26	1.69544491485312\\
27	1.70697594064711\\
28	1.71720583993623\\
29	1.7262809444819\\
30	1.73433115727287\\
31	1.74147179531066\\
32	1.7478052257929\\
33	1.75342231885168\\
34	1.75840373740865\\
35	1.7628210824046\\
36	1.76673790961448\\
37	1.77021063244259\\
38	1.77328932347903\\
39	1.77601842616608\\
40	1.77843738665143\\
41	1.78058121477559\\
42	1.78248098213803\\
43	1.7841642642963\\
44	1.78565553336142\\
45	1.78697650655121\\
46	1.78814645563942\\
47	1.78918248168534\\
48	1.79009975893699\\
49	1.7909117513645\\
50	1.791630404893\\
51	1.79226631806014\\
52	1.7928288935179\\
53	1.7933264725271\\
54	1.79376645435222\\
55	1.79415540225022\\
56	1.79449913755705\\
57	1.79480282320733\\
58	1.7950710378724\\
59	1.79530784176955\\
60	1.79551683507687\\
61	1.79570120978355\\
62	1.79586379571225\\
63	1.79600710136771\\
64	1.79613335019233\\
65	1.79624451274432\\
66	1.79634233525612\\
67	1.79642836497965\\
68	1.79650397267911\\
69	1.79657037259177\\
70	1.79662864014108\\
71	1.7966797276547\\
72	1.79672447831156\\
73	1.79676363851692\\
74	1.79679786888231\\
75	1.79682775396694\\
76	1.79685381091999\\
77	1.79687649714742\\
78	1.79689621711284\\
79	1.79691332837016\\
80	1.79692814691417\\
81	1.79694095192608\\
82	1.79695198998197\\
83	1.79696147878477\\
84	1.79696961047339\\
85	1.79697655455666\\
86	1.79698246051441\\
87	1.79698746010321\\
88	1.79699166940006\\
89	1.79699519061374\\
90	1.79699811368989\\
91	1.79700051773321\\
92	1.79700247226746\\
93	1.79700403835162\\
94	1.79700526956837\\
95	1.79700621289954\\
96	1.79700690950111\\
97	1.79700739538939\\
98	1.79700770204824\\
99	1.79700785696642\\
100	1.797007884113\\
101	1.79700780435784\\
102	1.79700763584335\\
103	1.79700739431327\\
104	1.79700709340309\\
105	1.79700674489673\\
106	1.79700635895323\\
107	1.79700594430688\\
108	1.79700550844388\\
109	1.79700505775816\\
110	1.79700459768881\\
111	1.79700413284126\\
112	1.79700366709395\\
113	1.79700320369234\\
114	1.79700274533158\\
115	1.79700229422931\\
116	1.79700185218951\\
117	1.7970014206587\\
118	1.79700100077522\\
119	1.79700059341239\\
120	1.79700019921639\\
121	1.79699981863939\\
122	1.7969994519685\\
123	1.79699909935105\\
124	1.79699876081664\\
125	1.79699843629634\\
126	1.79699812563939\\
127	1.79699782862767\\
128	1.79699754498826\\
129	1.79699727440427\\
130	1.7969970165242\\
131	1.79699677096999\\
132	1.79699653734388\\
133	1.79699631523434\\
134	1.79699610422104\\
135	1.79699590387911\\
136	1.79699571378272\\
137	1.79699553350805\\
138	1.7969953626358\\
139	1.79699520075323\\
140	1.7969950474558\\
141	1.79699490234855\\
142	1.79699476504709\\
143	1.79699463517849\\
144	1.7969945123818\\
145	1.79699439630853\\
146	1.79699428662288\\
147	1.7969941830019\\
148	1.79699408513551\\
149	1.79699399272646\\
150	1.79699390549016\\
151	1.79699382315454\\
152	1.79699374545974\\
153	1.79699367215784\\
154	1.79699360301259\\
155	1.79699353779898\\
156	1.79699347630295\\
157	1.79699341832097\\
158	1.79699336365969\\
159	1.79699331213552\\
160	1.79699326357426\\
161	1.79699321781069\\
162	1.79699317468824\\
163	1.79699313405853\\
164	1.79699309578107\\
165	1.79699305972288\\
166	1.79699302575812\\
167	1.79699299376777\\
168	1.79699296363927\\
169	1.79699293526624\\
170	1.79699290854814\\
171	1.79699288339001\\
172	1.79699285970213\\
173	1.79699283739981\\
174	1.7969928164031\\
175	1.79699279663651\\
176	1.79699277802883\\
177	1.79699276051287\\
178	1.79699274402526\\
179	1.79699272850622\\
180	1.79699271389937\\
181	1.79699270015157\\
182	1.79699268721271\\
183	1.79699267503556\\
184	1.79699266357561\\
185	1.79699265279091\\
186	1.79699264264193\\
187	1.79699263309143\\
188	1.79699262410431\\
189	1.79699261564753\\
190	1.79699260768995\\
191	1.79699260020224\\
192	1.79699259315678\\
193	1.79699258652757\\
194	1.79699258029012\\
195	1.79699257442135\\
196	1.79699256889957\\
197	1.79699256370431\\
198	1.79699255881634\\
199	1.79699255421754\\
200	1.79699254989084\\
201	1.79699254582019\\
202	1.79699254199048\\
203	1.79699253838747\\
204	1.79699253499778\\
205	1.79699253180881\\
206	1.7969925288087\\
207	1.79699252598627\\
208	1.79699252333103\\
209	1.79699252083309\\
210	1.79699251848315\\
211	1.79699251627244\\
212	1.79699251419274\\
213	1.79699251223629\\
214	1.79699251039579\\
215	1.79699250866438\\
216	1.7969925070356\\
217	1.79699250550338\\
218	1.796992504062\\
219	1.79699250270607\\
220	1.79699250143053\\
221	1.79699250023063\\
222	1.79699249910187\\
223	1.79699249804005\\
224	1.7969924970412\\
225	1.79699249610159\\
226	1.7969924952177\\
227	1.79699249438624\\
228	1.79699249360409\\
229	1.79699249286833\\
230	1.79699249217621\\
231	1.79699249152515\\
232	1.79699249091271\\
233	1.7969924903366\\
234	1.79699248979466\\
235	1.79699248928488\\
236	1.79699248880533\\
237	1.79699248835424\\
238	1.79699248792991\\
239	1.79699248753075\\
240	1.79699248715528\\
241	1.79699248680208\\
242	1.79699248646984\\
243	1.79699248615731\\
244	1.79699248586332\\
245	1.79699248558678\\
246	1.79699248532664\\
247	1.79699248508194\\
248	1.79699248485176\\
249	1.79699248463524\\
250	1.79699248443157\\
251	1.79699248423998\\
252	1.79699248405976\\
253	1.79699248389023\\
254	1.79699248373076\\
255	1.79699248358076\\
256	1.79699248343965\\
257	1.79699248330692\\
258	1.79699248318207\\
259	1.79699248306463\\
260	1.79699248295415\\
261	1.79699248285023\\
262	1.79699248275248\\
263	1.79699248266053\\
264	1.79699248257403\\
265	1.79699248249267\\
266	1.79699248241613\\
267	1.79699248234414\\
268	1.79699248227642\\
269	1.79699248221272\\
270	1.7969924821528\\
271	1.79699248209644\\
272	1.79699248204342\\
273	1.79699248199354\\
274	1.79699248194663\\
275	1.7969924819025\\
276	1.79699248186099\\
277	1.79699248182194\\
278	1.79699248178521\\
279	1.79699248175066\\
280	1.79699248171816\\
281	1.79699248168759\\
282	1.79699248165883\\
283	1.79699248163178\\
284	1.79699248160634\\
285	1.7969924815824\\
286	1.79699248155989\\
287	1.79699248153871\\
288	1.79699248151878\\
289	1.79699248150005\\
290	1.79699248148242\\
291	1.79699248148242\\
292	1.79699248148242\\
293	1.79699248148242\\
294	1.79699248148242\\
295	1.79699248148242\\
296	1.79699248148242\\
297	1.79699248148242\\
298	1.79699248148242\\
299	1.79699248148242\\
300	1.79699248148242\\
};
\end{axis}
\end{tikzpicture}%
\caption{}
\end{figure}

\end{document}

