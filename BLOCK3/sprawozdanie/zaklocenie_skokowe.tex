%! TEX encoding = utf8
\chapter{Regulacja z skokowym zakłóceniem}

Parametr $D^Z$ został dobrany analogicznie do parametru D: przyjęto go jako wartość k, dla której odpowiedź skokowa toru zakłócenie-wyjście stabilizuje się:
$D^Z=68$

W doświadczeniach skok wartości zakłócenia następowował w chwili k=60 i wynosił 1. Użyto regulatora o najlepszych parametrach z poprzedniego podpunktu ($N_U=50$ $N=2$ $\lambda=0,4$)

\section{Bez pomiaru zakłócenia}

\begin{figure}[H]
\centering
% This file was created by matlab2tikz.
%
%The latest updates can be retrieved from
%  http://www.mathworks.com/matlabcentral/fileexchange/22022-matlab2tikz-matlab2tikz
%where you can also make suggestions and rate matlab2tikz.
%
\definecolor{mycolor1}{rgb}{0.00000,0.44700,0.74100}%
%
\begin{tikzpicture}

\begin{axis}[%
width=4.272in,
height=1.075in,
at={(0.717in,1.839in)},
scale only axis,
xmin=0,
xmax=150,
xlabel style={font=\color{white!15!black}},
xlabel={k},
ymin=-1,
ymax=2,
ylabel style={font=\color{white!15!black}},
ylabel={U(k)},
axis background/.style={fill=white}
]
\addplot[const plot, color=mycolor1, forget plot] table[row sep=crcr] {%
1	0\\
2	0\\
3	0\\
4	0\\
5	0\\
6	0\\
7	0\\
8	0\\
9	0\\
10	0\\
11	0\\
12	0\\
13	0\\
14	0\\
15	0\\
16	0\\
17	0\\
18	0\\
19	0\\
20	0\\
21	0\\
22	0\\
23	0\\
24	0\\
25	0\\
26	0\\
27	0\\
28	0\\
29	0\\
30	1.20577393725138\\
31	1.538576433506\\
32	1.46712127199355\\
33	1.24868968296282\\
34	1.01114447937986\\
35	0.808413088923491\\
36	0.655593901391529\\
37	0.54975127059157\\
38	0.481347710796871\\
39	0.4399381217831\\
40	0.416581469792359\\
41	0.404525444762055\\
42	0.399086664830369\\
43	0.397234397851693\\
44	0.397131373389285\\
45	0.397739526004153\\
46	0.39852169972768\\
47	0.39923446155087\\
48	0.39979385972787\\
49	0.400193942132608\\
50	0.400460651022531\\
51	0.400627917253023\\
52	0.40072678887092\\
53	0.400781648510957\\
54	0.400809906317153\\
55	0.40082312096247\\
56	0.400828485130777\\
57	0.40083019020383\\
58	0.400830499953435\\
59	0.400830517872924\\
60	0.400830696699621\\
61	0.400831155277304\\
62	0.400831862848653\\
63	0.157242287350464\\
64	-0.126237559941116\\
65	-0.363459520974733\\
66	-0.529921027129546\\
67	-0.629701331604633\\
68	-0.677317062855556\\
69	-0.6887063241844\\
70	-0.677425709267158\\
71	-0.653583721587496\\
72	-0.624044297714149\\
73	-0.593094737885873\\
74	-0.563177017411293\\
75	-0.535512839880241\\
76	-0.510574454643001\\
77	-0.488409874901524\\
78	-0.468851951370847\\
79	-0.451643720698643\\
80	-0.436507830034587\\
81	-0.423181049970157\\
82	-0.411428452429936\\
83	-0.401046682468229\\
84	-0.391862032768448\\
85	-0.383726537435592\\
86	-0.376513739545895\\
87	-0.370114871271103\\
88	-0.36443568884242\\
89	-0.359393962541464\\
90	-0.354917523842451\\
91	-0.35094274842261\\
92	-0.34741336425909\\
93	-0.344279496267529\\
94	-0.34149688203281\\
95	-0.339026212664641\\
96	-0.336832567538444\\
97	-0.33488492203586\\
98	-0.333155714294796\\
99	-0.331620461381576\\
100	-0.330257418021999\\
101	-0.329047272684659\\
102	-0.327972876821419\\
103	-0.327019003710337\\
104	-0.326172133782785\\
105	-0.325420263645123\\
106	-0.32475273627751\\
107	-0.324160090133692\\
108	-0.323633925086959\\
109	-0.32316678337286\\
110	-0.322752043869628\\
111	-0.322383828232466\\
112	-0.322056917557802\\
113	-0.321766678398499\\
114	-0.321508997081482\\
115	-0.321280221396033\\
116	-0.321077108825327\\
117	-0.32089678058657\\
118	-0.320736680827671\\
119	-0.320594540401633\\
120	-0.320468344704919\\
121	-0.320356305123746\\
122	-0.320256833683494\\
123	-0.320168520541867\\
124	-0.320090114006785\\
125	-0.320020502795781\\
126	-0.319958700285458\\
127	-0.319903830527761\\
128	-0.319855115834888\\
129	-0.319811865756863\\
130	-0.31977346729556\\
131	-0.319739376216471\\
132	-0.319709109335089\\
133	-0.319682237668568\\
134	-0.3196583803556\\
135	-0.319637199258342\\
136	-0.319618970545337\\
137	-0.319603693852693\\
138	-0.319591029401118\\
139	-0.319580465741697\\
140	-0.319571482995506\\
141	-0.319563647754877\\
142	-0.319556644418344\\
143	-0.31955026668201\\
144	-0.319544391412479\\
145	-0.319538949671546\\
146	-0.319533902432941\\
147	-0.319529223467581\\
148	-0.319524889033341\\
149	-0.319520872804399\\
150	-0.319517144261884\\
};
\end{axis}

\begin{axis}[%
width=4.272in,
height=1.075in,
at={(0.717in,0.346in)},
scale only axis,
xmin=0,
xmax=150,
xlabel style={font=\color{white!15!black}},
xlabel={k},
ymin=0,
ymax=1,
ylabel style={font=\color{white!15!black}},
ylabel={Z(k)},
axis background/.style={fill=white}
]
\addplot[const plot, color=mycolor1, forget plot] table[row sep=crcr] {%
1	0\\
2	0\\
3	0\\
4	0\\
5	0\\
6	0\\
7	0\\
8	0\\
9	0\\
10	0\\
11	0\\
12	0\\
13	0\\
14	0\\
15	0\\
16	0\\
17	0\\
18	0\\
19	0\\
20	0\\
21	0\\
22	0\\
23	0\\
24	0\\
25	0\\
26	0\\
27	0\\
28	0\\
29	0\\
30	0\\
31	0\\
32	0\\
33	0\\
34	0\\
35	0\\
36	0\\
37	0\\
38	0\\
39	0\\
40	0\\
41	0\\
42	0\\
43	0\\
44	0\\
45	0\\
46	0\\
47	0\\
48	0\\
49	0\\
50	0\\
51	0\\
52	0\\
53	0\\
54	0\\
55	0\\
56	0\\
57	0\\
58	0\\
59	0\\
60	1\\
61	1\\
62	1\\
63	1\\
64	1\\
65	1\\
66	1\\
67	1\\
68	1\\
69	1\\
70	1\\
71	1\\
72	1\\
73	1\\
74	1\\
75	1\\
76	1\\
77	1\\
78	1\\
79	1\\
80	1\\
81	1\\
82	1\\
83	1\\
84	1\\
85	1\\
86	1\\
87	1\\
88	1\\
89	1\\
90	1\\
91	1\\
92	1\\
93	1\\
94	1\\
95	1\\
96	1\\
97	1\\
98	1\\
99	1\\
100	1\\
101	1\\
102	1\\
103	1\\
104	1\\
105	1\\
106	1\\
107	1\\
108	1\\
109	1\\
110	1\\
111	1\\
112	1\\
113	1\\
114	1\\
115	1\\
116	1\\
117	1\\
118	1\\
119	1\\
120	1\\
121	1\\
122	1\\
123	1\\
124	1\\
125	1\\
126	1\\
127	1\\
128	1\\
129	1\\
130	1\\
131	1\\
132	1\\
133	1\\
134	1\\
135	1\\
136	1\\
137	1\\
138	1\\
139	1\\
140	1\\
141	1\\
142	1\\
143	1\\
144	1\\
145	1\\
146	1\\
147	1\\
148	1\\
149	1\\
150	1\\
};
\end{axis}
\end{tikzpicture}%
\caption{Zakłócenie i sygnał sterujący}
\end{figure}

\begin{figure}[H]
\centering
\input{rysunki/podpunkt5_bez_2.006103e+01.tex}
\caption{Wyjście obiektu bez pomiaru zakłócenia błąd $E=20,06103$}
\end{figure}

\section{Z pomiaru zakłócenia}

\begin{figure}[H]
\centering
% This file was created by matlab2tikz.
%
%The latest updates can be retrieved from
%  http://www.mathworks.com/matlabcentral/fileexchange/22022-matlab2tikz-matlab2tikz
%where you can also make suggestions and rate matlab2tikz.
%
\definecolor{mycolor1}{rgb}{0.00000,0.44700,0.74100}%
%
\begin{tikzpicture}

\begin{axis}[%
width=4.272in,
height=1.075in,
at={(0.717in,1.839in)},
scale only axis,
xmin=0,
xmax=150,
xlabel style={font=\color{white!15!black}},
xlabel={k},
ymin=-2,
ymax=2,
ylabel style={font=\color{white!15!black}},
ylabel={U(k)},
axis background/.style={fill=white}
]
\addplot[const plot, color=mycolor1, forget plot] table[row sep=crcr] {%
1	0\\
2	0\\
3	0\\
4	0\\
5	0\\
6	0\\
7	0\\
8	0\\
9	0\\
10	0\\
11	0\\
12	0\\
13	0\\
14	0\\
15	0\\
16	0\\
17	0\\
18	0\\
19	0\\
20	0\\
21	0\\
22	0\\
23	0\\
24	0\\
25	0\\
26	0\\
27	0\\
28	0\\
29	0\\
30	1.20577393725138\\
31	1.538576433506\\
32	1.46712127199355\\
33	1.24868968296282\\
34	1.01114447937986\\
35	0.808413088923491\\
36	0.655593901391529\\
37	0.54975127059157\\
38	0.481347710796871\\
39	0.4399381217831\\
40	0.416581469792359\\
41	0.404525444762055\\
42	0.399086664830369\\
43	0.397234397851693\\
44	0.397131373389285\\
45	0.397739526004153\\
46	0.39852169972768\\
47	0.39923446155087\\
48	0.39979385972787\\
49	0.400193942132608\\
50	0.400460651022531\\
51	0.400627917253023\\
52	0.40072678887092\\
53	0.400781648510957\\
54	0.400809906317153\\
55	0.40082312096247\\
56	0.400828485130777\\
57	0.40083019020383\\
58	0.400830499953435\\
59	0.400830517872924\\
60	0.400830696699621\\
61	-0.956483506635671\\
62	-1.42231705273603\\
63	-1.69160005144651\\
64	-1.57440118893383\\
65	-1.32303372579229\\
66	-1.06137980039552\\
67	-0.84045625110547\\
68	-0.673366200972875\\
69	-0.555872660126145\\
70	-0.477622141501927\\
71	-0.427672269010304\\
72	-0.396782870956662\\
73	-0.378019506667135\\
74	-0.36658239054199\\
75	-0.359359587761817\\
76	-0.354450780966218\\
77	-0.350764189059999\\
78	-0.347714140184759\\
79	-0.345012230373121\\
80	-0.34253454725985\\
81	-0.340236507617432\\
82	-0.338107839188851\\
83	-0.336149233661171\\
84	-0.334361662771399\\
85	-0.332742547278446\\
86	-0.331285267318234\\
87	-0.329980036413333\\
88	-0.328815119465039\\
89	-0.327777934873299\\
90	-0.326855883632153\\
91	-0.326036895437746\\
92	-0.325309741392093\\
93	-0.324664177228084\\
94	-0.324090974943965\\
95	-0.323581887756868\\
96	-0.32312957990848\\
97	-0.322727541734464\\
98	-0.322370002185988\\
99	-0.322051845401083\\
100	-0.321768534388962\\
101	-0.321516042815174\\
102	-0.321290794761743\\
103	-0.321089611814396\\
104	-0.320909666645529\\
105	-0.320748442254937\\
106	-0.32060369610405\\
107	-0.320473428479832\\
108	-0.320355854524882\\
109	-0.320249379459184\\
110	-0.320152576593155\\
111	-0.32006416779192\\
112	-0.31998300609928\\
113	-0.319908060269044\\
114	-0.319838400983534\\
115	-0.319773188565772\\
116	-0.319711662014469\\
117	-0.31965312921043\\
118	-0.319596958159853\\
119	-0.319542569154915\\
120	-0.319489427745129\\
121	-0.319437038424593\\
122	-0.319384938950585\\
123	-0.319332695218109\\
124	-0.319291037900742\\
125	-0.319262170092156\\
126	-0.319244549398616\\
127	-0.319235443425775\\
128	-0.319232150017157\\
129	-0.319232472512029\\
130	-0.319314879354428\\
131	-0.319410536080885\\
132	-0.319490386170859\\
133	-0.319546214751958\\
134	-0.319579456452548\\
135	-0.319595058163842\\
136	-0.319599009208302\\
137	-0.319596172496547\\
138	-0.319589868275447\\
139	-0.319582116847833\\
140	-0.319574032655541\\
141	-0.319566170441952\\
142	-0.319558771107165\\
143	-0.319551914946694\\
144	-0.319545607575466\\
145	-0.319539823385151\\
146	-0.319534525122145\\
147	-0.319529671522524\\
148	-0.319525219785899\\
149	-0.319521126280475\\
150	-0.319517346909615\\
};
\end{axis}

\begin{axis}[%
width=4.272in,
height=1.075in,
at={(0.717in,0.346in)},
scale only axis,
xmin=0,
xmax=150,
xlabel style={font=\color{white!15!black}},
xlabel={k},
ymin=0,
ymax=1,
ylabel style={font=\color{white!15!black}},
ylabel={Z(k)},
axis background/.style={fill=white}
]
\addplot[const plot, color=mycolor1, forget plot] table[row sep=crcr] {%
1	0\\
2	0\\
3	0\\
4	0\\
5	0\\
6	0\\
7	0\\
8	0\\
9	0\\
10	0\\
11	0\\
12	0\\
13	0\\
14	0\\
15	0\\
16	0\\
17	0\\
18	0\\
19	0\\
20	0\\
21	0\\
22	0\\
23	0\\
24	0\\
25	0\\
26	0\\
27	0\\
28	0\\
29	0\\
30	0\\
31	0\\
32	0\\
33	0\\
34	0\\
35	0\\
36	0\\
37	0\\
38	0\\
39	0\\
40	0\\
41	0\\
42	0\\
43	0\\
44	0\\
45	0\\
46	0\\
47	0\\
48	0\\
49	0\\
50	0\\
51	0\\
52	0\\
53	0\\
54	0\\
55	0\\
56	0\\
57	0\\
58	0\\
59	0\\
60	1\\
61	1\\
62	1\\
63	1\\
64	1\\
65	1\\
66	1\\
67	1\\
68	1\\
69	1\\
70	1\\
71	1\\
72	1\\
73	1\\
74	1\\
75	1\\
76	1\\
77	1\\
78	1\\
79	1\\
80	1\\
81	1\\
82	1\\
83	1\\
84	1\\
85	1\\
86	1\\
87	1\\
88	1\\
89	1\\
90	1\\
91	1\\
92	1\\
93	1\\
94	1\\
95	1\\
96	1\\
97	1\\
98	1\\
99	1\\
100	1\\
101	1\\
102	1\\
103	1\\
104	1\\
105	1\\
106	1\\
107	1\\
108	1\\
109	1\\
110	1\\
111	1\\
112	1\\
113	1\\
114	1\\
115	1\\
116	1\\
117	1\\
118	1\\
119	1\\
120	1\\
121	1\\
122	1\\
123	1\\
124	1\\
125	1\\
126	1\\
127	1\\
128	1\\
129	1\\
130	1\\
131	1\\
132	1\\
133	1\\
134	1\\
135	1\\
136	1\\
137	1\\
138	1\\
139	1\\
140	1\\
141	1\\
142	1\\
143	1\\
144	1\\
145	1\\
146	1\\
147	1\\
148	1\\
149	1\\
150	1\\
};
\end{axis}
\end{tikzpicture}%
\caption{Zakłócenie i sygnał sterujący}
\end{figure}

\begin{figure}[H]
\centering
% This file was created by matlab2tikz.
%
%The latest updates can be retrieved from
%  http://www.mathworks.com/matlabcentral/fileexchange/22022-matlab2tikz-matlab2tikz
%where you can also make suggestions and rate matlab2tikz.
%
\definecolor{mycolor1}{rgb}{0.00000,0.44700,0.74100}%
\definecolor{mycolor2}{rgb}{0.85000,0.32500,0.09800}%
%
\begin{tikzpicture}

\begin{axis}[%
width=4.272in,
height=3.472in,
at={(0.717in,0.441in)},
scale only axis,
xmin=0,
xmax=150,
xlabel style={font=\color{white!15!black}},
xlabel={k},
ymin=0,
ymax=2,
ylabel style={font=\color{white!15!black}},
ylabel={Y(k)},
axis background/.style={fill=white},
legend style={legend cell align=left, align=left, draw=white!15!black}
]
\addplot[const plot, color=mycolor1] table[row sep=crcr] {%
1	0\\
2	0\\
3	0\\
4	0\\
5	0\\
6	0\\
7	0\\
8	0\\
9	0\\
10	0\\
11	0\\
12	0\\
13	0\\
14	0\\
15	0\\
16	0\\
17	0\\
18	0\\
19	0\\
20	0\\
21	0\\
22	0\\
23	0\\
24	0\\
25	0\\
26	0\\
27	0\\
28	0\\
29	0\\
30	0\\
31	0\\
32	0\\
33	0\\
34	0\\
35	0\\
36	0\\
37	0.178309849840735\\
38	0.3952760141839\\
39	0.588826611204609\\
40	0.738611928918155\\
41	0.844395923345594\\
42	0.913931740693088\\
43	0.956746557749988\\
44	0.981369563669113\\
45	0.994414676921337\\
46	1.00055965866249\\
47	1.00288312467568\\
48	1.00328286848648\\
49	1.00285171378131\\
50	1.00216956244591\\
51	1.0015102513556\\
52	1.00097786168913\\
53	1.00059084702661\\
54	1.00033054211392\\
55	1.00016694785008\\
56	1.0000709398569\\
57	1.00001893584117\\
58	0.999993751094316\\
59	0.999983795371811\\
60	0.999981757686305\\
61	0.999983323334048\\
62	0.999986132468706\\
63	1.20200901984749\\
64	1.38135527326315\\
65	1.54056803752264\\
66	1.68190550206501\\
67	1.8073727524717\\
68	1.71803040143897\\
69	1.55917677439691\\
70	1.36465794710609\\
71	1.19432884419806\\
72	1.06712207890516\\
73	0.982470519346334\\
74	0.93224375885162\\
75	0.906808543559815\\
76	0.897687302718937\\
77	0.898398925006169\\
78	0.904431790486833\\
79	0.912879111214112\\
80	0.922006188243962\\
81	0.9308682152983\\
82	0.939016455979743\\
83	0.946291953364617\\
84	0.952690805820321\\
85	0.958281916806407\\
86	0.963160298348053\\
87	0.967422597423599\\
88	0.971156508416861\\
89	0.974437330618857\\
90	0.977328058647748\\
91	0.979880937840613\\
92	0.982139396763533\\
93	0.984139851291592\\
94	0.985913195886471\\
95	0.987485958320915\\
96	0.988881162546951\\
97	0.990118964143804\\
98	0.99121711968075\\
99	0.992191339692066\\
100	0.993055561966445\\
101	0.993822170587604\\
102	0.994502177542927\\
103	0.995105377638433\\
104	0.995640483431401\\
105	0.996115244361213\\
106	0.996536552741922\\
107	0.996910538408444\\
108	0.99724265332441\\
109	0.9975377471952\\
110	0.997800134981086\\
111	0.998033657113194\\
112	0.998241733147466\\
113	0.998427409533445\\
114	0.99859340211934\\
115	0.998742133960592\\
116	0.998875768946308\\
117	0.998996241707235\\
118	0.999105284221292\\
119	0.999204449488526\\
120	0.999295132607073\\
121	0.99937858954528\\
122	0.999455953872464\\
123	0.999528251681592\\
124	0.999596414911225\\
125	0.999661293250975\\
126	0.99972366479431\\
127	0.999784245584398\\
128	0.999843698182612\\
129	0.999902639375107\\
130	0.999961647120263\\
131	1.00001961925648\\
132	1.00007514066796\\
133	1.00012706158779\\
134	1.00017466309325\\
135	1.00021763176396\\
136	1.00025596503561\\
137	1.00027802769044\\
138	1.00028302707354\\
139	1.00027449165336\\
140	1.000256935637\\
141	1.00023437586694\\
142	1.00020984385587\\
143	1.00018529173658\\
144	1.00016182431907\\
145	1.00013997879337\\
146	1.00011995182548\\
147	1.00010175479276\\
148	1.0000853085142\\
149	1.00007049590441\\
150	1.00005718875066\\
};
\addlegendentry{Y}

\addplot[const plot, color=mycolor2] table[row sep=crcr] {%
1	0\\
2	0\\
3	0\\
4	0\\
5	0\\
6	0\\
7	0\\
8	0\\
9	0\\
10	0\\
11	0\\
12	0\\
13	0\\
14	0\\
15	0\\
16	0\\
17	0\\
18	0\\
19	0\\
20	0\\
21	0\\
22	0\\
23	0\\
24	0\\
25	0\\
26	0\\
27	0\\
28	0\\
29	0\\
30	1\\
31	1\\
32	1\\
33	1\\
34	1\\
35	1\\
36	1\\
37	1\\
38	1\\
39	1\\
40	1\\
41	1\\
42	1\\
43	1\\
44	1\\
45	1\\
46	1\\
47	1\\
48	1\\
49	1\\
50	1\\
51	1\\
52	1\\
53	1\\
54	1\\
55	1\\
56	1\\
57	1\\
58	1\\
59	1\\
60	1\\
61	1\\
62	1\\
63	1\\
64	1\\
65	1\\
66	1\\
67	1\\
68	1\\
69	1\\
70	1\\
71	1\\
72	1\\
73	1\\
74	1\\
75	1\\
76	1\\
77	1\\
78	1\\
79	1\\
80	1\\
81	1\\
82	1\\
83	1\\
84	1\\
85	1\\
86	1\\
87	1\\
88	1\\
89	1\\
90	1\\
91	1\\
92	1\\
93	1\\
94	1\\
95	1\\
96	1\\
97	1\\
98	1\\
99	1\\
100	1\\
101	1\\
102	1\\
103	1\\
104	1\\
105	1\\
106	1\\
107	1\\
108	1\\
109	1\\
110	1\\
111	1\\
112	1\\
113	1\\
114	1\\
115	1\\
116	1\\
117	1\\
118	1\\
119	1\\
120	1\\
121	1\\
122	1\\
123	1\\
124	1\\
125	1\\
126	1\\
127	1\\
128	1\\
129	1\\
130	1\\
131	1\\
132	1\\
133	1\\
134	1\\
135	1\\
136	1\\
137	1\\
138	1\\
139	1\\
140	1\\
141	1\\
142	1\\
143	1\\
144	1\\
145	1\\
146	1\\
147	1\\
148	1\\
149	1\\
150	1\\
151	1\\
152	1\\
153	1\\
154	1\\
155	1\\
156	1\\
157	1\\
158	1\\
159	1\\
160	1\\
161	1\\
162	1\\
163	1\\
164	1\\
165	1\\
166	1\\
167	1\\
168	1\\
169	1\\
170	1\\
171	1\\
172	1\\
173	1\\
174	1\\
175	1\\
176	1\\
177	1\\
178	1\\
179	1\\
180	1\\
181	1\\
182	1\\
183	1\\
184	1\\
185	1\\
186	1\\
187	1\\
188	1\\
189	1\\
190	1\\
191	1\\
192	1\\
193	1\\
194	1\\
195	1\\
196	1\\
197	1\\
198	1\\
199	1\\
200	1\\
};
\addlegendentry{Yzad}

\end{axis}
\end{tikzpicture}%
\caption{Wyjście obiektu z pomiarem zakłócenia błąd $E=10,98983$}
\end{figure}

\section{Wnioski}
Pomiar zakłócen znacznie polepszył regulację. Bez niego błąd wynosił E=20,06103, natomiast z pomiarem błąd zmalał prawie dwukrotnie (E=10,98983).
\smallskip

