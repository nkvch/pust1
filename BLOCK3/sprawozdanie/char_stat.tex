%! TEX encoding = utf8
\chapter{Odpowiedzi skokowe}

Rozważamy punkt pracy oraz 5 różnych wartości skoku, z zera do: $-1$, $-0,75$, $-0,5$, $-0,25$, $0,25$.

\section{Opowiedzi skokowe}

\begin{figure}[H]
\centering
% This file was created by matlab2tikz.
%
%The latest updates can be retrieved from
%  http://www.mathworks.com/matlabcentral/fileexchange/22022-matlab2tikz-matlab2tikz
%where you can also make suggestions and rate matlab2tikz.
%
\definecolor{mycolor1}{rgb}{0.00000,0.44700,0.74100}%
\definecolor{mycolor2}{rgb}{0.85000,0.32500,0.09800}%
\definecolor{mycolor3}{rgb}{0.92900,0.69400,0.12500}%
\definecolor{mycolor4}{rgb}{0.49400,0.18400,0.55600}%
\definecolor{mycolor5}{rgb}{0.46600,0.67400,0.18800}%
\definecolor{mycolor6}{rgb}{0.30100,0.74500,0.93300}%
\definecolor{mycolor7}{rgb}{0.63500,0.07800,0.18400}%
%
\begin{tikzpicture}

\begin{axis}[%
width=6.102in,
height=6.417in,
at={(1.024in,0.866in)},
scale only axis,
xmin=0,
xmax=300,
xlabel style={font=\color{white!15!black}},
xlabel={k},
ymin=-0.8,
ymax=0.1,
ylabel style={font=\color{white!15!black}},
ylabel={Y(k)},
axis background/.style={fill=white},
legend style={at={(0.97,0.03)}, anchor=south east, legend cell align=left, align=left, draw=white!15!black}
]
\addplot[const plot, color=mycolor1] table[row sep=crcr] {%
1	0\\
2	0\\
3	0\\
4	0\\
5	0\\
6	0\\
7	0\\
8	0\\
9	0\\
10	0\\
11	0\\
12	0\\
13	0\\
14	0\\
15	-0.0335675877192982\\
16	-0.112965625142105\\
17	-0.212521981792259\\
18	-0.31513883462998\\
19	-0.41036001081684\\
20	-0.492623525565041\\
21	-0.5597813733745\\
22	-0.611911428472393\\
23	-0.650410015074312\\
24	-0.677333122976111\\
25	-0.694944862934428\\
26	-0.705429889631609\\
27	-0.71072929563746\\
28	-0.712464770401887\\
29	-0.711922131286509\\
30	-0.710071688301026\\
31	-0.707608712847838\\
32	-0.70500224248527\\
33	-0.702544465392307\\
34	-0.700396013802861\\
35	-0.698624752625891\\
36	-0.697237210244591\\
37	-0.696202804510307\\
38	-0.695471602257874\\
39	-0.69498663288036\\
40	-0.694691852874273\\
41	-0.694536805208319\\
42	-0.694478892004551\\
43	-0.694484021861918\\
44	-0.694526231178164\\
45	-0.694586728534931\\
46	-0.694652681377337\\
47	-0.694715958230594\\
48	-0.694771957427408\\
49	-0.694818592565868\\
50	-0.694855462432398\\
51	-0.694883205334342\\
52	-0.694903021243572\\
53	-0.694916336792617\\
54	-0.694924585428763\\
55	-0.694929075890161\\
56	-0.69493092508991\\
57	-0.69493103538224\\
58	-0.694930100299094\\
59	-0.69492862672469\\
60	-0.694926964866447\\
61	-0.694925340176121\\
62	-0.694923883564747\\
63	-0.694922657887373\\
64	-0.69492167982743\\
65	-0.694920937075836\\
66	-0.694920401165058\\
67	-0.694920036562932\\
68	-0.694919806722385\\
69	-0.694919677775033\\
70	-0.69491962048998\\
71	-0.694919611023644\\
72	-0.694919630882301\\
73	-0.694919666419212\\
74	-0.694919708099766\\
75	-0.694919749694496\\
76	-0.69491978750165\\
77	-0.694919819657184\\
78	-0.694919845558801\\
79	-0.694919865409643\\
80	-0.69491987987422\\
81	-0.694919889832076\\
82	-0.694919896211796\\
83	-0.694919899887748\\
84	-0.694919901623475\\
85	-0.694919902047935\\
86	-0.694919901653444\\
87	-0.694919900806729\\
88	-0.694919899766789\\
89	-0.694919898705208\\
90	-0.694919897726102\\
91	-0.694919896884061\\
92	-0.694919896199279\\
93	-0.694919895669655\\
94	-0.694919895279998\\
95	-0.694919895008691\\
96	-0.69491989483223\\
97	-0.694919894728098\\
98	-0.69491989467639\\
99	-0.694919894660544\\
100	-0.694919894667477\\
101	-0.69491989468735\\
102	-0.694919894713144\\
103	-0.694919894740146\\
104	-0.694919894765443\\
105	-0.694919894787452\\
106	-0.694919894805525\\
107	-0.694919894819632\\
108	-0.694919894830109\\
109	-0.694919894837484\\
110	-0.694919894842348\\
111	-0.694919894845282\\
112	-0.694919894846802\\
113	-0.694919894847343\\
114	-0.694919894847253\\
115	-0.694919894846796\\
116	-0.69491989484616\\
117	-0.694919894845476\\
118	-0.694919894844824\\
119	-0.69491989484425\\
120	-0.694919894843774\\
121	-0.694919894843398\\
122	-0.694919894843117\\
123	-0.694919894842917\\
124	-0.694919894842784\\
125	-0.694919894842702\\
126	-0.694919894842658\\
127	-0.69491989484264\\
128	-0.69491989484264\\
129	-0.69491989484265\\
130	-0.694919894842665\\
131	-0.694919894842683\\
132	-0.6949198948427\\
133	-0.694919894842715\\
134	-0.694919894842727\\
135	-0.694919894842737\\
136	-0.694919894842745\\
137	-0.69491989484275\\
138	-0.694919894842754\\
139	-0.694919894842756\\
140	-0.694919894842757\\
141	-0.694919894842757\\
142	-0.694919894842757\\
143	-0.694919894842757\\
144	-0.694919894842757\\
145	-0.694919894842756\\
146	-0.694919894842756\\
147	-0.694919894842756\\
148	-0.694919894842755\\
149	-0.694919894842755\\
150	-0.694919894842755\\
151	-0.694919894842755\\
152	-0.694919894842755\\
153	-0.694919894842755\\
154	-0.694919894842755\\
155	-0.694919894842755\\
156	-0.694919894842755\\
157	-0.694919894842755\\
158	-0.694919894842755\\
159	-0.694919894842755\\
160	-0.694919894842755\\
161	-0.694919894842755\\
162	-0.694919894842755\\
163	-0.694919894842755\\
164	-0.694919894842755\\
165	-0.694919894842755\\
166	-0.694919894842755\\
167	-0.694919894842755\\
168	-0.694919894842755\\
169	-0.694919894842755\\
170	-0.694919894842755\\
171	-0.694919894842755\\
172	-0.694919894842755\\
173	-0.694919894842755\\
174	-0.694919894842755\\
175	-0.694919894842755\\
176	-0.694919894842755\\
177	-0.694919894842755\\
178	-0.694919894842755\\
179	-0.694919894842755\\
180	-0.694919894842755\\
181	-0.694919894842755\\
182	-0.694919894842755\\
183	-0.694919894842755\\
184	-0.694919894842755\\
185	-0.694919894842755\\
186	-0.694919894842755\\
187	-0.694919894842755\\
188	-0.694919894842755\\
189	-0.694919894842755\\
190	-0.694919894842755\\
191	-0.694919894842755\\
192	-0.694919894842755\\
193	-0.694919894842755\\
194	-0.694919894842755\\
195	-0.694919894842755\\
196	-0.694919894842755\\
197	-0.694919894842755\\
198	-0.694919894842755\\
199	-0.694919894842755\\
200	-0.694919894842755\\
201	-0.694919894842755\\
202	-0.694919894842755\\
203	-0.694919894842755\\
204	-0.694919894842755\\
205	-0.694919894842755\\
206	-0.694919894842755\\
207	-0.694919894842755\\
208	-0.694919894842755\\
209	-0.694919894842755\\
210	-0.694919894842755\\
211	-0.694919894842755\\
212	-0.694919894842755\\
213	-0.694919894842755\\
214	-0.694919894842755\\
215	-0.694919894842755\\
216	-0.694919894842755\\
217	-0.694919894842755\\
218	-0.694919894842755\\
219	-0.694919894842755\\
220	-0.694919894842755\\
221	-0.694919894842755\\
222	-0.694919894842755\\
223	-0.694919894842755\\
224	-0.694919894842755\\
225	-0.694919894842755\\
226	-0.694919894842755\\
227	-0.694919894842755\\
228	-0.694919894842755\\
229	-0.694919894842755\\
230	-0.694919894842755\\
231	-0.694919894842755\\
232	-0.694919894842755\\
233	-0.694919894842755\\
234	-0.694919894842755\\
235	-0.694919894842755\\
236	-0.694919894842755\\
237	-0.694919894842755\\
238	-0.694919894842755\\
239	-0.694919894842755\\
240	-0.694919894842755\\
241	-0.694919894842755\\
242	-0.694919894842755\\
243	-0.694919894842755\\
244	-0.694919894842755\\
245	-0.694919894842755\\
246	-0.694919894842755\\
247	-0.694919894842755\\
248	-0.694919894842755\\
249	-0.694919894842755\\
250	-0.694919894842755\\
251	-0.694919894842755\\
252	-0.694919894842755\\
253	-0.694919894842755\\
254	-0.694919894842755\\
255	-0.694919894842755\\
256	-0.694919894842755\\
257	-0.694919894842755\\
258	-0.694919894842755\\
259	-0.694919894842755\\
260	-0.694919894842755\\
261	-0.694919894842755\\
262	-0.694919894842755\\
263	-0.694919894842755\\
264	-0.694919894842755\\
265	-0.694919894842755\\
266	-0.694919894842755\\
267	-0.694919894842755\\
268	-0.694919894842755\\
269	-0.694919894842755\\
270	-0.694919894842755\\
271	-0.694919894842755\\
272	-0.694919894842755\\
273	-0.694919894842755\\
274	-0.694919894842755\\
275	-0.694919894842755\\
276	-0.694919894842755\\
277	-0.694919894842755\\
278	-0.694919894842755\\
279	-0.694919894842755\\
280	-0.694919894842755\\
281	-0.694919894842755\\
282	-0.694919894842755\\
283	-0.694919894842755\\
284	-0.694919894842755\\
285	-0.694919894842755\\
286	-0.694919894842755\\
287	-0.694919894842755\\
288	-0.694919894842755\\
289	-0.694919894842755\\
290	-0.694919894842755\\
291	-0.694919894842755\\
292	-0.694919894842755\\
293	-0.694919894842755\\
294	-0.694919894842755\\
295	-0.694919894842755\\
296	-0.694919894842755\\
297	-0.694919894842755\\
298	-0.694919894842755\\
299	-0.694919894842755\\
300	-0.694919894842755\\
};
\addlegendentry{Skok U z 0.00 do -0.50}

\addplot[const plot, color=mycolor2] table[row sep=crcr] {%
1	0\\
2	0\\
3	0\\
4	0\\
5	0\\
6	0\\
7	0\\
8	0\\
9	0\\
10	0\\
11	0\\
12	0\\
13	0\\
14	0\\
15	-0.0150905587027915\\
16	-0.0484195449261625\\
17	-0.0864115139873757\\
18	-0.121395119761953\\
19	-0.149839723783237\\
20	-0.170855068765489\\
21	-0.185040572876755\\
22	-0.193679087044566\\
23	-0.198223356168033\\
24	-0.200006749973101\\
25	-0.200110976760491\\
26	-0.19933367609057\\
27	-0.198212181266038\\
28	-0.197072885924991\\
29	-0.196086789406331\\
30	-0.195320314175638\\
31	-0.194776439892249\\
32	-0.194424976657385\\
33	-0.194222917734573\\
34	-0.194126771051117\\
35	-0.194098993056308\\
36	-0.19411046158561\\
37	-0.194140551866487\\
38	-0.194175964595706\\
39	-0.194209076957524\\
40	-0.194236282449366\\
41	-0.19425656155894\\
42	-0.194270375315654\\
43	-0.194278883192304\\
44	-0.194283439888477\\
45	-0.194285307778275\\
46	-0.194285521654744\\
47	-0.194284851357772\\
48	-0.194283820249498\\
49	-0.194282749881076\\
50	-0.194281811811171\\
51	-0.1942810757299\\
52	-0.194280548816684\\
53	-0.194280204963872\\
54	-0.194280004604477\\
55	-0.194279906864281\\
56	-0.194279876020257\\
57	-0.194279884099752\\
58	-0.194279911117332\\
59	-0.194279944058589\\
60	-0.194279975362056\\
61	-0.194280001358456\\
62	-0.194280020910386\\
63	-0.19428003434934\\
64	-0.194280042717708\\
65	-0.194280047276223\\
66	-0.194280049218616\\
67	-0.194280049533891\\
68	-0.194280048964422\\
69	-0.194280048019476\\
70	-0.194280047015419\\
71	-0.194280046123964\\
72	-0.194280045417687\\
73	-0.194280044907653\\
74	-0.194280044571604\\
75	-0.194280044373251\\
76	-0.194280044274239\\
77	-0.194280044240613\\
78	-0.194280044245527\\
79	-0.194280044269632\\
80	-0.194280044300208\\
81	-0.194280044329764\\
82	-0.194280044354582\\
83	-0.194280044373416\\
84	-0.194280044386478\\
85	-0.194280044394699\\
86	-0.19428004439925\\
87	-0.194280044401257\\
88	-0.194280044401666\\
89	-0.194280044401192\\
90	-0.194280044400329\\
91	-0.194280044399388\\
92	-0.194280044398542\\
93	-0.194280044397865\\
94	-0.194280044397372\\
95	-0.194280044397044\\
96	-0.194280044396848\\
97	-0.194280044396748\\
98	-0.194280044396711\\
99	-0.194280044396713\\
100	-0.194280044396735\\
101	-0.194280044396763\\
102	-0.194280044396791\\
103	-0.194280044396815\\
104	-0.194280044396833\\
105	-0.194280044396845\\
106	-0.194280044396854\\
107	-0.194280044396858\\
108	-0.19428004439686\\
109	-0.194280044396861\\
110	-0.19428004439686\\
111	-0.194280044396859\\
112	-0.194280044396859\\
113	-0.194280044396858\\
114	-0.194280044396857\\
115	-0.194280044396857\\
116	-0.194280044396856\\
117	-0.194280044396856\\
118	-0.194280044396856\\
119	-0.194280044396856\\
120	-0.194280044396856\\
121	-0.194280044396856\\
122	-0.194280044396856\\
123	-0.194280044396856\\
124	-0.194280044396856\\
125	-0.194280044396856\\
126	-0.194280044396856\\
127	-0.194280044396856\\
128	-0.194280044396856\\
129	-0.194280044396856\\
130	-0.194280044396856\\
131	-0.194280044396856\\
132	-0.194280044396856\\
133	-0.194280044396856\\
134	-0.194280044396856\\
135	-0.194280044396856\\
136	-0.194280044396856\\
137	-0.194280044396856\\
138	-0.194280044396856\\
139	-0.194280044396856\\
140	-0.194280044396856\\
141	-0.194280044396856\\
142	-0.194280044396856\\
143	-0.194280044396856\\
144	-0.194280044396856\\
145	-0.194280044396856\\
146	-0.194280044396856\\
147	-0.194280044396856\\
148	-0.194280044396856\\
149	-0.194280044396856\\
150	-0.194280044396856\\
151	-0.194280044396856\\
152	-0.194280044396856\\
153	-0.194280044396856\\
154	-0.194280044396856\\
155	-0.194280044396856\\
156	-0.194280044396856\\
157	-0.194280044396856\\
158	-0.194280044396856\\
159	-0.194280044396856\\
160	-0.194280044396856\\
161	-0.194280044396856\\
162	-0.194280044396856\\
163	-0.194280044396856\\
164	-0.194280044396856\\
165	-0.194280044396856\\
166	-0.194280044396856\\
167	-0.194280044396856\\
168	-0.194280044396856\\
169	-0.194280044396856\\
170	-0.194280044396856\\
171	-0.194280044396856\\
172	-0.194280044396856\\
173	-0.194280044396856\\
174	-0.194280044396856\\
175	-0.194280044396856\\
176	-0.194280044396856\\
177	-0.194280044396856\\
178	-0.194280044396856\\
179	-0.194280044396856\\
180	-0.194280044396856\\
181	-0.194280044396856\\
182	-0.194280044396856\\
183	-0.194280044396856\\
184	-0.194280044396856\\
185	-0.194280044396856\\
186	-0.194280044396856\\
187	-0.194280044396856\\
188	-0.194280044396856\\
189	-0.194280044396856\\
190	-0.194280044396856\\
191	-0.194280044396856\\
192	-0.194280044396856\\
193	-0.194280044396856\\
194	-0.194280044396856\\
195	-0.194280044396856\\
196	-0.194280044396856\\
197	-0.194280044396856\\
198	-0.194280044396856\\
199	-0.194280044396856\\
200	-0.194280044396856\\
201	-0.194280044396856\\
202	-0.194280044396856\\
203	-0.194280044396856\\
204	-0.194280044396856\\
205	-0.194280044396856\\
206	-0.194280044396856\\
207	-0.194280044396856\\
208	-0.194280044396856\\
209	-0.194280044396856\\
210	-0.194280044396856\\
211	-0.194280044396856\\
212	-0.194280044396856\\
213	-0.194280044396856\\
214	-0.194280044396856\\
215	-0.194280044396856\\
216	-0.194280044396856\\
217	-0.194280044396856\\
218	-0.194280044396856\\
219	-0.194280044396856\\
220	-0.194280044396856\\
221	-0.194280044396856\\
222	-0.194280044396856\\
223	-0.194280044396856\\
224	-0.194280044396856\\
225	-0.194280044396856\\
226	-0.194280044396856\\
227	-0.194280044396856\\
228	-0.194280044396856\\
229	-0.194280044396856\\
230	-0.194280044396856\\
231	-0.194280044396856\\
232	-0.194280044396856\\
233	-0.194280044396856\\
234	-0.194280044396856\\
235	-0.194280044396856\\
236	-0.194280044396856\\
237	-0.194280044396856\\
238	-0.194280044396856\\
239	-0.194280044396856\\
240	-0.194280044396856\\
241	-0.194280044396856\\
242	-0.194280044396856\\
243	-0.194280044396856\\
244	-0.194280044396856\\
245	-0.194280044396856\\
246	-0.194280044396856\\
247	-0.194280044396856\\
248	-0.194280044396856\\
249	-0.194280044396856\\
250	-0.194280044396856\\
251	-0.194280044396856\\
252	-0.194280044396856\\
253	-0.194280044396856\\
254	-0.194280044396856\\
255	-0.194280044396856\\
256	-0.194280044396856\\
257	-0.194280044396856\\
258	-0.194280044396856\\
259	-0.194280044396856\\
260	-0.194280044396856\\
261	-0.194280044396856\\
262	-0.194280044396856\\
263	-0.194280044396856\\
264	-0.194280044396856\\
265	-0.194280044396856\\
266	-0.194280044396856\\
267	-0.194280044396856\\
268	-0.194280044396856\\
269	-0.194280044396856\\
270	-0.194280044396856\\
271	-0.194280044396856\\
272	-0.194280044396856\\
273	-0.194280044396856\\
274	-0.194280044396856\\
275	-0.194280044396856\\
276	-0.194280044396856\\
277	-0.194280044396856\\
278	-0.194280044396856\\
279	-0.194280044396856\\
280	-0.194280044396856\\
281	-0.194280044396856\\
282	-0.194280044396856\\
283	-0.194280044396856\\
284	-0.194280044396856\\
285	-0.194280044396856\\
286	-0.194280044396856\\
287	-0.194280044396856\\
288	-0.194280044396856\\
289	-0.194280044396856\\
290	-0.194280044396856\\
291	-0.194280044396856\\
292	-0.194280044396856\\
293	-0.194280044396856\\
294	-0.194280044396856\\
295	-0.194280044396856\\
296	-0.194280044396856\\
297	-0.194280044396856\\
298	-0.194280044396856\\
299	-0.194280044396856\\
300	-0.194280044396856\\
};
\addlegendentry{Skok U z 0.00 do -0.25}

\addplot[const plot, color=mycolor3] table[row sep=crcr] {%
1	0\\
2	0\\
3	0\\
4	0\\
5	0\\
6	0\\
7	0\\
8	0\\
9	0\\
10	0\\
11	0\\
12	0\\
13	0\\
14	0\\
15	0\\
16	0\\
17	0\\
18	0\\
19	0\\
20	0\\
21	0\\
22	0\\
23	0\\
24	0\\
25	0\\
26	0\\
27	0\\
28	0\\
29	0\\
30	0\\
31	0\\
32	0\\
33	0\\
34	0\\
35	0\\
36	0\\
37	0\\
38	0\\
39	0\\
40	0\\
41	0\\
42	0\\
43	0\\
44	0\\
45	0\\
46	0\\
47	0\\
48	0\\
49	0\\
50	0\\
51	0\\
52	0\\
53	0\\
54	0\\
55	0\\
56	0\\
57	0\\
58	0\\
59	0\\
60	0\\
61	0\\
62	0\\
63	0\\
64	0\\
65	0\\
66	0\\
67	0\\
68	0\\
69	0\\
70	0\\
71	0\\
72	0\\
73	0\\
74	0\\
75	0\\
76	0\\
77	0\\
78	0\\
79	0\\
80	0\\
81	0\\
82	0\\
83	0\\
84	0\\
85	0\\
86	0\\
87	0\\
88	0\\
89	0\\
90	0\\
91	0\\
92	0\\
93	0\\
94	0\\
95	0\\
96	0\\
97	0\\
98	0\\
99	0\\
100	0\\
101	0\\
102	0\\
103	0\\
104	0\\
105	0\\
106	0\\
107	0\\
108	0\\
109	0\\
110	0\\
111	0\\
112	0\\
113	0\\
114	0\\
115	0\\
116	0\\
117	0\\
118	0\\
119	0\\
120	0\\
121	0\\
122	0\\
123	0\\
124	0\\
125	0\\
126	0\\
127	0\\
128	0\\
129	0\\
130	0\\
131	0\\
132	0\\
133	0\\
134	0\\
135	0\\
136	0\\
137	0\\
138	0\\
139	0\\
140	0\\
141	0\\
142	0\\
143	0\\
144	0\\
145	0\\
146	0\\
147	0\\
148	0\\
149	0\\
150	0\\
151	0\\
152	0\\
153	0\\
154	0\\
155	0\\
156	0\\
157	0\\
158	0\\
159	0\\
160	0\\
161	0\\
162	0\\
163	0\\
164	0\\
165	0\\
166	0\\
167	0\\
168	0\\
169	0\\
170	0\\
171	0\\
172	0\\
173	0\\
174	0\\
175	0\\
176	0\\
177	0\\
178	0\\
179	0\\
180	0\\
181	0\\
182	0\\
183	0\\
184	0\\
185	0\\
186	0\\
187	0\\
188	0\\
189	0\\
190	0\\
191	0\\
192	0\\
193	0\\
194	0\\
195	0\\
196	0\\
197	0\\
198	0\\
199	0\\
200	0\\
201	0\\
202	0\\
203	0\\
204	0\\
205	0\\
206	0\\
207	0\\
208	0\\
209	0\\
210	0\\
211	0\\
212	0\\
213	0\\
214	0\\
215	0\\
216	0\\
217	0\\
218	0\\
219	0\\
220	0\\
221	0\\
222	0\\
223	0\\
224	0\\
225	0\\
226	0\\
227	0\\
228	0\\
229	0\\
230	0\\
231	0\\
232	0\\
233	0\\
234	0\\
235	0\\
236	0\\
237	0\\
238	0\\
239	0\\
240	0\\
241	0\\
242	0\\
243	0\\
244	0\\
245	0\\
246	0\\
247	0\\
248	0\\
249	0\\
250	0\\
251	0\\
252	0\\
253	0\\
254	0\\
255	0\\
256	0\\
257	0\\
258	0\\
259	0\\
260	0\\
261	0\\
262	0\\
263	0\\
264	0\\
265	0\\
266	0\\
267	0\\
268	0\\
269	0\\
270	0\\
271	0\\
272	0\\
273	0\\
274	0\\
275	0\\
276	0\\
277	0\\
278	0\\
279	0\\
280	0\\
281	0\\
282	0\\
283	0\\
284	0\\
285	0\\
286	0\\
287	0\\
288	0\\
289	0\\
290	0\\
291	0\\
292	0\\
293	0\\
294	0\\
295	0\\
296	0\\
297	0\\
298	0\\
299	0\\
300	0\\
};
\addlegendentry{Skok U z 0.00 do 0.00}

\addplot[const plot, color=mycolor4] table[row sep=crcr] {%
1	0\\
2	0\\
3	0\\
4	0\\
5	0\\
6	0\\
7	0\\
8	0\\
9	0\\
10	0\\
11	0\\
12	0\\
13	0\\
14	0\\
15	0.00916194129720854\\
16	0.0237109142051116\\
17	0.0347854477739337\\
18	0.0414098483215953\\
19	0.0448070287362194\\
20	0.0463239521732835\\
21	0.0468966103926382\\
22	0.0470569189157674\\
23	0.0470665175232664\\
24	0.047037056606585\\
25	0.047007921724263\\
26	0.0469883933986156\\
27	0.0469775416307898\\
28	0.046972307011315\\
29	0.0469701232168774\\
30	0.046969380787595\\
31	0.046969224712116\\
32	0.0469692594342373\\
33	0.0469693282748527\\
34	0.0469693827322709\\
35	0.0469694159507636\\
36	0.0469694332427358\\
37	0.0469694410832711\\
38	0.0469694441068266\\
39	0.0469694449934194\\
40	0.0469694450807196\\
41	0.046969444948964\\
42	0.0469694448080906\\
43	0.0469694447111181\\
44	0.0469694446563247\\
45	0.0469694446295051\\
46	0.0469694446181243\\
47	0.0469694446141454\\
48	0.046969444613232\\
49	0.046969444613348\\
50	0.0469694446136714\\
51	0.0469694446139384\\
52	0.0469694446141047\\
53	0.0469694446141926\\
54	0.046969444614233\\
55	0.046969444614249\\
56	0.0469694446142538\\
57	0.0469694446142545\\
58	0.0469694446142539\\
59	0.0469694446142532\\
60	0.0469694446142527\\
61	0.0469694446142525\\
62	0.0469694446142523\\
63	0.0469694446142523\\
64	0.0469694446142522\\
65	0.0469694446142522\\
66	0.0469694446142522\\
67	0.0469694446142522\\
68	0.0469694446142522\\
69	0.0469694446142522\\
70	0.0469694446142522\\
71	0.0469694446142522\\
72	0.0469694446142522\\
73	0.0469694446142522\\
74	0.0469694446142522\\
75	0.0469694446142522\\
76	0.0469694446142522\\
77	0.0469694446142522\\
78	0.0469694446142522\\
79	0.0469694446142522\\
80	0.0469694446142522\\
81	0.0469694446142522\\
82	0.0469694446142522\\
83	0.0469694446142522\\
84	0.0469694446142522\\
85	0.0469694446142522\\
86	0.0469694446142522\\
87	0.0469694446142522\\
88	0.0469694446142522\\
89	0.0469694446142522\\
90	0.0469694446142522\\
91	0.0469694446142522\\
92	0.0469694446142522\\
93	0.0469694446142522\\
94	0.0469694446142522\\
95	0.0469694446142522\\
96	0.0469694446142522\\
97	0.0469694446142522\\
98	0.0469694446142522\\
99	0.0469694446142522\\
100	0.0469694446142522\\
101	0.0469694446142522\\
102	0.0469694446142522\\
103	0.0469694446142522\\
104	0.0469694446142522\\
105	0.0469694446142522\\
106	0.0469694446142522\\
107	0.0469694446142522\\
108	0.0469694446142522\\
109	0.0469694446142522\\
110	0.0469694446142522\\
111	0.0469694446142522\\
112	0.0469694446142522\\
113	0.0469694446142522\\
114	0.0469694446142522\\
115	0.0469694446142522\\
116	0.0469694446142522\\
117	0.0469694446142522\\
118	0.0469694446142522\\
119	0.0469694446142522\\
120	0.0469694446142522\\
121	0.0469694446142522\\
122	0.0469694446142522\\
123	0.0469694446142522\\
124	0.0469694446142522\\
125	0.0469694446142522\\
126	0.0469694446142522\\
127	0.0469694446142522\\
128	0.0469694446142522\\
129	0.0469694446142522\\
130	0.0469694446142522\\
131	0.0469694446142522\\
132	0.0469694446142522\\
133	0.0469694446142522\\
134	0.0469694446142522\\
135	0.0469694446142522\\
136	0.0469694446142522\\
137	0.0469694446142522\\
138	0.0469694446142522\\
139	0.0469694446142522\\
140	0.0469694446142522\\
141	0.0469694446142522\\
142	0.0469694446142522\\
143	0.0469694446142522\\
144	0.0469694446142522\\
145	0.0469694446142522\\
146	0.0469694446142522\\
147	0.0469694446142522\\
148	0.0469694446142522\\
149	0.0469694446142522\\
150	0.0469694446142522\\
151	0.0469694446142522\\
152	0.0469694446142522\\
153	0.0469694446142522\\
154	0.0469694446142522\\
155	0.0469694446142522\\
156	0.0469694446142522\\
157	0.0469694446142522\\
158	0.0469694446142522\\
159	0.0469694446142522\\
160	0.0469694446142522\\
161	0.0469694446142522\\
162	0.0469694446142522\\
163	0.0469694446142522\\
164	0.0469694446142522\\
165	0.0469694446142522\\
166	0.0469694446142522\\
167	0.0469694446142522\\
168	0.0469694446142522\\
169	0.0469694446142522\\
170	0.0469694446142522\\
171	0.0469694446142522\\
172	0.0469694446142522\\
173	0.0469694446142522\\
174	0.0469694446142522\\
175	0.0469694446142522\\
176	0.0469694446142522\\
177	0.0469694446142522\\
178	0.0469694446142522\\
179	0.0469694446142522\\
180	0.0469694446142522\\
181	0.0469694446142522\\
182	0.0469694446142522\\
183	0.0469694446142522\\
184	0.0469694446142522\\
185	0.0469694446142522\\
186	0.0469694446142522\\
187	0.0469694446142522\\
188	0.0469694446142522\\
189	0.0469694446142522\\
190	0.0469694446142522\\
191	0.0469694446142522\\
192	0.0469694446142522\\
193	0.0469694446142522\\
194	0.0469694446142522\\
195	0.0469694446142522\\
196	0.0469694446142522\\
197	0.0469694446142522\\
198	0.0469694446142522\\
199	0.0469694446142522\\
200	0.0469694446142522\\
201	0.0469694446142522\\
202	0.0469694446142522\\
203	0.0469694446142522\\
204	0.0469694446142522\\
205	0.0469694446142522\\
206	0.0469694446142522\\
207	0.0469694446142522\\
208	0.0469694446142522\\
209	0.0469694446142522\\
210	0.0469694446142522\\
211	0.0469694446142522\\
212	0.0469694446142522\\
213	0.0469694446142522\\
214	0.0469694446142522\\
215	0.0469694446142522\\
216	0.0469694446142522\\
217	0.0469694446142522\\
218	0.0469694446142522\\
219	0.0469694446142522\\
220	0.0469694446142522\\
221	0.0469694446142522\\
222	0.0469694446142522\\
223	0.0469694446142522\\
224	0.0469694446142522\\
225	0.0469694446142522\\
226	0.0469694446142522\\
227	0.0469694446142522\\
228	0.0469694446142522\\
229	0.0469694446142522\\
230	0.0469694446142522\\
231	0.0469694446142522\\
232	0.0469694446142522\\
233	0.0469694446142522\\
234	0.0469694446142522\\
235	0.0469694446142522\\
236	0.0469694446142522\\
237	0.0469694446142522\\
238	0.0469694446142522\\
239	0.0469694446142522\\
240	0.0469694446142522\\
241	0.0469694446142522\\
242	0.0469694446142522\\
243	0.0469694446142522\\
244	0.0469694446142522\\
245	0.0469694446142522\\
246	0.0469694446142522\\
247	0.0469694446142522\\
248	0.0469694446142522\\
249	0.0469694446142522\\
250	0.0469694446142522\\
251	0.0469694446142522\\
252	0.0469694446142522\\
253	0.0469694446142522\\
254	0.0469694446142522\\
255	0.0469694446142522\\
256	0.0469694446142522\\
257	0.0469694446142522\\
258	0.0469694446142522\\
259	0.0469694446142522\\
260	0.0469694446142522\\
261	0.0469694446142522\\
262	0.0469694446142522\\
263	0.0469694446142522\\
264	0.0469694446142522\\
265	0.0469694446142522\\
266	0.0469694446142522\\
267	0.0469694446142522\\
268	0.0469694446142522\\
269	0.0469694446142522\\
270	0.0469694446142522\\
271	0.0469694446142522\\
272	0.0469694446142522\\
273	0.0469694446142522\\
274	0.0469694446142522\\
275	0.0469694446142522\\
276	0.0469694446142522\\
277	0.0469694446142522\\
278	0.0469694446142522\\
279	0.0469694446142522\\
280	0.0469694446142522\\
281	0.0469694446142522\\
282	0.0469694446142522\\
283	0.0469694446142522\\
284	0.0469694446142522\\
285	0.0469694446142522\\
286	0.0469694446142522\\
287	0.0469694446142522\\
288	0.0469694446142522\\
289	0.0469694446142522\\
290	0.0469694446142522\\
291	0.0469694446142522\\
292	0.0469694446142522\\
293	0.0469694446142522\\
294	0.0469694446142522\\
295	0.0469694446142522\\
296	0.0469694446142522\\
297	0.0469694446142522\\
298	0.0469694446142522\\
299	0.0469694446142522\\
300	0.0469694446142522\\
};
\addlegendentry{Skok U z 0.00 do 0.25}

\addplot[const plot, color=mycolor5] table[row sep=crcr] {%
1	0\\
2	0\\
3	0\\
4	0\\
5	0\\
6	0\\
7	0\\
8	0\\
9	0\\
10	0\\
11	0\\
12	0\\
13	0\\
14	0\\
15	0.0149374122807018\\
16	0.035320856777193\\
17	0.0488293794044997\\
18	0.0561147600020654\\
19	0.0596439430779413\\
20	0.0612356364241139\\
21	0.0619148017418959\\
22	0.062190952713034\\
23	0.0622981522140596\\
24	0.0623377779031105\\
25	0.0623516109313598\\
26	0.0623560896317763\\
27	0.0623573801404703\\
28	0.06235767363587\\
29	0.06235769714626\\
30	0.062357668800254\\
31	0.0623576429737223\\
32	0.0623576273557165\\
33	0.0623576192920019\\
34	0.062357615491721\\
35	0.0623576138125338\\
36	0.0623576131082947\\
37	0.062357612826517\\
38	0.0623576127189196\\
39	0.0623576126798786\\
40	0.0623576126665644\\
41	0.062357612662397\\
42	0.0623576126612666\\
43	0.0623576126610484\\
44	0.0623576126610581\\
45	0.0623576126611003\\
46	0.0623576126611317\\
47	0.0623576126611494\\
48	0.0623576126611583\\
49	0.0623576126611623\\
50	0.0623576126611641\\
51	0.0623576126611648\\
52	0.0623576126611651\\
53	0.0623576126611652\\
54	0.0623576126611652\\
55	0.0623576126611653\\
56	0.0623576126611652\\
57	0.0623576126611652\\
58	0.0623576126611652\\
59	0.0623576126611652\\
60	0.0623576126611652\\
61	0.0623576126611652\\
62	0.0623576126611652\\
63	0.0623576126611652\\
64	0.0623576126611652\\
65	0.0623576126611652\\
66	0.0623576126611652\\
67	0.0623576126611652\\
68	0.0623576126611652\\
69	0.0623576126611652\\
70	0.0623576126611652\\
71	0.0623576126611652\\
72	0.0623576126611652\\
73	0.0623576126611652\\
74	0.0623576126611652\\
75	0.0623576126611652\\
76	0.0623576126611652\\
77	0.0623576126611652\\
78	0.0623576126611652\\
79	0.0623576126611652\\
80	0.0623576126611652\\
81	0.0623576126611652\\
82	0.0623576126611652\\
83	0.0623576126611652\\
84	0.0623576126611652\\
85	0.0623576126611652\\
86	0.0623576126611652\\
87	0.0623576126611652\\
88	0.0623576126611652\\
89	0.0623576126611652\\
90	0.0623576126611652\\
91	0.0623576126611652\\
92	0.0623576126611652\\
93	0.0623576126611652\\
94	0.0623576126611652\\
95	0.0623576126611652\\
96	0.0623576126611652\\
97	0.0623576126611652\\
98	0.0623576126611652\\
99	0.0623576126611652\\
100	0.0623576126611652\\
101	0.0623576126611652\\
102	0.0623576126611652\\
103	0.0623576126611652\\
104	0.0623576126611652\\
105	0.0623576126611652\\
106	0.0623576126611652\\
107	0.0623576126611652\\
108	0.0623576126611652\\
109	0.0623576126611652\\
110	0.0623576126611652\\
111	0.0623576126611652\\
112	0.0623576126611652\\
113	0.0623576126611652\\
114	0.0623576126611652\\
115	0.0623576126611652\\
116	0.0623576126611652\\
117	0.0623576126611652\\
118	0.0623576126611652\\
119	0.0623576126611652\\
120	0.0623576126611652\\
121	0.0623576126611652\\
122	0.0623576126611652\\
123	0.0623576126611652\\
124	0.0623576126611652\\
125	0.0623576126611652\\
126	0.0623576126611652\\
127	0.0623576126611652\\
128	0.0623576126611652\\
129	0.0623576126611652\\
130	0.0623576126611652\\
131	0.0623576126611652\\
132	0.0623576126611652\\
133	0.0623576126611652\\
134	0.0623576126611652\\
135	0.0623576126611652\\
136	0.0623576126611652\\
137	0.0623576126611652\\
138	0.0623576126611652\\
139	0.0623576126611652\\
140	0.0623576126611652\\
141	0.0623576126611652\\
142	0.0623576126611652\\
143	0.0623576126611652\\
144	0.0623576126611652\\
145	0.0623576126611652\\
146	0.0623576126611652\\
147	0.0623576126611652\\
148	0.0623576126611652\\
149	0.0623576126611652\\
150	0.0623576126611652\\
151	0.0623576126611652\\
152	0.0623576126611652\\
153	0.0623576126611652\\
154	0.0623576126611652\\
155	0.0623576126611652\\
156	0.0623576126611652\\
157	0.0623576126611652\\
158	0.0623576126611652\\
159	0.0623576126611652\\
160	0.0623576126611652\\
161	0.0623576126611652\\
162	0.0623576126611652\\
163	0.0623576126611652\\
164	0.0623576126611652\\
165	0.0623576126611652\\
166	0.0623576126611652\\
167	0.0623576126611652\\
168	0.0623576126611652\\
169	0.0623576126611652\\
170	0.0623576126611652\\
171	0.0623576126611652\\
172	0.0623576126611652\\
173	0.0623576126611652\\
174	0.0623576126611652\\
175	0.0623576126611652\\
176	0.0623576126611652\\
177	0.0623576126611652\\
178	0.0623576126611652\\
179	0.0623576126611652\\
180	0.0623576126611652\\
181	0.0623576126611652\\
182	0.0623576126611652\\
183	0.0623576126611652\\
184	0.0623576126611652\\
185	0.0623576126611652\\
186	0.0623576126611652\\
187	0.0623576126611652\\
188	0.0623576126611652\\
189	0.0623576126611652\\
190	0.0623576126611652\\
191	0.0623576126611652\\
192	0.0623576126611652\\
193	0.0623576126611652\\
194	0.0623576126611652\\
195	0.0623576126611652\\
196	0.0623576126611652\\
197	0.0623576126611652\\
198	0.0623576126611652\\
199	0.0623576126611652\\
200	0.0623576126611652\\
201	0.0623576126611652\\
202	0.0623576126611652\\
203	0.0623576126611652\\
204	0.0623576126611652\\
205	0.0623576126611652\\
206	0.0623576126611652\\
207	0.0623576126611652\\
208	0.0623576126611652\\
209	0.0623576126611652\\
210	0.0623576126611652\\
211	0.0623576126611652\\
212	0.0623576126611652\\
213	0.0623576126611652\\
214	0.0623576126611652\\
215	0.0623576126611652\\
216	0.0623576126611652\\
217	0.0623576126611652\\
218	0.0623576126611652\\
219	0.0623576126611652\\
220	0.0623576126611652\\
221	0.0623576126611652\\
222	0.0623576126611652\\
223	0.0623576126611652\\
224	0.0623576126611652\\
225	0.0623576126611652\\
226	0.0623576126611652\\
227	0.0623576126611652\\
228	0.0623576126611652\\
229	0.0623576126611652\\
230	0.0623576126611652\\
231	0.0623576126611652\\
232	0.0623576126611652\\
233	0.0623576126611652\\
234	0.0623576126611652\\
235	0.0623576126611652\\
236	0.0623576126611652\\
237	0.0623576126611652\\
238	0.0623576126611652\\
239	0.0623576126611652\\
240	0.0623576126611652\\
241	0.0623576126611652\\
242	0.0623576126611652\\
243	0.0623576126611652\\
244	0.0623576126611652\\
245	0.0623576126611652\\
246	0.0623576126611652\\
247	0.0623576126611652\\
248	0.0623576126611652\\
249	0.0623576126611652\\
250	0.0623576126611652\\
251	0.0623576126611652\\
252	0.0623576126611652\\
253	0.0623576126611652\\
254	0.0623576126611652\\
255	0.0623576126611652\\
256	0.0623576126611652\\
257	0.0623576126611652\\
258	0.0623576126611652\\
259	0.0623576126611652\\
260	0.0623576126611652\\
261	0.0623576126611652\\
262	0.0623576126611652\\
263	0.0623576126611652\\
264	0.0623576126611652\\
265	0.0623576126611652\\
266	0.0623576126611652\\
267	0.0623576126611652\\
268	0.0623576126611652\\
269	0.0623576126611652\\
270	0.0623576126611652\\
271	0.0623576126611652\\
272	0.0623576126611652\\
273	0.0623576126611652\\
274	0.0623576126611652\\
275	0.0623576126611652\\
276	0.0623576126611652\\
277	0.0623576126611652\\
278	0.0623576126611652\\
279	0.0623576126611652\\
280	0.0623576126611652\\
281	0.0623576126611652\\
282	0.0623576126611652\\
283	0.0623576126611652\\
284	0.0623576126611652\\
285	0.0623576126611652\\
286	0.0623576126611652\\
287	0.0623576126611652\\
288	0.0623576126611652\\
289	0.0623576126611652\\
290	0.0623576126611652\\
291	0.0623576126611652\\
292	0.0623576126611652\\
293	0.0623576126611652\\
294	0.0623576126611652\\
295	0.0623576126611652\\
296	0.0623576126611652\\
297	0.0623576126611652\\
298	0.0623576126611652\\
299	0.0623576126611652\\
300	0.0623576126611652\\
};
\addlegendentry{Skok U z 0.00 do 0.50}

\addplot[const plot, color=mycolor6] table[row sep=crcr] {%
1	0\\
2	0\\
3	0\\
4	0\\
5	0\\
6	0\\
7	0\\
8	0\\
9	0\\
10	0\\
11	0\\
12	0\\
13	0\\
14	0\\
15	0.0199798481468771\\
16	0.0447848083375242\\
17	0.0601297465363197\\
18	0.0681153267352949\\
19	0.0719760077764885\\
20	0.0737738156815136\\
21	0.0745938026972925\\
22	0.0749633286294141\\
23	0.0751286671851614\\
24	0.0752023261210796\\
25	0.075235055052244\\
26	0.0752495740370943\\
27	0.0752560084486156\\
28	0.0752588582538279\\
29	0.0752601199569219\\
30	0.0752606784240591\\
31	0.075260925582425\\
32	0.075261034956452\\
33	0.0752610833546389\\
34	0.0752611047701843\\
35	0.0752611142460742\\
36	0.0752611184388843\\
37	0.0752611202940678\\
38	0.0752611211149227\\
39	0.0752611214781216\\
40	0.0752611216388238\\
41	0.0752611217099285\\
42	0.0752611217413897\\
43	0.0752611217553101\\
44	0.0752611217614694\\
45	0.0752611217641946\\
46	0.0752611217654004\\
47	0.075261121765934\\
48	0.07526112176617\\
49	0.0752611217662745\\
50	0.0752611217663207\\
51	0.0752611217663411\\
52	0.0752611217663502\\
53	0.0752611217663542\\
54	0.075261121766356\\
55	0.0752611217663567\\
56	0.0752611217663571\\
57	0.0752611217663572\\
58	0.0752611217663573\\
59	0.0752611217663573\\
60	0.0752611217663573\\
61	0.0752611217663573\\
62	0.0752611217663573\\
63	0.0752611217663573\\
64	0.0752611217663573\\
65	0.0752611217663573\\
66	0.0752611217663573\\
67	0.0752611217663573\\
68	0.0752611217663573\\
69	0.0752611217663573\\
70	0.0752611217663573\\
71	0.0752611217663573\\
72	0.0752611217663573\\
73	0.0752611217663573\\
74	0.0752611217663573\\
75	0.0752611217663573\\
76	0.0752611217663573\\
77	0.0752611217663573\\
78	0.0752611217663573\\
79	0.0752611217663573\\
80	0.0752611217663573\\
81	0.0752611217663573\\
82	0.0752611217663573\\
83	0.0752611217663573\\
84	0.0752611217663573\\
85	0.0752611217663573\\
86	0.0752611217663573\\
87	0.0752611217663573\\
88	0.0752611217663573\\
89	0.0752611217663573\\
90	0.0752611217663573\\
91	0.0752611217663573\\
92	0.0752611217663573\\
93	0.0752611217663573\\
94	0.0752611217663573\\
95	0.0752611217663573\\
96	0.0752611217663573\\
97	0.0752611217663573\\
98	0.0752611217663573\\
99	0.0752611217663573\\
100	0.0752611217663573\\
101	0.0752611217663573\\
102	0.0752611217663573\\
103	0.0752611217663573\\
104	0.0752611217663573\\
105	0.0752611217663573\\
106	0.0752611217663573\\
107	0.0752611217663573\\
108	0.0752611217663573\\
109	0.0752611217663573\\
110	0.0752611217663573\\
111	0.0752611217663573\\
112	0.0752611217663573\\
113	0.0752611217663573\\
114	0.0752611217663573\\
115	0.0752611217663573\\
116	0.0752611217663573\\
117	0.0752611217663573\\
118	0.0752611217663573\\
119	0.0752611217663573\\
120	0.0752611217663573\\
121	0.0752611217663573\\
122	0.0752611217663573\\
123	0.0752611217663573\\
124	0.0752611217663573\\
125	0.0752611217663573\\
126	0.0752611217663573\\
127	0.0752611217663573\\
128	0.0752611217663573\\
129	0.0752611217663573\\
130	0.0752611217663573\\
131	0.0752611217663573\\
132	0.0752611217663573\\
133	0.0752611217663573\\
134	0.0752611217663573\\
135	0.0752611217663573\\
136	0.0752611217663573\\
137	0.0752611217663573\\
138	0.0752611217663573\\
139	0.0752611217663573\\
140	0.0752611217663573\\
141	0.0752611217663573\\
142	0.0752611217663573\\
143	0.0752611217663573\\
144	0.0752611217663573\\
145	0.0752611217663573\\
146	0.0752611217663573\\
147	0.0752611217663573\\
148	0.0752611217663573\\
149	0.0752611217663573\\
150	0.0752611217663573\\
151	0.0752611217663573\\
152	0.0752611217663573\\
153	0.0752611217663573\\
154	0.0752611217663573\\
155	0.0752611217663573\\
156	0.0752611217663573\\
157	0.0752611217663573\\
158	0.0752611217663573\\
159	0.0752611217663573\\
160	0.0752611217663573\\
161	0.0752611217663573\\
162	0.0752611217663573\\
163	0.0752611217663573\\
164	0.0752611217663573\\
165	0.0752611217663573\\
166	0.0752611217663573\\
167	0.0752611217663573\\
168	0.0752611217663573\\
169	0.0752611217663573\\
170	0.0752611217663573\\
171	0.0752611217663573\\
172	0.0752611217663573\\
173	0.0752611217663573\\
174	0.0752611217663573\\
175	0.0752611217663573\\
176	0.0752611217663573\\
177	0.0752611217663573\\
178	0.0752611217663573\\
179	0.0752611217663573\\
180	0.0752611217663573\\
181	0.0752611217663573\\
182	0.0752611217663573\\
183	0.0752611217663573\\
184	0.0752611217663573\\
185	0.0752611217663573\\
186	0.0752611217663573\\
187	0.0752611217663573\\
188	0.0752611217663573\\
189	0.0752611217663573\\
190	0.0752611217663573\\
191	0.0752611217663573\\
192	0.0752611217663573\\
193	0.0752611217663573\\
194	0.0752611217663573\\
195	0.0752611217663573\\
196	0.0752611217663573\\
197	0.0752611217663573\\
198	0.0752611217663573\\
199	0.0752611217663573\\
200	0.0752611217663573\\
201	0.0752611217663573\\
202	0.0752611217663573\\
203	0.0752611217663573\\
204	0.0752611217663573\\
205	0.0752611217663573\\
206	0.0752611217663573\\
207	0.0752611217663573\\
208	0.0752611217663573\\
209	0.0752611217663573\\
210	0.0752611217663573\\
211	0.0752611217663573\\
212	0.0752611217663573\\
213	0.0752611217663573\\
214	0.0752611217663573\\
215	0.0752611217663573\\
216	0.0752611217663573\\
217	0.0752611217663573\\
218	0.0752611217663573\\
219	0.0752611217663573\\
220	0.0752611217663573\\
221	0.0752611217663573\\
222	0.0752611217663573\\
223	0.0752611217663573\\
224	0.0752611217663573\\
225	0.0752611217663573\\
226	0.0752611217663573\\
227	0.0752611217663573\\
228	0.0752611217663573\\
229	0.0752611217663573\\
230	0.0752611217663573\\
231	0.0752611217663573\\
232	0.0752611217663573\\
233	0.0752611217663573\\
234	0.0752611217663573\\
235	0.0752611217663573\\
236	0.0752611217663573\\
237	0.0752611217663573\\
238	0.0752611217663573\\
239	0.0752611217663573\\
240	0.0752611217663573\\
241	0.0752611217663573\\
242	0.0752611217663573\\
243	0.0752611217663573\\
244	0.0752611217663573\\
245	0.0752611217663573\\
246	0.0752611217663573\\
247	0.0752611217663573\\
248	0.0752611217663573\\
249	0.0752611217663573\\
250	0.0752611217663573\\
251	0.0752611217663573\\
252	0.0752611217663573\\
253	0.0752611217663573\\
254	0.0752611217663573\\
255	0.0752611217663573\\
256	0.0752611217663573\\
257	0.0752611217663573\\
258	0.0752611217663573\\
259	0.0752611217663573\\
260	0.0752611217663573\\
261	0.0752611217663573\\
262	0.0752611217663573\\
263	0.0752611217663573\\
264	0.0752611217663573\\
265	0.0752611217663573\\
266	0.0752611217663573\\
267	0.0752611217663573\\
268	0.0752611217663573\\
269	0.0752611217663573\\
270	0.0752611217663573\\
271	0.0752611217663573\\
272	0.0752611217663573\\
273	0.0752611217663573\\
274	0.0752611217663573\\
275	0.0752611217663573\\
276	0.0752611217663573\\
277	0.0752611217663573\\
278	0.0752611217663573\\
279	0.0752611217663573\\
280	0.0752611217663573\\
281	0.0752611217663573\\
282	0.0752611217663573\\
283	0.0752611217663573\\
284	0.0752611217663573\\
285	0.0752611217663573\\
286	0.0752611217663573\\
287	0.0752611217663573\\
288	0.0752611217663573\\
289	0.0752611217663573\\
290	0.0752611217663573\\
291	0.0752611217663573\\
292	0.0752611217663573\\
293	0.0752611217663573\\
294	0.0752611217663573\\
295	0.0752611217663573\\
296	0.0752611217663573\\
297	0.0752611217663573\\
298	0.0752611217663573\\
299	0.0752611217663573\\
300	0.0752611217663573\\
};
\addlegendentry{Skok U z 0.00 do 0.75}

\addplot[const plot, color=mycolor7] table[row sep=crcr] {%
1	0\\
2	0\\
3	0\\
4	0\\
5	0\\
6	0\\
7	0\\
8	0\\
9	0\\
10	0\\
11	0\\
12	0\\
13	0\\
14	0\\
15	0.0249067777777778\\
16	0.0540264030651852\\
17	0.0713864999938143\\
18	0.0802973721942573\\
19	0.0846280344990221\\
20	0.0866849533107856\\
21	0.0876520115004241\\
22	0.0881045698142786\\
23	0.0883159045205792\\
24	0.0884144960135448\\
25	0.0884604697743512\\
26	0.0884819030636113\\
27	0.0884918944337098\\
28	0.0884965518120525\\
29	0.0884987227571013\\
30	0.0884997346904105\\
31	0.0885002063764041\\
32	0.0885004262399126\\
33	0.0885005287231659\\
34	0.0885005764928595\\
35	0.0885005987593575\\
36	0.0885006091382576\\
37	0.0885006139760891\\
38	0.0885006162311077\\
39	0.0885006172822209\\
40	0.0885006177721677\\
41	0.0885006180005426\\
42	0.0885006181069931\\
43	0.0885006181566119\\
44	0.0885006181797404\\
45	0.088500618190521\\
46	0.0885006181955462\\
47	0.0885006181978885\\
48	0.0885006181989803\\
49	0.0885006181994892\\
50	0.0885006181997264\\
51	0.088500618199837\\
52	0.0885006181998885\\
53	0.0885006181999125\\
54	0.0885006181999237\\
55	0.088500618199929\\
56	0.0885006181999314\\
57	0.0885006181999325\\
58	0.0885006181999331\\
59	0.0885006181999333\\
60	0.0885006181999334\\
61	0.0885006181999334\\
62	0.0885006181999335\\
63	0.0885006181999335\\
64	0.0885006181999335\\
65	0.0885006181999335\\
66	0.0885006181999335\\
67	0.0885006181999335\\
68	0.0885006181999335\\
69	0.0885006181999335\\
70	0.0885006181999335\\
71	0.0885006181999335\\
72	0.0885006181999335\\
73	0.0885006181999335\\
74	0.0885006181999335\\
75	0.0885006181999335\\
76	0.0885006181999335\\
77	0.0885006181999335\\
78	0.0885006181999335\\
79	0.0885006181999335\\
80	0.0885006181999335\\
81	0.0885006181999335\\
82	0.0885006181999335\\
83	0.0885006181999335\\
84	0.0885006181999335\\
85	0.0885006181999335\\
86	0.0885006181999335\\
87	0.0885006181999335\\
88	0.0885006181999335\\
89	0.0885006181999335\\
90	0.0885006181999335\\
91	0.0885006181999335\\
92	0.0885006181999335\\
93	0.0885006181999335\\
94	0.0885006181999335\\
95	0.0885006181999335\\
96	0.0885006181999335\\
97	0.0885006181999335\\
98	0.0885006181999335\\
99	0.0885006181999335\\
100	0.0885006181999335\\
101	0.0885006181999335\\
102	0.0885006181999335\\
103	0.0885006181999335\\
104	0.0885006181999335\\
105	0.0885006181999335\\
106	0.0885006181999335\\
107	0.0885006181999335\\
108	0.0885006181999335\\
109	0.0885006181999335\\
110	0.0885006181999335\\
111	0.0885006181999335\\
112	0.0885006181999335\\
113	0.0885006181999335\\
114	0.0885006181999335\\
115	0.0885006181999335\\
116	0.0885006181999335\\
117	0.0885006181999335\\
118	0.0885006181999335\\
119	0.0885006181999335\\
120	0.0885006181999335\\
121	0.0885006181999335\\
122	0.0885006181999335\\
123	0.0885006181999335\\
124	0.0885006181999335\\
125	0.0885006181999335\\
126	0.0885006181999335\\
127	0.0885006181999335\\
128	0.0885006181999335\\
129	0.0885006181999335\\
130	0.0885006181999335\\
131	0.0885006181999335\\
132	0.0885006181999335\\
133	0.0885006181999335\\
134	0.0885006181999335\\
135	0.0885006181999335\\
136	0.0885006181999335\\
137	0.0885006181999335\\
138	0.0885006181999335\\
139	0.0885006181999335\\
140	0.0885006181999335\\
141	0.0885006181999335\\
142	0.0885006181999335\\
143	0.0885006181999335\\
144	0.0885006181999335\\
145	0.0885006181999335\\
146	0.0885006181999335\\
147	0.0885006181999335\\
148	0.0885006181999335\\
149	0.0885006181999335\\
150	0.0885006181999335\\
151	0.0885006181999335\\
152	0.0885006181999335\\
153	0.0885006181999335\\
154	0.0885006181999335\\
155	0.0885006181999335\\
156	0.0885006181999335\\
157	0.0885006181999335\\
158	0.0885006181999335\\
159	0.0885006181999335\\
160	0.0885006181999335\\
161	0.0885006181999335\\
162	0.0885006181999335\\
163	0.0885006181999335\\
164	0.0885006181999335\\
165	0.0885006181999335\\
166	0.0885006181999335\\
167	0.0885006181999335\\
168	0.0885006181999335\\
169	0.0885006181999335\\
170	0.0885006181999335\\
171	0.0885006181999335\\
172	0.0885006181999335\\
173	0.0885006181999335\\
174	0.0885006181999335\\
175	0.0885006181999335\\
176	0.0885006181999335\\
177	0.0885006181999335\\
178	0.0885006181999335\\
179	0.0885006181999335\\
180	0.0885006181999335\\
181	0.0885006181999335\\
182	0.0885006181999335\\
183	0.0885006181999335\\
184	0.0885006181999335\\
185	0.0885006181999335\\
186	0.0885006181999335\\
187	0.0885006181999335\\
188	0.0885006181999335\\
189	0.0885006181999335\\
190	0.0885006181999335\\
191	0.0885006181999335\\
192	0.0885006181999335\\
193	0.0885006181999335\\
194	0.0885006181999335\\
195	0.0885006181999335\\
196	0.0885006181999335\\
197	0.0885006181999335\\
198	0.0885006181999335\\
199	0.0885006181999335\\
200	0.0885006181999335\\
201	0.0885006181999335\\
202	0.0885006181999335\\
203	0.0885006181999335\\
204	0.0885006181999335\\
205	0.0885006181999335\\
206	0.0885006181999335\\
207	0.0885006181999335\\
208	0.0885006181999335\\
209	0.0885006181999335\\
210	0.0885006181999335\\
211	0.0885006181999335\\
212	0.0885006181999335\\
213	0.0885006181999335\\
214	0.0885006181999335\\
215	0.0885006181999335\\
216	0.0885006181999335\\
217	0.0885006181999335\\
218	0.0885006181999335\\
219	0.0885006181999335\\
220	0.0885006181999335\\
221	0.0885006181999335\\
222	0.0885006181999335\\
223	0.0885006181999335\\
224	0.0885006181999335\\
225	0.0885006181999335\\
226	0.0885006181999335\\
227	0.0885006181999335\\
228	0.0885006181999335\\
229	0.0885006181999335\\
230	0.0885006181999335\\
231	0.0885006181999335\\
232	0.0885006181999335\\
233	0.0885006181999335\\
234	0.0885006181999335\\
235	0.0885006181999335\\
236	0.0885006181999335\\
237	0.0885006181999335\\
238	0.0885006181999335\\
239	0.0885006181999335\\
240	0.0885006181999335\\
241	0.0885006181999335\\
242	0.0885006181999335\\
243	0.0885006181999335\\
244	0.0885006181999335\\
245	0.0885006181999335\\
246	0.0885006181999335\\
247	0.0885006181999335\\
248	0.0885006181999335\\
249	0.0885006181999335\\
250	0.0885006181999335\\
251	0.0885006181999335\\
252	0.0885006181999335\\
253	0.0885006181999335\\
254	0.0885006181999335\\
255	0.0885006181999335\\
256	0.0885006181999335\\
257	0.0885006181999335\\
258	0.0885006181999335\\
259	0.0885006181999335\\
260	0.0885006181999335\\
261	0.0885006181999335\\
262	0.0885006181999335\\
263	0.0885006181999335\\
264	0.0885006181999335\\
265	0.0885006181999335\\
266	0.0885006181999335\\
267	0.0885006181999335\\
268	0.0885006181999335\\
269	0.0885006181999335\\
270	0.0885006181999335\\
271	0.0885006181999335\\
272	0.0885006181999335\\
273	0.0885006181999335\\
274	0.0885006181999335\\
275	0.0885006181999335\\
276	0.0885006181999335\\
277	0.0885006181999335\\
278	0.0885006181999335\\
279	0.0885006181999335\\
280	0.0885006181999335\\
281	0.0885006181999335\\
282	0.0885006181999335\\
283	0.0885006181999335\\
284	0.0885006181999335\\
285	0.0885006181999335\\
286	0.0885006181999335\\
287	0.0885006181999335\\
288	0.0885006181999335\\
289	0.0885006181999335\\
290	0.0885006181999335\\
291	0.0885006181999335\\
292	0.0885006181999335\\
293	0.0885006181999335\\
294	0.0885006181999335\\
295	0.0885006181999335\\
296	0.0885006181999335\\
297	0.0885006181999335\\
298	0.0885006181999335\\
299	0.0885006181999335\\
300	0.0885006181999335\\
};
\addlegendentry{Skok U z 0.00 do 1.00}

\end{axis}
\end{tikzpicture}%
\caption{Wykresy odpowiedzi skokowych}
\end{figure}

Jak widać wartość skoku na wyjściu jest nieproporcjonalna wartości skoku wejścia.

Rozważane były też większe skoki dodatnie, natomiast wyniki leżały bardzo blisko wyjściu ze skoku do $0,25$, więc dla lepszej czytelności wykresu nie są one na nim pokazane.

\section{Charakterystyka statyczna}

Po przeprowadzeniu badań o większej częstotliwości próbek dostajemy wykres charakterystyki statycznej:

\begin{figure}[H]
\centering
% This file was created by matlab2tikz.
%
%The latest updates can be retrieved from
%  http://www.mathworks.com/matlabcentral/fileexchange/22022-matlab2tikz-matlab2tikz
%where you can also make suggestions and rate matlab2tikz.
%
\definecolor{mycolor1}{rgb}{0.00000,0.44700,0.74100}%
%
\begin{tikzpicture}

\begin{axis}[%
width=4.577in,
height=3.209in,
at={(0.768in,0.433in)},
scale only axis,
xmin=-1,
xmax=1,
xlabel style={font=\color{white!15!black}},
xlabel={u},
ymin=-3,
ymax=0.5,
ylabel style={font=\color{white!15!black}},
ylabel={y},
axis background/.style={fill=white}
]
\addplot [color=mycolor1, forget plot]
  table[row sep=crcr]{%
-1	-2.64166344592188\\
-0.9	-2.16440050489791\\
-0.8	-1.7280133231197\\
-0.7	-1.33583324714136\\
-0.6	-0.990455737715096\\
-0.5	-0.694919894842755\\
-0.4	-0.452828270800773\\
-0.3	-0.266659921667101\\
-0.2	-0.13482948618586\\
-0.1	-0.0498921814515263\\
0	0\\
0.1	0.0273436107653566\\
0.2	0.0421318974433539\\
0.3	0.050844335685454\\
0.4	0.0570292739041871\\
0.5	0.0623576126611652\\
0.6	0.0674971164230519\\
0.7	0.0726580236832028\\
0.8	0.0778792673693276\\
0.9	0.0831583733324931\\
1	0.0885006181999335\\
};
\end{axis}
\end{tikzpicture}%
\caption{Wykresy odpowiedzi skokowych}
\end{figure}

Nie jest to wykres prostej, więc charakterystyka jest nieliniowa.




