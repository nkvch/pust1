%! TEX encoding = utf8
\chapter{Laboratorium}

\section{Określenie wartości pomiaru temperatury w punkcie pracy}

W celu określenia wartości pomiaru temperatury w punkcie pracy ustawiono moc wentylatora  $W1 = 50\%$,a moc grzałki $G1 = 25\%$.
Po czasie około 5 minut temperatura odczytywana przez czujnik temperatury zaczeła się stabilizować  na poziomie  $T1 = 28,2^{\circ} C$.

Niestety z powodu ciągłego ruchu powietrza związanego z przemieszczaniem się osób w sali i dużej ilości tych osób wpływających na temperaturę sali oraz czułość stanowiska pomiarowego temperatura odczytywana przez czujnik zaczeła odbiegać i lekko oscylować wokół tej temperatury.

\begin{figure}[H]
\centering
% This file was created by matlab2tikz.
%
%The latest updates can be retrieved from
%  http://www.mathworks.com/matlabcentral/fileexchange/22022-matlab2tikz-matlab2tikz
%where you can also make suggestions and rate matlab2tikz.
%
\definecolor{mycolor1}{rgb}{0.00000,0.44700,0.74100}%
%
\begin{tikzpicture}

\begin{axis}[%
width=4.521in,
height=3.566in,
at={(0.758in,0.481in)},
scale only axis,
xmin=0,
xmax=450,
xlabel style={font=\color{white!15!black}},
xlabel={k},
ymin=25,
ymax=31,
ylabel style={font=\color{white!15!black}},
ylabel={$\text{T[}^\circ\text{C]}$},
axis background/.style={fill=white}
]
\addplot[const plot, color=mycolor1, forget plot] table[row sep=crcr] {%
1	25.12\\
2	25.12\\
3	25.18\\
4	25.25\\
5	25.37\\
6	25.43\\
7	25.5\\
8	25.62\\
9	25.68\\
10	25.75\\
11	25.81\\
12	25.87\\
13	26\\
14	26.06\\
15	26.12\\
16	26.18\\
17	26.25\\
18	26.31\\
19	26.43\\
20	26.5\\
21	26.56\\
22	26.62\\
23	26.68\\
24	26.68\\
25	26.81\\
26	26.81\\
27	26.87\\
28	26.93\\
29	27\\
30	27.06\\
31	27.12\\
32	27.18\\
33	27.18\\
34	27.25\\
35	27.31\\
36	27.37\\
37	27.37\\
38	27.43\\
39	27.5\\
40	27.5\\
41	27.56\\
42	27.62\\
43	27.62\\
44	27.68\\
45	27.68\\
46	27.75\\
47	27.81\\
48	27.81\\
49	27.87\\
50	27.87\\
51	27.93\\
52	28\\
53	28\\
54	28\\
55	28.06\\
56	28.06\\
57	28.12\\
58	28.12\\
59	28.18\\
60	28.18\\
61	28.25\\
62	28.25\\
63	28.25\\
64	28.31\\
65	28.37\\
66	28.37\\
67	28.37\\
68	28.43\\
69	28.43\\
70	28.5\\
71	28.5\\
72	28.5\\
73	28.56\\
74	28.56\\
75	28.56\\
76	28.62\\
77	28.62\\
78	28.68\\
79	28.68\\
80	28.68\\
81	28.75\\
82	28.75\\
83	28.81\\
84	28.81\\
85	28.81\\
86	28.87\\
87	28.87\\
88	28.93\\
89	28.93\\
90	29\\
91	29\\
92	29\\
93	29\\
94	29.06\\
95	29.06\\
96	29.06\\
97	29.12\\
98	29.12\\
99	29.12\\
100	29.12\\
101	29.18\\
102	29.18\\
103	29.18\\
104	29.25\\
105	29.25\\
106	29.25\\
107	29.25\\
108	29.31\\
109	29.31\\
110	29.31\\
111	29.31\\
112	29.37\\
113	29.31\\
114	29.37\\
115	29.37\\
116	29.37\\
117	29.43\\
118	29.43\\
119	29.43\\
120	29.5\\
121	29.5\\
122	29.5\\
123	29.5\\
124	29.5\\
125	29.56\\
126	29.56\\
127	29.56\\
128	29.56\\
129	29.56\\
130	29.56\\
131	29.62\\
132	29.62\\
133	29.62\\
134	29.62\\
135	29.62\\
136	29.62\\
137	29.62\\
138	29.62\\
139	29.68\\
140	29.68\\
141	29.68\\
142	29.68\\
143	29.75\\
144	29.75\\
145	29.75\\
146	29.75\\
147	29.75\\
148	29.75\\
149	29.75\\
150	29.75\\
151	29.75\\
152	29.75\\
153	29.75\\
154	29.81\\
155	29.81\\
156	29.81\\
157	29.81\\
158	29.81\\
159	29.81\\
160	29.81\\
161	29.87\\
162	29.87\\
163	29.87\\
164	29.87\\
165	29.87\\
166	29.87\\
167	29.93\\
168	29.93\\
169	29.93\\
170	29.93\\
171	29.93\\
172	29.93\\
173	29.93\\
174	29.93\\
175	30\\
176	29.93\\
177	30\\
178	30\\
179	30\\
180	30\\
181	30\\
182	30\\
183	30\\
184	30\\
185	30\\
186	30.06\\
187	30.06\\
188	30.06\\
189	30.06\\
190	30.06\\
191	30.06\\
192	30.06\\
193	30.06\\
194	30.06\\
195	30.06\\
196	30.06\\
197	30.06\\
198	30.06\\
199	30.06\\
200	30.06\\
201	30.06\\
202	30.12\\
203	30.12\\
204	30.12\\
205	30.12\\
206	30.12\\
207	30.12\\
208	30.12\\
209	30.18\\
210	30.18\\
211	30.18\\
212	30.18\\
213	30.18\\
214	30.18\\
215	30.18\\
216	30.18\\
217	30.18\\
218	30.25\\
219	30.25\\
220	30.25\\
221	30.25\\
222	30.25\\
223	30.25\\
224	30.25\\
225	30.31\\
226	30.25\\
227	30.25\\
228	30.31\\
229	30.31\\
230	30.25\\
231	30.31\\
232	30.31\\
233	30.31\\
234	30.31\\
235	30.31\\
236	30.31\\
237	30.31\\
238	30.31\\
239	30.31\\
240	30.31\\
241	30.31\\
242	30.37\\
243	30.37\\
244	30.37\\
245	30.37\\
246	30.37\\
247	30.43\\
248	30.43\\
249	30.43\\
250	30.43\\
251	30.43\\
252	30.43\\
253	30.43\\
254	30.43\\
255	30.43\\
256	30.5\\
257	30.5\\
258	30.5\\
259	30.5\\
260	30.5\\
261	30.56\\
262	30.56\\
263	30.56\\
264	30.56\\
265	30.56\\
266	30.62\\
267	30.56\\
268	30.62\\
269	30.56\\
270	30.56\\
271	30.56\\
272	30.56\\
273	30.56\\
274	30.56\\
275	30.56\\
276	30.56\\
277	30.56\\
278	30.56\\
279	30.56\\
280	30.56\\
281	30.56\\
282	30.56\\
283	30.56\\
284	30.56\\
285	30.56\\
286	30.56\\
287	30.62\\
288	30.62\\
289	30.56\\
290	30.56\\
291	30.62\\
292	30.62\\
293	30.62\\
294	30.62\\
295	30.62\\
296	30.62\\
297	30.62\\
298	30.62\\
299	30.62\\
300	30.62\\
301	30.62\\
302	30.62\\
303	30.62\\
304	30.62\\
305	30.62\\
306	30.62\\
307	30.62\\
308	30.62\\
309	30.62\\
310	30.62\\
311	30.68\\
312	30.68\\
313	30.62\\
314	30.68\\
315	30.68\\
316	30.68\\
317	30.68\\
318	30.68\\
319	30.68\\
320	30.68\\
321	30.68\\
322	30.68\\
323	30.68\\
324	30.68\\
325	30.68\\
326	30.68\\
327	30.68\\
328	30.68\\
329	30.68\\
330	30.68\\
331	30.68\\
332	30.68\\
333	30.68\\
334	30.68\\
335	30.68\\
336	30.68\\
337	30.68\\
338	30.68\\
339	30.68\\
340	30.68\\
341	30.68\\
342	30.68\\
343	30.68\\
344	30.68\\
345	30.75\\
346	30.75\\
347	30.75\\
348	30.75\\
349	30.75\\
350	30.75\\
351	30.75\\
352	30.81\\
353	30.75\\
354	30.75\\
355	30.81\\
356	30.75\\
357	30.81\\
358	30.81\\
359	30.81\\
360	30.81\\
361	30.81\\
362	30.81\\
363	30.81\\
364	30.75\\
365	30.81\\
366	30.75\\
367	30.81\\
368	30.81\\
369	30.81\\
370	30.81\\
371	30.81\\
372	30.81\\
373	30.81\\
374	30.81\\
375	30.81\\
376	30.87\\
377	30.87\\
378	30.87\\
379	30.87\\
380	30.87\\
381	30.93\\
382	30.93\\
383	30.93\\
384	30.93\\
385	30.93\\
386	30.93\\
387	30.93\\
388	30.93\\
389	30.93\\
390	30.93\\
391	30.93\\
392	30.87\\
393	30.87\\
394	30.87\\
395	30.87\\
396	30.87\\
397	30.87\\
398	30.87\\
399	30.87\\
400	30.87\\
401	30.87\\
402	30.87\\
403	30.87\\
404	30.87\\
405	30.87\\
406	30.87\\
407	30.87\\
408	30.87\\
409	30.87\\
410	30.87\\
411	30.87\\
412	30.87\\
413	30.87\\
414	30.87\\
415	30.87\\
416	30.87\\
417	30.87\\
418	30.87\\
419	30.87\\
420	30.93\\
421	30.93\\
422	30.93\\
423	30.93\\
424	30.93\\
425	30.93\\
426	30.87\\
427	30.93\\
428	30.93\\
429	30.87\\
430	30.93\\
431	30.93\\
432	30.93\\
433	30.93\\
434	30.93\\
435	30.93\\
436	30.93\\
437	30.93\\
438	30.93\\
439	30.93\\
440	30.93\\
441	30.93\\
442	30.93\\
443	30.93\\
444	31\\
445	30.93\\
446	30.93\\
447	30.93\\
448	30.93\\
449	30.93\\
450	30.93\\
};
\end{axis}
\end{tikzpicture}%
\caption{Pomiar temperatury w punkcie pracy}
\end{figure}

\section{Wyznaczenie odpowiedzi skokowych}

Rozpoczynając z punktu pracy wyznaczono odpowiedzi skokowe dla trzech różnych skoków sygnału zakłócenia  $Z = 10\%$  $Z = 20\%$ i $Z = 30\%$.

\begin{figure}[H]
\centering
% This file was created by matlab2tikz.
%
%The latest updates can be retrieved from
%  http://www.mathworks.com/matlabcentral/fileexchange/22022-matlab2tikz-matlab2tikz
%where you can also make suggestions and rate matlab2tikz.
%
\definecolor{mycolor1}{rgb}{0.00000,0.44700,0.74100}%
\definecolor{mycolor2}{rgb}{0.85000,0.32500,0.09800}%
\definecolor{mycolor3}{rgb}{0.92900,0.69400,0.12500}%
%
\begin{tikzpicture}

\begin{axis}[%
width=4.521in,
height=3.566in,
at={(0.758in,0.481in)},
scale only axis,
xmin=0,
xmax=350,
xlabel style={font=\color{white!15!black}},
xlabel={k},
ymin=28,
ymax=34,
ylabel style={font=\color{white!15!black}},
ylabel={$\text{T[}^\circ\text{C]}$},
axis background/.style={fill=white}
]
\addplot[const plot, color=mycolor1, forget plot] table[row sep=crcr] {%
1	28.18\\
2	28.18\\
3	28.18\\
4	28.18\\
5	28.12\\
6	28.12\\
7	28.12\\
8	28.12\\
9	28.12\\
10	28.12\\
11	28.12\\
12	28.12\\
13	28.06\\
14	28.12\\
15	28.12\\
16	28.12\\
17	28.12\\
18	28.18\\
19	28.18\\
20	28.18\\
21	28.18\\
22	28.25\\
23	28.25\\
24	28.25\\
25	28.25\\
26	28.31\\
27	28.31\\
28	28.31\\
29	28.37\\
30	28.37\\
31	28.43\\
32	28.43\\
33	28.43\\
34	28.5\\
35	28.56\\
36	28.56\\
37	28.62\\
38	28.62\\
39	28.68\\
40	28.68\\
41	28.68\\
42	28.75\\
43	28.75\\
44	28.81\\
45	28.87\\
46	28.87\\
47	28.87\\
48	28.93\\
49	28.93\\
50	29\\
51	29.06\\
52	29.06\\
53	29.12\\
54	29.18\\
55	29.18\\
56	29.25\\
57	29.25\\
58	29.31\\
59	29.31\\
60	29.37\\
61	29.37\\
62	29.37\\
63	29.37\\
64	29.43\\
65	29.43\\
66	29.5\\
67	29.5\\
68	29.5\\
69	29.56\\
70	29.56\\
71	29.56\\
72	29.62\\
73	29.62\\
74	29.62\\
75	29.62\\
76	29.62\\
77	29.62\\
78	29.62\\
79	29.62\\
80	29.62\\
81	29.62\\
82	29.62\\
83	29.62\\
84	29.62\\
85	29.62\\
86	29.62\\
87	29.62\\
88	29.62\\
89	29.68\\
90	29.68\\
91	29.68\\
92	29.68\\
93	29.75\\
94	29.75\\
95	29.75\\
96	29.75\\
97	29.81\\
98	29.81\\
99	29.81\\
100	29.81\\
101	29.81\\
102	29.81\\
103	29.87\\
104	29.87\\
105	29.87\\
106	29.87\\
107	29.87\\
108	29.87\\
109	29.87\\
110	29.93\\
111	29.93\\
112	29.87\\
113	29.87\\
114	29.87\\
115	29.87\\
116	29.93\\
117	29.87\\
118	29.93\\
119	29.93\\
120	29.93\\
121	29.93\\
122	29.93\\
123	29.93\\
124	29.93\\
125	29.87\\
126	29.87\\
127	29.93\\
128	29.93\\
129	29.93\\
130	29.93\\
131	29.93\\
132	29.93\\
133	30\\
134	30\\
135	30\\
136	30\\
137	30\\
138	30\\
139	30\\
140	30\\
141	30\\
142	30.06\\
143	30.06\\
144	30.06\\
145	30.06\\
146	30.06\\
147	30.12\\
148	30.12\\
149	30.12\\
150	30.12\\
151	30.12\\
152	30.12\\
153	30.12\\
154	30.12\\
155	30.12\\
156	30.18\\
157	30.18\\
158	30.18\\
159	30.18\\
160	30.18\\
161	30.18\\
162	30.25\\
163	30.25\\
164	30.25\\
165	30.25\\
166	30.25\\
167	30.25\\
168	30.31\\
169	30.31\\
170	30.31\\
171	30.31\\
172	30.31\\
173	30.31\\
174	30.37\\
175	30.37\\
176	30.37\\
177	30.37\\
178	30.37\\
179	30.37\\
180	30.43\\
181	30.43\\
182	30.43\\
183	30.43\\
184	30.43\\
185	30.43\\
186	30.43\\
187	30.37\\
188	30.43\\
189	30.37\\
190	30.37\\
191	30.37\\
192	30.37\\
193	30.37\\
194	30.37\\
195	30.37\\
196	30.37\\
197	30.37\\
198	30.37\\
199	30.37\\
200	30.37\\
201	30.37\\
202	30.37\\
203	30.37\\
204	30.37\\
205	30.37\\
206	30.37\\
207	30.37\\
208	30.37\\
209	30.37\\
210	30.37\\
211	30.37\\
212	30.37\\
213	30.37\\
214	30.37\\
215	30.37\\
216	30.37\\
217	30.37\\
218	30.37\\
219	30.37\\
220	30.43\\
221	30.43\\
222	30.43\\
223	30.43\\
224	30.43\\
225	30.43\\
226	30.43\\
227	30.43\\
228	30.43\\
229	30.43\\
230	30.43\\
231	30.43\\
232	30.43\\
233	30.37\\
234	30.37\\
235	30.37\\
236	30.37\\
237	30.43\\
238	30.43\\
239	30.43\\
240	30.43\\
241	30.43\\
242	30.43\\
243	30.43\\
244	30.43\\
245	30.43\\
246	30.43\\
247	30.43\\
248	30.5\\
249	30.5\\
250	30.5\\
251	30.5\\
252	30.5\\
253	30.5\\
254	30.5\\
255	30.5\\
256	30.5\\
257	30.5\\
258	30.5\\
259	30.5\\
260	30.5\\
261	30.5\\
262	30.5\\
263	30.5\\
264	30.5\\
265	30.56\\
266	30.5\\
267	30.5\\
268	30.5\\
269	30.5\\
270	30.5\\
271	30.5\\
272	30.5\\
273	30.5\\
274	30.5\\
275	30.5\\
276	30.5\\
277	30.5\\
278	30.5\\
279	30.5\\
280	30.5\\
281	30.5\\
282	30.5\\
283	30.5\\
284	30.5\\
285	30.5\\
286	30.5\\
287	30.56\\
288	30.56\\
289	30.56\\
290	30.56\\
291	30.56\\
292	30.56\\
293	30.56\\
294	30.56\\
295	30.56\\
296	30.56\\
297	30.56\\
298	30.56\\
299	30.56\\
300	30.56\\
301	30.5\\
302	30.5\\
303	30.5\\
304	30.5\\
305	30.5\\
306	30.5\\
307	30.43\\
308	30.43\\
309	30.43\\
310	30.43\\
311	30.43\\
312	30.43\\
313	30.43\\
314	30.43\\
315	30.37\\
316	30.37\\
317	30.37\\
318	30.37\\
319	30.37\\
320	30.37\\
321	30.37\\
322	30.37\\
323	30.37\\
324	30.37\\
325	30.37\\
326	30.43\\
327	30.43\\
328	30.43\\
329	30.5\\
330	30.43\\
331	30.5\\
332	30.5\\
333	30.5\\
334	30.56\\
335	30.56\\
336	30.56\\
337	30.56\\
338	30.56\\
339	30.56\\
340	30.56\\
341	30.56\\
342	30.62\\
343	30.62\\
344	30.62\\
345	30.62\\
346	30.62\\
347	30.62\\
348	30.62\\
349	30.62\\
350	30.62\\
};
\addplot[const plot, color=mycolor2, forget plot] table[row sep=crcr] {%
1	28.12\\
2	28.18\\
3	28.18\\
4	28.18\\
5	28.18\\
6	28.18\\
7	28.18\\
8	28.18\\
9	28.18\\
10	28.18\\
11	28.18\\
12	28.18\\
13	28.18\\
14	28.18\\
15	28.18\\
16	28.18\\
17	28.18\\
18	28.18\\
19	28.18\\
20	28.25\\
21	28.25\\
22	28.25\\
23	28.31\\
24	28.31\\
25	28.31\\
26	28.37\\
27	28.37\\
28	28.43\\
29	28.5\\
30	28.5\\
31	28.56\\
32	28.56\\
33	28.62\\
34	28.62\\
35	28.68\\
36	28.75\\
37	28.81\\
38	28.81\\
39	28.87\\
40	28.93\\
41	28.93\\
42	29\\
43	29.06\\
44	29.06\\
45	29.06\\
46	29.12\\
47	29.18\\
48	29.18\\
49	29.25\\
50	29.31\\
51	29.31\\
52	29.31\\
53	29.37\\
54	29.43\\
55	29.43\\
56	29.5\\
57	29.5\\
58	29.56\\
59	29.62\\
60	29.62\\
61	29.68\\
62	29.68\\
63	29.75\\
64	29.75\\
65	29.81\\
66	29.81\\
67	29.81\\
68	29.87\\
69	29.87\\
70	29.93\\
71	29.93\\
72	30\\
73	30\\
74	30.06\\
75	30.06\\
76	30.12\\
77	30.12\\
78	30.18\\
79	30.18\\
80	30.18\\
81	30.25\\
82	30.31\\
83	30.31\\
84	30.31\\
85	30.37\\
86	30.37\\
87	30.37\\
88	30.37\\
89	30.43\\
90	30.43\\
91	30.43\\
92	30.43\\
93	30.43\\
94	30.5\\
95	30.5\\
96	30.5\\
97	30.5\\
98	30.5\\
99	30.56\\
100	30.62\\
101	30.62\\
102	30.62\\
103	30.62\\
104	30.68\\
105	30.68\\
106	30.75\\
107	30.75\\
108	30.81\\
109	30.81\\
110	30.87\\
111	30.87\\
112	30.87\\
113	30.93\\
114	30.93\\
115	31\\
116	31\\
117	31.06\\
118	31\\
119	31.06\\
120	31.06\\
121	31\\
122	31\\
123	31\\
124	31.06\\
125	31\\
126	31\\
127	31.06\\
128	31.06\\
129	31.06\\
130	31.06\\
131	31.06\\
132	31.12\\
133	31.12\\
134	31.12\\
135	31.12\\
136	31.18\\
137	31.18\\
138	31.18\\
139	31.18\\
140	31.18\\
141	31.18\\
142	31.25\\
143	31.25\\
144	31.25\\
145	31.25\\
146	31.25\\
147	31.31\\
148	31.31\\
149	31.31\\
150	31.31\\
151	31.31\\
152	31.37\\
153	31.37\\
154	31.37\\
155	31.43\\
156	31.43\\
157	31.5\\
158	31.5\\
159	31.5\\
160	31.5\\
161	31.5\\
162	31.5\\
163	31.5\\
164	31.5\\
165	31.56\\
166	31.56\\
167	31.56\\
168	31.56\\
169	31.56\\
170	31.56\\
171	31.56\\
172	31.56\\
173	31.62\\
174	31.62\\
175	31.62\\
176	31.62\\
177	31.62\\
178	31.68\\
179	31.68\\
180	31.68\\
181	31.62\\
182	31.68\\
183	31.68\\
184	31.68\\
185	31.68\\
186	31.68\\
187	31.68\\
188	31.68\\
189	31.68\\
190	31.68\\
191	31.62\\
192	31.62\\
193	31.68\\
194	31.68\\
195	31.68\\
196	31.68\\
197	31.68\\
198	31.68\\
199	31.68\\
200	31.68\\
201	31.68\\
202	31.68\\
203	31.62\\
204	31.68\\
205	31.68\\
206	31.68\\
207	31.68\\
208	31.68\\
209	31.68\\
210	31.75\\
211	31.75\\
212	31.75\\
213	31.75\\
214	31.75\\
215	31.75\\
216	31.81\\
217	31.81\\
218	31.81\\
219	31.81\\
220	31.81\\
221	31.81\\
222	31.87\\
223	31.87\\
224	31.87\\
225	31.87\\
226	31.87\\
227	31.93\\
228	31.93\\
229	31.93\\
230	31.93\\
231	32\\
232	32\\
233	32\\
234	32\\
235	32\\
236	32\\
237	32\\
238	32\\
239	32\\
240	32\\
241	32\\
242	32\\
243	32\\
244	32.06\\
245	32.06\\
246	32.06\\
247	32.06\\
248	32.06\\
249	32.06\\
250	32.06\\
251	32.06\\
252	32\\
253	32\\
254	32\\
255	32\\
256	32\\
257	32\\
258	32\\
259	32\\
260	32\\
261	32\\
262	32\\
263	32\\
264	32\\
265	31.93\\
266	32\\
267	32\\
268	32\\
269	32\\
270	32\\
271	32\\
272	32\\
273	31.93\\
274	31.93\\
275	32\\
276	31.93\\
277	31.93\\
278	31.93\\
279	31.93\\
280	31.93\\
281	31.93\\
282	31.93\\
283	31.93\\
284	32\\
285	32\\
286	32\\
287	32\\
288	32\\
289	32\\
290	32\\
291	32\\
292	32\\
293	32\\
294	32\\
295	32\\
296	32\\
297	32\\
298	32\\
299	32\\
300	32\\
301	32.06\\
302	32\\
303	32\\
304	32\\
305	32.06\\
306	32.06\\
307	32.06\\
308	32.06\\
309	32.12\\
310	32.12\\
311	32.12\\
312	32.12\\
313	32.12\\
314	32.18\\
315	32.18\\
316	32.18\\
317	32.18\\
318	32.18\\
319	32.25\\
320	32.18\\
321	32.25\\
322	32.25\\
323	32.25\\
324	32.25\\
325	32.25\\
326	32.25\\
327	32.25\\
328	32.25\\
329	32.25\\
330	32.25\\
331	32.18\\
332	32.18\\
333	32.25\\
334	32.25\\
335	32.25\\
336	32.25\\
337	32.25\\
338	32.25\\
339	32.25\\
340	32.31\\
341	32.31\\
342	32.31\\
343	32.31\\
344	32.31\\
345	32.25\\
346	32.25\\
347	32.31\\
348	32.31\\
349	32.31\\
350	32.31\\
};
\addplot[const plot, color=mycolor3, forget plot] table[row sep=crcr] {%
1	28.18\\
2	28.18\\
3	28.18\\
4	28.25\\
5	28.25\\
6	28.25\\
7	28.25\\
8	28.25\\
9	28.31\\
10	28.25\\
11	28.25\\
12	28.25\\
13	28.25\\
14	28.25\\
15	28.25\\
16	28.25\\
17	28.25\\
18	28.31\\
19	28.31\\
20	28.31\\
21	28.37\\
22	28.37\\
23	28.43\\
24	28.43\\
25	28.5\\
26	28.5\\
27	28.56\\
28	28.62\\
29	28.62\\
30	28.68\\
31	28.75\\
32	28.81\\
33	28.87\\
34	28.87\\
35	28.93\\
36	29\\
37	29.06\\
38	29.12\\
39	29.25\\
40	29.31\\
41	29.37\\
42	29.43\\
43	29.5\\
44	29.56\\
45	29.62\\
46	29.68\\
47	29.75\\
48	29.81\\
49	29.87\\
50	29.93\\
51	30\\
52	30.06\\
53	30.12\\
54	30.18\\
55	30.25\\
56	30.31\\
57	30.31\\
58	30.37\\
59	30.43\\
60	30.5\\
61	30.5\\
62	30.5\\
63	30.56\\
64	30.62\\
65	30.68\\
66	30.75\\
67	30.75\\
68	30.81\\
69	30.87\\
70	30.87\\
71	30.93\\
72	31\\
73	31\\
74	31.06\\
75	31.12\\
76	31.12\\
77	31.18\\
78	31.18\\
79	31.25\\
80	31.25\\
81	31.31\\
82	31.37\\
83	31.37\\
84	31.43\\
85	31.5\\
86	31.5\\
87	31.56\\
88	31.62\\
89	31.62\\
90	31.68\\
91	31.68\\
92	31.68\\
93	31.75\\
94	31.75\\
95	31.81\\
96	31.81\\
97	31.81\\
98	31.87\\
99	31.93\\
100	31.93\\
101	31.93\\
102	32\\
103	32.06\\
104	32.06\\
105	32.12\\
106	32.12\\
107	32.18\\
108	32.18\\
109	32.25\\
110	32.25\\
111	32.25\\
112	32.25\\
113	32.31\\
114	32.31\\
115	32.37\\
116	32.37\\
117	32.37\\
118	32.37\\
119	32.43\\
120	32.43\\
121	32.43\\
122	32.43\\
123	32.43\\
124	32.43\\
125	32.43\\
126	32.43\\
127	32.5\\
128	32.5\\
129	32.5\\
130	32.56\\
131	32.56\\
132	32.56\\
133	32.62\\
134	32.62\\
135	32.62\\
136	32.62\\
137	32.62\\
138	32.62\\
139	32.62\\
140	32.62\\
141	32.62\\
142	32.62\\
143	32.68\\
144	32.68\\
145	32.68\\
146	32.68\\
147	32.68\\
148	32.68\\
149	32.75\\
150	32.68\\
151	32.75\\
152	32.75\\
153	32.75\\
154	32.75\\
155	32.75\\
156	32.75\\
157	32.81\\
158	32.81\\
159	32.81\\
160	32.87\\
161	32.87\\
162	32.87\\
163	32.87\\
164	32.93\\
165	32.93\\
166	32.93\\
167	32.93\\
168	32.93\\
169	33\\
170	33\\
171	33.06\\
172	33.06\\
173	33.06\\
174	33.06\\
175	33.06\\
176	33.12\\
177	33.12\\
178	33.18\\
179	33.18\\
180	33.25\\
181	33.25\\
182	33.25\\
183	33.25\\
184	33.31\\
185	33.31\\
186	33.31\\
187	33.31\\
188	33.37\\
189	33.37\\
190	33.37\\
191	33.43\\
192	33.43\\
193	33.43\\
194	33.43\\
195	33.43\\
196	33.5\\
197	33.5\\
198	33.5\\
199	33.5\\
200	33.5\\
201	33.5\\
202	33.5\\
203	33.5\\
204	33.5\\
205	33.5\\
206	33.5\\
207	33.5\\
208	33.5\\
209	33.5\\
210	33.5\\
211	33.5\\
212	33.5\\
213	33.5\\
214	33.5\\
215	33.5\\
216	33.5\\
217	33.5\\
218	33.56\\
219	33.56\\
220	33.56\\
221	33.56\\
222	33.56\\
223	33.56\\
224	33.62\\
225	33.62\\
226	33.62\\
227	33.62\\
228	33.62\\
229	33.62\\
230	33.62\\
231	33.62\\
232	33.62\\
233	33.62\\
234	33.68\\
235	33.68\\
236	33.62\\
237	33.68\\
238	33.68\\
239	33.62\\
240	33.62\\
241	33.62\\
242	33.62\\
243	33.62\\
244	33.56\\
245	33.62\\
246	33.56\\
247	33.56\\
248	33.56\\
249	33.62\\
250	33.56\\
251	33.62\\
252	33.62\\
253	33.68\\
254	33.68\\
255	33.75\\
256	33.81\\
257	33.81\\
258	33.81\\
259	33.81\\
260	33.87\\
261	33.87\\
262	33.87\\
263	33.87\\
264	33.87\\
265	33.87\\
266	33.87\\
267	33.87\\
268	33.87\\
269	33.87\\
270	33.93\\
271	33.93\\
272	33.87\\
273	33.93\\
274	33.93\\
275	33.93\\
276	33.93\\
277	33.93\\
278	33.93\\
279	33.93\\
280	33.93\\
281	33.93\\
282	33.93\\
283	33.93\\
284	33.93\\
285	33.93\\
286	33.93\\
287	33.93\\
288	33.93\\
289	33.87\\
290	33.87\\
291	33.93\\
292	33.87\\
293	33.87\\
294	33.87\\
295	33.87\\
296	33.87\\
297	33.87\\
298	33.87\\
299	33.87\\
300	33.87\\
301	33.87\\
302	33.87\\
303	33.87\\
304	33.87\\
305	33.87\\
306	33.87\\
307	33.87\\
308	33.87\\
309	33.87\\
310	33.87\\
311	33.87\\
312	33.87\\
313	33.93\\
314	33.93\\
315	33.93\\
316	33.93\\
317	33.93\\
318	33.93\\
319	33.93\\
320	34\\
321	33.93\\
322	34\\
323	34\\
324	34\\
325	34\\
326	34\\
327	34\\
328	34\\
329	34\\
330	34\\
331	34\\
332	34\\
333	34\\
334	33.93\\
335	33.93\\
336	33.93\\
337	34\\
338	34\\
339	34\\
340	34\\
341	34\\
342	33.93\\
343	33.93\\
344	33.93\\
345	33.93\\
346	33.93\\
347	33.93\\
348	33.87\\
349	33.93\\
350	33.87\\
};
\end{axis}
\end{tikzpicture}%
\caption{Odpowiedzi skokowe dla trzech różnych wartości sygnału sterującego}
\end{figure}

Analizując otrzymane wykresy można wywnioskować, że właściwości statyczne procesu są w przybliżeniu liniowe, zmiany wartości odpowiedzi skokowej dla tych samych chwil są w przybliżeniu proporcjonalne jak również sam kształt wykresów jest w przybliżeniu podobny. W celu sprawdzenia założeń narysowano charakterystykę statyczną procesu.

\begin{figure}[H]
\centering
% This file was created by matlab2tikz.
%
%The latest updates can be retrieved from
%  http://www.mathworks.com/matlabcentral/fileexchange/22022-matlab2tikz-matlab2tikz
%where you can also make suggestions and rate matlab2tikz.
%
\definecolor{mycolor1}{rgb}{0.00000,0.44700,0.74100}%
%
\begin{tikzpicture}

\begin{axis}[%
width=4.521in,
height=3.566in,
at={(0.758in,0.481in)},
scale only axis,
xmin=0,
xmax=30,
xlabel style={font=\color{white!15!black}},
xlabel={u},
ymin=28,
ymax=34,
ylabel style={font=\color{white!15!black}},
ylabel={y},
axis background/.style={fill=white}
]
\addplot [color=mycolor1, forget plot]
  table[row sep=crcr]{%
0	28.2\\
10	30.62\\
20	32.31\\
30	33.87\\
};
\end{axis}
\end{tikzpicture}%
\caption{Charakterystyka statyczna procesu}
\end{figure}

Która potwierdziła przypuszczenia, na jej podstawie można stwierdzić, że właściwości statyczne procesu są w dobrym przybliżeniu liniowe i w konsekwencji postanowiono wyznaczyć wzmocnienie statyczne procesu.

\begin{equation}
K_{stat} = 0,1890
\end{equation}

\section{Przekształcenie i aproksymacja odpowiedzi skokowej}

W celu przekształcenia odpowiedzi skokowej w taki sposób aby można ją było wykorzystać w algorytmie DMC skorzystano z poniższego wzoru: 

\begin{equation}
S_i=\frac{Y(i)-Y_{pp}}{\triangle U} \textrm{ ,dla } i=1,2 \ldots D
\label{step_norm}
\end{equation}

Dla odpowiedzi skokowej na torze wejście-wyjście otrzymanej w wyniku zmiany sygnału sterującego z $G1 = 25\%$ na $G1 = 35\%$ a następnie dokonano jej aproksymacji używając członu inercyjnego drugiego rzędu z opóźnieniem.

\begin{figure}[H]
\centering
% This file was created by matlab2tikz.
%
%The latest updates can be retrieved from
%  http://www.mathworks.com/matlabcentral/fileexchange/22022-matlab2tikz-matlab2tikz
%where you can also make suggestions and rate matlab2tikz.
%
\definecolor{mycolor1}{rgb}{0.00000,0.44700,0.74100}%
\definecolor{mycolor2}{rgb}{0.85000,0.32500,0.09800}%
%
\begin{tikzpicture}

\begin{axis}[%
width=4.521in,
height=3.566in,
at={(0.758in,0.481in)},
scale only axis,
xmin=0,
xmax=350,
xlabel style={font=\color{white!15!black}},
xlabel={k},
ymin=-0.05,
ymax=0.4,
ylabel style={font=\color{white!15!black}},
ylabel={s},
axis background/.style={fill=white},
legend style={at={(0.03,0.97)}, anchor=north west, legend cell align=left, align=left, draw=white!15!black}
]
\addplot[const plot, color=mycolor1] table[row sep=crcr] {%
1	-0.00599999999999987\\
2	-0.00599999999999987\\
3	-0.00599999999999987\\
4	-0.00599999999999987\\
5	0\\
6	0\\
7	0.00700000000000003\\
8	0.00700000000000003\\
9	0.0129999999999999\\
10	0.0129999999999999\\
11	0.0129999999999999\\
12	0.0129999999999999\\
13	0.0129999999999999\\
14	0.0190000000000001\\
15	0.0190000000000001\\
16	0.025\\
17	0.025\\
18	0.025\\
19	0.032\\
20	0.032\\
21	0.032\\
22	0.0379999999999999\\
23	0.0379999999999999\\
24	0.0379999999999999\\
25	0.0440000000000001\\
26	0.05\\
27	0.05\\
28	0.057\\
29	0.057\\
30	0.0629999999999999\\
31	0.0629999999999999\\
32	0.0690000000000001\\
33	0.0690000000000001\\
34	0.075\\
35	0.075\\
36	0.082\\
37	0.0879999999999999\\
38	0.0879999999999999\\
39	0.0940000000000001\\
40	0.0940000000000001\\
41	0.1\\
42	0.107\\
43	0.113\\
44	0.119\\
45	0.119\\
46	0.125\\
47	0.132\\
48	0.132\\
49	0.138\\
50	0.138\\
51	0.144\\
52	0.15\\
53	0.15\\
54	0.157\\
55	0.163\\
56	0.169\\
57	0.169\\
58	0.169\\
59	0.175\\
60	0.175\\
61	0.175\\
62	0.182\\
63	0.182\\
64	0.188\\
65	0.188\\
66	0.188\\
67	0.194\\
68	0.194\\
69	0.2\\
70	0.2\\
71	0.2\\
72	0.2\\
73	0.2\\
74	0.207\\
75	0.207\\
76	0.207\\
77	0.207\\
78	0.213\\
79	0.213\\
80	0.219\\
81	0.219\\
82	0.225\\
83	0.225\\
84	0.225\\
85	0.225\\
86	0.232\\
87	0.232\\
88	0.232\\
89	0.232\\
90	0.232\\
91	0.232\\
92	0.238\\
93	0.238\\
94	0.238\\
95	0.244\\
96	0.244\\
97	0.25\\
98	0.257\\
99	0.257\\
100	0.257\\
101	0.263\\
102	0.269\\
103	0.269\\
104	0.269\\
105	0.275\\
106	0.275\\
107	0.282\\
108	0.282\\
109	0.282\\
110	0.282\\
111	0.282\\
112	0.282\\
113	0.282\\
114	0.282\\
115	0.282\\
116	0.282\\
117	0.282\\
118	0.282\\
119	0.288\\
120	0.288\\
121	0.288\\
122	0.288\\
123	0.288\\
124	0.288\\
125	0.288\\
126	0.288\\
127	0.288\\
128	0.288\\
129	0.288\\
130	0.288\\
131	0.282\\
132	0.282\\
133	0.282\\
134	0.282\\
135	0.282\\
136	0.282\\
137	0.282\\
138	0.282\\
139	0.282\\
140	0.282\\
141	0.282\\
142	0.282\\
143	0.282\\
144	0.282\\
145	0.288\\
146	0.288\\
147	0.288\\
148	0.288\\
149	0.288\\
150	0.294\\
151	0.288\\
152	0.294\\
153	0.294\\
154	0.3\\
155	0.294\\
156	0.3\\
157	0.3\\
158	0.3\\
159	0.3\\
160	0.3\\
161	0.307\\
162	0.3\\
163	0.307\\
164	0.307\\
165	0.313\\
166	0.313\\
167	0.313\\
168	0.319\\
169	0.319\\
170	0.319\\
171	0.325\\
172	0.325\\
173	0.325\\
174	0.332\\
175	0.332\\
176	0.332\\
177	0.332\\
178	0.338\\
179	0.338\\
180	0.338\\
181	0.338\\
182	0.338\\
183	0.338\\
184	0.338\\
185	0.338\\
186	0.338\\
187	0.338\\
188	0.344\\
189	0.344\\
190	0.344\\
191	0.344\\
192	0.338\\
193	0.338\\
194	0.338\\
195	0.344\\
196	0.344\\
197	0.344\\
198	0.344\\
199	0.344\\
200	0.344\\
201	0.35\\
202	0.35\\
203	0.35\\
204	0.35\\
205	0.35\\
206	0.357\\
207	0.35\\
208	0.357\\
209	0.357\\
210	0.357\\
211	0.357\\
212	0.357\\
213	0.357\\
214	0.357\\
215	0.35\\
216	0.35\\
217	0.35\\
218	0.35\\
219	0.35\\
220	0.35\\
221	0.357\\
222	0.357\\
223	0.357\\
224	0.357\\
225	0.357\\
226	0.357\\
227	0.363\\
228	0.363\\
229	0.363\\
230	0.363\\
231	0.363\\
232	0.363\\
233	0.363\\
234	0.363\\
235	0.363\\
236	0.363\\
237	0.363\\
238	0.363\\
239	0.363\\
240	0.363\\
241	0.363\\
242	0.363\\
243	0.363\\
244	0.363\\
245	0.357\\
246	0.357\\
247	0.363\\
248	0.363\\
249	0.363\\
250	0.363\\
251	0.363\\
252	0.357\\
253	0.357\\
254	0.357\\
255	0.357\\
256	0.357\\
257	0.357\\
258	0.357\\
259	0.357\\
260	0.35\\
261	0.357\\
262	0.357\\
263	0.357\\
264	0.357\\
265	0.357\\
266	0.357\\
267	0.357\\
268	0.363\\
269	0.363\\
270	0.363\\
271	0.363\\
272	0.369\\
273	0.369\\
274	0.369\\
275	0.369\\
276	0.369\\
277	0.375\\
278	0.369\\
279	0.369\\
280	0.369\\
281	0.369\\
282	0.369\\
283	0.369\\
284	0.369\\
285	0.369\\
286	0.375\\
287	0.375\\
288	0.375\\
289	0.382\\
290	0.382\\
291	0.382\\
292	0.382\\
293	0.382\\
294	0.382\\
295	0.382\\
296	0.382\\
297	0.382\\
298	0.382\\
299	0.382\\
300	0.382\\
301	0.382\\
302	0.382\\
303	0.382\\
304	0.382\\
305	0.382\\
306	0.382\\
307	0.382\\
308	0.388\\
309	0.388\\
310	0.388\\
311	0.382\\
312	0.388\\
313	0.388\\
314	0.388\\
315	0.388\\
316	0.388\\
317	0.388\\
318	0.388\\
319	0.388\\
320	0.388\\
321	0.388\\
322	0.388\\
323	0.388\\
324	0.382\\
325	0.388\\
326	0.388\\
327	0.382\\
328	0.388\\
329	0.388\\
330	0.388\\
331	0.388\\
332	0.388\\
333	0.388\\
334	0.388\\
335	0.388\\
336	0.382\\
337	0.388\\
338	0.382\\
339	0.382\\
340	0.382\\
341	0.382\\
342	0.382\\
343	0.382\\
344	0.382\\
345	0.382\\
346	0.382\\
347	0.382\\
348	0.382\\
349	0.382\\
350	0.382\\
};
\addlegendentry{odpowiedź układu}

\addplot[const plot, color=mycolor2] table[row sep=crcr] {%
1	0\\
2	0\\
3	0\\
4	0\\
5	0\\
6	0\\
7	0\\
8	0\\
9	0\\
10	0\\
11	0\\
12	0\\
13	0.00237557398565751\\
14	0.00586672969344894\\
15	0.00986718587969116\\
16	0.0140853817862893\\
17	0.0183812726549001\\
18	0.0226877561577816\\
19	0.0269728437687246\\
20	0.0312214484891394\\
21	0.0354266167254346\\
22	0.0395853069839662\\
23	0.0436963575977874\\
24	0.0477595083463587\\
25	0.051774929463866\\
26	0.0557429949255997\\
27	0.0596641733397384\\
28	0.0635389754589091\\
29	0.0673679289504457\\
30	0.0711515662896691\\
31	0.0748904189706795\\
32	0.0785850147582008\\
33	0.0822358764030559\\
34	0.08584352106184\\
35	0.089408460055174\\
36	0.0929311987885189\\
37	0.0964122367508143\\
38	0.0998520675501487\\
39	0.103251178966829\\
40	0.106610053014402\\
41	0.109929166004084\\
42	0.113208988610416\\
43	0.116449985937099\\
44	0.119652617582515\\
45	0.122817337704685\\
46	0.125944595085576\\
47	0.129034833194691\\
48	0.13208849025193\\
49	0.135105999289726\\
50	0.138087788214448\\
51	0.141034279867071\\
52	0.143945892083146\\
53	0.146823037752047\\
54	0.14966612487552\\
55	0.152475556625547\\
56	0.155251731401513\\
57	0.157995042886702\\
58	0.160705880104131\\
59	0.163384627471709\\
60	0.166031664856754\\
61	0.16864736762986\\
62	0.171232106718129\\
63	0.173786248657771\\
64	0.176310155646079\\
65	0.178804185592796\\
66	0.181268692170864\\
67	0.183704024866584\\
68	0.186110529029171\\
69	0.188488545919726\\
70	0.190838412759636\\
71	0.19316046277839\\
72	0.195455025260831\\
73	0.197722425593859\\
74	0.199962985312567\\
75	0.202177022145837\\
76	0.204364850061402\\
77	0.206526779310363\\
78	0.20866311647119\\
79	0.210774164493191\\
80	0.212860222739481\\
81	0.214921587029427\\
82	0.216958549680599\\
83	0.218971399550225\\
84	0.220960422076154\\
85	0.222925899317328\\
86	0.224868109993786\\
87	0.226787329526184\\
88	0.228683830074856\\
89	0.230557880578407\\
90	0.232409746791848\\
91	0.234239691324284\\
92	0.236047973676157\\
93	0.237834850276041\\
94	0.239600574517006\\
95	0.241345396792554\\
96	0.243069564532123\\
97	0.244773322236178\\
98	0.24645691151088\\
99	0.248120571102351\\
100	0.249764536930525\\
101	0.25138904212261\\
102	0.252994317046142\\
103	0.254580589341653\\
104	0.256148083954956\\
105	0.257697023169041\\
106	0.259227626635599\\
107	0.260740111406165\\
108	0.262234691962905\\
109	0.263711580249026\\
110	0.265170985698831\\
111	0.266613115267424\\
112	0.26803817346005\\
113	0.269446362361102\\
114	0.270837881662775\\
115	0.272212928693384\\
116	0.273571698445347\\
117	0.274914383602839\\
118	0.276241174569109\\
119	0.277552259493491\\
120	0.278847824298075\\
121	0.280128052704079\\
122	0.2813931262579\\
123	0.282643224356857\\
124	0.283878524274633\\
125	0.285099201186414\\
126	0.286305428193729\\
127	0.287497376348996\\
128	0.288675214679784\\
129	0.289839110212773\\
130	0.290989227997451\\
131	0.29212573112951\\
132	0.29324878077398\\
133	0.294358536188078\\
134	0.295455154743798\\
135	0.296538791950226\\
136	0.297609601475587\\
137	0.298667735169044\\
138	0.299713343082226\\
139	0.30074657349051\\
140	0.301767572914045\\
141	0.30277648613853\\
142	0.30377345623575\\
143	0.304758624583856\\
144	0.305732130887423\\
145	0.306694113197255\\
146	0.307644707929964\\
147	0.308584049887313\\
148	0.309512272275337\\
149	0.310429506723227\\
150	0.311335883301996\\
151	0.312231530542931\\
152	0.313116575455811\\
153	0.313991143546925\\
154	0.314855358836864\\
155	0.315709343878112\\
156	0.316553219772426\\
157	0.317387106188003\\
158	0.318211121376456\\
159	0.31902538218958\\
160	0.319830004095926\\
161	0.320625101197168\\
162	0.321410786244293\\
163	0.322187170653581\\
164	0.322954364522411\\
165	0.323712476644872\\
166	0.324461614527189\\
167	0.32520188440297\\
168	0.325933391248271\\
169	0.326656238796482\\
170	0.327370529553034\\
171	0.328076364809942\\
172	0.328773844660164\\
173	0.329463068011793\\
174	0.330144132602091\\
175	0.330817135011339\\
176	0.331482170676541\\
177	0.332139333904953\\
178	0.332788717887458\\
179	0.333430414711782\\
180	0.33406451537555\\
181	0.334691109799195\\
182	0.335310286838704\\
183	0.335922134298221\\
184	0.336526738942499\\
185	0.337124186509204\\
186	0.337714561721071\\
187	0.338297948297921\\
188	0.33887442896853\\
189	0.339444085482365\\
190	0.340006998621172\\
191	0.340563248210435\\
192	0.341112913130695\\
193	0.341656071328733\\
194	0.34219279982863\\
195	0.342723174742682\\
196	0.343247271282198\\
197	0.343765163768164\\
198	0.344276925641785\\
199	0.344782629474893\\
200	0.345282346980247\\
201	0.345776149021694\\
202	0.346264105624226\\
203	0.346746285983902\\
204	0.347222758477667\\
205	0.347693590673046\\
206	0.348158849337725\\
207	0.348618600449019\\
208	0.34907290920323\\
209	0.349521840024892\\
210	0.349965456575904\\
211	0.350403821764561\\
212	0.350836997754474\\
213	0.351265045973385\\
214	0.351688027121877\\
215	0.352106001181985\\
216	0.352519027425696\\
217	0.352927164423362\\
218	0.353330470052\\
219	0.3537290015035\\
220	0.354122815292737\\
221	0.354511967265587\\
222	0.35489651260684\\
223	0.355276505848033\\
224	0.355652000875177\\
225	0.356023050936403\\
226	0.356389708649511\\
227	0.35675202600943\\
228	0.357110054395594\\
229	0.357463844579228\\
230	0.357813446730546\\
231	0.358158910425864\\
232	0.358500284654638\\
233	0.358837617826399\\
234	0.35917095777763\\
235	0.359500351778542\\
236	0.359825846539778\\
237	0.360147488219041\\
238	0.360465322427635\\
239	0.360779394236935\\
240	0.36108974818478\\
241	0.361396428281786\\
242	0.361699478017587\\
243	0.361998940367006\\
244	0.362294857796144\\
245	0.362587272268406\\
246	0.362876225250452\\
247	0.363161757718071\\
248	0.363443910162001\\
249	0.363722722593664\\
250	0.363998234550841\\
251	0.364270485103284\\
252	0.364539512858247\\
253	0.364805355965971\\
254	0.365068052125085\\
255	0.365327638587959\\
256	0.365584152165983\\
257	0.365837629234788\\
258	0.366088105739403\\
259	0.366335617199354\\
260	0.366580198713702\\
261	0.366821884966018\\
262	0.367060710229302\\
263	0.367296708370842\\
264	0.36752991285702\\
265	0.367760356758056\\
266	0.367988072752697\\
267	0.368213093132849\\
268	0.368435449808163\\
269	0.368655174310554\\
270	0.368872297798673\\
271	0.369086851062328\\
272	0.36929886452685\\
273	0.369508368257405\\
274	0.36971539196326\\
275	0.369919965001996\\
276	0.370122116383667\\
277	0.370321874774921\\
278	0.370519268503058\\
279	0.370714325560053\\
280	0.370907073606522\\
281	0.371097539975643\\
282	0.371285751677038\\
283	0.371471735400597\\
284	0.371655517520265\\
285	0.371837124097784\\
286	0.372016580886387\\
287	0.372193913334449\\
288	0.372369146589097\\
289	0.372542305499777\\
290	0.372713414621776\\
291	0.372882498219705\\
292	0.37304958027094\\
293	0.373214684469023\\
294	0.373377834227019\\
295	0.37353905268084\\
296	0.373698362692522\\
297	0.373855786853471\\
298	0.374011347487663\\
299	0.374165066654812\\
300	0.374316966153498\\
301	0.374467067524258\\
302	0.374615392052638\\
303	0.374761960772218\\
304	0.374906794467586\\
305	0.375049913677294\\
306	0.375191338696764\\
307	0.375331089581169\\
308	0.375469186148278\\
309	0.375605647981263\\
310	0.375740494431479\\
311	0.375873744621206\\
312	0.376005417446364\\
313	0.376135531579188\\
314	0.376264105470879\\
315	0.376391157354221\\
316	0.376516705246164\\
317	0.37664076695038\\
318	0.376763360059789\\
319	0.376884501959052\\
320	0.377004209827037\\
321	0.377122500639257\\
322	0.377239391170272\\
323	0.377354897996075\\
324	0.377469037496435\\
325	0.377581825857227\\
326	0.377693279072721\\
327	0.377803412947855\\
328	0.377912243100474\\
329	0.378019784963543\\
330	0.37812605378734\\
331	0.378231064641615\\
332	0.378334832417727\\
333	0.378437371830759\\
334	0.378538697421601\\
335	0.378638823559014\\
336	0.378737764441667\\
337	0.378835534100151\\
338	0.378932146398966\\
339	0.379027615038494\\
340	0.379121953556933\\
341	0.379215175332222\\
342	0.379307293583938\\
343	0.37939832137517\\
344	0.379488271614369\\
345	0.379577157057183\\
346	0.379664990308263\\
347	0.379751783823049\\
348	0.379837549909542\\
349	0.379922300730042\\
350	0.380006048302879\\
};
\addlegendentry{odpowiedź aproksymowana}

\end{axis}
\end{tikzpicture}%
\caption{Aproksymacja odpowiedzi skokowej U-Y}
\end{figure}

Taki sam proces aproksymacji przeprowadzony został dla toru zakłócenie-wyjście dla skoku zakłócenia od $Z = 0$ do $Z = 10$.

\begin{figure}[H]
\centering
% This file was created by matlab2tikz.
%
%The latest updates can be retrieved from
%  http://www.mathworks.com/matlabcentral/fileexchange/22022-matlab2tikz-matlab2tikz
%where you can also make suggestions and rate matlab2tikz.
%
\definecolor{mycolor1}{rgb}{0.00000,0.44700,0.74100}%
\definecolor{mycolor2}{rgb}{0.85000,0.32500,0.09800}%
%
\begin{tikzpicture}

\begin{axis}[%
width=4.521in,
height=3.566in,
at={(0.758in,0.481in)},
scale only axis,
xmin=0,
xmax=350,
xlabel style={font=\color{white!15!black}},
xlabel={k},
ymin=-0.05,
ymax=0.25,
ylabel style={font=\color{white!15!black}},
ylabel={s},
axis background/.style={fill=white},
legend style={at={(0.03,0.97)}, anchor=north west, legend cell align=left, align=left, draw=white!15!black}
]
\addplot[const plot, color=mycolor1] table[row sep=crcr] {%
1	0\\
2	0\\
3	0\\
4	0\\
5	-0.00599999999999987\\
6	-0.00599999999999987\\
7	-0.00599999999999987\\
8	-0.00599999999999987\\
9	-0.00599999999999987\\
10	-0.00599999999999987\\
11	-0.00599999999999987\\
12	-0.00599999999999987\\
13	-0.0120000000000001\\
14	-0.00599999999999987\\
15	-0.00599999999999987\\
16	-0.00599999999999987\\
17	-0.00599999999999987\\
18	0\\
19	0\\
20	0\\
21	0\\
22	0.00700000000000003\\
23	0.00700000000000003\\
24	0.00700000000000003\\
25	0.00700000000000003\\
26	0.0129999999999999\\
27	0.0129999999999999\\
28	0.0129999999999999\\
29	0.0190000000000001\\
30	0.0190000000000001\\
31	0.025\\
32	0.025\\
33	0.025\\
34	0.032\\
35	0.0379999999999999\\
36	0.0379999999999999\\
37	0.0440000000000001\\
38	0.0440000000000001\\
39	0.05\\
40	0.05\\
41	0.05\\
42	0.057\\
43	0.057\\
44	0.0629999999999999\\
45	0.0690000000000001\\
46	0.0690000000000001\\
47	0.0690000000000001\\
48	0.075\\
49	0.075\\
50	0.082\\
51	0.0879999999999999\\
52	0.0879999999999999\\
53	0.0940000000000001\\
54	0.1\\
55	0.1\\
56	0.107\\
57	0.107\\
58	0.113\\
59	0.113\\
60	0.119\\
61	0.119\\
62	0.119\\
63	0.119\\
64	0.125\\
65	0.125\\
66	0.132\\
67	0.132\\
68	0.132\\
69	0.138\\
70	0.138\\
71	0.138\\
72	0.144\\
73	0.144\\
74	0.144\\
75	0.144\\
76	0.144\\
77	0.144\\
78	0.144\\
79	0.144\\
80	0.144\\
81	0.144\\
82	0.144\\
83	0.144\\
84	0.144\\
85	0.144\\
86	0.144\\
87	0.144\\
88	0.144\\
89	0.15\\
90	0.15\\
91	0.15\\
92	0.15\\
93	0.157\\
94	0.157\\
95	0.157\\
96	0.157\\
97	0.163\\
98	0.163\\
99	0.163\\
100	0.163\\
101	0.163\\
102	0.163\\
103	0.169\\
104	0.169\\
105	0.169\\
106	0.169\\
107	0.169\\
108	0.169\\
109	0.169\\
110	0.175\\
111	0.175\\
112	0.169\\
113	0.169\\
114	0.169\\
115	0.169\\
116	0.175\\
117	0.169\\
118	0.175\\
119	0.175\\
120	0.175\\
121	0.175\\
122	0.175\\
123	0.175\\
124	0.175\\
125	0.169\\
126	0.169\\
127	0.175\\
128	0.175\\
129	0.175\\
130	0.175\\
131	0.175\\
132	0.175\\
133	0.182\\
134	0.182\\
135	0.182\\
136	0.182\\
137	0.182\\
138	0.182\\
139	0.182\\
140	0.182\\
141	0.182\\
142	0.188\\
143	0.188\\
144	0.188\\
145	0.188\\
146	0.188\\
147	0.194\\
148	0.194\\
149	0.194\\
150	0.194\\
151	0.194\\
152	0.194\\
153	0.194\\
154	0.194\\
155	0.194\\
156	0.2\\
157	0.2\\
158	0.2\\
159	0.2\\
160	0.2\\
161	0.2\\
162	0.207\\
163	0.207\\
164	0.207\\
165	0.207\\
166	0.207\\
167	0.207\\
168	0.213\\
169	0.213\\
170	0.213\\
171	0.213\\
172	0.213\\
173	0.213\\
174	0.219\\
175	0.219\\
176	0.219\\
177	0.219\\
178	0.219\\
179	0.219\\
180	0.225\\
181	0.225\\
182	0.225\\
183	0.225\\
184	0.225\\
185	0.225\\
186	0.225\\
187	0.219\\
188	0.225\\
189	0.219\\
190	0.219\\
191	0.219\\
192	0.219\\
193	0.219\\
194	0.219\\
195	0.219\\
196	0.219\\
197	0.219\\
198	0.219\\
199	0.219\\
200	0.219\\
201	0.219\\
202	0.219\\
203	0.219\\
204	0.219\\
205	0.219\\
206	0.219\\
207	0.219\\
208	0.219\\
209	0.219\\
210	0.219\\
211	0.219\\
212	0.219\\
213	0.219\\
214	0.219\\
215	0.219\\
216	0.219\\
217	0.219\\
218	0.219\\
219	0.219\\
220	0.225\\
221	0.225\\
222	0.225\\
223	0.225\\
224	0.225\\
225	0.225\\
226	0.225\\
227	0.225\\
228	0.225\\
229	0.225\\
230	0.225\\
231	0.225\\
232	0.225\\
233	0.219\\
234	0.219\\
235	0.219\\
236	0.219\\
237	0.225\\
238	0.225\\
239	0.225\\
240	0.225\\
241	0.225\\
242	0.225\\
243	0.225\\
244	0.225\\
245	0.225\\
246	0.225\\
247	0.225\\
248	0.232\\
249	0.232\\
250	0.232\\
251	0.232\\
252	0.232\\
253	0.232\\
254	0.232\\
255	0.232\\
256	0.232\\
257	0.232\\
258	0.232\\
259	0.232\\
260	0.232\\
261	0.232\\
262	0.232\\
263	0.232\\
264	0.232\\
265	0.238\\
266	0.232\\
267	0.232\\
268	0.232\\
269	0.232\\
270	0.232\\
271	0.232\\
272	0.232\\
273	0.232\\
274	0.232\\
275	0.232\\
276	0.232\\
277	0.232\\
278	0.232\\
279	0.232\\
280	0.232\\
281	0.232\\
282	0.232\\
283	0.232\\
284	0.232\\
285	0.232\\
286	0.232\\
287	0.238\\
288	0.238\\
289	0.238\\
290	0.238\\
291	0.238\\
292	0.238\\
293	0.238\\
294	0.238\\
295	0.238\\
296	0.238\\
297	0.238\\
298	0.238\\
299	0.238\\
300	0.238\\
301	0.232\\
302	0.232\\
303	0.232\\
304	0.232\\
305	0.232\\
306	0.232\\
307	0.225\\
308	0.225\\
309	0.225\\
310	0.225\\
311	0.225\\
312	0.225\\
313	0.225\\
314	0.225\\
315	0.219\\
316	0.219\\
317	0.219\\
318	0.219\\
319	0.219\\
320	0.219\\
321	0.219\\
322	0.219\\
323	0.219\\
324	0.219\\
325	0.219\\
326	0.225\\
327	0.225\\
328	0.225\\
329	0.232\\
330	0.225\\
331	0.232\\
332	0.232\\
333	0.232\\
334	0.238\\
335	0.238\\
336	0.238\\
337	0.238\\
338	0.238\\
339	0.238\\
340	0.238\\
341	0.238\\
342	0.244\\
343	0.244\\
344	0.244\\
345	0.244\\
346	0.244\\
347	0.244\\
348	0.244\\
349	0.244\\
350	0.244\\
};
\addlegendentry{odpowiedź układu}

\addplot[const plot, color=mycolor2] table[row sep=crcr] {%
1	0\\
2	0\\
3	0\\
4	0\\
5	0\\
6	0\\
7	0\\
8	0\\
9	0\\
10	0\\
11	0\\
12	0\\
13	0.00031480694231052\\
14	0.000912719759194197\\
15	0.0017648005265903\\
16	0.00284463943136222\\
17	0.00412813963954528\\
18	0.0055933203819677\\
19	0.00722013671602903\\
20	0.00899031455278642\\
21	0.0108871996578383\\
22	0.0128956194437396\\
23	0.0150017564716886\\
24	0.0171930326717687\\
25	0.019458003374835\\
26	0.0217862603258448\\
27	0.0241683429186625\\
28	0.0265956569566483\\
29	0.029060400302194\\
30	0.0315554948322354\\
31	0.0340745241660874\\
32	0.0366116766770875\\
33	0.0391616933408615\\
34	0.0417198200108493\\
35	0.0442817637463627\\
36	0.0468436528501437\\
37	0.0494020003014105\\
38	0.0519536702969458\\
39	0.0544958476370943\\
40	0.0570260097158007\\
41	0.0595419008941949\\
42	0.0620415090558849\\
43	0.0645230441591944\\
44	0.0669849186172131\\
45	0.069425729350837\\
46	0.0718442413730761\\
47	0.0742393727748949\\
48	0.0766101809938317\\
49	0.0789558502566869\\
50	0.0812756800967716\\
51	0.0835690748546267\\
52	0.0858355340788298\\
53	0.0880746437505653\\
54	0.0902860682620895\\
55	0.092469543085137\\
56	0.0946248680707262\\
57	0.0967519013267771\\
58	0.0988505536244888\\
59	0.100920783288578\\
60	0.102962591530277\\
61	0.104976018185476\\
62	0.106961137823564\\
63	0.108918056195462\\
64	0.110846906991985\\
65	0.112747848886131\\
66	0.114621062835125\\
67	0.116466749620087\\
68	0.118285127603076\\
69	0.120076430682984\\
70	0.121840906433292\\
71	0.123578814406187\\
72	0.125290424588799\\
73	0.126976015998577\\
74	0.128635875405879\\
75	0.130270296172888\\
76	0.131879577198882\\
77	0.133464021962729\\
78	0.135023937654248\\
79	0.136559634386811\\
80	0.138071424484164\\
81	0.139559621835092\\
82	0.141024541310045\\
83	0.142466498234384\\
84	0.14388580791333\\
85	0.145282785204128\\
86	0.146657744131329\\
87	0.148010997541415\\
88	0.149342856793343\\
89	0.150653631481859\\
90	0.151943629190696\\
91	0.153213155273039\\
92	0.154462512656829\\
93	0.155692001672717\\
94	0.156901919902658\\
95	0.158092562047274\\
96	0.15926421981034\\
97	0.160417181798817\\
98	0.161551733437043\\
99	0.162668156893783\\
100	0.163766731020971\\
101	0.164847731303046\\
102	0.165911429815926\\
103	0.166958095194694\\
104	0.167987992609187\\
105	0.169001383746738\\
106	0.169998526801372\\
107	0.170979676468844\\
108	0.171945083946939\\
109	0.172894996940504\\
110	0.173829659670748\\
111	0.174749312888367\\
112	0.175654193890093\\
113	0.176544536538309\\
114	0.1774205712834\\
115	0.17828252518853\\
116	0.17913062195658\\
117	0.179965081958986\\
118	0.180786122266263\\
119	0.181593956679984\\
120	0.182388795766047\\
121	0.18317084688905\\
122	0.183940314247607\\
123	0.184697398910479\\
124	0.185442298853383\\
125	0.186175208996353\\
126	0.186896321241563\\
127	0.187605824511508\\
128	0.188303904787443\\
129	0.18899074514803\\
130	0.189666525808086\\
131	0.190331424157398\\
132	0.190985614799527\\
133	0.191629269590559\\
134	0.192262557677751\\
135	0.192885645538028\\
136	0.193498697016303\\
137	0.194101873363569\\
138	0.194695333274753\\
139	0.195279232926285\\
140	0.195853726013379\\
141	0.196418963786979\\
142	0.196975095090381\\
143	0.19752226639549\\
144	0.198060621838716\\
145	0.198590303256474\\
146	0.199111450220313\\
147	0.199624200071625\\
148	0.200128687955956\\
149	0.200625046856899\\
150	0.201113407629571\\
151	0.201593899033656\\
152	0.202066647766039\\
153	0.202531778492993\\
154	0.202989413881948\\
155	0.203439674632824\\
156	0.203882679508937\\
157	0.204318545367472\\
158	0.204747387189528\\
159	0.205169318109739\\
160	0.205584449445462\\
161	0.205992890725556\\
162	0.206394749718724\\
163	0.206790132461456\\
164	0.207179143285541\\
165	0.207561884845184\\
166	0.207938458143708\\
167	0.208308962559857\\
168	0.2086734958737\\
169	0.209032154292141\\
170	0.209385032474043\\
171	0.209732223554957\\
172	0.210073819171483\\
173	0.21040990948525\\
174	0.210740583206519\\
175	0.211065927617426\\
176	0.211386028594864\\
177	0.211700970633006\\
178	0.212010836865468\\
179	0.212315709087141\\
180	0.212615667775663\\
181	0.212910792112566\\
182	0.213201160004082\\
183	0.213486848101623\\
184	0.213767931821937\\
185	0.214044485366946\\
186	0.214316581743269\\
187	0.214584292781432\\
188	0.214847689154779\\
189	0.215106840398077\\
190	0.215361814925826\\
191	0.215612680050278\\
192	0.215859501999172\\
193	0.216102345933176\\
194	0.216341275963063\\
195	0.216576355166604\\
196	0.216807645605194\\
197	0.217035208340211\\
198	0.217259103449114\\
199	0.217479390041282\\
200	0.217696126273603\\
201	0.217909369365806\\
202	0.218119175615556\\
203	0.218325600413299\\
204	0.218528698256871\\
205	0.218728522765881\\
206	0.218925126695845\\
207	0.219118561952113\\
208	0.219308879603559\\
209	0.219496129896054\\
210	0.219680362265727\\
211	0.219861625352006\\
212	0.220039967010456\\
213	0.220215434325404\\
214	0.220388073622369\\
215	0.22055793048028\\
216	0.220725049743515\\
217	0.220889475533724\\
218	0.221051251261487\\
219	0.22121041963776\\
220	0.221367022685155\\
221	0.221521101749028\\
222	0.221672697508394\\
223	0.221821849986666\\
224	0.221968598562213\\
225	0.22211298197876\\
226	0.222255038355616\\
227	0.222394805197731\\
228	0.222532319405599\\
229	0.222667617285001\\
230	0.222800734556585\\
231	0.222931706365297\\
232	0.223060567289661\\
233	0.223187351350904\\
234	0.223312092021937\\
235	0.223434822236195\\
236	0.223555574396329\\
237	0.223674380382759\\
238	0.22379127156209\\
239	0.223906278795398\\
240	0.22401943244637\\
241	0.224130762389324\\
242	0.224240298017095\\
243	0.224348068248794\\
244	0.224454101537446\\
245	0.224558425877497\\
246	0.224661068812208\\
247	0.224762057440923\\
248	0.224861418426228\\
249	0.224959178000988\\
250	0.225055361975272\\
251	0.225149995743166\\
252	0.22524310428948\\
253	0.225334712196345\\
254	0.225424843649697\\
255	0.225513522445668\\
256	0.225600771996869\\
257	0.225686615338565\\
258	0.225771075134763\\
259	0.225854173684192\\
260	0.22593593292619\\
261	0.226016374446499\\
262	0.226095519482962\\
263	0.226173388931129\\
264	0.226250003349777\\
265	0.226325382966334\\
266	0.226399547682222\\
267	0.22647251707811\\
268	0.226544310419083\\
269	0.226614946659731\\
270	0.22668444444915\\
271	0.226752822135867\\
272	0.226820097772685\\
273	0.226886289121449\\
274	0.226951413657731\\
275	0.227015488575454\\
276	0.22707853079142\\
277	0.227140556949783\\
278	0.227201583426443\\
279	0.227261626333365\\
280	0.227320701522841\\
281	0.227378824591664\\
282	0.227436010885258\\
283	0.227492275501719\\
284	0.227547633295808\\
285	0.22760209888287\\
286	0.227655686642692\\
287	0.227708410723302\\
288	0.227760285044702\\
289	0.227811323302545\\
290	0.227861538971749\\
291	0.227910945310057\\
292	0.227959555361532\\
293	0.228007381960009\\
294	0.228054437732477\\
295	0.228100735102413\\
296	0.228146286293066\\
297	0.228191103330682\\
298	0.228235198047676\\
299	0.228278582085761\\
300	0.228321266899019\\
301	0.228363263756926\\
302	0.228404583747326\\
303	0.228445237779359\\
304	0.228485236586341\\
305	0.2285245907286\\
306	0.22856331059626\\
307	0.228601406411988\\
308	0.228638888233689\\
309	0.228675765957167\\
310	0.228712049318732\\
311	0.228747747897774\\
312	0.228782871119289\\
313	0.228817428256373\\
314	0.228851428432664\\
315	0.228884880624754\\
316	0.228917793664558\\
317	0.228950176241648\\
318	0.228982036905542\\
319	0.229013384067968\\
320	0.229044226005079\\
321	0.229074570859642\\
322	0.229104426643185\\
323	0.229133801238114\\
324	0.229162702399794\\
325	0.229191137758594\\
326	0.229219114821907\\
327	0.229246640976125\\
328	0.229273723488595\\
329	0.229300369509536\\
330	0.229326586073923\\
331	0.229352380103351\\
332	0.229377758407857\\
333	0.229402727687722\\
334	0.229427294535235\\
335	0.229451465436439\\
336	0.229475246772839\\
337	0.22949864482309\\
338	0.229521665764652\\
339	0.229544315675425\\
340	0.229566600535346\\
341	0.229588526227978\\
342	0.229610098542054\\
343	0.229631323173013\\
344	0.229652205724496\\
345	0.229672751709833\\
346	0.229692966553494\\
347	0.229712855592522\\
348	0.229732424077942\\
349	0.229751677176148\\
350	0.229770619970268\\
};
\addlegendentry{odpowiedź aproksymowana}

\end{axis}
\end{tikzpicture}%
\caption{Aproksymacja odpowiedzi skokowe Z-Y}
\end{figure}

Do wyznaczenia optymalnych parametrów aproksymacji posłużono się algorytmem genetycznym o losowej populacji początkowej tak aby zminimalizować błąd dopasowania.

\section{DMC}

Prawo regulacji DMC przedstawia się następująco:

\begin{equation}
\triangle U(k)=K(Y^{zad}(k)-Y^0(k))
\end{equation}

Gdzie $\triangle U(k)$ to wektor $N_u$ (horyzont sterowania) przyszłych wartości sterowania, $Y^0(k)$ to przewidywana odpowiedź z modelu procesu, $K$ - macierz policzona raz na początku ze współczynników odpowiedzi skokowej, uwzględniając wybrany współczynnik $\lambda$ oraz horyzonty predykcji i sterowania.

W przypadku algorytmu DMC z pomiarem zakłóceń $Y^0(k)$ oblicza się z następującego wzoru:
\begin{equation}
Y^0(k)=Y(k)+M^P \triangle U^P(k)+M^{Z^P}\triangle Z^P(k)
\end{equation}

W powyższym wzorze dwa pierwsze elementy sumy odnoszą się do toru sterowanie-wyjście a ostatni element do toru zakłócenie-wyjście: $M^{Z^P}$ macierz wyznaczana przy pomocy współczynników odpowiedzi skokowej dla zakłócenia, $\triangle Z^P(k)$ jest wektorem przyrostów mierzalnego zakłócenia.

Poniżej przedstawione są wyniki działania programu dla skoku wartości zadanej z punktu pracy 28,18 do 35 dla różnych parametrów regulatora:

\begin{figure}[H]
\centering
% This file was created by matlab2tikz.
%
%The latest updates can be retrieved from
%  http://www.mathworks.com/matlabcentral/fileexchange/22022-matlab2tikz-matlab2tikz
%where you can also make suggestions and rate matlab2tikz.
%
\definecolor{mycolor1}{rgb}{0.00000,0.44700,0.74100}%
\definecolor{mycolor2}{rgb}{0.85000,0.32500,0.09800}%
%
\begin{tikzpicture}

\begin{axis}[%
width=4.521in,
height=3.566in,
at={(0.758in,0.481in)},
scale only axis,
xmin=0,
xmax=600,
xlabel style={font=\color{white!15!black}},
xlabel={k},
ymin=18,
ymax=36,
ylabel style={font=\color{white!15!black}},
ylabel={$\text{T[}^\circ\text{C]}$},
axis background/.style={fill=white}
]
\addplot[const plot, color=mycolor1, forget plot] table[row sep=crcr] {%
1	19.93\\
2	19.93\\
3	19.93\\
4	19.93\\
5	19.93\\
6	19.93\\
7	19.93\\
8	19.93\\
9	20\\
10	20\\
11	20\\
12	20.12\\
13	20.18\\
14	20.31\\
15	20.43\\
16	20.56\\
17	20.68\\
18	20.87\\
19	21.06\\
20	21.25\\
21	21.43\\
22	21.62\\
23	21.87\\
24	22.12\\
25	22.31\\
26	22.56\\
27	22.87\\
28	23.12\\
29	23.37\\
30	23.62\\
31	23.87\\
32	24.12\\
33	24.43\\
34	24.68\\
35	24.93\\
36	25.18\\
37	25.43\\
38	25.68\\
39	25.93\\
40	26.18\\
41	26.37\\
42	26.56\\
43	26.81\\
44	27\\
45	27.18\\
46	27.37\\
47	27.62\\
48	27.81\\
49	28\\
50	28.12\\
51	28.31\\
52	28.43\\
53	28.56\\
54	28.68\\
55	28.87\\
56	28.93\\
57	29.12\\
58	29.18\\
59	29.31\\
60	29.37\\
61	29.5\\
62	29.56\\
63	29.62\\
64	29.68\\
65	29.75\\
66	29.81\\
67	29.87\\
68	29.93\\
69	29.93\\
70	30\\
71	30\\
72	30.06\\
73	30.12\\
74	30.06\\
75	30.12\\
76	30.12\\
77	30.12\\
78	30.12\\
79	30.12\\
80	30.12\\
81	30.06\\
82	30.06\\
83	30.06\\
84	30.06\\
85	30\\
86	30\\
87	29.93\\
88	29.93\\
89	29.87\\
90	29.87\\
91	29.81\\
92	29.81\\
93	29.75\\
94	29.75\\
95	29.75\\
96	29.68\\
97	29.68\\
98	29.62\\
99	29.62\\
100	29.62\\
101	29.56\\
102	29.56\\
103	29.5\\
104	29.5\\
105	29.5\\
106	29.43\\
107	29.43\\
108	29.43\\
109	29.43\\
110	29.43\\
111	29.37\\
112	29.37\\
113	29.37\\
114	29.37\\
115	29.31\\
116	29.31\\
117	29.31\\
118	29.25\\
119	29.25\\
120	29.25\\
121	29.18\\
122	29.18\\
123	29.18\\
124	29.18\\
125	29.18\\
126	29.12\\
127	29.12\\
128	29.12\\
129	29.12\\
130	29.12\\
131	29.12\\
132	29.12\\
133	29.12\\
134	29.12\\
135	29.18\\
136	29.12\\
137	29.12\\
138	29.12\\
139	29.12\\
140	29.12\\
141	29.06\\
142	29.12\\
143	29.12\\
144	29.06\\
145	29.12\\
146	29.06\\
147	29.06\\
148	29.12\\
149	29.12\\
150	29.12\\
151	29.12\\
152	29.12\\
153	29.12\\
154	29.12\\
155	29.12\\
156	29.12\\
157	29.12\\
158	29.12\\
159	29.12\\
160	29.12\\
161	29.12\\
162	29.12\\
163	29.12\\
164	29.12\\
165	29.12\\
166	29.12\\
167	29.12\\
168	29.06\\
169	29.06\\
170	29.06\\
171	29.06\\
172	29.06\\
173	29.06\\
174	29.06\\
175	29\\
176	29\\
177	29\\
178	29\\
179	29\\
180	28.93\\
181	28.93\\
182	28.93\\
183	28.93\\
184	28.93\\
185	28.93\\
186	28.93\\
187	28.93\\
188	28.87\\
189	28.87\\
190	28.87\\
191	28.87\\
192	28.87\\
193	28.87\\
194	28.87\\
195	28.87\\
196	28.87\\
197	28.81\\
198	28.81\\
199	28.81\\
200	28.81\\
201	28.81\\
202	28.81\\
203	28.81\\
204	28.81\\
205	28.81\\
206	28.75\\
207	28.75\\
208	28.75\\
209	28.75\\
210	28.75\\
211	28.81\\
212	28.81\\
213	28.87\\
214	28.87\\
215	28.93\\
216	29\\
217	29.06\\
218	29.18\\
219	29.25\\
220	29.37\\
221	29.43\\
222	29.56\\
223	29.68\\
224	29.81\\
225	29.93\\
226	30.12\\
227	30.25\\
228	30.43\\
229	30.56\\
230	30.68\\
231	30.87\\
232	31\\
233	31.18\\
234	31.31\\
235	31.5\\
236	31.62\\
237	31.81\\
238	31.93\\
239	32.06\\
240	32.25\\
241	32.37\\
242	32.5\\
243	32.62\\
244	32.75\\
245	32.87\\
246	33\\
247	33.12\\
248	33.31\\
249	33.37\\
250	33.5\\
251	33.56\\
252	33.68\\
253	33.81\\
254	33.87\\
255	34\\
256	34.06\\
257	34.12\\
258	34.18\\
259	34.31\\
260	34.37\\
261	34.43\\
262	34.5\\
263	34.56\\
264	34.62\\
265	34.68\\
266	34.75\\
267	34.81\\
268	34.87\\
269	34.93\\
270	34.93\\
271	34.93\\
272	35\\
273	35\\
274	35.06\\
275	35.06\\
276	35.06\\
277	35.12\\
278	35.12\\
279	35.12\\
280	35.12\\
281	35.12\\
282	35.12\\
283	35.12\\
284	35.12\\
285	35.12\\
286	35.12\\
287	35.18\\
288	35.18\\
289	35.18\\
290	35.18\\
291	35.18\\
292	35.18\\
293	35.18\\
294	35.18\\
295	35.18\\
296	35.18\\
297	35.18\\
298	35.18\\
299	35.18\\
300	35.12\\
301	35.12\\
302	35.12\\
303	35.12\\
304	35.12\\
305	35.06\\
306	35.06\\
307	35.06\\
308	35.06\\
309	35.06\\
310	35\\
311	35.06\\
312	35.06\\
313	35.06\\
314	35.06\\
315	35.06\\
316	35.06\\
317	35.06\\
318	35.06\\
319	35.12\\
320	35.12\\
321	35.12\\
322	35.12\\
323	35.12\\
324	35.12\\
325	35.12\\
326	35.12\\
327	35.18\\
328	35.18\\
329	35.18\\
330	35.18\\
331	35.18\\
332	35.18\\
333	35.18\\
334	35.18\\
335	35.18\\
336	35.18\\
337	35.18\\
338	35.18\\
339	35.18\\
340	35.18\\
341	35.18\\
342	35.25\\
343	35.25\\
344	35.25\\
345	35.25\\
346	35.25\\
347	35.25\\
348	35.25\\
349	35.25\\
350	35.25\\
351	35.25\\
352	35.25\\
353	35.25\\
354	35.25\\
355	35.25\\
356	35.25\\
357	35.25\\
358	35.25\\
359	35.25\\
360	35.25\\
361	35.25\\
362	35.25\\
363	35.31\\
364	35.31\\
365	35.31\\
366	35.31\\
367	35.31\\
368	35.31\\
369	35.31\\
370	35.31\\
371	35.31\\
372	35.31\\
373	35.31\\
374	35.31\\
375	35.31\\
376	35.25\\
377	35.25\\
378	35.25\\
379	35.25\\
380	35.18\\
381	35.18\\
382	35.18\\
383	35.18\\
384	35.18\\
385	35.18\\
386	35.18\\
387	35.18\\
388	35.12\\
389	35.18\\
390	35.18\\
391	35.18\\
392	35.18\\
393	35.18\\
394	35.18\\
395	35.25\\
396	35.18\\
397	35.18\\
398	35.18\\
399	35.18\\
400	35.18\\
401	35.18\\
402	35.18\\
403	35.18\\
404	35.18\\
405	35.18\\
406	35.12\\
407	35.12\\
408	35.12\\
409	35.12\\
410	35.12\\
411	35.12\\
412	35.06\\
413	35.06\\
414	35.12\\
415	35.06\\
416	35.06\\
417	35.06\\
418	35.06\\
419	35.06\\
420	35.06\\
421	35.06\\
422	35.06\\
423	35.06\\
424	35.06\\
425	35.06\\
426	35.12\\
427	35.12\\
428	35.12\\
429	35.12\\
430	35.12\\
431	35.12\\
432	35.12\\
433	35.18\\
434	35.18\\
435	35.18\\
436	35.18\\
437	35.18\\
438	35.18\\
439	35.18\\
440	35.18\\
441	35.18\\
442	35.18\\
443	35.18\\
444	35.18\\
445	35.18\\
446	35.18\\
447	35.18\\
448	35.18\\
449	35.18\\
450	35.12\\
451	35.18\\
452	35.12\\
453	35.12\\
454	35.18\\
455	35.18\\
456	35.18\\
457	35.18\\
458	35.18\\
459	35.18\\
460	35.18\\
461	35.18\\
462	35.18\\
463	35.18\\
464	35.18\\
465	35.25\\
466	35.18\\
467	35.18\\
468	35.18\\
469	35.18\\
470	35.25\\
471	35.18\\
472	35.18\\
473	35.18\\
474	35.18\\
475	35.18\\
476	35.18\\
477	35.18\\
478	35.18\\
479	35.25\\
480	35.25\\
481	35.18\\
482	35.25\\
483	35.25\\
484	35.25\\
485	35.25\\
486	35.25\\
487	35.25\\
488	35.25\\
489	35.25\\
490	35.25\\
491	35.25\\
492	35.25\\
493	35.25\\
494	35.25\\
495	35.25\\
496	35.25\\
497	35.25\\
498	35.25\\
499	35.25\\
500	35.25\\
501	35.25\\
502	35.25\\
503	35.18\\
504	35.18\\
505	35.18\\
506	35.18\\
507	35.18\\
508	35.25\\
509	35.25\\
510	35.18\\
511	35.25\\
512	35.25\\
513	35.18\\
514	35.25\\
515	35.18\\
516	35.25\\
517	35.18\\
518	35.25\\
519	35.18\\
520	35.18\\
521	35.25\\
522	35.25\\
523	35.18\\
524	35.18\\
525	35.18\\
526	35.25\\
527	35.25\\
528	35.25\\
529	35.25\\
530	35.25\\
531	35.25\\
532	35.25\\
533	35.31\\
534	35.25\\
535	35.31\\
536	35.31\\
537	35.31\\
538	35.31\\
539	35.31\\
540	35.31\\
541	35.31\\
542	35.25\\
543	35.31\\
544	35.25\\
545	35.25\\
546	35.25\\
547	35.25\\
548	35.25\\
549	35.25\\
550	35.25\\
551	35.18\\
552	35.18\\
553	35.18\\
554	35.18\\
555	35.18\\
556	35.18\\
557	35.12\\
558	35.12\\
559	35.12\\
560	35.12\\
561	35.12\\
562	35.12\\
563	35.12\\
564	35.06\\
565	35.06\\
566	35.06\\
567	35.06\\
568	35.06\\
569	35.12\\
570	35.12\\
571	35.12\\
572	35.12\\
573	35.12\\
574	35.12\\
575	35.12\\
576	35.12\\
577	35.12\\
578	35.12\\
579	35.12\\
580	35.12\\
581	35.12\\
582	35.12\\
583	35.12\\
584	35.12\\
585	35.12\\
586	35.12\\
587	35.12\\
588	35.12\\
589	35.06\\
590	35.06\\
591	35.06\\
592	35.06\\
593	35.06\\
594	35.06\\
595	35.06\\
596	35.06\\
597	35.06\\
598	35.06\\
599	35.06\\
600	35.06\\
};
\addplot[const plot, color=mycolor2, forget plot] table[row sep=crcr] {%
1	28.18\\
2	28.18\\
3	28.18\\
4	28.18\\
5	28.18\\
6	28.18\\
7	28.18\\
8	28.18\\
9	28.18\\
10	28.18\\
11	28.18\\
12	28.18\\
13	28.18\\
14	28.18\\
15	28.18\\
16	28.18\\
17	28.18\\
18	28.18\\
19	28.18\\
20	28.18\\
21	28.18\\
22	28.18\\
23	28.18\\
24	28.18\\
25	28.18\\
26	28.18\\
27	28.18\\
28	28.18\\
29	28.18\\
30	28.18\\
31	28.18\\
32	28.18\\
33	28.18\\
34	28.18\\
35	28.18\\
36	28.18\\
37	28.18\\
38	28.18\\
39	28.18\\
40	28.18\\
41	28.18\\
42	28.18\\
43	28.18\\
44	28.18\\
45	28.18\\
46	28.18\\
47	28.18\\
48	28.18\\
49	28.18\\
50	28.18\\
51	28.18\\
52	28.18\\
53	28.18\\
54	28.18\\
55	28.18\\
56	28.18\\
57	28.18\\
58	28.18\\
59	28.18\\
60	28.18\\
61	28.18\\
62	28.18\\
63	28.18\\
64	28.18\\
65	28.18\\
66	28.18\\
67	28.18\\
68	28.18\\
69	28.18\\
70	28.18\\
71	28.18\\
72	28.18\\
73	28.18\\
74	28.18\\
75	28.18\\
76	28.18\\
77	28.18\\
78	28.18\\
79	28.18\\
80	28.18\\
81	28.18\\
82	28.18\\
83	28.18\\
84	28.18\\
85	28.18\\
86	28.18\\
87	28.18\\
88	28.18\\
89	28.18\\
90	28.18\\
91	28.18\\
92	28.18\\
93	28.18\\
94	28.18\\
95	28.18\\
96	28.18\\
97	28.18\\
98	28.18\\
99	28.18\\
100	28.18\\
101	28.18\\
102	28.18\\
103	28.18\\
104	28.18\\
105	28.18\\
106	28.18\\
107	28.18\\
108	28.18\\
109	28.18\\
110	28.18\\
111	28.18\\
112	28.18\\
113	28.18\\
114	28.18\\
115	28.18\\
116	28.18\\
117	28.18\\
118	28.18\\
119	28.18\\
120	28.18\\
121	28.18\\
122	28.18\\
123	28.18\\
124	28.18\\
125	28.18\\
126	28.18\\
127	28.18\\
128	28.18\\
129	28.18\\
130	28.18\\
131	28.18\\
132	28.18\\
133	28.18\\
134	28.18\\
135	28.18\\
136	28.18\\
137	28.18\\
138	28.18\\
139	28.18\\
140	28.18\\
141	28.18\\
142	28.18\\
143	28.18\\
144	28.18\\
145	28.18\\
146	28.18\\
147	28.18\\
148	28.18\\
149	28.18\\
150	28.18\\
151	28.18\\
152	28.18\\
153	28.18\\
154	28.18\\
155	28.18\\
156	28.18\\
157	28.18\\
158	28.18\\
159	28.18\\
160	28.18\\
161	28.18\\
162	28.18\\
163	28.18\\
164	28.18\\
165	28.18\\
166	28.18\\
167	28.18\\
168	28.18\\
169	28.18\\
170	28.18\\
171	28.18\\
172	28.18\\
173	28.18\\
174	28.18\\
175	28.18\\
176	28.18\\
177	28.18\\
178	28.18\\
179	28.18\\
180	28.18\\
181	28.18\\
182	28.18\\
183	28.18\\
184	28.18\\
185	28.18\\
186	28.18\\
187	28.18\\
188	28.18\\
189	28.18\\
190	28.18\\
191	28.18\\
192	28.18\\
193	28.18\\
194	28.18\\
195	28.18\\
196	28.18\\
197	28.18\\
198	28.18\\
199	28.18\\
200	28.18\\
201	35\\
202	35\\
203	35\\
204	35\\
205	35\\
206	35\\
207	35\\
208	35\\
209	35\\
210	35\\
211	35\\
212	35\\
213	35\\
214	35\\
215	35\\
216	35\\
217	35\\
218	35\\
219	35\\
220	35\\
221	35\\
222	35\\
223	35\\
224	35\\
225	35\\
226	35\\
227	35\\
228	35\\
229	35\\
230	35\\
231	35\\
232	35\\
233	35\\
234	35\\
235	35\\
236	35\\
237	35\\
238	35\\
239	35\\
240	35\\
241	35\\
242	35\\
243	35\\
244	35\\
245	35\\
246	35\\
247	35\\
248	35\\
249	35\\
250	35\\
251	35\\
252	35\\
253	35\\
254	35\\
255	35\\
256	35\\
257	35\\
258	35\\
259	35\\
260	35\\
261	35\\
262	35\\
263	35\\
264	35\\
265	35\\
266	35\\
267	35\\
268	35\\
269	35\\
270	35\\
271	35\\
272	35\\
273	35\\
274	35\\
275	35\\
276	35\\
277	35\\
278	35\\
279	35\\
280	35\\
281	35\\
282	35\\
283	35\\
284	35\\
285	35\\
286	35\\
287	35\\
288	35\\
289	35\\
290	35\\
291	35\\
292	35\\
293	35\\
294	35\\
295	35\\
296	35\\
297	35\\
298	35\\
299	35\\
300	35\\
301	35\\
302	35\\
303	35\\
304	35\\
305	35\\
306	35\\
307	35\\
308	35\\
309	35\\
310	35\\
311	35\\
312	35\\
313	35\\
314	35\\
315	35\\
316	35\\
317	35\\
318	35\\
319	35\\
320	35\\
321	35\\
322	35\\
323	35\\
324	35\\
325	35\\
326	35\\
327	35\\
328	35\\
329	35\\
330	35\\
331	35\\
332	35\\
333	35\\
334	35\\
335	35\\
336	35\\
337	35\\
338	35\\
339	35\\
340	35\\
341	35\\
342	35\\
343	35\\
344	35\\
345	35\\
346	35\\
347	35\\
348	35\\
349	35\\
350	35\\
351	35\\
352	35\\
353	35\\
354	35\\
355	35\\
356	35\\
357	35\\
358	35\\
359	35\\
360	35\\
361	35\\
362	35\\
363	35\\
364	35\\
365	35\\
366	35\\
367	35\\
368	35\\
369	35\\
370	35\\
371	35\\
372	35\\
373	35\\
374	35\\
375	35\\
376	35\\
377	35\\
378	35\\
379	35\\
380	35\\
381	35\\
382	35\\
383	35\\
384	35\\
385	35\\
386	35\\
387	35\\
388	35\\
389	35\\
390	35\\
391	35\\
392	35\\
393	35\\
394	35\\
395	35\\
396	35\\
397	35\\
398	35\\
399	35\\
400	35\\
401	35\\
402	35\\
403	35\\
404	35\\
405	35\\
406	35\\
407	35\\
408	35\\
409	35\\
410	35\\
411	35\\
412	35\\
413	35\\
414	35\\
415	35\\
416	35\\
417	35\\
418	35\\
419	35\\
420	35\\
421	35\\
422	35\\
423	35\\
424	35\\
425	35\\
426	35\\
427	35\\
428	35\\
429	35\\
430	35\\
431	35\\
432	35\\
433	35\\
434	35\\
435	35\\
436	35\\
437	35\\
438	35\\
439	35\\
440	35\\
441	35\\
442	35\\
443	35\\
444	35\\
445	35\\
446	35\\
447	35\\
448	35\\
449	35\\
450	35\\
451	35\\
452	35\\
453	35\\
454	35\\
455	35\\
456	35\\
457	35\\
458	35\\
459	35\\
460	35\\
461	35\\
462	35\\
463	35\\
464	35\\
465	35\\
466	35\\
467	35\\
468	35\\
469	35\\
470	35\\
471	35\\
472	35\\
473	35\\
474	35\\
475	35\\
476	35\\
477	35\\
478	35\\
479	35\\
480	35\\
481	35\\
482	35\\
483	35\\
484	35\\
485	35\\
486	35\\
487	35\\
488	35\\
489	35\\
490	35\\
491	35\\
492	35\\
493	35\\
494	35\\
495	35\\
496	35\\
497	35\\
498	35\\
499	35\\
500	35\\
501	35\\
502	35\\
503	35\\
504	35\\
505	35\\
506	35\\
507	35\\
508	35\\
509	35\\
510	35\\
511	35\\
512	35\\
513	35\\
514	35\\
515	35\\
516	35\\
517	35\\
518	35\\
519	35\\
520	35\\
521	35\\
522	35\\
523	35\\
524	35\\
525	35\\
526	35\\
527	35\\
528	35\\
529	35\\
530	35\\
531	35\\
532	35\\
533	35\\
534	35\\
535	35\\
536	35\\
537	35\\
538	35\\
539	35\\
540	35\\
541	35\\
542	35\\
543	35\\
544	35\\
545	35\\
546	35\\
547	35\\
548	35\\
549	35\\
550	35\\
551	35\\
552	35\\
553	35\\
554	35\\
555	35\\
556	35\\
557	35\\
558	35\\
559	35\\
560	35\\
561	35\\
562	35\\
563	35\\
564	35\\
565	35\\
566	35\\
567	35\\
568	35\\
569	35\\
570	35\\
571	35\\
572	35\\
573	35\\
574	35\\
575	35\\
576	35\\
577	35\\
578	35\\
579	35\\
580	35\\
581	35\\
582	35\\
583	35\\
584	35\\
585	35\\
586	35\\
587	35\\
588	35\\
589	35\\
590	35\\
591	35\\
592	35\\
593	35\\
594	35\\
595	35\\
596	35\\
597	35\\
598	35\\
599	35\\
600	35\\
};
\end{axis}
\end{tikzpicture}%
\caption{Wyjście procesu z regulatorem DMC dla parametrów $D = 340$, $ N = 90$, $N_u = 10$,  $\lambda = 0.4$}
\end{figure}

\begin{figure}[H]
\centering
% This file was created by matlab2tikz.
%
%The latest updates can be retrieved from
%  http://www.mathworks.com/matlabcentral/fileexchange/22022-matlab2tikz-matlab2tikz
%where you can also make suggestions and rate matlab2tikz.
%
\definecolor{mycolor1}{rgb}{0.00000,0.44700,0.74100}%
%
\begin{tikzpicture}

\begin{axis}[%
width=4.521in,
height=3.566in,
at={(0.758in,0.481in)},
scale only axis,
xmin=0,
xmax=600,
xlabel style={font=\color{white!15!black}},
xlabel={k},
ymin=20,
ymax=100,
ylabel style={font=\color{white!15!black}},
ylabel={\%},
axis background/.style={fill=white}
]
\addplot[const plot, color=mycolor1, forget plot] table[row sep=crcr] {%
1	36.502827\\
2	46.504549\\
3	55.145484\\
4	62.555481\\
5	68.854561\\
6	74.153506\\
7	78.554421\\
8	82.151262\\
9	84.932748\\
10	87.088386\\
11	88.689473\\
12	89.633842\\
13	90.086405\\
14	90.050634\\
15	89.656399\\
16	88.992409\\
17	88.161616\\
18	87.137477\\
19	85.988929\\
20	84.77055\\
21	83.538436\\
22	82.30813\\
23	81.018508\\
24	79.69575\\
25	78.442223\\
26	77.175971\\
27	75.821918\\
28	74.479274\\
29	73.148783\\
30	71.829374\\
31	70.51863\\
32	69.213288\\
33	67.826101\\
34	66.44811\\
35	65.075065\\
36	63.702952\\
37	62.32807\\
38	60.947342\\
39	59.558161\\
40	58.158038\\
41	56.828454\\
42	55.556681\\
43	54.247737\\
44	52.986154\\
45	51.77599\\
46	50.592433\\
47	49.343928\\
48	48.118061\\
49	46.907683\\
50	45.804124\\
51	44.691538\\
52	43.663879\\
53	42.69136\\
54	41.775806\\
55	40.807348\\
56	39.969118\\
57	39.058304\\
58	38.260745\\
59	37.4598\\
60	36.748634\\
61	36.013146\\
62	35.348997\\
63	34.742341\\
64	34.180995\\
65	33.640561\\
66	33.127896\\
67	32.635336\\
68	32.156115\\
69	31.768403\\
70	31.358847\\
71	31.024038\\
72	30.667028\\
73	30.287245\\
74	30.051737\\
75	29.771573\\
76	29.533101\\
77	29.328309\\
78	29.150273\\
79	28.993031\\
80	28.851464\\
81	28.804829\\
82	28.755049\\
83	28.700161\\
84	28.638782\\
85	28.653499\\
86	28.648829\\
87	28.723105\\
88	28.766817\\
89	28.865822\\
90	28.927814\\
91	29.039782\\
92	29.110401\\
93	29.22747\\
94	29.300616\\
95	29.334871\\
96	29.432478\\
97	29.487368\\
98	29.588499\\
99	29.646194\\
100	29.666199\\
101	29.737411\\
102	29.770182\\
103	29.853602\\
104	29.898171\\
105	29.909387\\
106	29.989947\\
107	30.034018\\
108	30.046648\\
109	30.032543\\
110	29.995853\\
111	30.024071\\
112	30.025976\\
113	30.005414\\
114	29.965942\\
115	29.994336\\
116	29.998934\\
117	29.983295\\
118	30.034119\\
119	30.059749\\
120	30.063768\\
121	30.146892\\
122	30.201436\\
123	30.230837\\
124	30.238371\\
125	30.227003\\
126	30.282965\\
127	30.313919\\
128	30.323088\\
129	30.313402\\
130	30.287448\\
131	30.247751\\
132	30.196596\\
133	30.135954\\
134	30.067825\\
135	29.910326\\
136	29.843201\\
137	29.772077\\
138	29.697984\\
139	29.621989\\
140	29.545051\\
141	29.55157\\
142	29.46385\\
143	29.377859\\
144	29.377415\\
145	29.28431\\
146	29.277739\\
147	29.262939\\
148	29.156757\\
149	29.054395\\
150	28.955494\\
151	28.85972\\
152	28.766776\\
153	28.676382\\
154	28.588544\\
155	28.503079\\
156	28.419738\\
157	28.338537\\
158	28.259286\\
159	28.182005\\
160	28.106758\\
161	28.033328\\
162	27.961392\\
163	27.89062\\
164	27.8207\\
165	27.751362\\
166	27.682381\\
167	27.613591\\
168	27.628498\\
169	27.632407\\
170	27.626262\\
171	27.610962\\
172	27.587329\\
173	27.556164\\
174	27.518205\\
175	27.557805\\
176	27.581037\\
177	27.589528\\
178	27.584777\\
179	27.568179\\
180	27.638598\\
181	27.687162\\
182	27.716455\\
183	27.728856\\
184	27.726509\\
185	27.711327\\
186	27.684983\\
187	27.648972\\
188	27.688522\\
189	27.710525\\
190	27.717308\\
191	27.710917\\
192	27.693134\\
193	27.665839\\
194	27.630758\\
195	27.589402\\
196	27.543031\\
197	27.57634\\
198	27.595601\\
199	27.602509\\
200	27.598537\\
201	37.094233\\
202	45.341136\\
203	52.456517\\
204	58.548803\\
205	63.718131\\
206	68.140492\\
207	71.806329\\
208	74.794844\\
209	77.17875\\
210	79.024987\\
211	80.311234\\
212	81.188115\\
213	81.622012\\
214	81.784485\\
215	81.670174\\
216	81.346489\\
217	80.898094\\
218	80.297184\\
219	79.668533\\
220	78.976837\\
221	78.340022\\
222	77.675216\\
223	77.017579\\
224	76.366054\\
225	75.744291\\
226	75.058179\\
227	74.4046\\
228	73.712629\\
229	73.058174\\
230	72.450425\\
231	71.78455\\
232	71.149831\\
233	70.470335\\
234	69.819297\\
235	69.107136\\
236	68.437058\\
237	67.703965\\
238	67.011578\\
239	66.338725\\
240	65.597136\\
241	64.891149\\
242	64.200356\\
243	63.534953\\
244	62.875778\\
245	62.234009\\
246	61.591336\\
247	60.959729\\
248	60.237897\\
249	59.616635\\
250	58.983288\\
251	58.434672\\
252	57.873887\\
253	57.285859\\
254	56.769109\\
255	56.214353\\
256	55.721582\\
257	55.280406\\
258	54.881633\\
259	54.419486\\
260	53.997099\\
261	53.606594\\
262	53.227231\\
263	52.868813\\
264	52.526202\\
265	52.194971\\
266	51.857237\\
267	51.525553\\
268	51.196916\\
269	50.868915\\
270	50.623391\\
271	50.448122\\
272	50.234443\\
273	50.082845\\
274	49.899795\\
275	49.771133\\
276	49.688142\\
277	49.559445\\
278	49.472809\\
279	49.421138\\
280	49.398122\\
281	49.39816\\
282	49.416287\\
283	49.448352\\
284	49.491023\\
285	49.541405\\
286	49.597137\\
287	49.572474\\
288	49.560286\\
289	49.558271\\
290	49.564282\\
291	49.576489\\
292	49.593407\\
293	49.613877\\
294	49.636996\\
295	49.662081\\
296	49.688607\\
297	49.716205\\
298	49.744596\\
299	49.773574\\
300	49.886377\\
301	49.98798\\
302	50.076594\\
303	50.151098\\
304	50.210945\\
305	50.339684\\
306	50.443011\\
307	50.522513\\
308	50.579903\\
309	50.617003\\
310	50.719351\\
311	50.710593\\
312	50.688181\\
313	50.654159\\
314	50.610572\\
315	50.559367\\
316	50.502302\\
317	50.440927\\
318	50.376832\\
319	50.227819\\
320	50.089681\\
321	49.962372\\
322	49.84566\\
323	49.739446\\
324	49.643301\\
325	49.556629\\
326	49.4787\\
327	49.32509\\
328	49.18962\\
329	49.070523\\
330	48.966131\\
331	48.874882\\
332	48.795058\\
333	48.724995\\
334	48.663203\\
335	48.608363\\
336	48.559356\\
337	48.515238\\
338	48.475223\\
339	48.438662\\
340	48.404775\\
341	48.37282\\
342	48.244552\\
343	48.129813\\
344	48.027078\\
345	47.935002\\
346	47.852423\\
347	47.77834\\
348	47.711895\\
349	47.652341\\
350	47.599041\\
351	47.5514\\
352	47.50891\\
353	47.471099\\
354	47.437534\\
355	47.407496\\
356	47.380253\\
357	47.355123\\
358	47.33155\\
359	47.309063\\
360	47.2873\\
361	47.265986\\
362	47.244924\\
363	47.140322\\
364	47.046657\\
365	46.962852\\
366	46.887949\\
367	46.821098\\
368	46.76152\\
369	46.708513\\
370	46.661437\\
371	46.619724\\
372	46.582823\\
373	46.550261\\
374	46.521596\\
375	46.496399\\
376	46.557666\\
377	46.61013\\
378	46.654177\\
379	46.690219\\
380	46.816278\\
381	46.92249\\
382	47.010502\\
383	47.081896\\
384	47.13819\\
385	47.180827\\
386	47.211145\\
387	47.230423\\
388	47.323489\\
389	47.31349\\
390	47.297378\\
391	47.276329\\
392	47.251358\\
393	47.223678\\
394	47.194413\\
395	47.066932\\
396	47.049947\\
397	47.032507\\
398	47.015123\\
399	46.998144\\
400	46.981803\\
401	46.966499\\
402	46.952329\\
403	46.939242\\
404	46.927088\\
405	46.915697\\
406	46.988542\\
407	47.050842\\
408	47.103117\\
409	47.14608\\
410	47.180515\\
411	47.207195\\
412	47.310519\\
413	47.396603\\
414	47.383459\\
415	47.450874\\
416	47.504679\\
417	47.546273\\
418	47.576955\\
419	47.598181\\
420	47.611393\\
421	47.617902\\
422	47.618844\\
423	47.615216\\
424	47.607863\\
425	47.597776\\
426	47.502241\\
427	47.416349\\
428	47.33966\\
429	47.271759\\
430	47.212211\\
431	47.16053\\
432	47.116213\\
433	46.995062\\
434	46.891113\\
435	46.802789\\
436	46.72862\\
437	46.667212\\
438	46.617246\\
439	46.577232\\
440	46.545713\\
441	46.521366\\
442	46.50299\\
443	46.489581\\
444	46.480293\\
445	46.474396\\
446	46.471021\\
447	46.469342\\
448	46.468669\\
449	46.468476\\
450	46.552039\\
451	46.540867\\
452	46.614061\\
453	46.676925\\
454	46.646746\\
455	46.619038\\
456	46.593692\\
457	46.570618\\
458	46.549774\\
459	46.531117\\
460	46.514634\\
461	46.500295\\
462	46.488088\\
463	46.47827\\
464	46.47087\\
465	46.368494\\
466	46.379299\\
467	46.391665\\
468	46.405458\\
469	46.420481\\
470	46.338952\\
471	46.368645\\
472	46.397907\\
473	46.426723\\
474	46.455095\\
475	46.483051\\
476	46.51063\\
477	46.537879\\
478	46.564549\\
479	46.493052\\
480	46.433865\\
481	46.483571\\
482	46.433322\\
483	46.393478\\
484	46.363106\\
485	46.341434\\
486	46.32776\\
487	46.321444\\
488	46.321845\\
489	46.328319\\
490	46.340236\\
491	46.356978\\
492	46.377629\\
493	46.401244\\
494	46.427257\\
495	46.45496\\
496	46.483665\\
497	46.512761\\
498	46.541735\\
499	46.570168\\
500	46.597753\\
501	46.624273\\
502	46.647595\\
503	46.763793\\
504	46.861517\\
505	46.941361\\
506	47.004169\\
507	47.050956\\
508	46.985276\\
509	46.918712\\
510	46.94897\\
511	46.86833\\
512	46.788627\\
513	46.807724\\
514	46.717993\\
515	46.728866\\
516	46.632931\\
517	46.639873\\
518	46.542135\\
519	46.549104\\
520	46.550463\\
521	46.449382\\
522	46.356678\\
523	46.369436\\
524	46.376569\\
525	46.378451\\
526	46.278104\\
527	46.186219\\
528	46.102154\\
529	46.025069\\
530	45.954414\\
531	45.889424\\
532	45.829648\\
533	45.691099\\
534	45.651404\\
535	45.53065\\
536	45.423202\\
537	45.327958\\
538	45.24395\\
539	45.169943\\
540	45.10468\\
541	45.047004\\
542	45.079522\\
543	45.023081\\
544	45.056126\\
545	45.083311\\
546	45.104714\\
547	45.120648\\
548	45.13125\\
549	45.136657\\
550	45.137026\\
551	45.230174\\
552	45.306027\\
553	45.366089\\
554	45.411822\\
555	45.444888\\
556	45.466678\\
557	45.562337\\
558	45.638725\\
559	45.698168\\
560	45.742794\\
561	45.774505\\
562	45.794998\\
563	45.805775\\
564	45.892143\\
565	45.961088\\
566	46.015004\\
567	46.056003\\
568	46.085922\\
569	46.022694\\
570	45.962536\\
571	45.905919\\
572	45.853173\\
573	45.804499\\
574	45.75996\\
575	45.719514\\
576	45.683018\\
577	45.650537\\
578	45.622152\\
579	45.597874\\
580	45.577621\\
581	45.561229\\
582	45.548185\\
583	45.537866\\
584	45.529653\\
585	45.522978\\
586	45.51734\\
587	45.512318\\
588	45.507578\\
589	45.586508\\
590	45.654323\\
591	45.711884\\
592	45.760026\\
593	45.799551\\
594	45.831231\\
595	45.855807\\
596	45.873983\\
597	45.886424\\
598	45.893753\\
599	45.89655\\
600	45.895354\\
};
\end{axis}
\end{tikzpicture}%
\caption{Sterowanie procesu z regulatorem DMC dla parametrów $D = 340$, $ N = 90$, $N_u = 10$,  $\lambda = 0,4$}
\end{figure}

\begin{equation}
E = 3,1023 * 10^3
\end{equation}

\begin{figure}[H]
\centering
% This file was created by matlab2tikz.
%
%The latest updates can be retrieved from
%  http://www.mathworks.com/matlabcentral/fileexchange/22022-matlab2tikz-matlab2tikz
%where you can also make suggestions and rate matlab2tikz.
%
\definecolor{mycolor1}{rgb}{0.00000,0.44700,0.74100}%
\definecolor{mycolor2}{rgb}{0.85000,0.32500,0.09800}%
%
\begin{tikzpicture}

\begin{axis}[%
width=4.521in,
height=3.566in,
at={(0.758in,0.481in)},
scale only axis,
xmin=0,
xmax=600,
xlabel style={font=\color{white!15!black}},
xlabel={k},
ymin=28,
ymax=36,
ylabel style={font=\color{white!15!black}},
ylabel={$\text{T[}^\circ\text{C]}$},
axis background/.style={fill=white}
]
\addplot[const plot, color=mycolor1, forget plot] table[row sep=crcr] {%
1	28.25\\
2	28.25\\
3	28.25\\
4	28.25\\
5	28.18\\
6	28.18\\
7	28.18\\
8	28.18\\
9	28.18\\
10	28.18\\
11	28.18\\
12	28.18\\
13	28.18\\
14	28.18\\
15	28.18\\
16	28.18\\
17	28.18\\
18	28.18\\
19	28.18\\
20	28.18\\
21	28.18\\
22	28.18\\
23	28.18\\
24	28.18\\
25	28.18\\
26	28.18\\
27	28.18\\
28	28.18\\
29	28.18\\
30	28.18\\
31	28.18\\
32	28.18\\
33	28.25\\
34	28.18\\
35	28.18\\
36	28.18\\
37	28.18\\
38	28.18\\
39	28.18\\
40	28.18\\
41	28.18\\
42	28.12\\
43	28.12\\
44	28.12\\
45	28.12\\
46	28.12\\
47	28.12\\
48	28.12\\
49	28.18\\
50	28.18\\
51	28.18\\
52	28.18\\
53	28.25\\
54	28.25\\
55	28.25\\
56	28.25\\
57	28.25\\
58	28.18\\
59	28.18\\
60	28.25\\
61	28.18\\
62	28.25\\
63	28.18\\
64	28.18\\
65	28.18\\
66	28.18\\
67	28.12\\
68	28.12\\
69	28.12\\
70	28.12\\
71	28.18\\
72	28.12\\
73	28.12\\
74	28.12\\
75	28.12\\
76	28.18\\
77	28.12\\
78	28.18\\
79	28.18\\
80	28.18\\
81	28.18\\
82	28.18\\
83	28.18\\
84	28.18\\
85	28.18\\
86	28.12\\
87	28.12\\
88	28.12\\
89	28.18\\
90	28.18\\
91	28.18\\
92	28.18\\
93	28.18\\
94	28.12\\
95	28.12\\
96	28.12\\
97	28.12\\
98	28.12\\
99	28.12\\
100	28.18\\
101	28.18\\
102	28.18\\
103	28.18\\
104	28.18\\
105	28.25\\
106	28.25\\
107	28.25\\
108	28.25\\
109	28.25\\
110	28.25\\
111	28.18\\
112	28.25\\
113	28.25\\
114	28.25\\
115	28.25\\
116	28.25\\
117	28.25\\
118	28.25\\
119	28.25\\
120	28.25\\
121	28.25\\
122	28.18\\
123	28.18\\
124	28.18\\
125	28.18\\
126	28.18\\
127	28.18\\
128	28.18\\
129	28.18\\
130	28.18\\
131	28.18\\
132	28.12\\
133	28.18\\
134	28.12\\
135	28.18\\
136	28.18\\
137	28.18\\
138	28.18\\
139	28.18\\
140	28.18\\
141	28.18\\
142	28.12\\
143	28.12\\
144	28.12\\
145	28.12\\
146	28.12\\
147	28.12\\
148	28.12\\
149	28.12\\
150	28.18\\
151	28.12\\
152	28.18\\
153	28.18\\
154	28.18\\
155	28.18\\
156	28.12\\
157	28.18\\
158	28.18\\
159	28.18\\
160	28.18\\
161	28.18\\
162	28.18\\
163	28.18\\
164	28.18\\
165	28.18\\
166	28.18\\
167	28.18\\
168	28.18\\
169	28.25\\
170	28.18\\
171	28.25\\
172	28.18\\
173	28.25\\
174	28.18\\
175	28.18\\
176	28.18\\
177	28.18\\
178	28.18\\
179	28.18\\
180	28.18\\
181	28.18\\
182	28.18\\
183	28.18\\
184	28.18\\
185	28.18\\
186	28.18\\
187	28.18\\
188	28.18\\
189	28.12\\
190	28.18\\
191	28.12\\
192	28.18\\
193	28.12\\
194	28.12\\
195	28.12\\
196	28.12\\
197	28.12\\
198	28.12\\
199	28.18\\
200	28.18\\
201	28.18\\
202	28.18\\
203	28.18\\
204	28.18\\
205	28.18\\
206	28.18\\
207	28.18\\
208	28.12\\
209	28.18\\
210	28.18\\
211	28.25\\
212	28.25\\
213	28.25\\
214	28.31\\
215	28.37\\
216	28.43\\
217	28.5\\
218	28.56\\
219	28.68\\
220	28.75\\
221	28.87\\
222	28.93\\
223	29.06\\
224	29.18\\
225	29.31\\
226	29.37\\
227	29.56\\
228	29.62\\
229	29.75\\
230	29.93\\
231	30.06\\
232	30.18\\
233	30.31\\
234	30.5\\
235	30.62\\
236	30.75\\
237	30.87\\
238	31\\
239	31.12\\
240	31.25\\
241	31.37\\
242	31.56\\
243	31.68\\
244	31.75\\
245	31.93\\
246	32\\
247	32.12\\
248	32.25\\
249	32.31\\
250	32.43\\
251	32.56\\
252	32.62\\
253	32.75\\
254	32.81\\
255	32.87\\
256	33\\
257	33.06\\
258	33.12\\
259	33.18\\
260	33.25\\
261	33.31\\
262	33.37\\
263	33.43\\
264	33.5\\
265	33.56\\
266	33.62\\
267	33.68\\
268	33.75\\
269	33.81\\
270	33.87\\
271	33.93\\
272	33.93\\
273	34\\
274	34\\
275	34.06\\
276	34.06\\
277	34.06\\
278	34.12\\
279	34.12\\
280	34.18\\
281	34.18\\
282	34.25\\
283	34.25\\
284	34.31\\
285	34.37\\
286	34.37\\
287	34.37\\
288	34.43\\
289	34.43\\
290	34.43\\
291	34.43\\
292	34.5\\
293	34.5\\
294	34.5\\
295	34.5\\
296	34.5\\
297	34.5\\
298	34.5\\
299	34.5\\
300	34.5\\
301	34.5\\
302	34.5\\
303	34.5\\
304	34.5\\
305	34.5\\
306	34.5\\
307	34.5\\
308	34.5\\
309	34.5\\
310	34.5\\
311	34.5\\
312	34.5\\
313	34.5\\
314	34.5\\
315	34.5\\
316	34.5\\
317	34.56\\
318	34.5\\
319	34.5\\
320	34.5\\
321	34.5\\
322	34.5\\
323	34.5\\
324	34.5\\
325	34.5\\
326	34.56\\
327	34.5\\
328	34.56\\
329	34.56\\
330	34.56\\
331	34.56\\
332	34.56\\
333	34.62\\
334	34.62\\
335	34.62\\
336	34.62\\
337	34.62\\
338	34.62\\
339	34.62\\
340	34.62\\
341	34.62\\
342	34.62\\
343	34.62\\
344	34.62\\
345	34.62\\
346	34.62\\
347	34.62\\
348	34.68\\
349	34.68\\
350	34.68\\
351	34.75\\
352	34.81\\
353	34.75\\
354	34.81\\
355	34.81\\
356	34.81\\
357	34.81\\
358	34.81\\
359	34.81\\
360	34.81\\
361	34.81\\
362	34.87\\
363	34.81\\
364	34.81\\
365	34.81\\
366	34.81\\
367	34.87\\
368	34.87\\
369	34.81\\
370	34.81\\
371	34.81\\
372	34.81\\
373	34.81\\
374	34.81\\
375	34.87\\
376	34.87\\
377	34.87\\
378	34.87\\
379	34.87\\
380	34.87\\
381	34.87\\
382	34.87\\
383	34.87\\
384	34.87\\
385	34.87\\
386	34.81\\
387	34.81\\
388	34.81\\
389	34.81\\
390	34.81\\
391	34.81\\
392	34.81\\
393	34.87\\
394	34.87\\
395	34.87\\
396	34.87\\
397	34.87\\
398	34.93\\
399	34.93\\
400	34.93\\
401	34.93\\
402	35\\
403	35\\
404	35\\
405	35\\
406	35\\
407	35\\
408	35\\
409	35\\
410	35\\
411	35\\
412	35\\
413	35.06\\
414	35.06\\
415	35.06\\
416	35.06\\
417	35.12\\
418	35.12\\
419	35.12\\
420	35.12\\
421	35.12\\
422	35.12\\
423	35.18\\
424	35.18\\
425	35.18\\
426	35.18\\
427	35.18\\
428	35.12\\
429	35.12\\
430	35.12\\
431	35.12\\
432	35.12\\
433	35.12\\
434	35.12\\
435	35.12\\
436	35.12\\
437	35.12\\
438	35.12\\
439	35.12\\
440	35.12\\
441	35.12\\
442	35.12\\
443	35.12\\
444	35.12\\
445	35.12\\
446	35.12\\
447	35.06\\
448	35.06\\
449	35.06\\
450	35\\
451	35\\
452	35\\
453	35\\
454	35\\
455	35\\
456	35\\
457	35\\
458	35\\
459	35\\
460	35\\
461	35.06\\
462	35\\
463	35\\
464	35\\
465	35\\
466	35\\
467	35\\
468	34.93\\
469	34.93\\
470	35\\
471	34.93\\
472	35\\
473	34.93\\
474	34.93\\
475	35\\
476	34.93\\
477	35\\
478	34.93\\
479	35\\
480	34.93\\
481	34.93\\
482	34.93\\
483	34.93\\
484	34.93\\
485	35\\
486	35\\
487	35\\
488	35.06\\
489	35\\
490	35\\
491	35.06\\
492	35\\
493	35\\
494	35\\
495	35\\
496	34.93\\
497	35\\
498	34.93\\
499	35\\
500	35\\
501	35\\
502	35.06\\
503	35.06\\
504	35.06\\
505	35.12\\
506	35.12\\
507	35.06\\
508	35.12\\
509	35.06\\
510	35.12\\
511	35.12\\
512	35.12\\
513	35.18\\
514	35.18\\
515	35.18\\
516	35.18\\
517	35.18\\
518	35.12\\
519	35.12\\
520	35.12\\
521	35.12\\
522	35.12\\
523	35.12\\
524	35.12\\
525	35.18\\
526	35.18\\
527	35.18\\
528	35.18\\
529	35.18\\
530	35.18\\
531	35.18\\
532	35.18\\
533	35.18\\
534	35.18\\
535	35.18\\
536	35.18\\
537	35.18\\
538	35.18\\
539	35.18\\
540	35.25\\
541	35.18\\
542	35.18\\
543	35.25\\
544	35.18\\
545	35.18\\
546	35.18\\
547	35.18\\
548	35.18\\
549	35.18\\
550	35.12\\
551	35.12\\
552	35.12\\
553	35.12\\
554	35.12\\
555	35.12\\
556	35.12\\
557	35.12\\
558	35.12\\
559	35.12\\
560	35.06\\
561	35.06\\
562	35.06\\
563	35.06\\
564	35.06\\
565	35.06\\
566	35.06\\
567	35.06\\
568	35.06\\
569	35\\
570	35.06\\
571	35.06\\
572	35.06\\
573	35.06\\
574	35.06\\
575	35.06\\
576	35.06\\
577	35.06\\
578	35.12\\
579	35.12\\
580	35.12\\
581	35.12\\
582	35.12\\
583	35.12\\
584	35.12\\
585	35.12\\
586	35.12\\
587	35.18\\
588	35.12\\
589	35.12\\
590	35.12\\
591	35.12\\
592	35.12\\
593	35.12\\
594	35.12\\
595	35.12\\
596	35.12\\
597	35.12\\
598	35.12\\
599	35.12\\
600	35.12\\
};
\addplot[const plot, color=mycolor2, forget plot] table[row sep=crcr] {%
1	28.18\\
2	28.18\\
3	28.18\\
4	28.18\\
5	28.18\\
6	28.18\\
7	28.18\\
8	28.18\\
9	28.18\\
10	28.18\\
11	28.18\\
12	28.18\\
13	28.18\\
14	28.18\\
15	28.18\\
16	28.18\\
17	28.18\\
18	28.18\\
19	28.18\\
20	28.18\\
21	28.18\\
22	28.18\\
23	28.18\\
24	28.18\\
25	28.18\\
26	28.18\\
27	28.18\\
28	28.18\\
29	28.18\\
30	28.18\\
31	28.18\\
32	28.18\\
33	28.18\\
34	28.18\\
35	28.18\\
36	28.18\\
37	28.18\\
38	28.18\\
39	28.18\\
40	28.18\\
41	28.18\\
42	28.18\\
43	28.18\\
44	28.18\\
45	28.18\\
46	28.18\\
47	28.18\\
48	28.18\\
49	28.18\\
50	28.18\\
51	28.18\\
52	28.18\\
53	28.18\\
54	28.18\\
55	28.18\\
56	28.18\\
57	28.18\\
58	28.18\\
59	28.18\\
60	28.18\\
61	28.18\\
62	28.18\\
63	28.18\\
64	28.18\\
65	28.18\\
66	28.18\\
67	28.18\\
68	28.18\\
69	28.18\\
70	28.18\\
71	28.18\\
72	28.18\\
73	28.18\\
74	28.18\\
75	28.18\\
76	28.18\\
77	28.18\\
78	28.18\\
79	28.18\\
80	28.18\\
81	28.18\\
82	28.18\\
83	28.18\\
84	28.18\\
85	28.18\\
86	28.18\\
87	28.18\\
88	28.18\\
89	28.18\\
90	28.18\\
91	28.18\\
92	28.18\\
93	28.18\\
94	28.18\\
95	28.18\\
96	28.18\\
97	28.18\\
98	28.18\\
99	28.18\\
100	28.18\\
101	28.18\\
102	28.18\\
103	28.18\\
104	28.18\\
105	28.18\\
106	28.18\\
107	28.18\\
108	28.18\\
109	28.18\\
110	28.18\\
111	28.18\\
112	28.18\\
113	28.18\\
114	28.18\\
115	28.18\\
116	28.18\\
117	28.18\\
118	28.18\\
119	28.18\\
120	28.18\\
121	28.18\\
122	28.18\\
123	28.18\\
124	28.18\\
125	28.18\\
126	28.18\\
127	28.18\\
128	28.18\\
129	28.18\\
130	28.18\\
131	28.18\\
132	28.18\\
133	28.18\\
134	28.18\\
135	28.18\\
136	28.18\\
137	28.18\\
138	28.18\\
139	28.18\\
140	28.18\\
141	28.18\\
142	28.18\\
143	28.18\\
144	28.18\\
145	28.18\\
146	28.18\\
147	28.18\\
148	28.18\\
149	28.18\\
150	28.18\\
151	28.18\\
152	28.18\\
153	28.18\\
154	28.18\\
155	28.18\\
156	28.18\\
157	28.18\\
158	28.18\\
159	28.18\\
160	28.18\\
161	28.18\\
162	28.18\\
163	28.18\\
164	28.18\\
165	28.18\\
166	28.18\\
167	28.18\\
168	28.18\\
169	28.18\\
170	28.18\\
171	28.18\\
172	28.18\\
173	28.18\\
174	28.18\\
175	28.18\\
176	28.18\\
177	28.18\\
178	28.18\\
179	28.18\\
180	28.18\\
181	28.18\\
182	28.18\\
183	28.18\\
184	28.18\\
185	28.18\\
186	28.18\\
187	28.18\\
188	28.18\\
189	28.18\\
190	28.18\\
191	28.18\\
192	28.18\\
193	28.18\\
194	28.18\\
195	28.18\\
196	28.18\\
197	28.18\\
198	28.18\\
199	28.18\\
200	28.18\\
201	35\\
202	35\\
203	35\\
204	35\\
205	35\\
206	35\\
207	35\\
208	35\\
209	35\\
210	35\\
211	35\\
212	35\\
213	35\\
214	35\\
215	35\\
216	35\\
217	35\\
218	35\\
219	35\\
220	35\\
221	35\\
222	35\\
223	35\\
224	35\\
225	35\\
226	35\\
227	35\\
228	35\\
229	35\\
230	35\\
231	35\\
232	35\\
233	35\\
234	35\\
235	35\\
236	35\\
237	35\\
238	35\\
239	35\\
240	35\\
241	35\\
242	35\\
243	35\\
244	35\\
245	35\\
246	35\\
247	35\\
248	35\\
249	35\\
250	35\\
251	35\\
252	35\\
253	35\\
254	35\\
255	35\\
256	35\\
257	35\\
258	35\\
259	35\\
260	35\\
261	35\\
262	35\\
263	35\\
264	35\\
265	35\\
266	35\\
267	35\\
268	35\\
269	35\\
270	35\\
271	35\\
272	35\\
273	35\\
274	35\\
275	35\\
276	35\\
277	35\\
278	35\\
279	35\\
280	35\\
281	35\\
282	35\\
283	35\\
284	35\\
285	35\\
286	35\\
287	35\\
288	35\\
289	35\\
290	35\\
291	35\\
292	35\\
293	35\\
294	35\\
295	35\\
296	35\\
297	35\\
298	35\\
299	35\\
300	35\\
301	35\\
302	35\\
303	35\\
304	35\\
305	35\\
306	35\\
307	35\\
308	35\\
309	35\\
310	35\\
311	35\\
312	35\\
313	35\\
314	35\\
315	35\\
316	35\\
317	35\\
318	35\\
319	35\\
320	35\\
321	35\\
322	35\\
323	35\\
324	35\\
325	35\\
326	35\\
327	35\\
328	35\\
329	35\\
330	35\\
331	35\\
332	35\\
333	35\\
334	35\\
335	35\\
336	35\\
337	35\\
338	35\\
339	35\\
340	35\\
341	35\\
342	35\\
343	35\\
344	35\\
345	35\\
346	35\\
347	35\\
348	35\\
349	35\\
350	35\\
351	35\\
352	35\\
353	35\\
354	35\\
355	35\\
356	35\\
357	35\\
358	35\\
359	35\\
360	35\\
361	35\\
362	35\\
363	35\\
364	35\\
365	35\\
366	35\\
367	35\\
368	35\\
369	35\\
370	35\\
371	35\\
372	35\\
373	35\\
374	35\\
375	35\\
376	35\\
377	35\\
378	35\\
379	35\\
380	35\\
381	35\\
382	35\\
383	35\\
384	35\\
385	35\\
386	35\\
387	35\\
388	35\\
389	35\\
390	35\\
391	35\\
392	35\\
393	35\\
394	35\\
395	35\\
396	35\\
397	35\\
398	35\\
399	35\\
400	35\\
401	35\\
402	35\\
403	35\\
404	35\\
405	35\\
406	35\\
407	35\\
408	35\\
409	35\\
410	35\\
411	35\\
412	35\\
413	35\\
414	35\\
415	35\\
416	35\\
417	35\\
418	35\\
419	35\\
420	35\\
421	35\\
422	35\\
423	35\\
424	35\\
425	35\\
426	35\\
427	35\\
428	35\\
429	35\\
430	35\\
431	35\\
432	35\\
433	35\\
434	35\\
435	35\\
436	35\\
437	35\\
438	35\\
439	35\\
440	35\\
441	35\\
442	35\\
443	35\\
444	35\\
445	35\\
446	35\\
447	35\\
448	35\\
449	35\\
450	35\\
451	35\\
452	35\\
453	35\\
454	35\\
455	35\\
456	35\\
457	35\\
458	35\\
459	35\\
460	35\\
461	35\\
462	35\\
463	35\\
464	35\\
465	35\\
466	35\\
467	35\\
468	35\\
469	35\\
470	35\\
471	35\\
472	35\\
473	35\\
474	35\\
475	35\\
476	35\\
477	35\\
478	35\\
479	35\\
480	35\\
481	35\\
482	35\\
483	35\\
484	35\\
485	35\\
486	35\\
487	35\\
488	35\\
489	35\\
490	35\\
491	35\\
492	35\\
493	35\\
494	35\\
495	35\\
496	35\\
497	35\\
498	35\\
499	35\\
500	35\\
501	35\\
502	35\\
503	35\\
504	35\\
505	35\\
506	35\\
507	35\\
508	35\\
509	35\\
510	35\\
511	35\\
512	35\\
513	35\\
514	35\\
515	35\\
516	35\\
517	35\\
518	35\\
519	35\\
520	35\\
521	35\\
522	35\\
523	35\\
524	35\\
525	35\\
526	35\\
527	35\\
528	35\\
529	35\\
530	35\\
531	35\\
532	35\\
533	35\\
534	35\\
535	35\\
536	35\\
537	35\\
538	35\\
539	35\\
540	35\\
541	35\\
542	35\\
543	35\\
544	35\\
545	35\\
546	35\\
547	35\\
548	35\\
549	35\\
550	35\\
551	35\\
552	35\\
553	35\\
554	35\\
555	35\\
556	35\\
557	35\\
558	35\\
559	35\\
560	35\\
561	35\\
562	35\\
563	35\\
564	35\\
565	35\\
566	35\\
567	35\\
568	35\\
569	35\\
570	35\\
571	35\\
572	35\\
573	35\\
574	35\\
575	35\\
576	35\\
577	35\\
578	35\\
579	35\\
580	35\\
581	35\\
582	35\\
583	35\\
584	35\\
585	35\\
586	35\\
587	35\\
588	35\\
589	35\\
590	35\\
591	35\\
592	35\\
593	35\\
594	35\\
595	35\\
596	35\\
597	35\\
598	35\\
599	35\\
600	35\\
};
\end{axis}
\end{tikzpicture}%
\caption{Wyjście procesu z regulatorem DMC dla parametrów $D = 340$, $ N = 60$, $N_u = 5$,  $\lambda = 1$}
\end{figure}

\begin{figure}[H]
\centering
% This file was created by matlab2tikz.
%
%The latest updates can be retrieved from
%  http://www.mathworks.com/matlabcentral/fileexchange/22022-matlab2tikz-matlab2tikz
%where you can also make suggestions and rate matlab2tikz.
%
\definecolor{mycolor1}{rgb}{0.00000,0.44700,0.74100}%
%
\begin{tikzpicture}

\begin{axis}[%
width=4.521in,
height=3.566in,
at={(0.758in,0.481in)},
scale only axis,
xmin=0,
xmax=600,
xlabel style={font=\color{white!15!black}},
xlabel={k},
ymin=20,
ymax=80,
ylabel style={font=\color{white!15!black}},
ylabel={\%},
axis background/.style={fill=white}
]
\addplot[const plot, color=mycolor1, forget plot] table[row sep=crcr] {%
1	24.902421\\
2	24.820495\\
3	24.752148\\
4	24.695564\\
5	24.746733\\
6	24.791032\\
7	24.829318\\
8	24.862343\\
9	24.890771\\
10	24.915181\\
11	24.936084\\
12	24.953929\\
13	24.969108\\
14	24.981644\\
15	24.991468\\
16	24.998535\\
17	25.002863\\
18	25.004856\\
19	25.005027\\
20	25.003868\\
21	25.00181\\
22	24.999207\\
23	24.996338\\
24	24.993419\\
25	24.990609\\
26	24.988021\\
27	24.985733\\
28	24.983787\\
29	24.982201\\
30	24.98097\\
31	24.980069\\
32	24.979464\\
33	24.881534\\
34	24.897044\\
35	24.910643\\
36	24.922548\\
37	24.932947\\
38	24.942005\\
39	24.949871\\
40	24.956676\\
41	24.962538\\
42	25.051204\\
43	25.125714\\
44	25.187933\\
45	25.239497\\
46	25.281517\\
47	25.315201\\
48	25.341711\\
49	25.278464\\
50	25.223453\\
51	25.175718\\
52	25.134409\\
53	25.001193\\
54	24.888631\\
55	24.794329\\
56	24.716259\\
57	24.652625\\
58	24.699381\\
59	24.741807\\
60	24.683035\\
61	24.734443\\
62	24.683318\\
63	24.740762\\
64	24.791708\\
65	24.83664\\
66	24.875699\\
67	24.992603\\
68	25.09042\\
69	25.171133\\
70	25.236584\\
71	25.20517\\
72	25.259552\\
73	25.302357\\
74	25.335332\\
75	25.359776\\
76	25.293442\\
77	25.31833\\
78	25.253122\\
79	25.196083\\
80	25.146732\\
81	25.104678\\
82	25.06952\\
83	25.040817\\
84	25.017807\\
85	24.999933\\
86	25.070356\\
87	25.131563\\
88	25.184887\\
89	25.147528\\
90	25.117437\\
91	25.093395\\
92	25.074263\\
93	25.059047\\
94	25.13055\\
95	25.191028\\
96	25.241709\\
97	25.283711\\
98	25.318061\\
99	25.345971\\
100	25.284973\\
101	25.233164\\
102	25.189376\\
103	25.152472\\
104	25.121425\\
105	24.997759\\
106	24.893942\\
107	24.807529\\
108	24.736473\\
109	24.678988\\
110	24.633478\\
111	24.696077\\
112	24.654666\\
113	24.623086\\
114	24.599688\\
115	24.583013\\
116	24.571812\\
117	24.565027\\
118	24.561453\\
119	24.55999\\
120	24.559726\\
121	24.559953\\
122	24.657731\\
123	24.739471\\
124	24.807346\\
125	24.8631\\
126	24.908226\\
127	24.944043\\
128	24.971734\\
129	24.992372\\
130	25.006923\\
131	25.016258\\
132	25.104796\\
133	25.092532\\
134	25.162545\\
135	25.134776\\
136	25.108693\\
137	25.084614\\
138	25.06275\\
139	25.043212\\
140	25.026021\\
141	25.011123\\
142	25.08205\\
143	25.1416\\
144	25.191373\\
145	25.233014\\
146	25.267728\\
147	25.296732\\
148	25.320837\\
149	25.340639\\
150	25.272974\\
151	25.298946\\
152	25.236444\\
153	25.183074\\
154	25.137556\\
155	25.099062\\
156	25.150598\\
157	25.110916\\
158	25.078332\\
159	25.052129\\
160	25.031632\\
161	25.01621\\
162	25.005274\\
163	24.998001\\
164	24.993843\\
165	24.992109\\
166	24.992137\\
167	24.99335\\
168	24.995282\\
169	24.900266\\
170	24.91888\\
171	24.837995\\
172	24.86847\\
173	24.79749\\
174	24.836216\\
175	24.869686\\
176	24.89849\\
177	24.923169\\
178	24.944221\\
179	24.962101\\
180	24.977217\\
181	24.989936\\
182	25.000263\\
183	25.00844\\
184	25.01446\\
185	25.018595\\
186	25.020872\\
187	25.021588\\
188	25.021112\\
189	25.103431\\
190	25.088154\\
191	25.157998\\
192	25.132238\\
193	25.193296\\
194	25.24383\\
195	25.285341\\
196	25.319129\\
197	25.34631\\
198	25.367844\\
199	25.300917\\
200	25.243294\\
201	34.700787\\
202	42.640631\\
203	49.263966\\
204	54.747138\\
205	59.244286\\
206	62.890457\\
207	65.803779\\
208	68.171103\\
209	69.901818\\
210	71.173312\\
211	71.956219\\
212	72.422859\\
213	72.622446\\
214	72.545557\\
215	72.278855\\
216	71.896351\\
217	71.443053\\
218	70.979711\\
219	70.454447\\
220	69.971788\\
221	69.475889\\
222	69.068663\\
223	68.650798\\
224	68.246591\\
225	67.84588\\
226	67.549346\\
227	67.160717\\
228	66.875585\\
229	66.578651\\
230	66.199759\\
231	65.819077\\
232	65.448135\\
233	65.068738\\
234	64.595826\\
235	64.139536\\
236	63.68134\\
237	63.233613\\
238	62.778951\\
239	62.330966\\
240	61.873125\\
241	61.4197\\
242	60.871308\\
243	60.339439\\
244	59.890141\\
245	59.355752\\
246	58.901769\\
247	58.444196\\
248	57.967809\\
249	57.571319\\
250	57.156521\\
251	56.710296\\
252	56.333261\\
253	55.914885\\
254	55.557232\\
255	55.248311\\
256	54.880144\\
257	54.55726\\
258	54.269962\\
259	54.010126\\
260	53.756938\\
261	53.520676\\
262	53.296277\\
263	53.079515\\
264	52.852795\\
265	52.629417\\
266	52.406633\\
267	52.182331\\
268	51.941058\\
269	51.697682\\
270	51.450947\\
271	51.199989\\
272	51.02794\\
273	50.823523\\
274	50.688222\\
275	50.526289\\
276	50.424374\\
277	50.371728\\
278	50.275441\\
279	50.22458\\
280	50.126865\\
281	50.071852\\
282	49.953717\\
283	49.878557\\
284	49.754494\\
285	49.588111\\
286	49.468829\\
287	49.388512\\
288	49.25651\\
289	49.164047\\
290	49.104574\\
291	49.072399\\
292	48.965165\\
293	48.891882\\
294	48.846944\\
295	48.825336\\
296	48.822776\\
297	48.83546\\
298	48.859897\\
299	48.893171\\
300	48.932949\\
301	48.977146\\
302	49.024068\\
303	49.072443\\
304	49.121374\\
305	49.169923\\
306	49.217349\\
307	49.263152\\
308	49.307045\\
309	49.348936\\
310	49.38887\\
311	49.42698\\
312	49.463468\\
313	49.498572\\
314	49.532538\\
315	49.565616\\
316	49.598034\\
317	49.546359\\
318	49.591455\\
319	49.634641\\
320	49.676256\\
321	49.716591\\
322	49.755874\\
323	49.794294\\
324	49.832005\\
325	49.869134\\
326	49.82213\\
327	49.871757\\
328	49.835603\\
329	49.810994\\
330	49.796105\\
331	49.789534\\
332	49.790152\\
333	49.713354\\
334	49.655371\\
335	49.61369\\
336	49.586124\\
337	49.570754\\
338	49.56589\\
339	49.569772\\
340	49.581079\\
341	49.598456\\
342	49.620679\\
343	49.646711\\
344	49.675706\\
345	49.706954\\
346	49.739605\\
347	49.772896\\
348	49.722565\\
349	49.685214\\
350	49.658774\\
351	49.543922\\
352	49.36881\\
353	49.31086\\
354	49.184779\\
355	49.085427\\
356	49.009143\\
357	48.952752\\
358	48.913501\\
359	48.888994\\
360	48.877143\\
361	48.875864\\
362	48.799614\\
363	48.827445\\
364	48.858803\\
365	48.892101\\
366	48.926226\\
367	48.876501\\
368	48.839133\\
369	48.895371\\
370	48.946221\\
371	48.991912\\
372	49.032744\\
373	49.069109\\
374	49.101445\\
375	49.046279\\
376	49.001077\\
377	48.964554\\
378	48.935651\\
379	48.913482\\
380	48.896986\\
381	48.885181\\
382	48.877463\\
383	48.873413\\
384	48.872699\\
385	48.875029\\
386	48.963755\\
387	49.041559\\
388	49.10967\\
389	49.169064\\
390	49.220609\\
391	49.265087\\
392	49.30322\\
393	49.25205\\
394	49.209273\\
395	49.17371\\
396	49.144346\\
397	49.12032\\
398	49.017264\\
399	48.93188\\
400	48.862175\\
401	48.806405\\
402	48.665428\\
403	48.551069\\
404	48.459953\\
405	48.389088\\
406	48.335528\\
407	48.296578\\
408	48.269857\\
409	48.253299\\
410	48.245144\\
411	48.243611\\
412	48.247137\\
413	48.170747\\
414	48.11042\\
415	48.063209\\
416	48.02654\\
417	47.914631\\
418	47.822778\\
419	47.747809\\
420	47.687076\\
421	47.638363\\
422	47.599815\\
423	47.486252\\
424	47.393434\\
425	47.318493\\
426	47.258685\\
427	47.211553\\
428	47.258611\\
429	47.301009\\
430	47.338758\\
431	47.371841\\
432	47.400322\\
433	47.424334\\
434	47.444101\\
435	47.459879\\
436	47.471691\\
437	47.479562\\
438	47.483565\\
439	47.483867\\
440	47.480706\\
441	47.474672\\
442	47.466404\\
443	47.456522\\
444	47.445566\\
445	47.433983\\
446	47.42213\\
447	47.49393\\
448	47.552541\\
449	47.599882\\
450	47.721242\\
451	47.820966\\
452	47.902019\\
453	47.967\\
454	48.01813\\
455	48.05735\\
456	48.086342\\
457	48.106548\\
458	48.119237\\
459	48.125496\\
460	48.12656\\
461	48.039928\\
462	48.047289\\
463	48.050891\\
464	48.051914\\
465	48.051323\\
466	48.049857\\
467	48.048071\\
468	48.143926\\
469	48.224435\\
470	48.194241\\
471	48.26603\\
472	48.228515\\
473	48.294101\\
474	48.348574\\
475	48.295738\\
476	48.347705\\
477	48.292745\\
478	48.342944\\
479	48.286527\\
480	48.335534\\
481	48.37606\\
482	48.409722\\
483	48.437611\\
484	48.460891\\
485	48.382737\\
486	48.317255\\
487	48.263215\\
488	48.135551\\
489	48.11391\\
490	48.098268\\
491	48.004402\\
492	48.01223\\
493	48.022493\\
494	48.035094\\
495	48.04995\\
496	48.164538\\
497	48.167913\\
498	48.272505\\
499	48.26671\\
500	48.26514\\
501	48.266677\\
502	48.186955\\
503	48.122524\\
504	48.070984\\
505	47.946846\\
506	47.845663\\
507	47.848044\\
508	47.770684\\
509	47.794106\\
510	47.735489\\
511	47.691993\\
512	47.6617\\
513	47.559137\\
514	47.479748\\
515	47.42002\\
516	47.376811\\
517	47.347396\\
518	47.4128\\
519	47.473665\\
520	47.53013\\
521	47.582199\\
522	47.630205\\
523	47.674283\\
524	47.714572\\
525	47.667592\\
526	47.630332\\
527	47.600937\\
528	47.577857\\
529	47.559832\\
530	47.545848\\
531	47.535382\\
532	47.528115\\
533	47.523834\\
534	47.52233\\
535	47.52342\\
536	47.526908\\
537	47.532592\\
538	47.539997\\
539	47.548592\\
540	47.460347\\
541	47.485686\\
542	47.507769\\
543	47.428224\\
544	47.457418\\
545	47.479877\\
546	47.495895\\
547	47.505892\\
548	47.510366\\
549	47.509858\\
550	47.588589\\
551	47.650089\\
552	47.696724\\
553	47.730287\\
554	47.752617\\
555	47.765453\\
556	47.770043\\
557	47.767718\\
558	47.759725\\
559	47.747164\\
560	47.814616\\
561	47.865811\\
562	47.903171\\
563	47.929046\\
564	47.945532\\
565	47.95442\\
566	47.957197\\
567	47.955072\\
568	47.949032\\
569	48.023497\\
570	47.998397\\
571	47.973068\\
572	47.947689\\
573	47.922656\\
574	47.898367\\
575	47.875145\\
576	47.853208\\
577	47.832679\\
578	47.729961\\
579	47.642092\\
580	47.567235\\
581	47.503745\\
582	47.45043\\
583	47.406036\\
584	47.369357\\
585	47.339279\\
586	47.31482\\
587	47.21147\\
588	47.209164\\
589	47.208421\\
590	47.208895\\
591	47.210023\\
592	47.211256\\
593	47.212132\\
594	47.212313\\
595	47.211567\\
596	47.209757\\
597	47.206833\\
598	47.202805\\
599	47.197722\\
600	47.191389\\
};
\end{axis}
\end{tikzpicture}%
\caption{Sterowanie procesu z regulatorem DMC dla parametrów $D = 340$, $ N = 60$, $N_u = 5$,  $\lambda = 1$}
\end{figure}

\begin{equation}
E = 1,5950 * 10^3
\end{equation}

\begin{figure}[H]
\centering
% This file was created by matlab2tikz.
%
%The latest updates can be retrieved from
%  http://www.mathworks.com/matlabcentral/fileexchange/22022-matlab2tikz-matlab2tikz
%where you can also make suggestions and rate matlab2tikz.
%
\definecolor{mycolor1}{rgb}{0.00000,0.44700,0.74100}%
\definecolor{mycolor2}{rgb}{0.85000,0.32500,0.09800}%
%
\begin{tikzpicture}

\begin{axis}[%
width=4.521in,
height=3.566in,
at={(0.758in,0.481in)},
scale only axis,
xmin=0,
xmax=600,
xlabel style={font=\color{white!15!black}},
xlabel={k},
ymin=27,
ymax=36,
ylabel style={font=\color{white!15!black}},
ylabel={$\text{T[}^\circ\text{C]}$},
axis background/.style={fill=white}
]
\addplot[const plot, color=mycolor1, forget plot] table[row sep=crcr] {%
1	28.31\\
2	28.31\\
3	28.31\\
4	28.31\\
5	28.37\\
6	28.31\\
7	28.31\\
8	28.31\\
9	28.31\\
10	28.31\\
11	28.31\\
12	28.31\\
13	28.31\\
14	28.31\\
15	28.31\\
16	28.31\\
17	28.31\\
18	28.31\\
19	28.31\\
20	28.31\\
21	28.31\\
22	28.25\\
23	28.25\\
24	28.25\\
25	28.25\\
26	28.25\\
27	28.25\\
28	28.25\\
29	28.25\\
30	28.25\\
31	28.25\\
32	28.25\\
33	28.25\\
34	28.25\\
35	28.25\\
36	28.25\\
37	28.25\\
38	28.25\\
39	28.25\\
40	28.25\\
41	28.25\\
42	28.25\\
43	28.25\\
44	28.25\\
45	28.25\\
46	28.25\\
47	28.25\\
48	28.25\\
49	28.25\\
50	28.25\\
51	28.25\\
52	28.25\\
53	28.25\\
54	28.25\\
55	28.25\\
56	28.25\\
57	28.31\\
58	28.31\\
59	28.31\\
60	28.31\\
61	28.31\\
62	28.31\\
63	28.31\\
64	28.31\\
65	28.31\\
66	28.31\\
67	28.31\\
68	28.31\\
69	28.31\\
70	28.37\\
71	28.37\\
72	28.37\\
73	28.37\\
74	28.37\\
75	28.43\\
76	28.43\\
77	28.43\\
78	28.43\\
79	28.43\\
80	28.43\\
81	28.43\\
82	28.43\\
83	28.43\\
84	28.43\\
85	28.37\\
86	28.37\\
87	28.37\\
88	28.37\\
89	28.37\\
90	28.37\\
91	28.37\\
92	28.37\\
93	28.37\\
94	28.37\\
95	28.31\\
96	28.31\\
97	28.31\\
98	28.31\\
99	28.25\\
100	28.25\\
101	28.25\\
102	28.25\\
103	28.18\\
104	28.18\\
105	28.18\\
106	28.12\\
107	28.12\\
108	28.12\\
109	28.12\\
110	28.12\\
111	28.12\\
112	28.06\\
113	28.06\\
114	28.06\\
115	28.06\\
116	28.06\\
117	28.06\\
118	28.06\\
119	28\\
120	28\\
121	28.06\\
122	28\\
123	28\\
124	28\\
125	28\\
126	28\\
127	28\\
128	27.93\\
129	28\\
130	28\\
131	28\\
132	28\\
133	27.93\\
134	27.93\\
135	28\\
136	27.93\\
137	27.93\\
138	28\\
139	28\\
140	28\\
141	27.93\\
142	27.93\\
143	27.93\\
144	27.93\\
145	28\\
146	28\\
147	28\\
148	28\\
149	28\\
150	28\\
151	28\\
152	28\\
153	28.06\\
154	28.06\\
155	28.06\\
156	28.06\\
157	28.06\\
158	28.06\\
159	28.06\\
160	28.06\\
161	28\\
162	28.06\\
163	28.06\\
164	28.06\\
165	28.06\\
166	28.06\\
167	28.06\\
168	28.06\\
169	28.06\\
170	28.06\\
171	28.06\\
172	28.06\\
173	28.06\\
174	28.06\\
175	28.06\\
176	28.06\\
177	28.06\\
178	28\\
179	28.06\\
180	28\\
181	28\\
182	28\\
183	28.06\\
184	28\\
185	28.06\\
186	28.06\\
187	28.06\\
188	28\\
189	28.06\\
190	28.06\\
191	28.06\\
192	28.06\\
193	28\\
194	28\\
195	28\\
196	28\\
197	28\\
198	28\\
199	28\\
200	28\\
201	28\\
202	28\\
203	28\\
204	28\\
205	28\\
206	28\\
207	28\\
208	28\\
209	28\\
210	28\\
211	28.06\\
212	28.06\\
213	28.12\\
214	28.12\\
215	28.18\\
216	28.25\\
217	28.31\\
218	28.43\\
219	28.5\\
220	28.56\\
221	28.68\\
222	28.81\\
223	28.93\\
224	29\\
225	29.18\\
226	29.31\\
227	29.37\\
228	29.5\\
229	29.68\\
230	29.75\\
231	29.93\\
232	30.06\\
233	30.18\\
234	30.31\\
235	30.43\\
236	30.56\\
237	30.68\\
238	30.87\\
239	31\\
240	31.12\\
241	31.25\\
242	31.37\\
243	31.5\\
244	31.68\\
245	31.81\\
246	31.93\\
247	32.06\\
248	32.18\\
249	32.31\\
250	32.43\\
251	32.56\\
252	32.68\\
253	32.75\\
254	32.87\\
255	33\\
256	33.06\\
257	33.18\\
258	33.25\\
259	33.31\\
260	33.43\\
261	33.5\\
262	33.56\\
263	33.62\\
264	33.68\\
265	33.75\\
266	33.81\\
267	33.87\\
268	34\\
269	34\\
270	34.06\\
271	34.12\\
272	34.18\\
273	34.18\\
274	34.25\\
275	34.25\\
276	34.31\\
277	34.31\\
278	34.37\\
279	34.37\\
280	34.43\\
281	34.43\\
282	34.5\\
283	34.5\\
284	34.5\\
285	34.5\\
286	34.5\\
287	34.5\\
288	34.5\\
289	34.5\\
290	34.56\\
291	34.56\\
292	34.56\\
293	34.56\\
294	34.56\\
295	34.56\\
296	34.56\\
297	34.56\\
298	34.62\\
299	34.56\\
300	34.56\\
301	34.62\\
302	34.62\\
303	34.68\\
304	34.62\\
305	34.62\\
306	34.62\\
307	34.62\\
308	34.62\\
309	34.56\\
310	34.62\\
311	34.62\\
312	34.62\\
313	34.56\\
314	34.56\\
315	34.56\\
316	34.56\\
317	34.56\\
318	34.5\\
319	34.5\\
320	34.5\\
321	34.5\\
322	34.43\\
323	34.5\\
324	34.5\\
325	34.43\\
326	34.43\\
327	34.43\\
328	34.43\\
329	34.43\\
330	34.43\\
331	34.43\\
332	34.43\\
333	34.43\\
334	34.43\\
335	34.43\\
336	34.43\\
337	34.5\\
338	34.5\\
339	34.5\\
340	34.5\\
341	34.5\\
342	34.56\\
343	34.5\\
344	34.56\\
345	34.56\\
346	34.56\\
347	34.56\\
348	34.56\\
349	34.56\\
350	34.62\\
351	34.62\\
352	34.62\\
353	34.62\\
354	34.62\\
355	34.62\\
356	34.68\\
357	34.68\\
358	34.75\\
359	34.75\\
360	34.75\\
361	34.75\\
362	34.75\\
363	34.75\\
364	34.75\\
365	34.81\\
366	34.75\\
367	34.81\\
368	34.75\\
369	34.81\\
370	34.75\\
371	34.75\\
372	34.75\\
373	34.75\\
374	34.75\\
375	34.75\\
376	34.81\\
377	34.75\\
378	34.81\\
379	34.81\\
380	34.81\\
381	34.87\\
382	34.87\\
383	34.87\\
384	34.87\\
385	34.87\\
386	34.93\\
387	34.93\\
388	34.93\\
389	34.87\\
390	34.87\\
391	34.87\\
392	34.87\\
393	34.87\\
394	34.87\\
395	34.87\\
396	34.87\\
397	34.93\\
398	34.93\\
399	34.93\\
400	34.93\\
401	34.93\\
402	34.93\\
403	34.93\\
404	34.93\\
405	35\\
406	35\\
407	35\\
408	35\\
409	35\\
410	35\\
411	35\\
412	35\\
413	35\\
414	35\\
415	35\\
416	35\\
417	35\\
418	35\\
419	35\\
420	35\\
421	35\\
422	35\\
423	35.06\\
424	35\\
425	35\\
426	35\\
427	35\\
428	35.06\\
429	35.06\\
430	35.06\\
431	35.06\\
432	35.06\\
433	35\\
434	35\\
435	35\\
436	35\\
437	34.93\\
438	34.93\\
439	35\\
440	35\\
441	35\\
442	35\\
443	35\\
444	35\\
445	35\\
446	35\\
447	35\\
448	35\\
449	35\\
450	35\\
451	35\\
452	35\\
453	34.93\\
454	34.93\\
455	35\\
456	35\\
457	35\\
458	35\\
459	35.06\\
460	35.06\\
461	35.06\\
462	35.12\\
463	35.12\\
464	35.12\\
465	35.12\\
466	35.12\\
467	35.12\\
468	35.12\\
469	35.18\\
470	35.18\\
471	35.18\\
472	35.18\\
473	35.18\\
474	35.18\\
475	35.25\\
476	35.25\\
477	35.25\\
478	35.25\\
479	35.25\\
480	35.25\\
481	35.25\\
482	35.25\\
483	35.25\\
484	35.25\\
485	35.25\\
486	35.25\\
487	35.18\\
488	35.25\\
489	35.18\\
490	35.18\\
491	35.18\\
492	35.18\\
493	35.18\\
494	35.18\\
495	35.18\\
496	35.18\\
497	35.18\\
498	35.18\\
499	35.18\\
500	35.18\\
501	35.18\\
502	35.18\\
503	35.18\\
504	35.18\\
505	35.18\\
506	35.18\\
507	35.18\\
508	35.18\\
509	35.12\\
510	35.12\\
511	35.12\\
512	35.12\\
513	35.12\\
514	35.06\\
515	35.12\\
516	35.12\\
517	35.12\\
518	35.12\\
519	35.18\\
520	35.18\\
521	35.18\\
522	35.18\\
523	35.18\\
524	35.18\\
525	35.18\\
526	35.12\\
527	35.12\\
528	35.18\\
529	35.18\\
530	35.18\\
531	35.12\\
532	35.12\\
533	35.12\\
534	35.12\\
535	35.12\\
536	35.12\\
537	35.06\\
538	35.06\\
539	35.12\\
540	35.06\\
541	35.06\\
542	35.06\\
543	35.06\\
544	35.06\\
545	35.06\\
546	35.06\\
547	35.06\\
548	35.06\\
549	35.06\\
550	35.12\\
551	35.12\\
552	35.06\\
553	35.06\\
554	35.06\\
555	35.06\\
556	35.06\\
557	35.06\\
558	35.06\\
559	35.06\\
560	35.06\\
561	35.06\\
562	35.06\\
563	35.06\\
564	35\\
565	35\\
566	35\\
567	35\\
568	35\\
569	35\\
570	35\\
571	35\\
572	35\\
573	35\\
574	35\\
575	35.06\\
576	35\\
577	35.06\\
578	35.06\\
579	35.06\\
580	35.06\\
581	35.12\\
582	35.12\\
583	35.12\\
584	35.12\\
585	35.12\\
586	35.12\\
587	35.12\\
588	35.18\\
589	35.18\\
590	35.18\\
591	35.18\\
592	35.18\\
593	35.18\\
594	35.18\\
595	35.18\\
596	35.18\\
597	35.18\\
598	35.18\\
599	35.18\\
600	35.18\\
};
\addplot[const plot, color=mycolor2, forget plot] table[row sep=crcr] {%
1	28.18\\
2	28.18\\
3	28.18\\
4	28.18\\
5	28.18\\
6	28.18\\
7	28.18\\
8	28.18\\
9	28.18\\
10	28.18\\
11	28.18\\
12	28.18\\
13	28.18\\
14	28.18\\
15	28.18\\
16	28.18\\
17	28.18\\
18	28.18\\
19	28.18\\
20	28.18\\
21	28.18\\
22	28.18\\
23	28.18\\
24	28.18\\
25	28.18\\
26	28.18\\
27	28.18\\
28	28.18\\
29	28.18\\
30	28.18\\
31	28.18\\
32	28.18\\
33	28.18\\
34	28.18\\
35	28.18\\
36	28.18\\
37	28.18\\
38	28.18\\
39	28.18\\
40	28.18\\
41	28.18\\
42	28.18\\
43	28.18\\
44	28.18\\
45	28.18\\
46	28.18\\
47	28.18\\
48	28.18\\
49	28.18\\
50	28.18\\
51	28.18\\
52	28.18\\
53	28.18\\
54	28.18\\
55	28.18\\
56	28.18\\
57	28.18\\
58	28.18\\
59	28.18\\
60	28.18\\
61	28.18\\
62	28.18\\
63	28.18\\
64	28.18\\
65	28.18\\
66	28.18\\
67	28.18\\
68	28.18\\
69	28.18\\
70	28.18\\
71	28.18\\
72	28.18\\
73	28.18\\
74	28.18\\
75	28.18\\
76	28.18\\
77	28.18\\
78	28.18\\
79	28.18\\
80	28.18\\
81	28.18\\
82	28.18\\
83	28.18\\
84	28.18\\
85	28.18\\
86	28.18\\
87	28.18\\
88	28.18\\
89	28.18\\
90	28.18\\
91	28.18\\
92	28.18\\
93	28.18\\
94	28.18\\
95	28.18\\
96	28.18\\
97	28.18\\
98	28.18\\
99	28.18\\
100	28.18\\
101	28.18\\
102	28.18\\
103	28.18\\
104	28.18\\
105	28.18\\
106	28.18\\
107	28.18\\
108	28.18\\
109	28.18\\
110	28.18\\
111	28.18\\
112	28.18\\
113	28.18\\
114	28.18\\
115	28.18\\
116	28.18\\
117	28.18\\
118	28.18\\
119	28.18\\
120	28.18\\
121	28.18\\
122	28.18\\
123	28.18\\
124	28.18\\
125	28.18\\
126	28.18\\
127	28.18\\
128	28.18\\
129	28.18\\
130	28.18\\
131	28.18\\
132	28.18\\
133	28.18\\
134	28.18\\
135	28.18\\
136	28.18\\
137	28.18\\
138	28.18\\
139	28.18\\
140	28.18\\
141	28.18\\
142	28.18\\
143	28.18\\
144	28.18\\
145	28.18\\
146	28.18\\
147	28.18\\
148	28.18\\
149	28.18\\
150	28.18\\
151	28.18\\
152	28.18\\
153	28.18\\
154	28.18\\
155	28.18\\
156	28.18\\
157	28.18\\
158	28.18\\
159	28.18\\
160	28.18\\
161	28.18\\
162	28.18\\
163	28.18\\
164	28.18\\
165	28.18\\
166	28.18\\
167	28.18\\
168	28.18\\
169	28.18\\
170	28.18\\
171	28.18\\
172	28.18\\
173	28.18\\
174	28.18\\
175	28.18\\
176	28.18\\
177	28.18\\
178	28.18\\
179	28.18\\
180	28.18\\
181	28.18\\
182	28.18\\
183	28.18\\
184	28.18\\
185	28.18\\
186	28.18\\
187	28.18\\
188	28.18\\
189	28.18\\
190	28.18\\
191	28.18\\
192	28.18\\
193	28.18\\
194	28.18\\
195	28.18\\
196	28.18\\
197	28.18\\
198	28.18\\
199	28.18\\
200	28.18\\
201	35\\
202	35\\
203	35\\
204	35\\
205	35\\
206	35\\
207	35\\
208	35\\
209	35\\
210	35\\
211	35\\
212	35\\
213	35\\
214	35\\
215	35\\
216	35\\
217	35\\
218	35\\
219	35\\
220	35\\
221	35\\
222	35\\
223	35\\
224	35\\
225	35\\
226	35\\
227	35\\
228	35\\
229	35\\
230	35\\
231	35\\
232	35\\
233	35\\
234	35\\
235	35\\
236	35\\
237	35\\
238	35\\
239	35\\
240	35\\
241	35\\
242	35\\
243	35\\
244	35\\
245	35\\
246	35\\
247	35\\
248	35\\
249	35\\
250	35\\
251	35\\
252	35\\
253	35\\
254	35\\
255	35\\
256	35\\
257	35\\
258	35\\
259	35\\
260	35\\
261	35\\
262	35\\
263	35\\
264	35\\
265	35\\
266	35\\
267	35\\
268	35\\
269	35\\
270	35\\
271	35\\
272	35\\
273	35\\
274	35\\
275	35\\
276	35\\
277	35\\
278	35\\
279	35\\
280	35\\
281	35\\
282	35\\
283	35\\
284	35\\
285	35\\
286	35\\
287	35\\
288	35\\
289	35\\
290	35\\
291	35\\
292	35\\
293	35\\
294	35\\
295	35\\
296	35\\
297	35\\
298	35\\
299	35\\
300	35\\
301	35\\
302	35\\
303	35\\
304	35\\
305	35\\
306	35\\
307	35\\
308	35\\
309	35\\
310	35\\
311	35\\
312	35\\
313	35\\
314	35\\
315	35\\
316	35\\
317	35\\
318	35\\
319	35\\
320	35\\
321	35\\
322	35\\
323	35\\
324	35\\
325	35\\
326	35\\
327	35\\
328	35\\
329	35\\
330	35\\
331	35\\
332	35\\
333	35\\
334	35\\
335	35\\
336	35\\
337	35\\
338	35\\
339	35\\
340	35\\
341	35\\
342	35\\
343	35\\
344	35\\
345	35\\
346	35\\
347	35\\
348	35\\
349	35\\
350	35\\
351	35\\
352	35\\
353	35\\
354	35\\
355	35\\
356	35\\
357	35\\
358	35\\
359	35\\
360	35\\
361	35\\
362	35\\
363	35\\
364	35\\
365	35\\
366	35\\
367	35\\
368	35\\
369	35\\
370	35\\
371	35\\
372	35\\
373	35\\
374	35\\
375	35\\
376	35\\
377	35\\
378	35\\
379	35\\
380	35\\
381	35\\
382	35\\
383	35\\
384	35\\
385	35\\
386	35\\
387	35\\
388	35\\
389	35\\
390	35\\
391	35\\
392	35\\
393	35\\
394	35\\
395	35\\
396	35\\
397	35\\
398	35\\
399	35\\
400	35\\
401	35\\
402	35\\
403	35\\
404	35\\
405	35\\
406	35\\
407	35\\
408	35\\
409	35\\
410	35\\
411	35\\
412	35\\
413	35\\
414	35\\
415	35\\
416	35\\
417	35\\
418	35\\
419	35\\
420	35\\
421	35\\
422	35\\
423	35\\
424	35\\
425	35\\
426	35\\
427	35\\
428	35\\
429	35\\
430	35\\
431	35\\
432	35\\
433	35\\
434	35\\
435	35\\
436	35\\
437	35\\
438	35\\
439	35\\
440	35\\
441	35\\
442	35\\
443	35\\
444	35\\
445	35\\
446	35\\
447	35\\
448	35\\
449	35\\
450	35\\
451	35\\
452	35\\
453	35\\
454	35\\
455	35\\
456	35\\
457	35\\
458	35\\
459	35\\
460	35\\
461	35\\
462	35\\
463	35\\
464	35\\
465	35\\
466	35\\
467	35\\
468	35\\
469	35\\
470	35\\
471	35\\
472	35\\
473	35\\
474	35\\
475	35\\
476	35\\
477	35\\
478	35\\
479	35\\
480	35\\
481	35\\
482	35\\
483	35\\
484	35\\
485	35\\
486	35\\
487	35\\
488	35\\
489	35\\
490	35\\
491	35\\
492	35\\
493	35\\
494	35\\
495	35\\
496	35\\
497	35\\
498	35\\
499	35\\
500	35\\
501	35\\
502	35\\
503	35\\
504	35\\
505	35\\
506	35\\
507	35\\
508	35\\
509	35\\
510	35\\
511	35\\
512	35\\
513	35\\
514	35\\
515	35\\
516	35\\
517	35\\
518	35\\
519	35\\
520	35\\
521	35\\
522	35\\
523	35\\
524	35\\
525	35\\
526	35\\
527	35\\
528	35\\
529	35\\
530	35\\
531	35\\
532	35\\
533	35\\
534	35\\
535	35\\
536	35\\
537	35\\
538	35\\
539	35\\
540	35\\
541	35\\
542	35\\
543	35\\
544	35\\
545	35\\
546	35\\
547	35\\
548	35\\
549	35\\
550	35\\
551	35\\
552	35\\
553	35\\
554	35\\
555	35\\
556	35\\
557	35\\
558	35\\
559	35\\
560	35\\
561	35\\
562	35\\
563	35\\
564	35\\
565	35\\
566	35\\
567	35\\
568	35\\
569	35\\
570	35\\
571	35\\
572	35\\
573	35\\
574	35\\
575	35\\
576	35\\
577	35\\
578	35\\
579	35\\
580	35\\
581	35\\
582	35\\
583	35\\
584	35\\
585	35\\
586	35\\
587	35\\
588	35\\
589	35\\
590	35\\
591	35\\
592	35\\
593	35\\
594	35\\
595	35\\
596	35\\
597	35\\
598	35\\
599	35\\
600	35\\
};
\end{axis}
\end{tikzpicture}%
\caption{Wyjście procesu z regulatorem DMC dla parametrów $D = 340$, $ N = 60$, $N_u = 5$,  $\lambda = 2$}
\end{figure}

\begin{figure}[H]
\centering
% This file was created by matlab2tikz.
%
%The latest updates can be retrieved from
%  http://www.mathworks.com/matlabcentral/fileexchange/22022-matlab2tikz-matlab2tikz
%where you can also make suggestions and rate matlab2tikz.
%
\definecolor{mycolor1}{rgb}{0.00000,0.44700,0.74100}%
%
\begin{tikzpicture}

\begin{axis}[%
width=4.521in,
height=3.566in,
at={(0.758in,0.481in)},
scale only axis,
xmin=0,
xmax=600,
xlabel style={font=\color{white!15!black}},
xlabel={k},
ymin=20,
ymax=80,
ylabel style={font=\color{white!15!black}},
ylabel={\%},
axis background/.style={fill=white}
]
\addplot[const plot, color=mycolor1, forget plot] table[row sep=crcr] {%
1	24.864342\\
2	24.745314\\
3	24.641328\\
4	24.550931\\
5	24.410182\\
6	24.350763\\
7	24.300545\\
8	24.258575\\
9	24.223983\\
10	24.19598\\
11	24.173845\\
12	24.156923\\
13	24.144619\\
14	24.136057\\
15	24.130336\\
16	24.126637\\
17	24.124261\\
18	24.122482\\
19	24.120808\\
20	24.118909\\
21	24.116561\\
22	24.176232\\
23	24.227557\\
24	24.271233\\
25	24.307937\\
26	24.338314\\
27	24.362978\\
28	24.382509\\
29	24.397447\\
30	24.408298\\
31	24.415523\\
32	24.419549\\
33	24.420761\\
34	24.419506\\
35	24.416254\\
36	24.411474\\
37	24.405591\\
38	24.398965\\
39	24.391888\\
40	24.384595\\
41	24.377265\\
42	24.37003\\
43	24.362985\\
44	24.356191\\
45	24.349683\\
46	24.343476\\
47	24.337565\\
48	24.331937\\
49	24.326568\\
50	24.321427\\
51	24.316484\\
52	24.311705\\
53	24.307061\\
54	24.30252\\
55	24.298058\\
56	24.293652\\
57	24.226671\\
58	24.167385\\
59	24.115049\\
60	24.068983\\
61	24.028567\\
62	23.993233\\
63	23.962467\\
64	23.935798\\
65	23.9128\\
66	23.893084\\
67	23.876295\\
68	23.862114\\
69	23.850248\\
70	23.777666\\
71	23.714225\\
72	23.658801\\
73	23.61039\\
74	23.568103\\
75	23.468548\\
76	23.381327\\
77	23.305112\\
78	23.238712\\
79	23.181063\\
80	23.13121\\
81	23.088301\\
82	23.051572\\
83	23.020185\\
84	22.993337\\
85	23.032918\\
86	23.068009\\
87	23.098803\\
88	23.125354\\
89	23.147699\\
90	23.165901\\
91	23.180065\\
92	23.190337\\
93	23.196901\\
94	23.199969\\
95	23.262384\\
96	23.314108\\
97	23.356129\\
98	23.389527\\
99	23.477973\\
100	23.55216\\
101	23.613708\\
102	23.664077\\
103	23.777623\\
104	23.873515\\
105	23.953649\\
106	24.082344\\
107	24.19085\\
108	24.281434\\
109	24.356198\\
110	24.41704\\
111	24.465655\\
112	24.566314\\
113	24.650251\\
114	24.719543\\
115	24.776034\\
116	24.821527\\
117	24.857667\\
118	24.885903\\
119	24.970252\\
120	25.041683\\
121	25.039392\\
122	25.097843\\
123	25.14736\\
124	25.189154\\
125	25.224415\\
126	25.254217\\
127	25.279487\\
128	25.374053\\
129	25.383515\\
130	25.391258\\
131	25.397644\\
132	25.40311\\
133	25.481115\\
134	25.549846\\
135	25.537347\\
136	25.599665\\
137	25.654832\\
138	25.630655\\
139	25.609882\\
140	25.592315\\
141	25.650971\\
142	25.703675\\
143	25.751048\\
144	25.793618\\
145	25.758804\\
146	25.7292\\
147	25.704545\\
148	25.684417\\
149	25.668528\\
150	25.65664\\
151	25.648342\\
152	25.643184\\
153	25.578116\\
154	25.523204\\
155	25.477581\\
156	25.44047\\
157	25.411149\\
158	25.388762\\
159	25.372449\\
160	25.361407\\
161	25.417523\\
162	25.407256\\
163	25.401059\\
164	25.39838\\
165	25.398738\\
166	25.401556\\
167	25.40628\\
168	25.412419\\
169	25.419563\\
170	25.427378\\
171	25.435605\\
172	25.444044\\
173	25.452549\\
174	25.461173\\
175	25.469882\\
176	25.478618\\
177	25.487329\\
178	25.558584\\
179	25.559458\\
180	25.623567\\
181	25.680558\\
182	25.731111\\
183	25.713239\\
184	25.760418\\
185	25.739545\\
186	25.721451\\
187	25.705903\\
188	25.755299\\
189	25.736542\\
190	25.720469\\
191	25.707005\\
192	25.695974\\
193	25.74993\\
194	25.79856\\
195	25.842562\\
196	25.8824\\
197	25.918572\\
198	25.951403\\
199	25.981135\\
200	26.007971\\
201	33.149073\\
202	39.415434\\
203	44.890475\\
204	49.650565\\
205	53.765581\\
206	57.299588\\
207	60.311182\\
208	62.853885\\
209	64.976515\\
210	66.723564\\
211	68.072926\\
212	69.131731\\
213	69.870113\\
214	70.405723\\
215	70.723765\\
216	70.864912\\
217	70.884909\\
218	70.75773\\
219	70.572976\\
220	70.36528\\
221	70.089602\\
222	69.756335\\
223	69.393551\\
224	69.065756\\
225	68.660932\\
226	68.245957\\
227	67.899308\\
228	67.542584\\
229	67.126611\\
230	66.774442\\
231	66.363918\\
232	65.954091\\
233	65.554947\\
234	65.154063\\
235	64.761091\\
236	64.363424\\
237	63.970851\\
238	63.50831\\
239	63.045476\\
240	62.591667\\
241	62.134235\\
242	61.6828\\
243	61.225055\\
244	60.708354\\
245	60.19085\\
246	59.682031\\
247	59.169379\\
248	58.662772\\
249	58.150027\\
250	57.641337\\
251	57.124638\\
252	56.610126\\
253	56.14852\\
254	55.679956\\
255	55.193494\\
256	54.763047\\
257	54.317729\\
258	53.909883\\
259	53.543755\\
260	53.1499\\
261	52.782061\\
262	52.445748\\
263	52.135309\\
264	51.845676\\
265	51.561905\\
266	51.291836\\
267	51.031965\\
268	50.706117\\
269	50.456283\\
270	50.209038\\
271	49.962434\\
272	49.714933\\
273	49.527819\\
274	49.319289\\
275	49.163591\\
276	48.990416\\
277	48.863244\\
278	48.712688\\
279	48.603017\\
280	48.465547\\
281	48.364978\\
282	48.222655\\
283	48.115611\\
284	48.038566\\
285	47.986802\\
286	47.956252\\
287	47.94329\\
288	47.944819\\
289	47.958103\\
290	47.918215\\
291	47.893386\\
292	47.88116\\
293	47.879318\\
294	47.885962\\
295	47.899341\\
296	47.917943\\
297	47.940544\\
298	47.903547\\
299	47.939093\\
300	47.975587\\
301	47.950049\\
302	47.932508\\
303	47.859297\\
304	47.862358\\
305	47.86983\\
306	47.881061\\
307	47.895506\\
308	47.912707\\
309	47.994897\\
310	48.008862\\
311	48.025196\\
312	48.043575\\
313	48.126379\\
314	48.202982\\
315	48.273779\\
316	48.33898\\
317	48.39891\\
318	48.516547\\
319	48.621993\\
320	48.716368\\
321	48.800722\\
322	48.94923\\
323	49.007783\\
324	49.060055\\
325	49.179762\\
326	49.285655\\
327	49.379328\\
328	49.462243\\
329	49.535719\\
330	49.600931\\
331	49.659074\\
332	49.711262\\
333	49.758479\\
334	49.801575\\
335	49.84145\\
336	49.878756\\
337	49.840934\\
338	49.810543\\
339	49.787243\\
340	49.770694\\
341	49.76053\\
342	49.693753\\
343	49.702864\\
344	49.653827\\
345	49.616568\\
346	49.590003\\
347	49.573107\\
348	49.564918\\
349	49.564539\\
350	49.508355\\
351	49.465662\\
352	49.434824\\
353	49.414363\\
354	49.40296\\
355	49.399295\\
356	49.339696\\
357	49.293411\\
358	49.18572\\
359	49.097131\\
360	49.025473\\
361	48.968818\\
362	48.925452\\
363	48.893695\\
364	48.872\\
365	48.796383\\
366	48.798539\\
367	48.743805\\
368	48.764237\\
369	48.725331\\
370	48.759246\\
371	48.794148\\
372	48.829339\\
373	48.864247\\
374	48.898421\\
375	48.93153\\
376	48.900736\\
377	48.9388\\
378	48.911868\\
379	48.89025\\
380	48.873162\\
381	48.797451\\
382	48.732828\\
383	48.678191\\
384	48.632609\\
385	48.595268\\
386	48.502814\\
387	48.424871\\
388	48.360087\\
389	48.369677\\
390	48.382182\\
391	48.397149\\
392	48.414124\\
393	48.4327\\
394	48.452361\\
395	48.472612\\
396	48.493015\\
397	48.450607\\
398	48.415408\\
399	48.386333\\
400	48.362389\\
401	48.342697\\
402	48.326676\\
403	48.313903\\
404	48.304044\\
405	48.223786\\
406	48.154902\\
407	48.096333\\
408	48.047122\\
409	48.006388\\
410	47.973176\\
411	47.946558\\
412	47.925698\\
413	47.909835\\
414	47.898302\\
415	47.890508\\
416	47.885947\\
417	47.884173\\
418	47.884616\\
419	47.886716\\
420	47.889981\\
421	47.893999\\
422	47.898434\\
423	47.84042\\
424	47.852662\\
425	47.863997\\
426	47.874387\\
427	47.883832\\
428	47.829739\\
429	47.782435\\
430	47.741245\\
431	47.70556\\
432	47.674844\\
433	47.711216\\
434	47.743955\\
435	47.773392\\
436	47.799672\\
437	47.89609\\
438	47.980912\\
439	47.982222\\
440	47.983051\\
441	47.98334\\
442	47.982973\\
443	47.981848\\
444	47.979883\\
445	47.977027\\
446	47.973409\\
447	47.96922\\
448	47.964664\\
449	47.959919\\
450	47.955322\\
451	47.951221\\
452	47.947737\\
453	48.017936\\
454	48.079775\\
455	48.061012\\
456	48.04437\\
457	48.029644\\
458	48.016648\\
459	47.942589\\
460	47.877587\\
461	47.820747\\
462	47.708654\\
463	47.610851\\
464	47.525929\\
465	47.452599\\
466	47.389869\\
467	47.336897\\
468	47.292736\\
469	47.193821\\
470	47.109556\\
471	47.038391\\
472	46.978762\\
473	46.929197\\
474	46.88835\\
475	46.781829\\
476	46.690405\\
477	46.612132\\
478	46.54529\\
479	46.488379\\
480	46.44009\\
481	46.399289\\
482	46.364861\\
483	46.3358\\
484	46.311248\\
485	46.290497\\
486	46.272971\\
487	46.331259\\
488	46.309765\\
489	46.363878\\
490	46.411094\\
491	46.45186\\
492	46.486643\\
493	46.515924\\
494	46.540188\\
495	46.559912\\
496	46.575559\\
497	46.587577\\
498	46.596394\\
499	46.602407\\
500	46.606172\\
501	46.608076\\
502	46.608591\\
503	46.608185\\
504	46.607276\\
505	46.606217\\
506	46.605285\\
507	46.604696\\
508	46.604613\\
509	46.667762\\
510	46.723933\\
511	46.77389\\
512	46.818319\\
513	46.857819\\
514	46.955538\\
515	46.979046\\
516	46.999779\\
517	47.018051\\
518	47.034134\\
519	46.985652\\
520	46.943116\\
521	46.905977\\
522	46.873895\\
523	46.846595\\
524	46.823835\\
525	46.805365\\
526	46.85354\\
527	46.897963\\
528	46.876515\\
529	46.859855\\
530	46.847553\\
531	46.901793\\
532	46.951721\\
533	46.997495\\
534	47.039232\\
535	47.077067\\
536	47.11114\\
537	47.204222\\
538	47.286199\\
539	47.295529\\
540	47.36617\\
541	47.427934\\
542	47.480907\\
543	47.525277\\
544	47.561485\\
545	47.59009\\
546	47.611697\\
547	47.626926\\
548	47.636387\\
549	47.640671\\
550	47.577886\\
551	47.519046\\
552	47.526554\\
553	47.530032\\
554	47.530159\\
555	47.527548\\
556	47.52274\\
557	47.516197\\
558	47.508311\\
559	47.499407\\
560	47.48975\\
561	47.479546\\
562	47.468957\\
563	47.457952\\
564	47.509048\\
565	47.552001\\
566	47.587645\\
567	47.616756\\
568	47.64004\\
569	47.658135\\
570	47.671621\\
571	47.681013\\
572	47.686781\\
573	47.689347\\
574	47.68909\\
575	47.623741\\
576	47.626506\\
577	47.564187\\
578	47.507571\\
579	47.456384\\
580	47.410353\\
581	47.306582\\
582	47.215066\\
583	47.134773\\
584	47.064734\\
585	47.00404\\
586	46.951844\\
587	46.907347\\
588	46.807043\\
589	46.720491\\
590	46.646181\\
591	46.582694\\
592	46.528742\\
593	46.483164\\
594	46.444767\\
595	46.412444\\
596	46.385216\\
597	46.362229\\
598	46.342762\\
599	46.326206\\
600	46.312051\\
};
\end{axis}
\end{tikzpicture}%
\caption{Sterowanie procesu z regulatorem DMC dla parametrów $D = 340$, $ N = 60$, $N_u = 5$,  $\lambda = 2$}
\end{figure}

\begin{equation}
E = 1,6698 * 10^3
\end{equation}

\begin{figure}[H]
\centering
% This file was created by matlab2tikz.
%
%The latest updates can be retrieved from
%  http://www.mathworks.com/matlabcentral/fileexchange/22022-matlab2tikz-matlab2tikz
%where you can also make suggestions and rate matlab2tikz.
%
\definecolor{mycolor1}{rgb}{0.00000,0.44700,0.74100}%
\definecolor{mycolor2}{rgb}{0.85000,0.32500,0.09800}%
%
\begin{tikzpicture}

\begin{axis}[%
width=4.521in,
height=3.566in,
at={(0.758in,0.481in)},
scale only axis,
xmin=0,
xmax=600,
xlabel style={font=\color{white!15!black}},
xlabel={k},
ymin=28,
ymax=36,
ylabel style={font=\color{white!15!black}},
ylabel={$\text{T[}^\circ\text{C]}$},
axis background/.style={fill=white}
]
\addplot[const plot, color=mycolor1, forget plot] table[row sep=crcr] {%
1	28.12\\
2	28.12\\
3	28.12\\
4	28.12\\
5	28.06\\
6	28.06\\
7	28.06\\
8	28.06\\
9	28.06\\
10	28.06\\
11	28\\
12	28.06\\
13	28\\
14	28.06\\
15	28.06\\
16	28.06\\
17	28\\
18	28\\
19	28.06\\
20	28.06\\
21	28.06\\
22	28.06\\
23	28.06\\
24	28.06\\
25	28.12\\
26	28.12\\
27	28.12\\
28	28.12\\
29	28.12\\
30	28.12\\
31	28.12\\
32	28.18\\
33	28.18\\
34	28.18\\
35	28.18\\
36	28.18\\
37	28.18\\
38	28.18\\
39	28.18\\
40	28.18\\
41	28.18\\
42	28.18\\
43	28.18\\
44	28.18\\
45	28.25\\
46	28.18\\
47	28.25\\
48	28.25\\
49	28.25\\
50	28.25\\
51	28.25\\
52	28.25\\
53	28.25\\
54	28.25\\
55	28.25\\
56	28.31\\
57	28.25\\
58	28.25\\
59	28.25\\
60	28.25\\
61	28.25\\
62	28.25\\
63	28.31\\
64	28.25\\
65	28.31\\
66	28.31\\
67	28.31\\
68	28.31\\
69	28.37\\
70	28.37\\
71	28.37\\
72	28.37\\
73	28.43\\
74	28.37\\
75	28.37\\
76	28.37\\
77	28.37\\
78	28.37\\
79	28.37\\
80	28.37\\
81	28.37\\
82	28.37\\
83	28.37\\
84	28.37\\
85	28.31\\
86	28.31\\
87	28.31\\
88	28.31\\
89	28.31\\
90	28.31\\
91	28.31\\
92	28.31\\
93	28.31\\
94	28.31\\
95	28.31\\
96	28.31\\
97	28.31\\
98	28.31\\
99	28.31\\
100	28.31\\
101	28.31\\
102	28.31\\
103	28.31\\
104	28.31\\
105	28.31\\
106	28.31\\
107	28.31\\
108	28.25\\
109	28.31\\
110	28.25\\
111	28.25\\
112	28.25\\
113	28.25\\
114	28.31\\
115	28.25\\
116	28.25\\
117	28.25\\
118	28.25\\
119	28.31\\
120	28.25\\
121	28.31\\
122	28.31\\
123	28.25\\
124	28.25\\
125	28.25\\
126	28.25\\
127	28.25\\
128	28.25\\
129	28.18\\
130	28.18\\
131	28.18\\
132	28.18\\
133	28.18\\
134	28.18\\
135	28.18\\
136	28.18\\
137	28.18\\
138	28.18\\
139	28.18\\
140	28.18\\
141	28.18\\
142	28.18\\
143	28.18\\
144	28.18\\
145	28.18\\
146	28.12\\
147	28.18\\
148	28.18\\
149	28.18\\
150	28.18\\
151	28.18\\
152	28.18\\
153	28.18\\
154	28.18\\
155	28.18\\
156	28.18\\
157	28.18\\
158	28.25\\
159	28.25\\
160	28.25\\
161	28.25\\
162	28.25\\
163	28.25\\
164	28.25\\
165	28.25\\
166	28.25\\
167	28.25\\
168	28.25\\
169	28.25\\
170	28.25\\
171	28.18\\
172	28.25\\
173	28.25\\
174	28.18\\
175	28.18\\
176	28.18\\
177	28.18\\
178	28.18\\
179	28.18\\
180	28.18\\
181	28.18\\
182	28.18\\
183	28.18\\
184	28.18\\
185	28.12\\
186	28.18\\
187	28.12\\
188	28.12\\
189	28.12\\
190	28.12\\
191	28.12\\
192	28.12\\
193	28.12\\
194	28.12\\
195	28.12\\
196	28.12\\
197	28.12\\
198	28.12\\
199	28.12\\
200	28.12\\
201	28.12\\
202	28.12\\
203	28.12\\
204	28.18\\
205	28.12\\
206	28.18\\
207	28.18\\
208	28.18\\
209	28.18\\
210	28.18\\
211	28.18\\
212	28.25\\
213	28.31\\
214	28.37\\
215	28.37\\
216	28.43\\
217	28.56\\
218	28.62\\
219	28.75\\
220	28.87\\
221	29\\
222	29.12\\
223	29.25\\
224	29.37\\
225	29.5\\
226	29.62\\
227	29.75\\
228	29.87\\
229	30\\
230	30.12\\
231	30.31\\
232	30.43\\
233	30.56\\
234	30.75\\
235	30.87\\
236	31\\
237	31.12\\
238	31.31\\
239	31.43\\
240	31.56\\
241	31.62\\
242	31.81\\
243	31.93\\
244	32.06\\
245	32.18\\
246	32.31\\
247	32.43\\
248	32.56\\
249	32.68\\
250	32.75\\
251	32.87\\
252	33\\
253	33.06\\
254	33.18\\
255	33.25\\
256	33.31\\
257	33.43\\
258	33.5\\
259	33.62\\
260	33.68\\
261	33.75\\
262	33.75\\
263	33.81\\
264	33.81\\
265	33.87\\
266	33.93\\
267	33.93\\
268	34\\
269	34.06\\
270	34.12\\
271	34.12\\
272	34.18\\
273	34.25\\
274	34.31\\
275	34.37\\
276	34.37\\
277	34.43\\
278	34.5\\
279	34.5\\
280	34.5\\
281	34.56\\
282	34.56\\
283	34.56\\
284	34.56\\
285	34.62\\
286	34.62\\
287	34.62\\
288	34.62\\
289	34.62\\
290	34.62\\
291	34.62\\
292	34.68\\
293	34.68\\
294	34.68\\
295	34.68\\
296	34.68\\
297	34.68\\
298	34.68\\
299	34.68\\
300	34.75\\
301	34.75\\
302	34.75\\
303	34.75\\
304	34.75\\
305	34.75\\
306	34.75\\
307	34.75\\
308	34.75\\
309	34.68\\
310	34.68\\
311	34.68\\
312	34.68\\
313	34.68\\
314	34.68\\
315	34.62\\
316	34.62\\
317	34.62\\
318	34.62\\
319	34.56\\
320	34.56\\
321	34.56\\
322	34.5\\
323	34.5\\
324	34.5\\
325	34.5\\
326	34.5\\
327	34.5\\
328	34.43\\
329	34.43\\
330	34.43\\
331	34.43\\
332	34.43\\
333	34.5\\
334	34.5\\
335	34.5\\
336	34.5\\
337	34.5\\
338	34.56\\
339	34.5\\
340	34.56\\
341	34.56\\
342	34.56\\
343	34.56\\
344	34.62\\
345	34.62\\
346	34.62\\
347	34.62\\
348	34.62\\
349	34.62\\
350	34.62\\
351	34.68\\
352	34.62\\
353	34.68\\
354	34.68\\
355	34.68\\
356	34.68\\
357	34.75\\
358	34.75\\
359	34.75\\
360	34.81\\
361	34.81\\
362	34.81\\
363	34.81\\
364	34.87\\
365	34.87\\
366	34.87\\
367	34.87\\
368	34.87\\
369	34.93\\
370	34.93\\
371	34.93\\
372	34.93\\
373	34.93\\
374	34.93\\
375	34.93\\
376	34.93\\
377	35\\
378	35\\
379	35\\
380	35\\
381	35\\
382	35\\
383	35.06\\
384	35.06\\
385	35.06\\
386	35.06\\
387	35.12\\
388	35.12\\
389	35.12\\
390	35.12\\
391	35.12\\
392	35.12\\
393	35.12\\
394	35.12\\
395	35.12\\
396	35.12\\
397	35.06\\
398	35.06\\
399	35\\
400	35\\
401	35\\
402	35\\
403	34.93\\
404	34.93\\
405	34.87\\
406	34.87\\
407	34.81\\
408	34.81\\
409	34.81\\
410	34.81\\
411	34.81\\
412	34.81\\
413	34.81\\
414	34.87\\
415	34.87\\
416	34.87\\
417	34.93\\
418	34.93\\
419	35\\
420	35\\
421	35\\
422	35.06\\
423	35.06\\
424	35.06\\
425	35.06\\
426	35.06\\
427	35.06\\
428	35.06\\
429	35.06\\
430	35.06\\
431	35.06\\
432	35.06\\
433	35.06\\
434	35\\
435	35\\
436	35\\
437	35\\
438	35\\
439	35\\
440	35\\
441	35\\
442	35\\
443	35\\
444	35\\
445	34.93\\
446	34.93\\
447	34.93\\
448	34.93\\
449	34.87\\
450	34.87\\
451	34.87\\
452	34.87\\
453	34.87\\
454	34.81\\
455	34.81\\
456	34.81\\
457	34.81\\
458	34.81\\
459	34.81\\
460	34.81\\
461	34.81\\
462	34.81\\
463	34.81\\
464	34.81\\
465	34.81\\
466	34.81\\
467	34.81\\
468	34.81\\
469	34.81\\
470	34.81\\
471	34.81\\
472	34.87\\
473	34.87\\
474	34.93\\
475	34.93\\
476	35\\
477	35\\
478	35\\
479	35\\
480	35\\
481	35.06\\
482	35.06\\
483	35.06\\
484	35.06\\
485	35.06\\
486	35.06\\
487	35.06\\
488	35.06\\
489	35.12\\
490	35.12\\
491	35.12\\
492	35.12\\
493	35.12\\
494	35.12\\
495	35.12\\
496	35.12\\
497	35.12\\
498	35.12\\
499	35.12\\
500	35.12\\
501	35.12\\
502	35.18\\
503	35.18\\
504	35.18\\
505	35.18\\
506	35.18\\
507	35.18\\
508	35.18\\
509	35.18\\
510	35.18\\
511	35.18\\
512	35.18\\
513	35.18\\
514	35.18\\
515	35.25\\
516	35.18\\
517	35.18\\
518	35.12\\
519	35.12\\
520	35.12\\
521	35.06\\
522	35.06\\
523	35.06\\
524	35.06\\
525	35.06\\
526	35\\
527	35\\
528	35\\
529	34.93\\
530	34.93\\
531	34.93\\
532	34.87\\
533	34.87\\
534	34.87\\
535	34.87\\
536	34.87\\
537	34.87\\
538	34.87\\
539	34.93\\
540	34.93\\
541	35\\
542	35\\
543	35\\
544	35\\
545	35.06\\
546	35.06\\
547	35.06\\
548	35.12\\
549	35.12\\
550	35.12\\
551	35.12\\
552	35.12\\
553	35.12\\
554	35.12\\
555	35.12\\
556	35.18\\
557	35.18\\
558	35.18\\
559	35.25\\
560	35.25\\
561	35.25\\
562	35.25\\
563	35.25\\
564	35.25\\
565	35.25\\
566	35.25\\
567	35.25\\
568	35.25\\
569	35.25\\
570	35.25\\
571	35.25\\
572	35.25\\
573	35.25\\
574	35.25\\
575	35.25\\
576	35.25\\
577	35.25\\
578	35.25\\
579	35.25\\
580	35.25\\
581	35.25\\
582	35.25\\
583	35.31\\
584	35.31\\
585	35.31\\
586	35.25\\
587	35.31\\
588	35.31\\
589	35.31\\
590	35.31\\
591	35.31\\
592	35.31\\
593	35.31\\
594	35.31\\
595	35.37\\
596	35.37\\
597	35.37\\
598	35.37\\
599	35.37\\
600	35.37\\
};
\addplot[const plot, color=mycolor2, forget plot] table[row sep=crcr] {%
1	28.18\\
2	28.18\\
3	28.18\\
4	28.18\\
5	28.18\\
6	28.18\\
7	28.18\\
8	28.18\\
9	28.18\\
10	28.18\\
11	28.18\\
12	28.18\\
13	28.18\\
14	28.18\\
15	28.18\\
16	28.18\\
17	28.18\\
18	28.18\\
19	28.18\\
20	28.18\\
21	28.18\\
22	28.18\\
23	28.18\\
24	28.18\\
25	28.18\\
26	28.18\\
27	28.18\\
28	28.18\\
29	28.18\\
30	28.18\\
31	28.18\\
32	28.18\\
33	28.18\\
34	28.18\\
35	28.18\\
36	28.18\\
37	28.18\\
38	28.18\\
39	28.18\\
40	28.18\\
41	28.18\\
42	28.18\\
43	28.18\\
44	28.18\\
45	28.18\\
46	28.18\\
47	28.18\\
48	28.18\\
49	28.18\\
50	28.18\\
51	28.18\\
52	28.18\\
53	28.18\\
54	28.18\\
55	28.18\\
56	28.18\\
57	28.18\\
58	28.18\\
59	28.18\\
60	28.18\\
61	28.18\\
62	28.18\\
63	28.18\\
64	28.18\\
65	28.18\\
66	28.18\\
67	28.18\\
68	28.18\\
69	28.18\\
70	28.18\\
71	28.18\\
72	28.18\\
73	28.18\\
74	28.18\\
75	28.18\\
76	28.18\\
77	28.18\\
78	28.18\\
79	28.18\\
80	28.18\\
81	28.18\\
82	28.18\\
83	28.18\\
84	28.18\\
85	28.18\\
86	28.18\\
87	28.18\\
88	28.18\\
89	28.18\\
90	28.18\\
91	28.18\\
92	28.18\\
93	28.18\\
94	28.18\\
95	28.18\\
96	28.18\\
97	28.18\\
98	28.18\\
99	28.18\\
100	28.18\\
101	28.18\\
102	28.18\\
103	28.18\\
104	28.18\\
105	28.18\\
106	28.18\\
107	28.18\\
108	28.18\\
109	28.18\\
110	28.18\\
111	28.18\\
112	28.18\\
113	28.18\\
114	28.18\\
115	28.18\\
116	28.18\\
117	28.18\\
118	28.18\\
119	28.18\\
120	28.18\\
121	28.18\\
122	28.18\\
123	28.18\\
124	28.18\\
125	28.18\\
126	28.18\\
127	28.18\\
128	28.18\\
129	28.18\\
130	28.18\\
131	28.18\\
132	28.18\\
133	28.18\\
134	28.18\\
135	28.18\\
136	28.18\\
137	28.18\\
138	28.18\\
139	28.18\\
140	28.18\\
141	28.18\\
142	28.18\\
143	28.18\\
144	28.18\\
145	28.18\\
146	28.18\\
147	28.18\\
148	28.18\\
149	28.18\\
150	28.18\\
151	28.18\\
152	28.18\\
153	28.18\\
154	28.18\\
155	28.18\\
156	28.18\\
157	28.18\\
158	28.18\\
159	28.18\\
160	28.18\\
161	28.18\\
162	28.18\\
163	28.18\\
164	28.18\\
165	28.18\\
166	28.18\\
167	28.18\\
168	28.18\\
169	28.18\\
170	28.18\\
171	28.18\\
172	28.18\\
173	28.18\\
174	28.18\\
175	28.18\\
176	28.18\\
177	28.18\\
178	28.18\\
179	28.18\\
180	28.18\\
181	28.18\\
182	28.18\\
183	28.18\\
184	28.18\\
185	28.18\\
186	28.18\\
187	28.18\\
188	28.18\\
189	28.18\\
190	28.18\\
191	28.18\\
192	28.18\\
193	28.18\\
194	28.18\\
195	28.18\\
196	28.18\\
197	28.18\\
198	28.18\\
199	28.18\\
200	28.18\\
201	35\\
202	35\\
203	35\\
204	35\\
205	35\\
206	35\\
207	35\\
208	35\\
209	35\\
210	35\\
211	35\\
212	35\\
213	35\\
214	35\\
215	35\\
216	35\\
217	35\\
218	35\\
219	35\\
220	35\\
221	35\\
222	35\\
223	35\\
224	35\\
225	35\\
226	35\\
227	35\\
228	35\\
229	35\\
230	35\\
231	35\\
232	35\\
233	35\\
234	35\\
235	35\\
236	35\\
237	35\\
238	35\\
239	35\\
240	35\\
241	35\\
242	35\\
243	35\\
244	35\\
245	35\\
246	35\\
247	35\\
248	35\\
249	35\\
250	35\\
251	35\\
252	35\\
253	35\\
254	35\\
255	35\\
256	35\\
257	35\\
258	35\\
259	35\\
260	35\\
261	35\\
262	35\\
263	35\\
264	35\\
265	35\\
266	35\\
267	35\\
268	35\\
269	35\\
270	35\\
271	35\\
272	35\\
273	35\\
274	35\\
275	35\\
276	35\\
277	35\\
278	35\\
279	35\\
280	35\\
281	35\\
282	35\\
283	35\\
284	35\\
285	35\\
286	35\\
287	35\\
288	35\\
289	35\\
290	35\\
291	35\\
292	35\\
293	35\\
294	35\\
295	35\\
296	35\\
297	35\\
298	35\\
299	35\\
300	35\\
301	35\\
302	35\\
303	35\\
304	35\\
305	35\\
306	35\\
307	35\\
308	35\\
309	35\\
310	35\\
311	35\\
312	35\\
313	35\\
314	35\\
315	35\\
316	35\\
317	35\\
318	35\\
319	35\\
320	35\\
321	35\\
322	35\\
323	35\\
324	35\\
325	35\\
326	35\\
327	35\\
328	35\\
329	35\\
330	35\\
331	35\\
332	35\\
333	35\\
334	35\\
335	35\\
336	35\\
337	35\\
338	35\\
339	35\\
340	35\\
341	35\\
342	35\\
343	35\\
344	35\\
345	35\\
346	35\\
347	35\\
348	35\\
349	35\\
350	35\\
351	35\\
352	35\\
353	35\\
354	35\\
355	35\\
356	35\\
357	35\\
358	35\\
359	35\\
360	35\\
361	35\\
362	35\\
363	35\\
364	35\\
365	35\\
366	35\\
367	35\\
368	35\\
369	35\\
370	35\\
371	35\\
372	35\\
373	35\\
374	35\\
375	35\\
376	35\\
377	35\\
378	35\\
379	35\\
380	35\\
381	35\\
382	35\\
383	35\\
384	35\\
385	35\\
386	35\\
387	35\\
388	35\\
389	35\\
390	35\\
391	35\\
392	35\\
393	35\\
394	35\\
395	35\\
396	35\\
397	35\\
398	35\\
399	35\\
400	35\\
401	35\\
402	35\\
403	35\\
404	35\\
405	35\\
406	35\\
407	35\\
408	35\\
409	35\\
410	35\\
411	35\\
412	35\\
413	35\\
414	35\\
415	35\\
416	35\\
417	35\\
418	35\\
419	35\\
420	35\\
421	35\\
422	35\\
423	35\\
424	35\\
425	35\\
426	35\\
427	35\\
428	35\\
429	35\\
430	35\\
431	35\\
432	35\\
433	35\\
434	35\\
435	35\\
436	35\\
437	35\\
438	35\\
439	35\\
440	35\\
441	35\\
442	35\\
443	35\\
444	35\\
445	35\\
446	35\\
447	35\\
448	35\\
449	35\\
450	35\\
451	35\\
452	35\\
453	35\\
454	35\\
455	35\\
456	35\\
457	35\\
458	35\\
459	35\\
460	35\\
461	35\\
462	35\\
463	35\\
464	35\\
465	35\\
466	35\\
467	35\\
468	35\\
469	35\\
470	35\\
471	35\\
472	35\\
473	35\\
474	35\\
475	35\\
476	35\\
477	35\\
478	35\\
479	35\\
480	35\\
481	35\\
482	35\\
483	35\\
484	35\\
485	35\\
486	35\\
487	35\\
488	35\\
489	35\\
490	35\\
491	35\\
492	35\\
493	35\\
494	35\\
495	35\\
496	35\\
497	35\\
498	35\\
499	35\\
500	35\\
501	35\\
502	35\\
503	35\\
504	35\\
505	35\\
506	35\\
507	35\\
508	35\\
509	35\\
510	35\\
511	35\\
512	35\\
513	35\\
514	35\\
515	35\\
516	35\\
517	35\\
518	35\\
519	35\\
520	35\\
521	35\\
522	35\\
523	35\\
524	35\\
525	35\\
526	35\\
527	35\\
528	35\\
529	35\\
530	35\\
531	35\\
532	35\\
533	35\\
534	35\\
535	35\\
536	35\\
537	35\\
538	35\\
539	35\\
540	35\\
541	35\\
542	35\\
543	35\\
544	35\\
545	35\\
546	35\\
547	35\\
548	35\\
549	35\\
550	35\\
551	35\\
552	35\\
553	35\\
554	35\\
555	35\\
556	35\\
557	35\\
558	35\\
559	35\\
560	35\\
561	35\\
562	35\\
563	35\\
564	35\\
565	35\\
566	35\\
567	35\\
568	35\\
569	35\\
570	35\\
571	35\\
572	35\\
573	35\\
574	35\\
575	35\\
576	35\\
577	35\\
578	35\\
579	35\\
580	35\\
581	35\\
582	35\\
583	35\\
584	35\\
585	35\\
586	35\\
587	35\\
588	35\\
589	35\\
590	35\\
591	35\\
592	35\\
593	35\\
594	35\\
595	35\\
596	35\\
597	35\\
598	35\\
599	35\\
600	35\\
};
\end{axis}
\end{tikzpicture}%
\caption{Wyjście procesu z regulatorem DMC dla parametrów $D = 340$, $ N = 60$, $N_u = 5$,  $\lambda = 0,4$}
\end{figure}

\begin{figure}[H]
\centering
% This file was created by matlab2tikz.
%
%The latest updates can be retrieved from
%  http://www.mathworks.com/matlabcentral/fileexchange/22022-matlab2tikz-matlab2tikz
%where you can also make suggestions and rate matlab2tikz.
%
\definecolor{mycolor1}{rgb}{0.00000,0.44700,0.74100}%
%
\begin{tikzpicture}

\begin{axis}[%
width=4.521in,
height=3.566in,
at={(0.758in,0.481in)},
scale only axis,
xmin=0,
xmax=600,
xlabel style={font=\color{white!15!black}},
xlabel={k},
ymin=20,
ymax=80,
ylabel style={font=\color{white!15!black}},
ylabel={\%},
axis background/.style={fill=white}
]
\addplot[const plot, color=mycolor1, forget plot] table[row sep=crcr] {%
1	25.109392\\
2	25.197141\\
3	25.266871\\
4	25.32163\\
5	25.473365\\
6	25.593187\\
7	25.686522\\
8	25.757923\\
9	25.811209\\
10	25.849583\\
11	25.985118\\
12	25.97963\\
13	26.079048\\
14	26.044405\\
15	26.011686\\
16	25.981372\\
17	26.063147\\
18	26.126572\\
19	26.066376\\
20	26.016916\\
21	25.977167\\
22	25.946085\\
23	25.922631\\
24	25.906266\\
25	25.78667\\
26	25.694329\\
27	25.624665\\
28	25.573622\\
29	25.53771\\
30	25.514457\\
31	25.501915\\
32	25.388575\\
33	25.303531\\
34	25.241503\\
35	25.198103\\
36	25.169709\\
37	25.153352\\
38	25.14614\\
39	25.145601\\
40	25.14976\\
41	25.157112\\
42	25.166541\\
43	25.177255\\
44	25.188706\\
45	25.072442\\
46	25.108233\\
47	25.010847\\
48	24.933556\\
49	24.872611\\
50	24.824991\\
51	24.788265\\
52	24.760473\\
53	24.740033\\
54	24.725658\\
55	24.7163\\
56	24.601706\\
57	24.621596\\
58	24.640165\\
59	24.657456\\
60	24.673129\\
61	24.686825\\
62	24.698297\\
63	24.598044\\
64	24.626505\\
65	24.539733\\
66	24.469516\\
67	24.412933\\
68	24.367585\\
69	24.221641\\
70	24.104489\\
71	24.011302\\
72	23.938105\\
73	23.772196\\
74	23.751223\\
75	23.738186\\
76	23.731212\\
77	23.729075\\
78	23.730415\\
79	23.734021\\
80	23.738922\\
81	23.744386\\
82	23.749414\\
83	23.753141\\
84	23.754956\\
85	23.863897\\
86	23.948323\\
87	24.011861\\
88	24.057839\\
89	24.089209\\
90	24.108549\\
91	24.118072\\
92	24.11967\\
93	24.114944\\
94	24.105239\\
95	24.091676\\
96	24.075178\\
97	24.056499\\
98	24.036717\\
99	24.016819\\
100	23.997556\\
101	23.979432\\
102	23.962732\\
103	23.947566\\
104	23.933919\\
105	23.921684\\
106	23.910701\\
107	23.900777\\
108	24.001102\\
109	23.971042\\
110	24.054458\\
111	24.11956\\
112	24.169332\\
113	24.206287\\
114	24.123149\\
115	24.162133\\
116	24.190092\\
117	24.208888\\
118	24.220095\\
119	24.115653\\
120	24.137117\\
121	24.041858\\
122	23.962327\\
123	24.005802\\
124	24.039526\\
125	24.065624\\
126	24.085785\\
127	24.100857\\
128	24.111819\\
129	24.247193\\
130	24.354844\\
131	24.43958\\
132	24.504916\\
133	24.554116\\
134	24.589559\\
135	24.613153\\
136	24.627003\\
137	24.63306\\
138	24.633006\\
139	24.628238\\
140	24.619891\\
141	24.608872\\
142	24.596455\\
143	24.583798\\
144	24.571798\\
145	24.561061\\
146	24.661336\\
147	24.632349\\
148	24.608764\\
149	24.589891\\
150	24.575063\\
151	24.56365\\
152	24.555066\\
153	24.548782\\
154	24.544324\\
155	24.541276\\
156	24.539285\\
157	24.538058\\
158	24.409733\\
159	24.307473\\
160	24.226947\\
161	24.164377\\
162	24.116555\\
163	24.080791\\
164	24.054848\\
165	24.036877\\
166	24.025354\\
167	24.019024\\
168	24.016855\\
169	24.018\\
170	24.021765\\
171	24.154649\\
172	24.135094\\
173	24.119378\\
174	24.233878\\
175	24.324807\\
176	24.395758\\
177	24.449858\\
178	24.489825\\
179	24.518013\\
180	24.536455\\
181	24.546899\\
182	24.550846\\
183	24.549574\\
184	24.544724\\
185	24.646642\\
186	24.615515\\
187	24.696389\\
188	24.758579\\
189	24.805943\\
190	24.8416\\
191	24.868037\\
192	24.88722\\
193	24.900685\\
194	24.90963\\
195	24.914987\\
196	24.91747\\
197	24.917638\\
198	24.916394\\
199	24.914111\\
200	24.911405\\
201	37.343084\\
202	47.315152\\
203	55.240013\\
204	61.354588\\
205	66.189995\\
206	69.746052\\
207	72.358836\\
208	74.1965\\
209	75.399787\\
210	76.086379\\
211	76.354567\\
212	76.158707\\
213	75.610564\\
214	74.838189\\
215	74.070371\\
216	73.270794\\
217	72.382632\\
218	71.605321\\
219	70.833293\\
220	70.117528\\
221	69.454327\\
222	68.87013\\
223	68.343766\\
224	67.889725\\
225	67.478628\\
226	67.119358\\
227	66.779676\\
228	66.468659\\
229	66.15633\\
230	65.85426\\
231	65.426094\\
232	65.017774\\
233	64.601662\\
234	64.06473\\
235	63.553959\\
236	63.042643\\
237	62.546288\\
238	61.931961\\
239	61.348661\\
240	60.771075\\
241	60.324851\\
242	59.746231\\
243	59.187615\\
244	58.625703\\
245	58.078017\\
246	57.522051\\
247	56.97561\\
248	56.416397\\
249	55.862868\\
250	55.402918\\
251	54.924212\\
252	54.408917\\
253	53.988733\\
254	53.533086\\
255	53.136942\\
256	52.803753\\
257	52.408283\\
258	52.050323\\
259	51.628017\\
260	51.260062\\
261	50.914422\\
262	50.71096\\
263	50.509146\\
264	50.414323\\
265	50.292148\\
266	50.144006\\
267	50.080648\\
268	49.954342\\
269	49.792396\\
270	49.598808\\
271	49.486533\\
272	49.327423\\
273	49.109527\\
274	48.860178\\
275	48.584166\\
276	48.395257\\
277	48.166728\\
278	47.887748\\
279	47.695695\\
280	47.57381\\
281	47.398631\\
282	47.289301\\
283	47.232078\\
284	47.215768\\
285	47.121635\\
286	47.072703\\
287	47.058454\\
288	47.069949\\
289	47.100108\\
290	47.143062\\
291	47.193774\\
292	47.139026\\
293	47.107187\\
294	47.092477\\
295	47.090414\\
296	47.097644\\
297	47.1118\\
298	47.130728\\
299	47.152808\\
300	47.0493\\
301	46.972374\\
302	46.917391\\
303	46.880617\\
304	46.859043\\
305	46.849703\\
306	46.850039\\
307	46.857966\\
308	46.871797\\
309	47.017858\\
310	47.142254\\
311	47.248407\\
312	47.339253\\
313	47.416739\\
314	47.48239\\
315	47.646958\\
316	47.780661\\
317	47.888319\\
318	47.974082\\
319	48.150957\\
320	48.291016\\
321	48.40057\\
322	48.594891\\
323	48.748046\\
324	48.867913\\
325	48.961022\\
326	49.032692\\
327	49.087268\\
328	49.256377\\
329	49.390508\\
330	49.496563\\
331	49.580147\\
332	49.646266\\
333	49.571424\\
334	49.511778\\
335	49.466085\\
336	49.433233\\
337	49.412056\\
338	49.291936\\
339	49.312038\\
340	49.227095\\
341	49.168081\\
342	49.131296\\
343	49.113535\\
344	49.002558\\
345	48.926817\\
346	48.879828\\
347	48.855921\\
348	48.850299\\
349	48.859009\\
350	48.87886\\
351	48.797397\\
352	48.852948\\
353	48.799492\\
354	48.766998\\
355	48.751121\\
356	48.748383\\
357	48.627932\\
358	48.539949\\
359	48.478111\\
360	48.327931\\
361	48.216371\\
362	48.136527\\
363	48.082749\\
364	47.940508\\
365	47.836983\\
366	47.765063\\
367	47.718787\\
368	47.693198\\
369	47.574847\\
370	47.490934\\
371	47.43459\\
372	47.400128\\
373	47.38253\\
374	47.377621\\
375	47.382101\\
376	47.393433\\
377	47.281607\\
378	47.197579\\
379	47.135727\\
380	47.091528\\
381	47.061436\\
382	47.042149\\
383	46.921555\\
384	46.828577\\
385	46.758008\\
386	46.705648\\
387	46.558727\\
388	46.445554\\
389	46.360302\\
390	46.297606\\
391	46.252865\\
392	46.222287\\
393	46.20283\\
394	46.192067\\
395	46.188095\\
396	46.188929\\
397	46.302291\\
398	46.395834\\
399	46.581625\\
400	46.73055\\
401	46.847848\\
402	46.938099\\
403	47.132955\\
404	47.283115\\
405	47.505346\\
406	47.674934\\
407	47.910287\\
408	48.088012\\
409	48.218186\\
410	48.309791\\
411	48.370467\\
412	48.407075\\
413	48.425345\\
414	48.32053\\
415	48.227298\\
416	48.145383\\
417	47.965067\\
418	47.817338\\
419	47.570731\\
420	47.37508\\
421	47.223371\\
422	46.999964\\
423	46.830311\\
424	46.705768\\
425	46.618808\\
426	46.562908\\
427	46.531983\\
428	46.520703\\
429	46.524584\\
430	46.539467\\
431	46.561839\\
432	46.588394\\
433	46.61643\\
434	46.753386\\
435	46.866421\\
436	46.957869\\
437	47.030054\\
438	47.085246\\
439	47.125697\\
440	47.153559\\
441	47.170872\\
442	47.179483\\
443	47.181081\\
444	47.17718\\
445	47.29671\\
446	47.387897\\
447	47.456379\\
448	47.50699\\
449	47.653064\\
450	47.766764\\
451	47.854205\\
452	47.920378\\
453	47.969363\\
454	48.113814\\
455	48.225379\\
456	48.30983\\
457	48.371925\\
458	48.416183\\
459	48.446461\\
460	48.465959\\
461	48.477178\\
462	48.482596\\
463	48.484269\\
464	48.483747\\
465	48.482148\\
466	48.480228\\
467	48.47896\\
468	48.479131\\
469	48.481246\\
470	48.485538\\
471	48.492023\\
472	48.391185\\
473	48.313828\\
474	48.146635\\
475	48.017292\\
476	47.791701\\
477	47.61726\\
478	47.485139\\
479	47.387908\\
480	47.319348\\
481	47.164876\\
482	47.051262\\
483	46.97136\\
484	46.919245\\
485	46.889527\\
486	46.8776\\
487	46.87922\\
488	46.89084\\
489	46.799722\\
490	46.734186\\
491	46.688597\\
492	46.658526\\
493	46.640567\\
494	46.631607\\
495	46.629133\\
496	46.631247\\
497	46.636589\\
498	46.644256\\
499	46.653663\\
500	46.664479\\
501	46.676525\\
502	46.579849\\
503	46.504943\\
504	46.447804\\
505	46.405167\\
506	46.374382\\
507	46.35331\\
508	46.340241\\
509	46.333782\\
510	46.332814\\
511	46.336387\\
512	46.343741\\
513	46.354236\\
514	46.367384\\
515	46.254615\\
516	46.295546\\
517	46.332382\\
518	46.474694\\
519	46.591694\\
520	46.687346\\
521	46.874419\\
522	47.02482\\
523	47.144658\\
524	47.239146\\
525	47.312672\\
526	47.478321\\
527	47.608174\\
528	47.707905\\
529	47.910322\\
530	48.067064\\
531	48.186646\\
532	48.38572\\
533	48.539273\\
534	48.656305\\
535	48.744358\\
536	48.809555\\
537	48.856814\\
538	48.890042\\
539	48.803385\\
540	48.730816\\
541	48.54334\\
542	48.390915\\
543	48.2673\\
544	48.167457\\
545	47.978345\\
546	47.828065\\
547	47.710533\\
548	47.511171\\
549	47.356569\\
550	47.239295\\
551	47.153044\\
552	47.092005\\
553	47.051184\\
554	47.025881\\
555	47.012109\\
556	46.897254\\
557	46.809847\\
558	46.743905\\
559	46.56684\\
560	46.427575\\
561	46.318325\\
562	46.232673\\
563	46.165476\\
564	46.112689\\
565	46.07117\\
566	46.038501\\
567	46.012829\\
568	45.992742\\
569	45.976688\\
570	45.963331\\
571	45.951646\\
572	45.940373\\
573	45.928446\\
574	45.915128\\
575	45.900016\\
576	45.882954\\
577	45.863969\\
578	45.843217\\
579	45.820948\\
580	45.797447\\
581	45.773015\\
582	45.747921\\
583	45.613054\\
584	45.499698\\
585	45.404441\\
586	45.433802\\
587	45.344929\\
588	45.270431\\
589	45.207988\\
590	45.155634\\
591	45.111684\\
592	45.074712\\
593	45.043519\\
594	45.017071\\
595	44.885112\\
596	44.777465\\
597	44.689368\\
598	44.61688\\
599	44.557262\\
600	44.507981\\
};
\end{axis}
\end{tikzpicture}%
\caption{Sterowanie procesu z regulatorem DMC dla parametrów $D = 340$, $ N = 60$, $N_u = 5$,  $\lambda = 0,4$}
\end{figure}

\begin{equation}
E = 1,5041 * 10^3
\end{equation}

\begin{figure}[H]
\centering
% This file was created by matlab2tikz.
%
%The latest updates can be retrieved from
%  http://www.mathworks.com/matlabcentral/fileexchange/22022-matlab2tikz-matlab2tikz
%where you can also make suggestions and rate matlab2tikz.
%
\definecolor{mycolor1}{rgb}{0.00000,0.44700,0.74100}%
\definecolor{mycolor2}{rgb}{0.85000,0.32500,0.09800}%
%
\begin{tikzpicture}

\begin{axis}[%
width=4.521in,
height=3.566in,
at={(0.758in,0.481in)},
scale only axis,
xmin=0,
xmax=600,
xlabel style={font=\color{white!15!black}},
xlabel={k},
ymin=28,
ymax=36,
ylabel style={font=\color{white!15!black}},
ylabel={$\text{T[}^\circ\text{C]}$},
axis background/.style={fill=white}
]
\addplot[const plot, color=mycolor1, forget plot] table[row sep=crcr] {%
1	28.62\\
2	28.62\\
3	28.62\\
4	28.62\\
5	28.62\\
6	28.62\\
7	28.62\\
8	28.62\\
9	28.62\\
10	28.62\\
11	28.62\\
12	28.62\\
13	28.62\\
14	28.62\\
15	28.56\\
16	28.56\\
17	28.56\\
18	28.56\\
19	28.5\\
20	28.56\\
21	28.5\\
22	28.5\\
23	28.5\\
24	28.43\\
25	28.43\\
26	28.43\\
27	28.37\\
28	28.37\\
29	28.37\\
30	28.31\\
31	28.31\\
32	28.31\\
33	28.25\\
34	28.25\\
35	28.25\\
36	28.18\\
37	28.18\\
38	28.18\\
39	28.18\\
40	28.12\\
41	28.12\\
42	28.12\\
43	28.12\\
44	28.12\\
45	28.12\\
46	28.12\\
47	28.06\\
48	28.06\\
49	28.06\\
50	28.12\\
51	28.12\\
52	28.06\\
53	28.12\\
54	28.06\\
55	28.06\\
56	28.06\\
57	28.06\\
58	28.06\\
59	28.06\\
60	28.06\\
61	28.06\\
62	28.06\\
63	28.06\\
64	28.06\\
65	28.06\\
66	28.06\\
67	28.06\\
68	28.12\\
69	28.12\\
70	28.12\\
71	28.12\\
72	28.12\\
73	28.12\\
74	28.12\\
75	28.12\\
76	28.12\\
77	28.12\\
78	28.12\\
79	28.12\\
80	28.18\\
81	28.18\\
82	28.18\\
83	28.18\\
84	28.18\\
85	28.18\\
86	28.25\\
87	28.25\\
88	28.25\\
89	28.25\\
90	28.25\\
91	28.25\\
92	28.25\\
93	28.25\\
94	28.25\\
95	28.25\\
96	28.25\\
97	28.25\\
98	28.25\\
99	28.25\\
100	28.25\\
101	28.25\\
102	28.25\\
103	28.25\\
104	28.25\\
105	28.25\\
106	28.25\\
107	28.25\\
108	28.25\\
109	28.31\\
110	28.25\\
111	28.25\\
112	28.31\\
113	28.31\\
114	28.31\\
115	28.31\\
116	28.31\\
117	28.31\\
118	28.31\\
119	28.31\\
120	28.31\\
121	28.31\\
122	28.25\\
123	28.31\\
124	28.25\\
125	28.25\\
126	28.25\\
127	28.25\\
128	28.25\\
129	28.25\\
130	28.25\\
131	28.25\\
132	28.25\\
133	28.25\\
134	28.25\\
135	28.25\\
136	28.25\\
137	28.25\\
138	28.25\\
139	28.25\\
140	28.25\\
141	28.25\\
142	28.25\\
143	28.25\\
144	28.25\\
145	28.25\\
146	28.25\\
147	28.25\\
148	28.25\\
149	28.25\\
150	28.18\\
151	28.25\\
152	28.18\\
153	28.18\\
154	28.18\\
155	28.18\\
156	28.18\\
157	28.18\\
158	28.18\\
159	28.18\\
160	28.18\\
161	28.18\\
162	28.18\\
163	28.18\\
164	28.18\\
165	28.18\\
166	28.18\\
167	28.18\\
168	28.25\\
169	28.25\\
170	28.18\\
171	28.25\\
172	28.18\\
173	28.25\\
174	28.18\\
175	28.25\\
176	28.25\\
177	28.25\\
178	28.25\\
179	28.25\\
180	28.25\\
181	28.25\\
182	28.25\\
183	28.25\\
184	28.25\\
185	28.18\\
186	28.25\\
187	28.25\\
188	28.25\\
189	28.25\\
190	28.25\\
191	28.25\\
192	28.25\\
193	28.18\\
194	28.25\\
195	28.18\\
196	28.18\\
197	28.18\\
198	28.18\\
199	28.18\\
200	28.18\\
201	28.18\\
202	28.18\\
203	28.18\\
204	28.18\\
205	28.18\\
206	28.18\\
207	28.18\\
208	28.18\\
209	28.18\\
210	28.25\\
211	28.25\\
212	28.25\\
213	28.31\\
214	28.37\\
215	28.43\\
216	28.56\\
217	28.62\\
218	28.75\\
219	28.87\\
220	29\\
221	29.18\\
222	29.31\\
223	29.5\\
224	29.68\\
225	29.87\\
226	30.06\\
227	30.25\\
228	30.43\\
229	30.68\\
230	30.87\\
231	31.12\\
232	31.31\\
233	31.5\\
234	31.75\\
235	32\\
236	32.18\\
237	32.37\\
238	32.56\\
239	32.75\\
240	32.93\\
241	33.06\\
242	33.25\\
243	33.43\\
244	33.62\\
245	33.75\\
246	33.87\\
247	34.06\\
248	34.18\\
249	34.31\\
250	34.43\\
251	34.56\\
252	34.68\\
253	34.81\\
254	34.87\\
255	35\\
256	35.12\\
257	35.18\\
258	35.25\\
259	35.31\\
260	35.37\\
261	35.43\\
262	35.5\\
263	35.56\\
264	35.56\\
265	35.62\\
266	35.62\\
267	35.68\\
268	35.68\\
269	35.68\\
270	35.68\\
271	35.68\\
272	35.68\\
273	35.62\\
274	35.68\\
275	35.62\\
276	35.62\\
277	35.56\\
278	35.56\\
279	35.56\\
280	35.5\\
281	35.5\\
282	35.43\\
283	35.43\\
284	35.37\\
285	35.37\\
286	35.31\\
287	35.31\\
288	35.25\\
289	35.18\\
290	35.12\\
291	35.12\\
292	35.06\\
293	35.06\\
294	35\\
295	34.93\\
296	34.87\\
297	34.87\\
298	34.81\\
299	34.81\\
300	34.75\\
301	34.75\\
302	34.68\\
303	34.68\\
304	34.68\\
305	34.62\\
306	34.62\\
307	34.62\\
308	34.62\\
309	34.56\\
310	34.56\\
311	34.56\\
312	34.56\\
313	34.56\\
314	34.56\\
315	34.56\\
316	34.5\\
317	34.5\\
318	34.5\\
319	34.56\\
320	34.5\\
321	34.56\\
322	34.56\\
323	34.56\\
324	34.5\\
325	34.5\\
326	34.5\\
327	34.5\\
328	34.56\\
329	34.56\\
330	34.56\\
331	34.56\\
332	34.56\\
333	34.62\\
334	34.68\\
335	34.68\\
336	34.75\\
337	34.75\\
338	34.75\\
339	34.81\\
340	34.81\\
341	34.81\\
342	34.87\\
343	34.87\\
344	34.93\\
345	34.93\\
346	35\\
347	35\\
348	35\\
349	35\\
350	35.06\\
351	35.06\\
352	35.06\\
353	35.06\\
354	35.06\\
355	35.06\\
356	35.06\\
357	35.06\\
358	35.06\\
359	35.06\\
360	35.06\\
361	35.06\\
362	35.06\\
363	35.06\\
364	35.06\\
365	35.06\\
366	35\\
367	35\\
368	35.06\\
369	35.06\\
370	35\\
371	35.06\\
372	35.06\\
373	35.06\\
374	35.06\\
375	35.06\\
376	35.06\\
377	35.06\\
378	35.06\\
379	35.06\\
380	35.06\\
381	35.06\\
382	35.06\\
383	35.06\\
384	35.06\\
385	35.06\\
386	35.06\\
387	35.06\\
388	35.06\\
389	35.06\\
390	35.06\\
391	35.06\\
392	35.06\\
393	35.06\\
394	35.06\\
395	35.06\\
396	35.06\\
397	35.06\\
398	35.06\\
399	35.06\\
400	35.06\\
401	35.06\\
402	35.06\\
403	35.06\\
404	35.06\\
405	35.06\\
406	35.06\\
407	35.06\\
408	35.06\\
409	35.06\\
410	35.06\\
411	35.06\\
412	35.06\\
413	35.06\\
414	35.06\\
415	35.06\\
416	35.06\\
417	35.06\\
418	35.06\\
419	35\\
420	35.06\\
421	35.06\\
422	35\\
423	35\\
424	35.06\\
425	35.06\\
426	35.06\\
427	35.06\\
428	35.06\\
429	35.06\\
430	35.06\\
431	35.06\\
432	35\\
433	35\\
434	35\\
435	35\\
436	35\\
437	35\\
438	35\\
439	35\\
440	35\\
441	35\\
442	35\\
443	35\\
444	34.93\\
445	34.93\\
446	35\\
447	35\\
448	35\\
449	35\\
450	35\\
451	34.93\\
452	35\\
453	35\\
454	35\\
455	35\\
456	35\\
457	35\\
458	35\\
459	35\\
460	35\\
461	35\\
462	35\\
463	35\\
464	34.93\\
465	34.93\\
466	35\\
467	34.93\\
468	35\\
469	35\\
470	35\\
471	35\\
472	35\\
473	35\\
474	34.93\\
475	34.93\\
476	35\\
477	35\\
478	35\\
479	35\\
480	35\\
481	35\\
482	35.06\\
483	35.06\\
484	35.06\\
485	35.06\\
486	35.06\\
487	35.12\\
488	35.12\\
489	35.12\\
490	35.12\\
491	35.12\\
492	35.12\\
493	35.12\\
494	35.18\\
495	35.18\\
496	35.18\\
497	35.18\\
498	35.18\\
499	35.18\\
500	35.18\\
501	35.18\\
502	35.18\\
503	35.18\\
504	35.18\\
505	35.18\\
506	35.18\\
507	35.18\\
508	35.18\\
509	35.18\\
510	35.18\\
511	35.18\\
512	35.25\\
513	35.18\\
514	35.25\\
515	35.25\\
516	35.25\\
517	35.18\\
518	35.18\\
519	35.18\\
520	35.18\\
521	35.18\\
522	35.18\\
523	35.12\\
524	35.12\\
525	35.12\\
526	35.12\\
527	35.12\\
528	35.12\\
529	35.12\\
530	35.12\\
531	35.12\\
532	35.12\\
533	35.06\\
534	35.06\\
535	35.06\\
536	35\\
537	35.06\\
538	35.06\\
539	35.06\\
540	35.06\\
541	35.06\\
542	35\\
543	35\\
544	35.06\\
545	35\\
546	35\\
547	35\\
548	35\\
549	35\\
550	35\\
551	35\\
552	35\\
553	35\\
554	35\\
555	35\\
556	34.93\\
557	34.93\\
558	34.93\\
559	34.93\\
560	34.93\\
561	34.93\\
562	34.93\\
563	34.93\\
564	35\\
565	35\\
566	35\\
567	35\\
568	35.06\\
569	35.06\\
570	35.06\\
571	35.06\\
572	35.12\\
573	35.12\\
574	35.12\\
575	35.12\\
576	35.12\\
577	35.12\\
578	35.18\\
579	35.12\\
580	35.12\\
581	35.18\\
582	35.18\\
583	35.18\\
584	35.18\\
585	35.18\\
586	35.18\\
587	35.18\\
588	35.18\\
589	35.18\\
590	35.18\\
591	35.18\\
592	35.18\\
593	35.18\\
594	35.18\\
595	35.18\\
596	35.18\\
597	35.18\\
598	35.18\\
599	35.18\\
600	35.18\\
};
\addplot[const plot, color=mycolor2, forget plot] table[row sep=crcr] {%
1	28.18\\
2	28.18\\
3	28.18\\
4	28.18\\
5	28.18\\
6	28.18\\
7	28.18\\
8	28.18\\
9	28.18\\
10	28.18\\
11	28.18\\
12	28.18\\
13	28.18\\
14	28.18\\
15	28.18\\
16	28.18\\
17	28.18\\
18	28.18\\
19	28.18\\
20	28.18\\
21	28.18\\
22	28.18\\
23	28.18\\
24	28.18\\
25	28.18\\
26	28.18\\
27	28.18\\
28	28.18\\
29	28.18\\
30	28.18\\
31	28.18\\
32	28.18\\
33	28.18\\
34	28.18\\
35	28.18\\
36	28.18\\
37	28.18\\
38	28.18\\
39	28.18\\
40	28.18\\
41	28.18\\
42	28.18\\
43	28.18\\
44	28.18\\
45	28.18\\
46	28.18\\
47	28.18\\
48	28.18\\
49	28.18\\
50	28.18\\
51	28.18\\
52	28.18\\
53	28.18\\
54	28.18\\
55	28.18\\
56	28.18\\
57	28.18\\
58	28.18\\
59	28.18\\
60	28.18\\
61	28.18\\
62	28.18\\
63	28.18\\
64	28.18\\
65	28.18\\
66	28.18\\
67	28.18\\
68	28.18\\
69	28.18\\
70	28.18\\
71	28.18\\
72	28.18\\
73	28.18\\
74	28.18\\
75	28.18\\
76	28.18\\
77	28.18\\
78	28.18\\
79	28.18\\
80	28.18\\
81	28.18\\
82	28.18\\
83	28.18\\
84	28.18\\
85	28.18\\
86	28.18\\
87	28.18\\
88	28.18\\
89	28.18\\
90	28.18\\
91	28.18\\
92	28.18\\
93	28.18\\
94	28.18\\
95	28.18\\
96	28.18\\
97	28.18\\
98	28.18\\
99	28.18\\
100	28.18\\
101	28.18\\
102	28.18\\
103	28.18\\
104	28.18\\
105	28.18\\
106	28.18\\
107	28.18\\
108	28.18\\
109	28.18\\
110	28.18\\
111	28.18\\
112	28.18\\
113	28.18\\
114	28.18\\
115	28.18\\
116	28.18\\
117	28.18\\
118	28.18\\
119	28.18\\
120	28.18\\
121	28.18\\
122	28.18\\
123	28.18\\
124	28.18\\
125	28.18\\
126	28.18\\
127	28.18\\
128	28.18\\
129	28.18\\
130	28.18\\
131	28.18\\
132	28.18\\
133	28.18\\
134	28.18\\
135	28.18\\
136	28.18\\
137	28.18\\
138	28.18\\
139	28.18\\
140	28.18\\
141	28.18\\
142	28.18\\
143	28.18\\
144	28.18\\
145	28.18\\
146	28.18\\
147	28.18\\
148	28.18\\
149	28.18\\
150	28.18\\
151	28.18\\
152	28.18\\
153	28.18\\
154	28.18\\
155	28.18\\
156	28.18\\
157	28.18\\
158	28.18\\
159	28.18\\
160	28.18\\
161	28.18\\
162	28.18\\
163	28.18\\
164	28.18\\
165	28.18\\
166	28.18\\
167	28.18\\
168	28.18\\
169	28.18\\
170	28.18\\
171	28.18\\
172	28.18\\
173	28.18\\
174	28.18\\
175	28.18\\
176	28.18\\
177	28.18\\
178	28.18\\
179	28.18\\
180	28.18\\
181	28.18\\
182	28.18\\
183	28.18\\
184	28.18\\
185	28.18\\
186	28.18\\
187	28.18\\
188	28.18\\
189	28.18\\
190	28.18\\
191	28.18\\
192	28.18\\
193	28.18\\
194	28.18\\
195	28.18\\
196	28.18\\
197	28.18\\
198	28.18\\
199	28.18\\
200	28.18\\
201	35\\
202	35\\
203	35\\
204	35\\
205	35\\
206	35\\
207	35\\
208	35\\
209	35\\
210	35\\
211	35\\
212	35\\
213	35\\
214	35\\
215	35\\
216	35\\
217	35\\
218	35\\
219	35\\
220	35\\
221	35\\
222	35\\
223	35\\
224	35\\
225	35\\
226	35\\
227	35\\
228	35\\
229	35\\
230	35\\
231	35\\
232	35\\
233	35\\
234	35\\
235	35\\
236	35\\
237	35\\
238	35\\
239	35\\
240	35\\
241	35\\
242	35\\
243	35\\
244	35\\
245	35\\
246	35\\
247	35\\
248	35\\
249	35\\
250	35\\
251	35\\
252	35\\
253	35\\
254	35\\
255	35\\
256	35\\
257	35\\
258	35\\
259	35\\
260	35\\
261	35\\
262	35\\
263	35\\
264	35\\
265	35\\
266	35\\
267	35\\
268	35\\
269	35\\
270	35\\
271	35\\
272	35\\
273	35\\
274	35\\
275	35\\
276	35\\
277	35\\
278	35\\
279	35\\
280	35\\
281	35\\
282	35\\
283	35\\
284	35\\
285	35\\
286	35\\
287	35\\
288	35\\
289	35\\
290	35\\
291	35\\
292	35\\
293	35\\
294	35\\
295	35\\
296	35\\
297	35\\
298	35\\
299	35\\
300	35\\
301	35\\
302	35\\
303	35\\
304	35\\
305	35\\
306	35\\
307	35\\
308	35\\
309	35\\
310	35\\
311	35\\
312	35\\
313	35\\
314	35\\
315	35\\
316	35\\
317	35\\
318	35\\
319	35\\
320	35\\
321	35\\
322	35\\
323	35\\
324	35\\
325	35\\
326	35\\
327	35\\
328	35\\
329	35\\
330	35\\
331	35\\
332	35\\
333	35\\
334	35\\
335	35\\
336	35\\
337	35\\
338	35\\
339	35\\
340	35\\
341	35\\
342	35\\
343	35\\
344	35\\
345	35\\
346	35\\
347	35\\
348	35\\
349	35\\
350	35\\
351	35\\
352	35\\
353	35\\
354	35\\
355	35\\
356	35\\
357	35\\
358	35\\
359	35\\
360	35\\
361	35\\
362	35\\
363	35\\
364	35\\
365	35\\
366	35\\
367	35\\
368	35\\
369	35\\
370	35\\
371	35\\
372	35\\
373	35\\
374	35\\
375	35\\
376	35\\
377	35\\
378	35\\
379	35\\
380	35\\
381	35\\
382	35\\
383	35\\
384	35\\
385	35\\
386	35\\
387	35\\
388	35\\
389	35\\
390	35\\
391	35\\
392	35\\
393	35\\
394	35\\
395	35\\
396	35\\
397	35\\
398	35\\
399	35\\
400	35\\
401	35\\
402	35\\
403	35\\
404	35\\
405	35\\
406	35\\
407	35\\
408	35\\
409	35\\
410	35\\
411	35\\
412	35\\
413	35\\
414	35\\
415	35\\
416	35\\
417	35\\
418	35\\
419	35\\
420	35\\
421	35\\
422	35\\
423	35\\
424	35\\
425	35\\
426	35\\
427	35\\
428	35\\
429	35\\
430	35\\
431	35\\
432	35\\
433	35\\
434	35\\
435	35\\
436	35\\
437	35\\
438	35\\
439	35\\
440	35\\
441	35\\
442	35\\
443	35\\
444	35\\
445	35\\
446	35\\
447	35\\
448	35\\
449	35\\
450	35\\
451	35\\
452	35\\
453	35\\
454	35\\
455	35\\
456	35\\
457	35\\
458	35\\
459	35\\
460	35\\
461	35\\
462	35\\
463	35\\
464	35\\
465	35\\
466	35\\
467	35\\
468	35\\
469	35\\
470	35\\
471	35\\
472	35\\
473	35\\
474	35\\
475	35\\
476	35\\
477	35\\
478	35\\
479	35\\
480	35\\
481	35\\
482	35\\
483	35\\
484	35\\
485	35\\
486	35\\
487	35\\
488	35\\
489	35\\
490	35\\
491	35\\
492	35\\
493	35\\
494	35\\
495	35\\
496	35\\
497	35\\
498	35\\
499	35\\
500	35\\
501	35\\
502	35\\
503	35\\
504	35\\
505	35\\
506	35\\
507	35\\
508	35\\
509	35\\
510	35\\
511	35\\
512	35\\
513	35\\
514	35\\
515	35\\
516	35\\
517	35\\
518	35\\
519	35\\
520	35\\
521	35\\
522	35\\
523	35\\
524	35\\
525	35\\
526	35\\
527	35\\
528	35\\
529	35\\
530	35\\
531	35\\
532	35\\
533	35\\
534	35\\
535	35\\
536	35\\
537	35\\
538	35\\
539	35\\
540	35\\
541	35\\
542	35\\
543	35\\
544	35\\
545	35\\
546	35\\
547	35\\
548	35\\
549	35\\
550	35\\
551	35\\
552	35\\
553	35\\
554	35\\
555	35\\
556	35\\
557	35\\
558	35\\
559	35\\
560	35\\
561	35\\
562	35\\
563	35\\
564	35\\
565	35\\
566	35\\
567	35\\
568	35\\
569	35\\
570	35\\
571	35\\
572	35\\
573	35\\
574	35\\
575	35\\
576	35\\
577	35\\
578	35\\
579	35\\
580	35\\
581	35\\
582	35\\
583	35\\
584	35\\
585	35\\
586	35\\
587	35\\
588	35\\
589	35\\
590	35\\
591	35\\
592	35\\
593	35\\
594	35\\
595	35\\
596	35\\
597	35\\
598	35\\
599	35\\
600	35\\
};
\end{axis}
\end{tikzpicture}%
\caption{Wyjście procesu z regulatorem DMC dla parametrów $D = 340$, $ N = 30$, $N_u = 1$,  $\lambda = 0,4$}
\end{figure}

\begin{figure}[H]
\centering
% This file was created by matlab2tikz.
%
%The latest updates can be retrieved from
%  http://www.mathworks.com/matlabcentral/fileexchange/22022-matlab2tikz-matlab2tikz
%where you can also make suggestions and rate matlab2tikz.
%
\definecolor{mycolor1}{rgb}{0.00000,0.44700,0.74100}%
%
\begin{tikzpicture}

\begin{axis}[%
width=4.521in,
height=3.566in,
at={(0.758in,0.481in)},
scale only axis,
xmin=0,
xmax=600,
xlabel style={font=\color{white!15!black}},
xlabel={k},
ymin=20,
ymax=100,
ylabel style={font=\color{white!15!black}},
ylabel={\%},
axis background/.style={fill=white}
]
\addplot[const plot, color=mycolor1, forget plot] table[row sep=crcr] {%
1	24.321874\\
2	23.699428\\
3	23.132209\\
4	22.619422\\
5	22.159955\\
6	21.752421\\
7	21.395185\\
8	21.0864\\
9	20.824038\\
10	20.605919\\
11	20.42974\\
12	20.293102\\
13	20.193536\\
14	20.126041\\
15	20.177306\\
16	20.241526\\
17	20.31344\\
18	20.38842\\
19	20.554985\\
20	20.617326\\
21	20.765419\\
22	20.897227\\
23	21.011399\\
24	21.215014\\
25	21.391018\\
26	21.53961\\
27	21.753885\\
28	21.935092\\
29	22.084986\\
30	22.298043\\
31	22.47633\\
32	22.622432\\
33	22.831209\\
34	23.005206\\
35	23.147105\\
36	23.367422\\
37	23.552335\\
38	23.704813\\
39	23.827719\\
40	24.016551\\
41	24.174171\\
42	24.303353\\
43	24.407002\\
44	24.487905\\
45	24.548626\\
46	24.591815\\
47	24.712435\\
48	24.812661\\
49	24.894912\\
50	24.869023\\
51	24.83715\\
52	24.893533\\
53	24.847442\\
54	24.893103\\
55	24.93187\\
56	24.964942\\
57	24.993296\\
58	25.01771\\
59	25.038786\\
60	25.057314\\
61	25.074063\\
62	25.089699\\
63	25.104415\\
64	25.118171\\
65	25.131152\\
66	25.14326\\
67	25.154617\\
68	25.072941\\
69	24.99849\\
70	24.93137\\
71	24.871608\\
72	24.819156\\
73	24.773897\\
74	24.735661\\
75	24.704226\\
76	24.679338\\
77	24.660707\\
78	24.648019\\
79	24.64094\\
80	24.546645\\
81	24.464499\\
82	24.393627\\
83	24.333112\\
84	24.282054\\
85	24.239593\\
86	24.097041\\
87	23.970394\\
88	23.8589\\
89	23.761808\\
90	23.67838\\
91	23.607887\\
92	23.549615\\
93	23.502526\\
94	23.465494\\
95	23.437412\\
96	23.417243\\
97	23.404032\\
98	23.396918\\
99	23.394729\\
100	23.396256\\
101	23.400374\\
102	23.406093\\
103	23.412574\\
104	23.419128\\
105	23.425213\\
106	23.430418\\
107	23.434449\\
108	23.437119\\
109	23.345854\\
110	23.353163\\
111	23.358957\\
112	23.270802\\
113	23.188839\\
114	23.113114\\
115	23.043642\\
116	22.980405\\
117	22.923348\\
118	22.872381\\
119	22.827372\\
120	22.788155\\
121	22.754531\\
122	22.818401\\
123	22.786867\\
124	22.85225\\
125	22.913947\\
126	22.971379\\
127	23.024012\\
128	23.071395\\
129	23.113179\\
130	23.149121\\
131	23.179078\\
132	23.203003\\
133	23.220935\\
134	23.232995\\
135	23.239708\\
136	23.241426\\
137	23.238777\\
138	23.232484\\
139	23.223278\\
140	23.211844\\
141	23.198807\\
142	23.184723\\
143	23.170068\\
144	23.155243\\
145	23.140573\\
146	23.126315\\
147	23.112656\\
148	23.099726\\
149	23.087598\\
150	23.184186\\
151	23.164855\\
152	23.254262\\
153	23.335658\\
154	23.40909\\
155	23.474642\\
156	23.532441\\
157	23.582649\\
158	23.625471\\
159	23.661144\\
160	23.689938\\
161	23.712153\\
162	23.728112\\
163	23.738557\\
164	23.743971\\
165	23.74512\\
166	23.742838\\
167	23.737937\\
168	23.623273\\
169	23.516243\\
170	23.525219\\
171	23.425913\\
172	23.443225\\
173	23.352659\\
174	23.378918\\
175	23.297332\\
176	23.224567\\
177	23.160445\\
178	23.104707\\
179	23.057028\\
180	23.017033\\
181	22.983909\\
182	22.956698\\
183	22.934823\\
184	22.91749\\
185	23.012139\\
186	22.993468\\
187	22.977961\\
188	22.965102\\
189	22.954317\\
190	22.94506\\
191	22.936846\\
192	22.929264\\
193	23.029872\\
194	23.013788\\
195	23.105524\\
196	23.188264\\
197	23.262049\\
198	23.327409\\
199	23.384677\\
200	23.434124\\
201	33.986977\\
202	43.669529\\
203	52.489109\\
204	60.458419\\
205	67.595027\\
206	73.921234\\
207	79.462878\\
208	84.24937\\
209	88.312998\\
210	91.580482\\
211	94.204996\\
212	96.224296\\
213	97.584615\\
214	98.371326\\
215	98.681739\\
216	98.504402\\
217	98.045714\\
218	97.278071\\
219	96.295792\\
220	95.150738\\
221	93.82434\\
222	92.448992\\
223	90.971881\\
224	89.447351\\
225	87.89013\\
226	86.32448\\
227	84.768288\\
228	83.249409\\
229	81.666277\\
230	80.122903\\
231	78.527667\\
232	76.979135\\
233	75.474073\\
234	73.915337\\
235	72.305129\\
236	70.752895\\
237	69.235615\\
238	67.74686\\
239	66.280348\\
240	64.845519\\
241	63.512713\\
242	62.176626\\
243	60.847977\\
244	59.505695\\
245	58.238433\\
246	57.050834\\
247	55.823387\\
248	54.661484\\
249	53.538972\\
250	52.462526\\
251	51.407464\\
252	50.381892\\
253	49.362581\\
254	48.451847\\
255	47.528078\\
256	46.602475\\
257	45.762525\\
258	44.980996\\
259	44.263689\\
260	43.600275\\
261	42.98121\\
262	42.382313\\
263	41.812322\\
264	41.35647\\
265	40.908147\\
266	40.553931\\
267	40.188827\\
268	39.901049\\
269	39.679314\\
270	39.513194\\
271	39.393142\\
272	39.310551\\
273	39.350175\\
274	39.312563\\
275	39.384094\\
276	39.459171\\
277	39.625725\\
278	39.77983\\
279	39.918531\\
280	40.131903\\
281	40.318193\\
282	40.584377\\
283	40.813479\\
284	41.098374\\
285	41.339953\\
286	41.632482\\
287	41.877815\\
288	42.170923\\
289	42.522439\\
290	42.911444\\
291	43.241495\\
292	43.608817\\
293	43.917437\\
294	44.264032\\
295	44.660955\\
296	45.088488\\
297	45.451407\\
298	45.846981\\
299	46.180128\\
300	46.548217\\
301	46.856188\\
302	47.217188\\
303	47.519945\\
304	47.769803\\
305	48.064562\\
306	48.309197\\
307	48.508738\\
308	48.668289\\
309	48.885498\\
310	49.065128\\
311	49.211913\\
312	49.330196\\
313	49.424172\\
314	49.49762\\
315	49.554242\\
316	49.689826\\
317	49.807156\\
318	49.908985\\
319	49.90531\\
320	49.990762\\
321	49.974475\\
322	49.958187\\
323	49.943326\\
324	50.023493\\
325	50.099468\\
326	50.171866\\
327	50.241125\\
328	50.215065\\
329	50.194266\\
330	50.179199\\
331	50.170218\\
332	50.167212\\
333	50.077694\\
334	49.908899\\
335	49.760304\\
336	49.523296\\
337	49.314122\\
338	49.132263\\
339	48.884632\\
340	48.670577\\
341	48.488845\\
342	48.245533\\
343	48.039138\\
344	47.775497\\
345	47.55287\\
346	47.261124\\
347	47.014238\\
348	46.809205\\
349	46.642589\\
350	46.418439\\
351	46.233422\\
352	46.083949\\
353	45.966499\\
354	45.877747\\
355	45.814287\\
356	45.772876\\
357	45.750208\\
358	45.743196\\
359	45.748685\\
360	45.763741\\
361	45.785825\\
362	45.81277\\
363	45.842473\\
364	45.873076\\
365	45.903041\\
366	46.023648\\
367	46.133978\\
368	46.141019\\
369	46.144595\\
370	46.237133\\
371	46.226219\\
372	46.212196\\
373	46.195451\\
374	46.176399\\
375	46.155496\\
376	46.133209\\
377	46.109982\\
378	46.08624\\
379	46.062723\\
380	46.040212\\
381	46.019112\\
382	45.999592\\
383	45.982054\\
384	45.966598\\
385	45.953171\\
386	45.941623\\
387	45.931778\\
388	45.923408\\
389	45.916278\\
390	45.910169\\
391	45.904874\\
392	45.90019\\
393	45.895926\\
394	45.891918\\
395	45.888015\\
396	45.884084\\
397	45.880018\\
398	45.87574\\
399	45.871192\\
400	45.866342\\
401	45.861152\\
402	45.85562\\
403	45.849751\\
404	45.843556\\
405	45.837063\\
406	45.830304\\
407	45.8233\\
408	45.816089\\
409	45.808709\\
410	45.801206\\
411	45.793606\\
412	45.785957\\
413	45.778284\\
414	45.770613\\
415	45.762962\\
416	45.755349\\
417	45.747785\\
418	45.740273\\
419	45.825286\\
420	45.810294\\
421	45.795436\\
422	45.873236\\
423	45.943677\\
424	45.914384\\
425	45.885548\\
426	45.857278\\
427	45.82969\\
428	45.80287\\
429	45.776898\\
430	45.751839\\
431	45.727733\\
432	45.797438\\
433	45.861012\\
434	45.918539\\
435	45.970412\\
436	46.017143\\
437	46.058942\\
438	46.095869\\
439	46.127956\\
440	46.155234\\
441	46.177765\\
442	46.195611\\
443	46.208877\\
444	46.325591\\
445	46.429509\\
446	46.413438\\
447	46.394892\\
448	46.374588\\
449	46.353168\\
450	46.331218\\
451	46.417104\\
452	46.386612\\
453	46.356929\\
454	46.328379\\
455	46.301238\\
456	46.275716\\
457	46.252335\\
458	46.231706\\
459	46.213971\\
460	46.199063\\
461	46.186787\\
462	46.176908\\
463	46.169151\\
464	46.271528\\
465	46.367141\\
466	46.347904\\
467	46.438271\\
468	46.413412\\
469	46.389941\\
470	46.367733\\
471	46.346678\\
472	46.326693\\
473	46.307704\\
474	46.397535\\
475	46.479407\\
476	46.44549\\
477	46.413034\\
478	46.382664\\
479	46.354582\\
480	46.329204\\
481	46.306589\\
482	46.194158\\
483	46.091784\\
484	45.999188\\
485	45.916042\\
486	45.841966\\
487	45.684492\\
488	45.543772\\
489	45.419415\\
490	45.310791\\
491	45.217126\\
492	45.137602\\
493	45.071373\\
494	44.92509\\
495	44.797609\\
496	44.687564\\
497	44.59357\\
498	44.514286\\
499	44.448412\\
500	44.394396\\
501	44.350662\\
502	44.315718\\
503	44.288208\\
504	44.266945\\
505	44.250887\\
506	44.239141\\
507	44.230608\\
508	44.224226\\
509	44.219058\\
510	44.21433\\
511	44.209383\\
512	44.096028\\
513	44.098808\\
514	43.993187\\
515	43.896135\\
516	43.807952\\
517	43.836853\\
518	43.86641\\
519	43.896967\\
520	43.928893\\
521	43.962504\\
522	43.998094\\
523	44.128337\\
524	44.253324\\
525	44.372779\\
526	44.486649\\
527	44.594577\\
528	44.696103\\
529	44.790763\\
530	44.878529\\
531	44.959545\\
532	45.034038\\
533	45.194733\\
534	45.341797\\
535	45.475567\\
536	45.689206\\
537	45.791026\\
538	45.882042\\
539	45.963164\\
540	46.03528\\
541	46.099205\\
542	46.246657\\
543	46.377023\\
544	46.39753\\
545	46.500636\\
546	46.586675\\
547	46.656344\\
548	46.710505\\
549	46.750453\\
550	46.777304\\
551	46.792118\\
552	46.795926\\
553	46.789744\\
554	46.774599\\
555	46.751844\\
556	46.830794\\
557	46.895761\\
558	46.947985\\
559	46.988732\\
560	47.019213\\
561	47.040551\\
562	47.053768\\
563	47.059778\\
564	46.95151\\
565	46.846424\\
566	46.745076\\
567	46.647883\\
568	46.462669\\
569	46.29008\\
570	46.130678\\
571	45.984894\\
572	45.760527\\
573	45.557737\\
574	45.376388\\
575	45.216163\\
576	45.076581\\
577	44.956612\\
578	44.762505\\
579	44.685417\\
580	44.623822\\
581	44.483481\\
582	44.362626\\
583	44.259438\\
584	44.172186\\
585	44.09892\\
586	44.037708\\
587	43.986752\\
588	43.944454\\
589	43.909416\\
590	43.880441\\
591	43.856186\\
592	43.835721\\
593	43.818386\\
594	43.803373\\
595	43.789902\\
596	43.777308\\
597	43.765067\\
598	43.752782\\
599	43.740172\\
600	43.727058\\
};
\end{axis}
\end{tikzpicture}%
\caption{Sterowanie procesu z regulatorem DMC dla parametrów $D = 340$, $ N = 30$, $N_u = 1$,  $\lambda = 0,4$}
\end{figure}

\begin{equation}
E = 1,2676 * 10^3
\end{equation}

\section{Dobór parametru $D^z$}

W wyniku eksperymentów dobraliśmy wartość $D^z = 320$.

Poniżej są przedstawione wyniki regulacji najpierw bez pomiaru zakłócenia, a potem z pomiarem dla skoków sygnału zakłócenia w chwili k=350 z wartości 0 do 30 oraz w k=600 z 30 do 10.

\begin{figure}[H]
\centering
% This file was created by matlab2tikz.
%
%The latest updates can be retrieved from
%  http://www.mathworks.com/matlabcentral/fileexchange/22022-matlab2tikz-matlab2tikz
%where you can also make suggestions and rate matlab2tikz.
%
\definecolor{mycolor1}{rgb}{0.00000,0.44700,0.74100}%
%
\begin{tikzpicture}

\begin{axis}[%
width=4.521in,
height=3.566in,
at={(0.758in,0.481in)},
scale only axis,
xmin=0,
xmax=800,
xlabel style={font=\color{white!15!black}},
xlabel={k},
ymin=28,
ymax=37,
ylabel style={font=\color{white!15!black}},
ylabel={Y(k)]},
axis background/.style={fill=white}
]
\addplot[const plot, color=mycolor1, forget plot] table[row sep=crcr] {%
1	28.25\\
2	28.25\\
3	28.18\\
4	28.18\\
5	28.18\\
6	28.18\\
7	28.12\\
8	28.12\\
9	28.12\\
10	28.12\\
11	28.06\\
12	28.12\\
13	28.12\\
14	28.12\\
15	28.12\\
16	28.12\\
17	28.12\\
18	28.12\\
19	28.12\\
20	28.12\\
21	28.12\\
22	28.12\\
23	28.12\\
24	28.12\\
25	28.18\\
26	28.18\\
27	28.18\\
28	28.18\\
29	28.25\\
30	28.25\\
31	28.18\\
32	28.25\\
33	28.25\\
34	28.25\\
35	28.31\\
36	28.31\\
37	28.31\\
38	28.31\\
39	28.31\\
40	28.31\\
41	28.37\\
42	28.37\\
43	28.37\\
44	28.37\\
45	28.37\\
46	28.37\\
47	28.37\\
48	28.43\\
49	28.37\\
50	28.43\\
51	28.37\\
52	28.43\\
53	28.43\\
54	28.43\\
55	28.43\\
56	28.43\\
57	28.43\\
58	28.43\\
59	28.43\\
60	28.43\\
61	28.43\\
62	28.43\\
63	28.43\\
64	28.43\\
65	28.43\\
66	28.43\\
67	28.43\\
68	28.43\\
69	28.43\\
70	28.43\\
71	28.43\\
72	28.43\\
73	28.43\\
74	28.43\\
75	28.43\\
76	28.43\\
77	28.43\\
78	28.43\\
79	28.43\\
80	28.43\\
81	28.37\\
82	28.37\\
83	28.37\\
84	28.37\\
85	28.37\\
86	28.37\\
87	28.37\\
88	28.37\\
89	28.37\\
90	28.31\\
91	28.31\\
92	28.31\\
93	28.31\\
94	28.31\\
95	28.31\\
96	28.31\\
97	28.31\\
98	28.31\\
99	28.31\\
100	28.31\\
101	28.31\\
102	28.31\\
103	28.31\\
104	28.31\\
105	28.25\\
106	28.25\\
107	28.25\\
108	28.31\\
109	28.25\\
110	28.31\\
111	28.31\\
112	28.37\\
113	28.37\\
114	28.43\\
115	28.5\\
116	28.56\\
117	28.62\\
118	28.81\\
119	28.93\\
120	29.06\\
121	29.18\\
122	29.31\\
123	29.5\\
124	29.68\\
125	29.87\\
126	30.06\\
127	30.25\\
128	30.5\\
129	30.68\\
130	30.87\\
131	31.12\\
132	31.31\\
133	31.5\\
134	31.75\\
135	31.93\\
136	32.18\\
137	32.37\\
138	32.56\\
139	32.75\\
140	32.93\\
141	33.12\\
142	33.31\\
143	33.5\\
144	33.68\\
145	33.81\\
146	34\\
147	34.12\\
148	34.31\\
149	34.43\\
150	34.56\\
151	34.68\\
152	34.81\\
153	34.93\\
154	35.06\\
155	35.12\\
156	35.25\\
157	35.31\\
158	35.37\\
159	35.43\\
160	35.5\\
161	35.56\\
162	35.56\\
163	35.62\\
164	35.62\\
165	35.68\\
166	35.68\\
167	35.68\\
168	35.75\\
169	35.75\\
170	35.75\\
171	35.75\\
172	35.75\\
173	35.75\\
174	35.68\\
175	35.68\\
176	35.62\\
177	35.62\\
178	35.56\\
179	35.56\\
180	35.56\\
181	35.5\\
182	35.5\\
183	35.5\\
184	35.43\\
185	35.37\\
186	35.37\\
187	35.31\\
188	35.25\\
189	35.25\\
190	35.18\\
191	35.12\\
192	35.12\\
193	35.12\\
194	35.06\\
195	35.06\\
196	35\\
197	34.93\\
198	34.93\\
199	34.87\\
200	34.81\\
201	34.75\\
202	34.75\\
203	34.68\\
204	34.68\\
205	34.62\\
206	34.56\\
207	34.56\\
208	34.56\\
209	34.5\\
210	34.56\\
211	34.56\\
212	34.56\\
213	34.5\\
214	34.5\\
215	34.5\\
216	34.56\\
217	34.5\\
218	34.56\\
219	34.56\\
220	34.56\\
221	34.56\\
222	34.56\\
223	34.56\\
224	34.56\\
225	34.56\\
226	34.56\\
227	34.56\\
228	34.62\\
229	34.62\\
230	34.62\\
231	34.62\\
232	34.68\\
233	34.68\\
234	34.68\\
235	34.68\\
236	34.68\\
237	34.68\\
238	34.75\\
239	34.75\\
240	34.75\\
241	34.75\\
242	34.81\\
243	34.81\\
244	34.81\\
245	34.87\\
246	34.87\\
247	34.87\\
248	34.93\\
249	34.93\\
250	34.93\\
251	35\\
252	35\\
253	35\\
254	35\\
255	35\\
256	35\\
257	35.06\\
258	35.06\\
259	35.06\\
260	35.06\\
261	35.06\\
262	35.06\\
263	35.06\\
264	35.06\\
265	35.06\\
266	35.06\\
267	35.06\\
268	35.06\\
269	35.06\\
270	35.06\\
271	35.06\\
272	35.06\\
273	35.06\\
274	35.06\\
275	35.06\\
276	35.06\\
277	35.06\\
278	35.06\\
279	35.06\\
280	35.06\\
281	35.06\\
282	35.06\\
283	35.06\\
284	35.06\\
285	35.06\\
286	35.06\\
287	35.06\\
288	35.06\\
289	35.06\\
290	35.06\\
291	35.06\\
292	35\\
293	35\\
294	35\\
295	35\\
296	35\\
297	35\\
298	35.06\\
299	35\\
300	35.06\\
301	35\\
302	35.06\\
303	35.06\\
304	35.06\\
305	35.06\\
306	35.06\\
307	35.12\\
308	35.12\\
309	35.12\\
310	35.12\\
311	35.12\\
312	35.12\\
313	35.12\\
314	35.12\\
315	35.18\\
316	35.12\\
317	35.18\\
318	35.18\\
319	35.18\\
320	35.18\\
321	35.18\\
322	35.18\\
323	35.18\\
324	35.18\\
325	35.18\\
326	35.18\\
327	35.12\\
328	35.12\\
329	35.12\\
330	35.12\\
331	35.12\\
332	35.12\\
333	35.12\\
334	35.12\\
335	35.12\\
336	35.12\\
337	35.12\\
338	35.12\\
339	35.06\\
340	35.06\\
341	35.06\\
342	35.06\\
343	35.06\\
344	35.06\\
345	35.06\\
346	35.06\\
347	35.06\\
348	35.06\\
349	35.06\\
350	35.06\\
351	35.06\\
352	35.06\\
353	35.06\\
354	35.06\\
355	35.06\\
356	35.06\\
357	35.06\\
358	35.06\\
359	35.06\\
360	35.06\\
361	35.12\\
362	35.12\\
363	35.12\\
364	35.12\\
365	35.18\\
366	35.18\\
367	35.18\\
368	35.25\\
369	35.31\\
370	35.31\\
371	35.37\\
372	35.43\\
373	35.43\\
374	35.5\\
375	35.5\\
376	35.56\\
377	35.62\\
378	35.62\\
379	35.68\\
380	35.68\\
381	35.75\\
382	35.81\\
383	35.87\\
384	35.93\\
385	35.93\\
386	36\\
387	36.06\\
388	36.06\\
389	36.12\\
390	36.12\\
391	36.18\\
392	36.18\\
393	36.25\\
394	36.25\\
395	36.31\\
396	36.31\\
397	36.37\\
398	36.37\\
399	36.37\\
400	36.43\\
401	36.5\\
402	36.5\\
403	36.5\\
404	36.56\\
405	36.56\\
406	36.56\\
407	36.56\\
408	36.56\\
409	36.56\\
410	36.56\\
411	36.56\\
412	36.56\\
413	36.56\\
414	36.5\\
415	36.5\\
416	36.5\\
417	36.5\\
418	36.5\\
419	36.5\\
420	36.5\\
421	36.5\\
422	36.5\\
423	36.43\\
424	36.43\\
425	36.43\\
426	36.43\\
427	36.43\\
428	36.43\\
429	36.43\\
430	36.37\\
431	36.37\\
432	36.31\\
433	36.31\\
434	36.31\\
435	36.25\\
436	36.25\\
437	36.25\\
438	36.18\\
439	36.18\\
440	36.12\\
441	36.12\\
442	36.06\\
443	36\\
444	36\\
445	35.93\\
446	35.93\\
447	35.87\\
448	35.87\\
449	35.81\\
450	35.81\\
451	35.81\\
452	35.75\\
453	35.75\\
454	35.75\\
455	35.68\\
456	35.75\\
457	35.68\\
458	35.68\\
459	35.68\\
460	35.68\\
461	35.68\\
462	35.68\\
463	35.62\\
464	35.62\\
465	35.62\\
466	35.62\\
467	35.56\\
468	35.62\\
469	35.62\\
470	35.62\\
471	35.62\\
472	35.62\\
473	35.62\\
474	35.62\\
475	35.62\\
476	35.62\\
477	35.62\\
478	35.62\\
479	35.62\\
480	35.62\\
481	35.68\\
482	35.68\\
483	35.68\\
484	35.68\\
485	35.68\\
486	35.68\\
487	35.68\\
488	35.68\\
489	35.68\\
490	35.68\\
491	35.68\\
492	35.68\\
493	35.68\\
494	35.68\\
495	35.68\\
496	35.68\\
497	35.68\\
498	35.62\\
499	35.62\\
500	35.62\\
501	35.62\\
502	35.68\\
503	35.62\\
504	35.62\\
505	35.62\\
506	35.62\\
507	35.62\\
508	35.62\\
509	35.56\\
510	35.56\\
511	35.56\\
512	35.56\\
513	35.56\\
514	35.56\\
515	35.56\\
516	35.56\\
517	35.56\\
518	35.56\\
519	35.56\\
520	35.56\\
521	35.56\\
522	35.56\\
523	35.56\\
524	35.56\\
525	35.5\\
526	35.5\\
527	35.56\\
528	35.5\\
529	35.5\\
530	35.5\\
531	35.5\\
532	35.43\\
533	35.43\\
534	35.43\\
535	35.43\\
536	35.43\\
537	35.43\\
538	35.37\\
539	35.37\\
540	35.31\\
541	35.37\\
542	35.31\\
543	35.31\\
544	35.31\\
545	35.31\\
546	35.25\\
547	35.25\\
548	35.25\\
549	35.18\\
550	35.18\\
551	35.18\\
552	35.18\\
553	35.12\\
554	35.18\\
555	35.18\\
556	35.18\\
557	35.18\\
558	35.12\\
559	35.12\\
560	35.12\\
561	35.12\\
562	35.12\\
563	35.06\\
564	35.06\\
565	35.06\\
566	35.06\\
567	35.06\\
568	35.06\\
569	35.06\\
570	35.06\\
571	35.06\\
572	35.06\\
573	35.06\\
574	35.06\\
575	35.12\\
576	35.12\\
577	35.12\\
578	35.18\\
579	35.18\\
580	35.18\\
581	35.18\\
582	35.25\\
583	35.31\\
584	35.31\\
585	35.31\\
586	35.31\\
587	35.31\\
588	35.31\\
589	35.31\\
590	35.37\\
591	35.37\\
592	35.31\\
593	35.31\\
594	35.31\\
595	35.31\\
596	35.31\\
597	35.31\\
598	35.31\\
599	35.31\\
600	35.31\\
601	35.31\\
602	35.31\\
603	35.25\\
604	35.31\\
605	35.31\\
606	35.25\\
607	35.25\\
608	35.18\\
609	35.25\\
610	35.18\\
611	35.18\\
612	35.18\\
613	35.12\\
614	35.12\\
615	35.06\\
616	35.06\\
617	35\\
618	35\\
619	35\\
620	34.93\\
621	34.87\\
622	34.87\\
623	34.87\\
624	34.81\\
625	34.81\\
626	34.75\\
627	34.75\\
628	34.68\\
629	34.62\\
630	34.56\\
631	34.56\\
632	34.5\\
633	34.5\\
634	34.5\\
635	34.43\\
636	34.43\\
637	34.37\\
638	34.37\\
639	34.31\\
640	34.31\\
641	34.31\\
642	34.31\\
643	34.25\\
644	34.25\\
645	34.18\\
646	34.18\\
647	34.12\\
648	34.12\\
649	34.12\\
650	34.12\\
651	34.06\\
652	34.06\\
653	34.06\\
654	34.06\\
655	34.06\\
656	34.06\\
657	34.06\\
658	34.12\\
659	34.06\\
660	34.12\\
661	34.12\\
662	34.12\\
663	34.12\\
664	34.12\\
665	34.12\\
666	34.12\\
667	34.12\\
668	34.12\\
669	34.12\\
670	34.12\\
671	34.12\\
672	34.18\\
673	34.12\\
674	34.18\\
675	34.18\\
676	34.18\\
677	34.18\\
678	34.18\\
679	34.25\\
680	34.25\\
681	34.25\\
682	34.25\\
683	34.25\\
684	34.25\\
685	34.25\\
686	34.25\\
687	34.25\\
688	34.25\\
689	34.25\\
690	34.25\\
691	34.31\\
692	34.31\\
693	34.31\\
694	34.31\\
695	34.37\\
696	34.37\\
697	34.31\\
698	34.37\\
699	34.37\\
700	34.37\\
701	34.37\\
702	34.37\\
703	34.37\\
704	34.37\\
705	34.37\\
706	34.37\\
707	34.37\\
708	34.37\\
709	34.43\\
710	34.43\\
711	34.5\\
712	34.5\\
713	34.5\\
714	34.5\\
715	34.56\\
716	34.56\\
717	34.56\\
718	34.62\\
719	34.62\\
720	34.68\\
721	34.68\\
722	34.68\\
723	34.75\\
724	34.75\\
725	34.75\\
726	34.75\\
727	34.81\\
728	34.81\\
729	34.81\\
730	34.81\\
731	34.81\\
732	34.81\\
733	34.87\\
734	34.87\\
735	34.87\\
736	34.87\\
737	34.87\\
738	34.87\\
739	34.87\\
740	34.87\\
741	34.87\\
742	34.87\\
743	34.87\\
744	34.93\\
745	34.93\\
746	34.93\\
747	34.93\\
748	34.93\\
749	34.93\\
750	34.93\\
751	34.93\\
752	34.87\\
753	34.93\\
754	34.87\\
755	34.87\\
756	34.87\\
757	34.87\\
758	34.87\\
759	34.81\\
760	34.81\\
761	34.81\\
762	34.81\\
763	34.81\\
764	34.81\\
765	34.75\\
766	34.75\\
767	34.75\\
768	34.75\\
769	34.75\\
770	34.68\\
771	34.75\\
772	34.68\\
773	34.68\\
774	34.68\\
775	34.68\\
776	34.68\\
777	34.68\\
778	34.62\\
779	34.62\\
780	34.62\\
781	34.62\\
782	34.62\\
783	34.62\\
784	34.62\\
785	34.62\\
786	34.56\\
787	34.56\\
788	34.62\\
789	34.62\\
790	34.62\\
791	34.62\\
792	34.62\\
793	34.62\\
794	34.62\\
795	34.62\\
796	34.62\\
797	34.62\\
798	34.62\\
799	34.62\\
800	34.62\\
};
\end{axis}
\end{tikzpicture}%
\caption{Wyjście procesu z regulatorem DMC dla parametrów $D = 340$, $ N = 30$, $N_u = 1$,  $\lambda = 0,4$ bez pomiaru zakłócenia}
\end{figure}

\begin{figure}[H]
\centering
% This file was created by matlab2tikz.
%
%The latest updates can be retrieved from
%  http://www.mathworks.com/matlabcentral/fileexchange/22022-matlab2tikz-matlab2tikz
%where you can also make suggestions and rate matlab2tikz.
%
\definecolor{mycolor1}{rgb}{0.00000,0.44700,0.74100}%
%
\begin{tikzpicture}

\begin{axis}[%
width=4.521in,
height=3.566in,
at={(0.758in,0.481in)},
scale only axis,
xmin=0,
xmax=800,
xlabel style={font=\color{white!15!black}},
xlabel={k},
ymin=20,
ymax=100,
ylabel style={font=\color{white!15!black}},
ylabel={U(k},
axis background/.style={fill=white}
]
\addplot[const plot, color=mycolor1, forget plot] table[row sep=crcr] {%
1	24.892116\\
2	24.793091\\
3	24.810735\\
4	24.828181\\
5	24.845323\\
6	24.862068\\
7	24.970804\\
8	25.071393\\
9	25.163834\\
10	25.248184\\
11	25.417021\\
12	25.477964\\
13	25.531336\\
14	25.57702\\
15	25.614811\\
16	25.644923\\
17	25.66778\\
18	25.683896\\
19	25.693827\\
20	25.698479\\
21	25.698856\\
22	25.695944\\
23	25.690657\\
24	25.684146\\
25	25.584754\\
26	25.493144\\
27	25.409699\\
28	25.334637\\
29	25.160149\\
30	25.002926\\
31	24.970624\\
32	24.838241\\
33	24.722088\\
34	24.621609\\
35	24.443688\\
36	24.28767\\
37	24.152722\\
38	24.037586\\
39	23.940836\\
40	23.861003\\
41	23.704167\\
42	23.568605\\
43	23.452415\\
44	23.35413\\
45	23.272114\\
46	23.204733\\
47	23.15044\\
48	23.014996\\
49	22.989198\\
50	22.878849\\
51	22.875017\\
52	22.783837\\
53	22.704246\\
54	22.634902\\
55	22.574474\\
56	22.521742\\
57	22.47563\\
58	22.435214\\
59	22.39971\\
60	22.36847\\
61	22.340619\\
62	22.315617\\
63	22.292775\\
64	22.271757\\
65	22.252067\\
66	22.233206\\
67	22.214728\\
68	22.196265\\
69	22.17753\\
70	22.15831\\
71	22.138455\\
72	22.117875\\
73	22.096525\\
74	22.0744\\
75	22.051524\\
76	22.027943\\
77	22.003723\\
78	21.978939\\
79	21.953673\\
80	21.92801\\
81	21.994504\\
82	22.053169\\
83	22.104138\\
84	22.147583\\
85	22.18371\\
86	22.212753\\
87	22.234971\\
88	22.250643\\
89	22.260066\\
90	22.35602\\
91	22.438763\\
92	22.508683\\
93	22.566224\\
94	22.612213\\
95	22.647618\\
96	22.673437\\
97	22.690636\\
98	22.700126\\
99	22.702755\\
100	22.699296\\
101	33.201414\\
102	42.835747\\
103	51.610214\\
104	59.538129\\
105	66.730052\\
106	73.108222\\
107	78.698684\\
108	83.438233\\
109	87.551417\\
110	90.879928\\
111	93.559925\\
112	95.536409\\
113	96.947932\\
114	97.779592\\
115	98.113161\\
116	98.061696\\
117	97.714862\\
118	96.953065\\
119	95.971387\\
120	94.82242\\
121	93.580452\\
122	92.278391\\
123	90.864178\\
124	89.392966\\
125	87.879949\\
126	86.350059\\
127	84.821926\\
128	83.216197\\
129	81.656956\\
130	80.132495\\
131	78.551843\\
132	77.014008\\
133	75.516086\\
134	73.961535\\
135	72.461099\\
136	70.900225\\
137	69.373255\\
138	67.874276\\
139	66.397428\\
140	64.952483\\
141	63.517161\\
142	62.08628\\
143	60.655211\\
144	59.234994\\
145	57.897555\\
146	56.539539\\
147	55.265885\\
148	53.957452\\
149	52.72003\\
150	51.527765\\
151	50.387496\\
152	49.274586\\
153	48.197191\\
154	47.131978\\
155	46.180751\\
156	45.221214\\
157	44.356295\\
158	43.573031\\
159	42.8591\\
160	42.187702\\
161	41.564739\\
162	41.072773\\
163	40.602732\\
164	40.239047\\
165	39.874474\\
166	39.595201\\
167	39.38812\\
168	39.133354\\
169	38.936493\\
170	38.787373\\
171	38.676878\\
172	38.596925\\
173	38.540326\\
174	38.60859\\
175	38.679704\\
176	38.841704\\
177	38.991072\\
178	39.217493\\
179	39.418907\\
180	39.594158\\
181	39.834904\\
182	40.040775\\
183	40.212029\\
184	40.457292\\
185	40.753474\\
186	41.002101\\
187	41.297986\\
188	41.63627\\
189	41.920222\\
190	42.261332\\
191	42.639349\\
192	42.958362\\
193	43.222531\\
194	43.52873\\
195	43.781247\\
196	44.076761\\
197	44.427511\\
198	44.72143\\
199	45.055649\\
200	45.427352\\
201	45.833928\\
202	46.180177\\
203	46.579029\\
204	46.918969\\
205	47.297539\\
206	47.711904\\
207	48.06679\\
208	48.366805\\
209	48.709096\\
210	48.905984\\
211	49.062244\\
212	49.182564\\
213	49.36414\\
214	49.511656\\
215	49.62959\\
216	49.629896\\
217	49.709184\\
218	49.678778\\
219	49.642578\\
220	49.603976\\
221	49.565826\\
222	49.530744\\
223	49.500584\\
224	49.476603\\
225	49.459598\\
226	49.450289\\
227	49.449197\\
228	49.36412\\
229	49.294808\\
230	49.240897\\
231	49.201565\\
232	49.083293\\
233	48.984978\\
234	48.905359\\
235	48.843128\\
236	48.796995\\
237	48.765695\\
238	48.640117\\
239	48.535827\\
240	48.451666\\
241	48.386139\\
242	48.245186\\
243	48.127336\\
244	48.03109\\
245	47.862176\\
246	47.718832\\
247	47.599251\\
248	47.409218\\
249	47.247151\\
250	47.111461\\
251	46.892311\\
252	46.704436\\
253	46.54581\\
254	46.414457\\
255	46.308153\\
256	46.224668\\
257	46.069398\\
258	45.94008\\
259	45.83438\\
260	45.750071\\
261	45.68476\\
262	45.636114\\
263	45.601991\\
264	45.580056\\
265	45.568087\\
266	45.564087\\
267	45.566341\\
268	45.573382\\
269	45.583998\\
270	45.59688\\
271	45.610834\\
272	45.624825\\
273	45.638052\\
274	45.649905\\
275	45.65996\\
276	45.667956\\
277	45.673769\\
278	45.677382\\
279	45.678877\\
280	45.67838\\
281	45.676091\\
282	45.672214\\
283	45.666981\\
284	45.660631\\
285	45.653405\\
286	45.645516\\
287	45.637175\\
288	45.628552\\
289	45.619794\\
290	45.611035\\
291	45.602375\\
292	45.68636\\
293	45.762984\\
294	45.83234\\
295	45.89456\\
296	45.94979\\
297	45.998211\\
298	45.947558\\
299	45.990601\\
300	45.934993\\
301	45.973479\\
302	45.913707\\
303	45.855954\\
304	45.800385\\
305	45.747491\\
306	45.69783\\
307	45.559448\\
308	45.432859\\
309	45.318343\\
310	45.216038\\
311	45.125592\\
312	45.046779\\
313	44.97901\\
314	44.921878\\
315	44.78221\\
316	44.751711\\
317	44.636912\\
318	44.536963\\
319	44.450919\\
320	44.377514\\
321	44.315385\\
322	44.263197\\
323	44.219709\\
324	44.183783\\
325	44.154393\\
326	44.130635\\
327	44.204201\\
328	44.274001\\
329	44.339411\\
330	44.399716\\
331	44.454254\\
332	44.502515\\
333	44.544125\\
334	44.578876\\
335	44.606684\\
336	44.627581\\
337	44.64172\\
338	44.649318\\
339	44.743152\\
340	44.823849\\
341	44.892322\\
342	44.949547\\
343	44.996513\\
344	45.034176\\
345	45.063432\\
346	45.085113\\
347	45.09999\\
348	45.108767\\
349	45.112067\\
350	45.110461\\
351	45.104445\\
352	45.09479\\
353	45.082351\\
354	45.067901\\
355	45.052138\\
356	45.035659\\
357	45.018935\\
358	45.002358\\
359	44.986212\\
360	44.970722\\
361	44.86355\\
362	44.764852\\
363	44.674602\\
364	44.592682\\
365	44.426419\\
366	44.275624\\
367	44.13994\\
368	43.911033\\
369	43.612642\\
370	43.344196\\
371	43.012411\\
372	42.62393\\
373	42.277698\\
374	41.864159\\
375	41.49812\\
376	41.084984\\
377	40.630088\\
378	40.230775\\
379	39.791585\\
380	39.409501\\
381	38.973082\\
382	38.502572\\
383	38.001362\\
384	37.472489\\
385	37.011054\\
386	36.504133\\
387	35.97072\\
388	35.505625\\
389	35.010956\\
390	34.581146\\
391	34.118224\\
392	33.716705\\
393	33.263385\\
394	32.869612\\
395	32.43738\\
396	32.061111\\
397	31.642584\\
398	31.276425\\
399	30.957175\\
400	30.586919\\
401	30.152757\\
402	29.766446\\
403	29.423313\\
404	29.02632\\
405	28.671298\\
406	28.353993\\
407	28.070424\\
408	27.816676\\
409	27.589135\\
410	27.384256\\
411	27.198777\\
412	27.03002\\
413	26.875467\\
414	26.825069\\
415	26.776493\\
416	26.727937\\
417	26.67771\\
418	26.624415\\
419	26.567001\\
420	26.504784\\
421	26.437356\\
422	26.36457\\
423	26.394332\\
424	26.410102\\
425	26.412241\\
426	26.401238\\
427	26.37799\\
428	26.343557\\
429	26.299015\\
430	26.337865\\
431	26.360961\\
432	26.461668\\
433	26.540773\\
434	26.59912\\
435	26.730048\\
436	26.834734\\
437	26.914599\\
438	27.078991\\
439	27.21259\\
440	27.40933\\
441	27.57059\\
442	27.788831\\
443	28.057091\\
444	28.276377\\
445	28.556344\\
446	28.782374\\
447	29.049344\\
448	29.26011\\
449	29.510355\\
450	29.703332\\
451	29.842869\\
452	30.025419\\
453	30.155245\\
454	30.236747\\
455	30.382453\\
456	30.372641\\
457	30.436772\\
458	30.462996\\
459	30.456017\\
460	30.420502\\
461	30.360785\\
462	30.28108\\
463	30.277683\\
464	30.253877\\
465	30.212906\\
466	30.157683\\
467	30.183179\\
468	30.099341\\
469	30.008124\\
470	29.911323\\
471	29.810489\\
472	29.7069\\
473	29.60155\\
474	29.495184\\
475	29.388343\\
476	29.281729\\
477	29.176019\\
478	29.071779\\
479	28.969442\\
480	28.86965\\
481	28.68023\\
482	28.501341\\
483	28.3329\\
484	28.174699\\
485	28.026443\\
486	27.887765\\
487	27.758251\\
488	27.637448\\
489	27.524886\\
490	27.420075\\
491	27.322516\\
492	27.231711\\
493	27.14717\\
494	27.06808\\
495	26.993542\\
496	26.922676\\
497	26.854684\\
498	26.881328\\
499	26.901931\\
500	26.916038\\
501	26.923337\\
502	26.831175\\
503	26.832011\\
504	26.825749\\
505	26.812458\\
506	26.792295\\
507	26.765483\\
508	26.732301\\
509	26.785558\\
510	26.82556\\
511	26.853098\\
512	26.86913\\
513	26.874656\\
514	26.870658\\
515	26.857719\\
516	26.836599\\
517	26.808092\\
518	26.772976\\
519	26.731972\\
520	26.685748\\
521	26.63491\\
522	26.580329\\
523	26.522923\\
524	26.463554\\
525	26.495438\\
526	26.51913\\
527	26.442728\\
528	26.459272\\
529	26.469072\\
530	26.472474\\
531	26.469776\\
532	26.569121\\
533	26.654012\\
534	26.724747\\
535	26.781663\\
536	26.825141\\
537	26.855601\\
538	26.966302\\
539	27.058121\\
540	27.224215\\
541	27.272988\\
542	27.398084\\
543	27.500568\\
544	27.581614\\
545	27.642818\\
546	27.778367\\
547	27.88989\\
548	27.979075\\
549	28.155438\\
550	28.303784\\
551	28.426009\\
552	28.524084\\
553	28.692755\\
554	28.741419\\
555	28.772254\\
556	28.787298\\
557	28.78845\\
558	28.869893\\
559	28.93344\\
560	28.980958\\
561	29.014188\\
562	29.035106\\
563	29.138105\\
564	29.224866\\
565	29.296983\\
566	29.356206\\
567	29.403882\\
568	29.441089\\
569	29.468728\\
570	29.487556\\
571	29.498555\\
572	29.502723\\
573	29.500993\\
574	29.494188\\
575	29.390544\\
576	29.291064\\
577	29.196635\\
578	29.015536\\
579	28.848437\\
580	28.695709\\
581	28.55752\\
582	28.325965\\
583	28.025137\\
584	27.754706\\
585	27.51403\\
586	27.302279\\
587	27.118448\\
588	26.961041\\
589	26.828356\\
590	26.626122\\
591	26.452254\\
592	26.396944\\
593	26.357857\\
594	26.332862\\
595	26.319589\\
596	26.315382\\
597	26.317667\\
598	26.324122\\
599	26.332733\\
600	26.341813\\
601	26.34999\\
602	26.356195\\
603	26.451764\\
604	26.443166\\
605	26.430095\\
606	26.504977\\
607	26.567922\\
608	26.727197\\
609	26.758699\\
610	26.887756\\
611	26.99837\\
612	27.091399\\
613	27.260235\\
614	27.405772\\
615	27.621528\\
616	27.808871\\
617	28.061565\\
618	28.28079\\
619	28.468185\\
620	28.733437\\
621	29.054525\\
622	29.333425\\
623	29.572529\\
624	29.866812\\
625	30.118715\\
626	30.423554\\
627	30.684184\\
628	31.011746\\
629	31.385262\\
630	31.800734\\
631	32.161746\\
632	32.564373\\
633	32.912416\\
634	33.210114\\
635	33.569574\\
636	33.878091\\
637	34.232324\\
638	34.536273\\
639	34.886651\\
640	35.187486\\
641	35.44309\\
642	35.657976\\
643	35.929316\\
644	36.161532\\
645	36.467007\\
646	36.733213\\
647	37.056504\\
648	37.340722\\
649	37.589586\\
650	37.806837\\
651	38.088452\\
652	38.337868\\
653	38.558387\\
654	38.752996\\
655	38.924344\\
656	39.075109\\
657	39.207761\\
658	39.23244\\
659	39.344138\\
660	39.352777\\
661	39.360698\\
662	39.369818\\
663	39.381688\\
664	39.397849\\
665	39.419626\\
666	39.448059\\
667	39.483893\\
668	39.52758\\
669	39.579326\\
670	39.639121\\
671	39.706437\\
672	39.688363\\
673	39.776652\\
674	39.777784\\
675	39.790823\\
676	39.814736\\
677	39.848507\\
678	39.891161\\
679	39.833894\\
680	39.79258\\
681	39.766352\\
682	39.754366\\
683	39.755803\\
684	39.769874\\
685	39.79548\\
686	39.831783\\
687	39.877747\\
688	39.932299\\
689	39.994415\\
690	40.063145\\
691	40.045159\\
692	40.039354\\
693	40.044461\\
694	40.059251\\
695	39.990111\\
696	39.936079\\
697	39.988655\\
698	39.954398\\
699	39.93259\\
700	39.922494\\
701	39.923424\\
702	39.934744\\
703	39.955856\\
704	39.985851\\
705	40.023739\\
706	40.068568\\
707	40.119451\\
708	40.17524\\
709	40.142313\\
710	40.120018\\
711	39.99958\\
712	39.896842\\
713	39.810888\\
714	39.740816\\
715	39.593291\\
716	39.467547\\
717	39.362733\\
718	39.185525\\
719	39.035114\\
720	38.81809\\
721	38.633515\\
722	38.479942\\
723	38.247847\\
724	38.051865\\
725	37.88968\\
726	37.758932\\
727	37.564841\\
728	37.404907\\
729	37.27649\\
730	37.177018\\
731	37.103695\\
732	37.053773\\
733	36.931844\\
734	36.835157\\
735	36.761061\\
736	36.706749\\
737	36.669532\\
738	36.646973\\
739	36.636919\\
740	36.637157\\
741	36.645664\\
742	36.660683\\
743	36.680733\\
744	36.612131\\
745	36.553988\\
746	36.505166\\
747	36.464565\\
748	36.4312\\
749	36.404237\\
750	36.382981\\
751	36.366868\\
752	36.447915\\
753	36.433224\\
754	36.515102\\
755	36.593284\\
756	36.667671\\
757	36.737896\\
758	36.80355\\
759	36.956747\\
760	37.097154\\
761	37.22466\\
762	37.339298\\
763	37.441229\\
764	37.530712\\
765	37.700918\\
766	37.852336\\
767	37.985879\\
768	38.102626\\
769	38.203723\\
770	38.398211\\
771	38.462597\\
772	38.62302\\
773	38.764282\\
774	38.887941\\
775	38.995502\\
776	39.088369\\
777	39.167852\\
778	39.327976\\
779	39.470217\\
780	39.596082\\
781	39.70697\\
782	39.804171\\
783	39.889248\\
784	39.963369\\
785	40.027876\\
786	40.176547\\
787	40.31052\\
788	40.338443\\
789	40.361411\\
790	40.38029\\
791	40.39616\\
792	40.410087\\
793	40.423018\\
794	40.43576\\
795	40.448975\\
796	40.463167\\
797	40.478732\\
798	40.495935\\
799	40.515287\\
800	40.537319\\
};
\end{axis}
\end{tikzpicture}%
\caption{Sterowanie procesu z regulatorem DMC dla parametrów $D = 340$, $ N = 30$, $N_u = 1$,  $\lambda = 0,4$ bez pomiaru zakłócenia}
\end{figure}

\begin{equation}
E = 1,4576 * 10^3
\end{equation}

Włączamy pomiar zakłóceń.

\begin{figure}[H]
\centering
% This file was created by matlab2tikz.
%
%The latest updates can be retrieved from
%  http://www.mathworks.com/matlabcentral/fileexchange/22022-matlab2tikz-matlab2tikz
%where you can also make suggestions and rate matlab2tikz.
%
\definecolor{mycolor1}{rgb}{0.00000,0.44700,0.74100}%
%
\begin{tikzpicture}

\begin{axis}[%
width=4.521in,
height=3.566in,
at={(0.758in,0.481in)},
scale only axis,
xmin=0,
xmax=800,
xlabel style={font=\color{white!15!black}},
xlabel={k},
ymin=28,
ymax=36,
ylabel style={font=\color{white!15!black}},
ylabel={Y(k)},
axis background/.style={fill=white}
]
\addplot[const plot, color=mycolor1, forget plot] table[row sep=crcr] {%
1	28.12\\
2	28.12\\
3	28.12\\
4	28.06\\
5	28.06\\
6	28.06\\
7	28\\
8	28.06\\
9	28.06\\
10	28\\
11	28\\
12	28.06\\
13	28\\
14	28.06\\
15	28.06\\
16	28.06\\
17	28.06\\
18	28.06\\
19	28.06\\
20	28.06\\
21	28.06\\
22	28.12\\
23	28.06\\
24	28.12\\
25	28.12\\
26	28.12\\
27	28.12\\
28	28.12\\
29	28.12\\
30	28.18\\
31	28.18\\
32	28.18\\
33	28.18\\
34	28.18\\
35	28.25\\
36	28.25\\
37	28.25\\
38	28.25\\
39	28.31\\
40	28.31\\
41	28.31\\
42	28.31\\
43	28.31\\
44	28.37\\
45	28.37\\
46	28.37\\
47	28.37\\
48	28.37\\
49	28.37\\
50	28.37\\
51	28.43\\
52	28.43\\
53	28.43\\
54	28.43\\
55	28.43\\
56	28.43\\
57	28.43\\
58	28.43\\
59	28.43\\
60	28.43\\
61	28.43\\
62	28.43\\
63	28.37\\
64	28.43\\
65	28.43\\
66	28.43\\
67	28.43\\
68	28.43\\
69	28.43\\
70	28.43\\
71	28.43\\
72	28.43\\
73	28.43\\
74	28.37\\
75	28.37\\
76	28.37\\
77	28.37\\
78	28.37\\
79	28.31\\
80	28.31\\
81	28.31\\
82	28.31\\
83	28.31\\
84	28.25\\
85	28.31\\
86	28.31\\
87	28.25\\
88	28.25\\
89	28.25\\
90	28.25\\
91	28.25\\
92	28.25\\
93	28.25\\
94	28.25\\
95	28.25\\
96	28.25\\
97	28.25\\
98	28.25\\
99	28.25\\
100	28.25\\
101	28.25\\
102	28.25\\
103	28.25\\
104	28.25\\
105	28.25\\
106	28.25\\
107	28.25\\
108	28.25\\
109	28.25\\
110	28.25\\
111	28.31\\
112	28.31\\
113	28.37\\
114	28.43\\
115	28.5\\
116	28.56\\
117	28.68\\
118	28.75\\
119	28.87\\
120	29.06\\
121	29.18\\
122	29.31\\
123	29.5\\
124	29.62\\
125	29.87\\
126	30.06\\
127	30.25\\
128	30.5\\
129	30.68\\
130	30.87\\
131	31.06\\
132	31.31\\
133	31.56\\
134	31.68\\
135	31.93\\
136	32.12\\
137	32.31\\
138	32.5\\
139	32.75\\
140	32.87\\
141	33.12\\
142	33.25\\
143	33.43\\
144	33.62\\
145	33.81\\
146	33.93\\
147	34.12\\
148	34.25\\
149	34.37\\
150	34.56\\
151	34.68\\
152	34.81\\
153	34.93\\
154	35\\
155	35.12\\
156	35.18\\
157	35.25\\
158	35.31\\
159	35.37\\
160	35.43\\
161	35.5\\
162	35.56\\
163	35.56\\
164	35.62\\
165	35.62\\
166	35.68\\
167	35.68\\
168	35.68\\
169	35.75\\
170	35.68\\
171	35.68\\
172	35.68\\
173	35.68\\
174	35.68\\
175	35.68\\
176	35.62\\
177	35.62\\
178	35.62\\
179	35.56\\
180	35.56\\
181	35.5\\
182	35.5\\
183	35.43\\
184	35.43\\
185	35.37\\
186	35.37\\
187	35.37\\
188	35.31\\
189	35.25\\
190	35.25\\
191	35.18\\
192	35.12\\
193	35.12\\
194	35.06\\
195	35.06\\
196	35.06\\
197	35\\
198	35\\
199	34.93\\
200	34.93\\
201	34.93\\
202	34.87\\
203	34.81\\
204	34.81\\
205	34.81\\
206	34.75\\
207	34.75\\
208	34.75\\
209	34.75\\
210	34.68\\
211	34.68\\
212	34.68\\
213	34.68\\
214	34.62\\
215	34.68\\
216	34.62\\
217	34.62\\
218	34.62\\
219	34.62\\
220	34.62\\
221	34.62\\
222	34.62\\
223	34.62\\
224	34.62\\
225	34.62\\
226	34.62\\
227	34.62\\
228	34.62\\
229	34.68\\
230	34.68\\
231	34.68\\
232	34.68\\
233	34.68\\
234	34.75\\
235	34.75\\
236	34.75\\
237	34.75\\
238	34.75\\
239	34.75\\
240	34.75\\
241	34.81\\
242	34.81\\
243	34.81\\
244	34.81\\
245	34.87\\
246	34.81\\
247	34.87\\
248	34.87\\
249	34.87\\
250	34.93\\
251	34.93\\
252	34.93\\
253	34.93\\
254	34.93\\
255	35\\
256	35\\
257	35\\
258	35\\
259	35\\
260	35\\
261	35.06\\
262	35.06\\
263	35.06\\
264	35.06\\
265	35.06\\
266	35.06\\
267	35.06\\
268	35.06\\
269	35.06\\
270	35.06\\
271	35.12\\
272	35.12\\
273	35.12\\
274	35.12\\
275	35.12\\
276	35.18\\
277	35.12\\
278	35.12\\
279	35.12\\
280	35.12\\
281	35.12\\
282	35.12\\
283	35.12\\
284	35.12\\
285	35.12\\
286	35.12\\
287	35.06\\
288	35.12\\
289	35.06\\
290	35.06\\
291	35.12\\
292	35.12\\
293	35.06\\
294	35.06\\
295	35.06\\
296	35.06\\
297	35.06\\
298	35.06\\
299	35.06\\
300	35.06\\
301	35.06\\
302	35.06\\
303	35\\
304	35\\
305	35.06\\
306	35\\
307	35\\
308	35\\
309	35\\
310	35\\
311	35\\
312	35.06\\
313	35\\
314	35\\
315	35\\
316	35.06\\
317	35\\
318	35\\
319	35\\
320	35\\
321	35\\
322	35\\
323	35.06\\
324	35.06\\
325	35.06\\
326	35.06\\
327	35.06\\
328	35.06\\
329	35.06\\
330	35.06\\
331	35.06\\
332	35.06\\
333	35.06\\
334	35.12\\
335	35.12\\
336	35.12\\
337	35.12\\
338	35.12\\
339	35.12\\
340	35.12\\
341	35.12\\
342	35.12\\
343	35.12\\
344	35.12\\
345	35.18\\
346	35.12\\
347	35.18\\
348	35.12\\
349	35.12\\
350	35.12\\
351	35.18\\
352	35.12\\
353	35.12\\
354	35.12\\
355	35.18\\
356	35.18\\
357	35.18\\
358	35.18\\
359	35.18\\
360	35.12\\
361	35.12\\
362	35.18\\
363	35.18\\
364	35.18\\
365	35.18\\
366	35.25\\
367	35.25\\
368	35.25\\
369	35.25\\
370	35.31\\
371	35.31\\
372	35.37\\
373	35.37\\
374	35.43\\
375	35.43\\
376	35.43\\
377	35.5\\
378	35.5\\
379	35.56\\
380	35.56\\
381	35.56\\
382	35.56\\
383	35.56\\
384	35.5\\
385	35.5\\
386	35.5\\
387	35.5\\
388	35.5\\
389	35.5\\
390	35.43\\
391	35.43\\
392	35.37\\
393	35.37\\
394	35.31\\
395	35.31\\
396	35.25\\
397	35.25\\
398	35.18\\
399	35.18\\
400	35.18\\
401	35.12\\
402	35.12\\
403	35.06\\
404	35.06\\
405	35.06\\
406	35\\
407	35\\
408	34.93\\
409	34.93\\
410	34.87\\
411	34.87\\
412	34.81\\
413	34.75\\
414	34.81\\
415	34.75\\
416	34.75\\
417	34.68\\
418	34.68\\
419	34.62\\
420	34.62\\
421	34.62\\
422	34.62\\
423	34.56\\
424	34.56\\
425	34.56\\
426	34.56\\
427	34.56\\
428	34.5\\
429	34.5\\
430	34.5\\
431	34.5\\
432	34.5\\
433	34.5\\
434	34.5\\
435	34.56\\
436	34.56\\
437	34.56\\
438	34.56\\
439	34.56\\
440	34.56\\
441	34.56\\
442	34.56\\
443	34.56\\
444	34.56\\
445	34.62\\
446	34.62\\
447	34.62\\
448	34.62\\
449	34.68\\
450	34.68\\
451	34.68\\
452	34.68\\
453	34.75\\
454	34.75\\
455	34.81\\
456	34.81\\
457	34.81\\
458	34.81\\
459	34.81\\
460	34.87\\
461	34.87\\
462	34.87\\
463	34.87\\
464	34.93\\
465	34.93\\
466	34.93\\
467	34.93\\
468	35\\
469	35\\
470	35\\
471	35\\
472	35\\
473	35\\
474	35.06\\
475	35.06\\
476	35.06\\
477	35.06\\
478	35.06\\
479	35.06\\
480	35.06\\
481	35.06\\
482	35.06\\
483	35.12\\
484	35.12\\
485	35.12\\
486	35.06\\
487	35.12\\
488	35.12\\
489	35.12\\
490	35.12\\
491	35.12\\
492	35.12\\
493	35.12\\
494	35.12\\
495	35.12\\
496	35.12\\
497	35.12\\
498	35.12\\
499	35.12\\
500	35.12\\
501	35.12\\
502	35.12\\
503	35.12\\
504	35.12\\
505	35.06\\
506	35.12\\
507	35.12\\
508	35.12\\
509	35.12\\
510	35.12\\
511	35.12\\
512	35.12\\
513	35.12\\
514	35.12\\
515	35.06\\
516	35.12\\
517	35.06\\
518	35.06\\
519	35.06\\
520	35.06\\
521	35.06\\
522	35\\
523	35.06\\
524	35\\
525	35\\
526	35\\
527	35.06\\
528	35\\
529	35\\
530	35\\
531	35\\
532	35\\
533	35\\
534	35\\
535	35\\
536	35\\
537	35\\
538	35\\
539	35\\
540	35\\
541	35\\
542	35\\
543	35\\
544	35\\
545	35\\
546	35\\
547	35\\
548	35\\
549	35\\
550	35\\
551	35\\
552	35\\
553	35\\
554	34.93\\
555	35\\
556	35\\
557	35\\
558	35\\
559	35\\
560	35\\
561	35.06\\
562	35.06\\
563	35.06\\
564	35.06\\
565	35.06\\
566	35.06\\
567	35.06\\
568	35.06\\
569	35.06\\
570	35.06\\
571	35.06\\
572	35.12\\
573	35.12\\
574	35.12\\
575	35.12\\
576	35.12\\
577	35.12\\
578	35.12\\
579	35.12\\
580	35.18\\
581	35.18\\
582	35.18\\
583	35.18\\
584	35.12\\
585	35.12\\
586	35.12\\
587	35.12\\
588	35.12\\
589	35.12\\
590	35.12\\
591	35.12\\
592	35.18\\
593	35.12\\
594	35.12\\
595	35.12\\
596	35.12\\
597	35.12\\
598	35.12\\
599	35.06\\
600	35.12\\
601	35.12\\
602	35.12\\
603	35.12\\
604	35.12\\
605	35.12\\
606	35.06\\
607	35.12\\
608	35.12\\
609	35.12\\
610	35.12\\
611	35.12\\
612	35.06\\
613	35.06\\
614	35\\
615	35\\
616	35\\
617	34.93\\
618	34.93\\
619	34.93\\
620	34.93\\
621	34.93\\
622	34.87\\
623	34.87\\
624	34.87\\
625	34.87\\
626	34.87\\
627	34.87\\
628	34.87\\
629	34.81\\
630	34.81\\
631	34.81\\
632	34.75\\
633	34.75\\
634	34.75\\
635	34.75\\
636	34.75\\
637	34.75\\
638	34.75\\
639	34.75\\
640	34.75\\
641	34.75\\
642	34.81\\
643	34.81\\
644	34.81\\
645	34.81\\
646	34.87\\
647	34.87\\
648	34.87\\
649	34.87\\
650	34.93\\
651	34.93\\
652	34.93\\
653	34.93\\
654	35\\
655	35\\
656	35\\
657	35\\
658	35.06\\
659	35.06\\
660	35.06\\
661	35.12\\
662	35.18\\
663	35.18\\
664	35.18\\
665	35.25\\
666	35.25\\
667	35.25\\
668	35.31\\
669	35.31\\
670	35.37\\
671	35.37\\
672	35.37\\
673	35.43\\
674	35.37\\
675	35.43\\
676	35.43\\
677	35.5\\
678	35.5\\
679	35.5\\
680	35.5\\
681	35.5\\
682	35.5\\
683	35.5\\
684	35.56\\
685	35.5\\
686	35.5\\
687	35.5\\
688	35.5\\
689	35.5\\
690	35.5\\
691	35.5\\
692	35.5\\
693	35.5\\
694	35.5\\
695	35.5\\
696	35.43\\
697	35.43\\
698	35.43\\
699	35.37\\
700	35.37\\
701	35.37\\
702	35.37\\
703	35.31\\
704	35.37\\
705	35.37\\
706	35.31\\
707	35.31\\
708	35.31\\
709	35.31\\
710	35.25\\
711	35.31\\
712	35.25\\
713	35.25\\
714	35.25\\
715	35.25\\
716	35.25\\
717	35.18\\
718	35.18\\
719	35.18\\
720	35.18\\
721	35.12\\
722	35.12\\
723	35.12\\
724	35.12\\
725	35.06\\
726	35.06\\
727	35.06\\
728	35.06\\
729	35.06\\
730	35\\
731	35\\
732	35\\
733	35\\
734	35\\
735	34.93\\
736	35\\
737	35\\
738	35\\
739	35\\
740	35\\
741	35\\
742	35\\
743	34.93\\
744	35\\
745	35\\
746	35\\
747	34.93\\
748	34.93\\
749	34.93\\
750	34.93\\
751	34.93\\
752	34.93\\
753	34.93\\
754	34.93\\
755	35\\
756	34.93\\
757	35\\
758	35\\
759	34.93\\
760	35\\
761	35\\
762	35\\
763	35\\
764	35\\
765	35\\
766	35\\
767	35.06\\
768	35\\
769	35\\
770	35\\
771	35.06\\
772	35\\
773	35.06\\
774	35.06\\
775	35\\
776	35.06\\
777	35\\
778	35.06\\
779	35\\
780	35.06\\
781	35.06\\
782	35\\
783	35\\
784	35\\
785	35\\
786	35\\
787	35\\
788	35\\
789	35\\
790	35\\
791	35\\
792	34.93\\
793	34.93\\
794	34.93\\
795	34.93\\
796	34.93\\
797	34.93\\
798	34.93\\
799	34.93\\
800	34.93\\
};
\end{axis}
\end{tikzpicture}%
\caption{Wyjście procesu z regulatorem DMC dla parametrów $D = 340$, $ N = 30$, $N_u = 1$,  $\lambda = 0,4$ z pomiarem zakłócenia}
\end{figure}

\begin{figure}[H]
\centering
% This file was created by matlab2tikz.
%
%The latest updates can be retrieved from
%  http://www.mathworks.com/matlabcentral/fileexchange/22022-matlab2tikz-matlab2tikz
%where you can also make suggestions and rate matlab2tikz.
%
\definecolor{mycolor1}{rgb}{0.00000,0.44700,0.74100}%
%
\begin{tikzpicture}

\begin{axis}[%
width=4.521in,
height=3.566in,
at={(0.758in,0.481in)},
scale only axis,
xmin=0,
xmax=800,
xlabel style={font=\color{white!15!black}},
xlabel={k},
ymin=10,
ymax=100,
ylabel style={font=\color{white!15!black}},
ylabel={U(k},
axis background/.style={fill=white}
]
\addplot[const plot, color=mycolor1, forget plot] table[row sep=crcr] {%
1	25.092472\\
2	25.177351\\
3	25.254699\\
4	25.417096\\
5	25.56463\\
6	25.69755\\
7	25.908662\\
8	26.005831\\
9	26.089649\\
10	26.253156\\
11	26.396895\\
12	26.429099\\
13	26.543079\\
14	26.547414\\
15	26.543452\\
16	26.53245\\
17	26.515939\\
18	26.495458\\
19	26.472437\\
20	26.448487\\
21	26.424803\\
22	26.30982\\
23	26.297096\\
24	26.194977\\
25	26.103926\\
26	26.024325\\
27	25.956097\\
28	25.898892\\
29	25.85219\\
30	25.722894\\
31	25.610375\\
32	25.513837\\
33	25.432422\\
34	25.365234\\
35	25.203128\\
36	25.061817\\
37	24.93998\\
38	24.836204\\
39	24.656611\\
40	24.499909\\
41	24.364753\\
42	24.249832\\
43	24.153536\\
44	23.981725\\
45	23.832844\\
46	23.705315\\
47	23.59763\\
48	23.507964\\
49	23.434455\\
50	23.375324\\
51	23.236458\\
52	23.116093\\
53	23.012395\\
54	22.923631\\
55	22.848224\\
56	22.784754\\
57	22.731625\\
58	22.68728\\
59	22.650309\\
60	22.619472\\
61	22.593714\\
62	22.572149\\
63	22.64652\\
64	22.623358\\
65	22.601875\\
66	22.581309\\
67	22.56103\\
68	22.540546\\
69	22.519493\\
70	22.497622\\
71	22.474779\\
72	22.450893\\
73	22.425958\\
74	22.49249\\
75	22.550509\\
76	22.60053\\
77	22.642897\\
78	22.6779\\
79	22.798291\\
80	22.904295\\
81	22.996277\\
82	23.074665\\
83	23.139937\\
84	23.285087\\
85	23.31814\\
86	23.339811\\
87	23.443567\\
88	23.530458\\
89	23.601659\\
90	23.65834\\
91	23.701642\\
92	23.732994\\
93	23.753879\\
94	23.765715\\
95	23.769807\\
96	23.767328\\
97	23.759653\\
98	23.747787\\
99	23.732497\\
100	23.714723\\
101	34.206339\\
102	43.834154\\
103	52.605852\\
104	60.534325\\
105	67.637173\\
106	73.9362\\
107	79.45691\\
108	84.22802\\
109	88.280988\\
110	91.649564\\
111	94.276888\\
112	96.300094\\
113	97.664794\\
114	98.455823\\
115	98.754632\\
116	98.673968\\
117	98.210728\\
118	97.529703\\
119	96.625065\\
120	95.456244\\
121	94.197354\\
122	92.880891\\
123	91.454694\\
124	90.066076\\
125	88.537663\\
126	86.994256\\
127	85.454204\\
128	83.83787\\
129	82.26908\\
130	80.735572\\
131	79.238907\\
132	77.685869\\
133	76.081258\\
134	74.62894\\
135	73.114849\\
136	71.633602\\
137	70.179763\\
138	68.747541\\
139	67.238669\\
140	65.855528\\
141	64.383336\\
142	63.009528\\
143	61.64488\\
144	60.268784\\
145	58.87763\\
146	57.57585\\
147	56.244144\\
148	54.97284\\
149	53.768132\\
150	52.512267\\
151	51.312723\\
152	50.145006\\
153	49.01754\\
154	47.999405\\
155	46.999987\\
156	46.105217\\
157	45.286055\\
158	44.54606\\
159	43.873007\\
160	43.255304\\
161	42.666798\\
162	42.114392\\
163	41.681379\\
164	41.259315\\
165	40.933168\\
166	40.596243\\
167	40.335572\\
168	40.139013\\
169	39.887692\\
170	39.796353\\
171	39.739384\\
172	39.708981\\
173	39.698283\\
174	39.701193\\
175	39.712349\\
176	39.819839\\
177	39.919884\\
178	40.009656\\
179	40.179314\\
180	40.327151\\
181	40.544716\\
182	40.731274\\
183	40.994901\\
184	41.219637\\
185	41.49926\\
186	41.735385\\
187	41.93\\
188	42.177789\\
189	42.473869\\
190	42.721131\\
191	43.030548\\
192	43.381309\\
193	43.676942\\
194	44.013892\\
195	44.296108\\
196	44.52793\\
197	44.80617\\
198	45.03523\\
199	45.327382\\
200	45.570021\\
201	45.767388\\
202	46.016362\\
203	46.313601\\
204	46.563451\\
205	46.770506\\
206	47.031661\\
207	47.251275\\
208	47.433528\\
209	47.582261\\
210	47.80914\\
211	48.000922\\
212	48.161222\\
213	48.293455\\
214	48.493188\\
215	48.570986\\
216	48.72264\\
217	48.851115\\
218	48.959145\\
219	49.049492\\
220	49.124699\\
221	49.187013\\
222	49.23836\\
223	49.280757\\
224	49.316073\\
225	49.345942\\
226	49.371706\\
227	49.394791\\
228	49.416158\\
229	49.344319\\
230	49.280204\\
231	49.224441\\
232	49.177445\\
233	49.139417\\
234	49.002498\\
235	48.883288\\
236	48.781464\\
237	48.696543\\
238	48.62791\\
239	48.574843\\
240	48.536567\\
241	48.419763\\
242	48.323259\\
243	48.245652\\
244	48.1855\\
245	48.048889\\
246	48.026969\\
247	47.925439\\
248	47.842556\\
249	47.77655\\
250	47.633296\\
251	47.511316\\
252	47.409159\\
253	47.325458\\
254	47.258586\\
255	47.099022\\
256	46.961961\\
257	46.845908\\
258	46.749117\\
259	46.670159\\
260	46.607456\\
261	46.46699\\
262	46.347363\\
263	46.246835\\
264	46.163631\\
265	46.096061\\
266	46.042567\\
267	46.001748\\
268	45.971928\\
269	45.951437\\
270	45.938748\\
271	45.840041\\
272	45.75419\\
273	45.680146\\
274	45.616621\\
275	45.562338\\
276	45.423644\\
277	45.392031\\
278	45.366468\\
279	45.346149\\
280	45.330376\\
281	45.318591\\
282	45.310306\\
283	45.305123\\
284	45.302377\\
285	45.301361\\
286	45.301402\\
287	45.394404\\
288	45.387348\\
289	45.472104\\
290	45.548264\\
291	45.523252\\
292	45.497184\\
293	45.562642\\
294	45.619709\\
295	45.668625\\
296	45.709686\\
297	45.743243\\
298	45.769689\\
299	45.78945\\
300	45.803306\\
301	45.811795\\
302	45.815705\\
303	45.908338\\
304	45.99008\\
305	45.968805\\
306	46.03769\\
307	46.097349\\
308	46.1485\\
309	46.191824\\
310	46.227971\\
311	46.257525\\
312	46.188573\\
313	46.214148\\
314	46.234577\\
315	46.250232\\
316	46.169322\\
317	46.185142\\
318	46.19796\\
319	46.208278\\
320	46.216638\\
321	46.223536\\
322	46.2294\\
323	46.142122\\
324	46.062054\\
325	45.989007\\
326	45.922928\\
327	45.863785\\
328	45.811495\\
329	45.765594\\
330	45.725798\\
331	45.69187\\
332	45.663601\\
333	45.640744\\
334	45.530581\\
335	45.43289\\
336	45.346969\\
337	45.271969\\
338	45.206976\\
339	45.151106\\
340	45.103508\\
341	45.063397\\
342	45.03001\\
343	45.002652\\
344	44.980703\\
345	44.871105\\
346	44.865859\\
347	44.771553\\
348	44.779806\\
349	44.789637\\
350	44.800252\\
351	44.718477\\
352	43.858132\\
353	42.967275\\
354	42.050126\\
355	41.018757\\
356	39.977971\\
357	38.932526\\
358	37.886836\\
359	36.84545\\
360	35.905029\\
361	34.969773\\
362	33.951452\\
363	32.95421\\
364	31.981476\\
365	31.047957\\
366	30.056207\\
367	29.130266\\
368	28.274559\\
369	27.491277\\
370	26.688465\\
371	25.965311\\
372	25.227243\\
373	24.571621\\
374	23.902537\\
375	23.31559\\
376	22.80567\\
377	22.259335\\
378	21.787518\\
379	21.291426\\
380	20.864768\\
381	20.501187\\
382	20.194686\\
383	19.93928\\
384	19.821634\\
385	19.736046\\
386	19.677288\\
387	19.640373\\
388	19.620773\\
389	19.614512\\
390	19.725754\\
391	19.834434\\
392	20.029999\\
393	20.20987\\
394	20.464601\\
395	20.69274\\
396	20.98605\\
397	21.244601\\
398	21.577003\\
399	21.86771\\
400	22.1183\\
401	22.423078\\
402	22.684001\\
403	22.996126\\
404	23.2623\\
405	23.485847\\
406	23.762697\\
407	23.996586\\
408	24.299156\\
409	24.557668\\
410	24.86856\\
411	25.135859\\
412	25.456283\\
413	25.826289\\
414	26.057639\\
415	26.347073\\
416	26.598804\\
417	26.924846\\
418	27.212279\\
419	27.557488\\
420	27.864069\\
421	28.13585\\
422	28.376535\\
423	28.682242\\
424	28.956313\\
425	29.202239\\
426	29.42356\\
427	29.623201\\
428	29.896478\\
429	30.145758\\
430	30.373659\\
431	30.582585\\
432	30.774938\\
433	30.953059\\
434	31.118863\\
435	31.181495\\
436	31.242794\\
437	31.304205\\
438	31.366933\\
439	31.431874\\
440	31.499708\\
441	31.571194\\
442	31.645477\\
443	31.721788\\
444	31.799497\\
445	31.785453\\
446	31.779179\\
447	31.780125\\
448	31.787367\\
449	31.707522\\
450	31.639779\\
451	31.583238\\
452	31.537176\\
453	31.393018\\
454	31.266894\\
455	31.065825\\
456	30.889319\\
457	30.736745\\
458	30.607169\\
459	30.499376\\
460	30.319668\\
461	30.166922\\
462	30.039544\\
463	29.935752\\
464	29.761367\\
465	29.614768\\
466	29.493828\\
467	29.396381\\
468	29.212068\\
469	29.055168\\
470	28.923318\\
471	28.814296\\
472	28.72604\\
473	28.656322\\
474	28.510517\\
475	28.386729\\
476	28.283147\\
477	28.197767\\
478	28.128642\\
479	28.074017\\
480	28.03229\\
481	28.001676\\
482	27.98046\\
483	27.874554\\
484	27.782752\\
485	27.703788\\
486	27.729157\\
487	27.665027\\
488	27.610294\\
489	27.563864\\
490	27.524773\\
491	27.49224\\
492	27.4656\\
493	27.444254\\
494	27.427737\\
495	27.415652\\
496	27.40737\\
497	27.402104\\
498	27.399198\\
499	27.398406\\
500	27.399334\\
501	27.401517\\
502	27.404576\\
503	27.408229\\
504	27.412191\\
505	27.508703\\
506	27.50504\\
507	27.501285\\
508	27.497363\\
509	27.49328\\
510	27.489016\\
511	27.484645\\
512	27.480193\\
513	27.47574\\
514	27.471313\\
515	27.55942\\
516	27.547597\\
517	27.628492\\
518	27.702421\\
519	27.769679\\
520	27.830374\\
521	27.884692\\
522	28.025236\\
523	28.059622\\
524	28.180649\\
525	28.288479\\
526	28.383453\\
527	28.373454\\
528	28.45182\\
529	28.519017\\
530	28.575853\\
531	28.623231\\
532	28.662044\\
533	28.693143\\
534	28.717332\\
535	28.735735\\
536	28.749127\\
537	28.758453\\
538	28.764738\\
539	28.76878\\
540	28.771048\\
541	28.772029\\
542	28.772244\\
543	28.772137\\
544	28.772056\\
545	28.772222\\
546	28.772874\\
547	28.774117\\
548	28.776048\\
549	28.778615\\
550	28.78183\\
551	28.785607\\
552	28.789894\\
553	28.794576\\
554	28.907463\\
555	28.903846\\
556	28.900416\\
557	28.897146\\
558	28.893966\\
559	28.890913\\
560	28.887924\\
561	28.792521\\
562	28.704766\\
563	28.62464\\
564	28.55199\\
565	28.486669\\
566	28.42854\\
567	28.37778\\
568	28.334254\\
569	28.29768\\
570	28.267755\\
571	28.24408\\
572	28.133804\\
573	28.036551\\
574	27.95152\\
575	27.877723\\
576	27.814188\\
577	27.75991\\
578	27.713985\\
579	27.675566\\
580	27.551353\\
581	27.440745\\
582	27.342998\\
583	27.257432\\
584	27.275848\\
585	27.297152\\
586	27.320395\\
587	27.344632\\
588	27.36904\\
589	27.392877\\
590	27.415624\\
591	27.436833\\
592	27.363692\\
593	27.388189\\
594	27.409536\\
595	27.427196\\
596	27.440718\\
597	27.450117\\
598	27.455681\\
599	27.55023\\
600	27.541628\\
601	27.530454\\
602	28.102592\\
603	28.692779\\
604	29.298313\\
605	29.916039\\
606	30.635342\\
607	31.260696\\
608	31.889218\\
609	32.518037\\
610	33.144283\\
611	33.765217\\
612	34.471022\\
613	35.159194\\
614	35.919958\\
615	36.643459\\
616	37.322998\\
617	38.061835\\
618	38.740372\\
619	39.357478\\
620	39.912964\\
621	40.407637\\
622	40.935498\\
623	41.398758\\
624	41.800202\\
625	42.143538\\
626	42.432928\\
627	42.673128\\
628	42.869153\\
629	43.118356\\
630	43.32578\\
631	43.496505\\
632	43.727743\\
633	43.923756\\
634	44.088566\\
635	44.226094\\
636	44.340023\\
637	44.433535\\
638	44.509503\\
639	44.570316\\
640	44.618039\\
641	44.654389\\
642	44.588681\\
643	44.522657\\
644	44.457771\\
645	44.395467\\
646	44.244604\\
647	44.106328\\
648	43.981328\\
649	43.870059\\
650	43.680197\\
651	43.51178\\
652	43.36444\\
653	43.237726\\
654	43.023057\\
655	42.835948\\
656	42.674969\\
657	42.538523\\
658	42.332426\\
659	42.154739\\
660	42.003315\\
661	41.78352\\
662	41.50083\\
663	41.252965\\
664	41.037424\\
665	40.74386\\
666	40.486701\\
667	40.263297\\
668	39.978435\\
669	39.72951\\
670	39.421561\\
671	39.151873\\
672	38.917664\\
673	38.622372\\
674	38.455799\\
675	38.222277\\
676	38.018849\\
677	37.734761\\
678	37.483657\\
679	37.262552\\
680	37.068596\\
681	36.898846\\
682	36.750542\\
683	36.620742\\
684	36.414283\\
685	36.321302\\
686	36.239165\\
687	36.165926\\
688	36.099682\\
689	36.038735\\
690	35.981225\\
691	35.92544\\
692	35.869946\\
693	35.813707\\
694	35.755902\\
695	35.69603\\
696	35.741648\\
697	35.775539\\
698	35.797719\\
699	35.900904\\
700	35.985468\\
701	36.052069\\
702	36.101446\\
703	36.226964\\
704	36.237058\\
705	36.232812\\
706	36.307762\\
707	36.362928\\
708	36.39938\\
709	36.41867\\
710	36.514918\\
711	36.497213\\
712	36.560039\\
713	36.605308\\
714	36.634866\\
715	36.650442\\
716	36.653984\\
717	36.754894\\
718	36.837705\\
719	36.90391\\
720	36.95499\\
721	37.084808\\
722	37.194575\\
723	37.285843\\
724	37.359858\\
725	37.510449\\
726	37.638914\\
727	37.746611\\
728	37.834849\\
729	37.904852\\
730	38.050688\\
731	38.173914\\
732	38.276164\\
733	38.35904\\
734	38.424308\\
735	38.581727\\
736	38.608366\\
737	38.622657\\
738	38.62644\\
739	38.621494\\
740	38.609458\\
741	38.59176\\
742	38.569623\\
743	38.652338\\
744	38.616581\\
745	38.580307\\
746	38.544494\\
747	38.617852\\
748	38.684619\\
749	38.745393\\
750	38.800558\\
751	38.850392\\
752	38.895064\\
753	38.934654\\
754	38.969239\\
755	38.891024\\
756	38.925103\\
757	38.846924\\
758	38.773266\\
759	38.811921\\
760	38.738544\\
761	38.670273\\
762	38.607511\\
763	38.550569\\
764	38.499697\\
765	38.455052\\
766	38.416679\\
767	38.292074\\
768	38.273308\\
769	38.259918\\
770	38.251267\\
771	38.154076\\
772	38.160502\\
773	38.077331\\
774	38.003875\\
775	38.03185\\
776	37.96796\\
777	38.004111\\
778	37.947133\\
779	37.989044\\
780	37.936482\\
781	37.889028\\
782	37.938806\\
783	37.985491\\
784	38.028591\\
785	38.067895\\
786	38.103073\\
787	38.13379\\
788	38.160047\\
789	38.1818\\
790	38.199245\\
791	38.212453\\
792	38.329634\\
793	38.434157\\
794	38.526149\\
795	38.606159\\
796	38.674865\\
797	38.733046\\
798	38.78147\\
799	38.82096\\
800	38.852245\\
};
\end{axis}
\end{tikzpicture}%
\caption{Sterowanie procesu z regulatorem DMC dla parametrów $D = 340$, $ N = 30$, $N_u = 1$,  $\lambda = 0,4$ z pomiarem zakłócenia}
\end{figure}

\begin{equation}
E = 1,2734 * 10^3
\end{equation}

Jak widać, regulacja z pomiarem jest bardziej precyzyjna.

