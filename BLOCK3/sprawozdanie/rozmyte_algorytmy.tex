\chapter{Algorytmy rozmytej regulacji}

W przypadku obiektów nieliniowych użycie zwykłego regulatora liniowego może nie przyność zadowalających wyników (co widać w poprzednim podpunkcie). Rozwiązaniem w takich obiektach może być użycie nieliniowego regulatora rozmytego, który swój sposób działania opiera na zastosowaniu wielu prostych regulatorów liniowych (PID, DMC) pracyjących lokalnie (działających w pewnym obszarze).

Pierwszym krokiem projektowania jest wybranie zmiennej wybierającej (w wykonanych doświadczeniach jest to wartość sterowania u(k-1), użyta ze względu na odgórną znajomość granic tejże wartości). Następnie należy określić bazę reguł rozmytych:

$Reg^1$: jeżeli $u(k-1)\in U^1$, to $u^1(k)= \dots$ 


\t\vdots 


$Reg^r$: jeżeli $u(k-1)\in U^r$, to $u^r(k)= \dots$ 


Gdzie $U^1, \dots U^r$ to przedziały, w których pracuje lokalny regulator, a sterowanie obliczane jest z odpowiedniego wzoru (w zależności od tego, jakich regulatorów użyto).

Kolejnym krokiem przy projektowaniu jest określenie współczynników przynależności ($w^i$) zależnych od wybranej funkcji przynależności (w eksperymentach jest to funkcja trapezoidalna użyta ze względu na fakt, iż chatakterystyka statyczna w pewnych przedziałach jest niemal liniowa i może wtedy działać tylko jeden regulator lokalny). Współczynniki te pozwalają określić w jakim stopniu dany lokalny regulator wpływa na wartość sterowania.

Ostatecznie wzór na wartość sterowania w chwili k, przedstawia się następująco:

\begin{equation}
u(k)= \frac{\sum_{i=1}^{r}w^i(k)u^(k)}{\sum_{i=1}^{r}w^i(k)}
\end{equation}

%\begin{verbatim}
%tu jakiś kod ewentualnie
%\end{verbatim}