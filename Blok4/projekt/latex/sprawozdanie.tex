%! TEX encoding = utf8
\documentclass[a4paper,titlepage,11pt,twosides,floatssmall]{mwrep}
\usepackage[left=2.5cm,right=2.5cm,top=2.5cm,bottom=2.5cm]{geometry}
\usepackage[OT1]{fontenc}
\usepackage{polski}
\usepackage{amsmath}
\usepackage{amsfonts}
\usepackage{amssymb}
\usepackage{graphicx}
\usepackage{float}
\usepackage{url}
\usepackage{tikz}
\usetikzlibrary{arrows,calc,decorations.markings,math,arrows.meta}
\usepackage{rotating}
\usepackage[percent]{overpic}
\usepackage[utf8]{inputenc}
\usepackage{xcolor}
\usepackage{colortbl}
\usepackage{pgfplots}
\usetikzlibrary{pgfplots.groupplots}
\usepackage{listings}
\usepackage{matlab-prettifier}
\usepackage{enumitem,amssymb}
\definecolor{szary}{rgb}{0.95,0.95,0.95}
\usepackage{siunitx}
\usepackage{etoolbox}
\makeatletter
\patchcmd{\chapter}{\if@openright\cleardoublepage\else\clearpage\fi}{}{}{}
\makeatother
\sisetup{detect-weight,exponent-product=\cdot,output-decimal-marker={,},per-mode=symbol,binary-units=true,range-phrase={-},range-units=single}
\SendSettingsToPgf
%konfiguracje pakietu listings
\lstset{
	backgroundcolor=\color{szary},
	frame=single,
	breaklines=true,
}
\lstdefinestyle{customlatex}{
	basicstyle=\footnotesize\ttfamily,
	%basicstyle=\small\ttfamily,
}
\lstdefinestyle{customc}{
	breaklines=true,
	frame=tb,
	language=C,
	xleftmargin=0pt,
	showstringspaces=false,
	basicstyle=\small\ttfamily,
	keywordstyle=\bfseries\color{green!40!black},
	commentstyle=\itshape\color{purple!40!black},
	identifierstyle=\color{blue},
	stringstyle=\color{orange},
}
\lstdefinestyle{custommatlab}{
	captionpos=t,
	breaklines=true,
	frame=tb,
	xleftmargin=0pt,
	language=matlab,
	showstringspaces=false,
	basicstyle=\footnotesize\ttfamily,
	%basicstyle=\scriptsize\ttfamily,
	keywordstyle=\bfseries\color{green!40!black},
	commentstyle=\itshape\color{purple!40!black},
	identifierstyle=\color{blue},
	stringstyle=\color{orange},
}

%wymiar tekstu (bez �ywej paginy)
\textwidth 160mm \textheight 247mm

%ustawienia pakietu pgfplots
\pgfplotsset{
tick label style={font=\scriptsize},
label style={font=\small},
legend style={font=\small},
title style={font=\small}
}

\def\figurename{Rys.}
\def\tablename{Tab.}

%konfiguracja liczby p�ywaj�cych element�w
\setcounter{topnumber}{0}%2
\setcounter{bottomnumber}{3}%1
\setcounter{totalnumber}{5}%3
\renewcommand{\textfraction}{0.01}%0.2
\renewcommand{\topfraction}{0.95}%0.7
\renewcommand{\bottomfraction}{0.95}%0.3
\renewcommand{\floatpagefraction}{0.35}%0.5

\begin{document}
\frenchspacing
\pagestyle{uheadings}

%strona tytu�owa
\title{\bf Sprawozdanie z projektu i ćwiczenia laboratoryjnego nr 3, zadanie nr 10\vskip 0.1cm}
\author{Stanislau Stankevich, Rafał Bednarz, Ostrysz Jakub}
\date{2021}

\makeatletter
\renewcommand{\maketitle}{\begin{titlepage}
\begin{center}{\LARGE {\bf
Wydział Elektroniki i Technik Informacyjnych}}\\
\vspace{0.4cm}
{\LARGE {\bf Politechnika Warszawska}}\\
\vspace{0.3cm}
\end{center}
\vspace{5cm}
\begin{center}
{\bf \LARGE Projektowanie układów sterowania\\ (projekt grupowy) \vskip 0.1cm}
\end{center}
\vspace{1cm}
\begin{center}
{\bf \LARGE \@title}
\end{center}
\vspace{2cm}
\begin{center}
{\bf \Large \@author \par}
\end{center}
\vspace*{\stretch{6}}
\begin{center}
\bf{\large{Warszawa, \@date\vskip 0.1cm}}
\end{center}
\end{titlepage}
}
\makeatother

\maketitle

\tableofcontents

\chapter{Sprawdzenie punktu pracy}

Podając za wejścia same zera, po 300 iteracjach dostajemy następujący przebieg wyjść:

\begin{figure}[H]
\centering
% This file was created by matlab2tikz.
%
%The latest updates can be retrieved from
%  http://www.mathworks.com/matlabcentral/fileexchange/22022-matlab2tikz-matlab2tikz
%where you can also make suggestions and rate matlab2tikz.
%
\definecolor{mycolor1}{rgb}{0.00000,0.44700,0.74100}%
%
\begin{tikzpicture}

\begin{axis}[%
width=4.577in,
height=1.274in,
at={(0.768in,4.189in)},
scale only axis,
xmin=0,
xmax=300,
ymin=-1,
ymax=1,
ylabel style={font=\color{white!15!black}},
ylabel={y1},
axis background/.style={fill=white}
]
\addplot[const plot, color=green, forget plot] table[row sep=crcr] {%
1	0\\
2	0\\
3	0\\
4	0\\
5	0\\
6	0\\
7	0\\
8	0\\
9	0\\
10	0\\
11	0\\
12	0\\
13	0\\
14	0\\
15	0\\
16	0\\
17	0\\
18	0\\
19	0\\
20	0\\
21	0\\
22	0\\
23	0\\
24	0\\
25	0\\
26	0\\
27	0\\
28	0\\
29	0\\
30	0\\
31	0\\
32	0\\
33	0\\
34	0\\
35	0\\
36	0\\
37	0\\
38	0\\
39	0\\
40	0\\
41	0\\
42	0\\
43	0\\
44	0\\
45	0\\
46	0\\
47	0\\
48	0\\
49	0\\
50	0\\
51	0\\
52	0\\
53	0\\
54	0\\
55	0\\
56	0\\
57	0\\
58	0\\
59	0\\
60	0\\
61	0\\
62	0\\
63	0\\
64	0\\
65	0\\
66	0\\
67	0\\
68	0\\
69	0\\
70	0\\
71	0\\
72	0\\
73	0\\
74	0\\
75	0\\
76	0\\
77	0\\
78	0\\
79	0\\
80	0\\
81	0\\
82	0\\
83	0\\
84	0\\
85	0\\
86	0\\
87	0\\
88	0\\
89	0\\
90	0\\
91	0\\
92	0\\
93	0\\
94	0\\
95	0\\
96	0\\
97	0\\
98	0\\
99	0\\
100	0\\
101	0\\
102	0\\
103	0\\
104	0\\
105	0\\
106	0\\
107	0\\
108	0\\
109	0\\
110	0\\
111	0\\
112	0\\
113	0\\
114	0\\
115	0\\
116	0\\
117	0\\
118	0\\
119	0\\
120	0\\
121	0\\
122	0\\
123	0\\
124	0\\
125	0\\
126	0\\
127	0\\
128	0\\
129	0\\
130	0\\
131	0\\
132	0\\
133	0\\
134	0\\
135	0\\
136	0\\
137	0\\
138	0\\
139	0\\
140	0\\
141	0\\
142	0\\
143	0\\
144	0\\
145	0\\
146	0\\
147	0\\
148	0\\
149	0\\
150	0\\
151	0\\
152	0\\
153	0\\
154	0\\
155	0\\
156	0\\
157	0\\
158	0\\
159	0\\
160	0\\
161	0\\
162	0\\
163	0\\
164	0\\
165	0\\
166	0\\
167	0\\
168	0\\
169	0\\
170	0\\
171	0\\
172	0\\
173	0\\
174	0\\
175	0\\
176	0\\
177	0\\
178	0\\
179	0\\
180	0\\
181	0\\
182	0\\
183	0\\
184	0\\
185	0\\
186	0\\
187	0\\
188	0\\
189	0\\
190	0\\
191	0\\
192	0\\
193	0\\
194	0\\
195	0\\
196	0\\
197	0\\
198	0\\
199	0\\
200	0\\
201	0\\
202	0\\
203	0\\
204	0\\
205	0\\
206	0\\
207	0\\
208	0\\
209	0\\
210	0\\
211	0\\
212	0\\
213	0\\
214	0\\
215	0\\
216	0\\
217	0\\
218	0\\
219	0\\
220	0\\
221	0\\
222	0\\
223	0\\
224	0\\
225	0\\
226	0\\
227	0\\
228	0\\
229	0\\
230	0\\
231	0\\
232	0\\
233	0\\
234	0\\
235	0\\
236	0\\
237	0\\
238	0\\
239	0\\
240	0\\
241	0\\
242	0\\
243	0\\
244	0\\
245	0\\
246	0\\
247	0\\
248	0\\
249	0\\
250	0\\
251	0\\
252	0\\
253	0\\
254	0\\
255	0\\
256	0\\
257	0\\
258	0\\
259	0\\
260	0\\
261	0\\
262	0\\
263	0\\
264	0\\
265	0\\
266	0\\
267	0\\
268	0\\
269	0\\
270	0\\
271	0\\
272	0\\
273	0\\
274	0\\
275	0\\
276	0\\
277	0\\
278	0\\
279	0\\
280	0\\
281	0\\
282	0\\
283	0\\
284	0\\
285	0\\
286	0\\
287	0\\
288	0\\
289	0\\
290	0\\
291	0\\
292	0\\
293	0\\
294	0\\
295	0\\
296	0\\
297	0\\
298	0\\
299	0\\
300	0\\
};
\end{axis}

\begin{axis}[%
width=4.577in,
height=1.274in,
at={(0.768in,2.419in)},
scale only axis,
xmin=0,
xmax=300,
ymin=-1,
ymax=1,
ylabel style={font=\color{white!15!black}},
ylabel={y2},
axis background/.style={fill=white}
]
\addplot[const plot, color=red, forget plot] table[row sep=crcr] {%
1	0\\
2	0\\
3	0\\
4	0\\
5	0\\
6	0\\
7	0\\
8	0\\
9	0\\
10	0\\
11	0\\
12	0\\
13	0\\
14	0\\
15	0\\
16	0\\
17	0\\
18	0\\
19	0\\
20	0\\
21	0\\
22	0\\
23	0\\
24	0\\
25	0\\
26	0\\
27	0\\
28	0\\
29	0\\
30	0\\
31	0\\
32	0\\
33	0\\
34	0\\
35	0\\
36	0\\
37	0\\
38	0\\
39	0\\
40	0\\
41	0\\
42	0\\
43	0\\
44	0\\
45	0\\
46	0\\
47	0\\
48	0\\
49	0\\
50	0\\
51	0\\
52	0\\
53	0\\
54	0\\
55	0\\
56	0\\
57	0\\
58	0\\
59	0\\
60	0\\
61	0\\
62	0\\
63	0\\
64	0\\
65	0\\
66	0\\
67	0\\
68	0\\
69	0\\
70	0\\
71	0\\
72	0\\
73	0\\
74	0\\
75	0\\
76	0\\
77	0\\
78	0\\
79	0\\
80	0\\
81	0\\
82	0\\
83	0\\
84	0\\
85	0\\
86	0\\
87	0\\
88	0\\
89	0\\
90	0\\
91	0\\
92	0\\
93	0\\
94	0\\
95	0\\
96	0\\
97	0\\
98	0\\
99	0\\
100	0\\
101	0\\
102	0\\
103	0\\
104	0\\
105	0\\
106	0\\
107	0\\
108	0\\
109	0\\
110	0\\
111	0\\
112	0\\
113	0\\
114	0\\
115	0\\
116	0\\
117	0\\
118	0\\
119	0\\
120	0\\
121	0\\
122	0\\
123	0\\
124	0\\
125	0\\
126	0\\
127	0\\
128	0\\
129	0\\
130	0\\
131	0\\
132	0\\
133	0\\
134	0\\
135	0\\
136	0\\
137	0\\
138	0\\
139	0\\
140	0\\
141	0\\
142	0\\
143	0\\
144	0\\
145	0\\
146	0\\
147	0\\
148	0\\
149	0\\
150	0\\
151	0\\
152	0\\
153	0\\
154	0\\
155	0\\
156	0\\
157	0\\
158	0\\
159	0\\
160	0\\
161	0\\
162	0\\
163	0\\
164	0\\
165	0\\
166	0\\
167	0\\
168	0\\
169	0\\
170	0\\
171	0\\
172	0\\
173	0\\
174	0\\
175	0\\
176	0\\
177	0\\
178	0\\
179	0\\
180	0\\
181	0\\
182	0\\
183	0\\
184	0\\
185	0\\
186	0\\
187	0\\
188	0\\
189	0\\
190	0\\
191	0\\
192	0\\
193	0\\
194	0\\
195	0\\
196	0\\
197	0\\
198	0\\
199	0\\
200	0\\
201	0\\
202	0\\
203	0\\
204	0\\
205	0\\
206	0\\
207	0\\
208	0\\
209	0\\
210	0\\
211	0\\
212	0\\
213	0\\
214	0\\
215	0\\
216	0\\
217	0\\
218	0\\
219	0\\
220	0\\
221	0\\
222	0\\
223	0\\
224	0\\
225	0\\
226	0\\
227	0\\
228	0\\
229	0\\
230	0\\
231	0\\
232	0\\
233	0\\
234	0\\
235	0\\
236	0\\
237	0\\
238	0\\
239	0\\
240	0\\
241	0\\
242	0\\
243	0\\
244	0\\
245	0\\
246	0\\
247	0\\
248	0\\
249	0\\
250	0\\
251	0\\
252	0\\
253	0\\
254	0\\
255	0\\
256	0\\
257	0\\
258	0\\
259	0\\
260	0\\
261	0\\
262	0\\
263	0\\
264	0\\
265	0\\
266	0\\
267	0\\
268	0\\
269	0\\
270	0\\
271	0\\
272	0\\
273	0\\
274	0\\
275	0\\
276	0\\
277	0\\
278	0\\
279	0\\
280	0\\
281	0\\
282	0\\
283	0\\
284	0\\
285	0\\
286	0\\
287	0\\
288	0\\
289	0\\
290	0\\
291	0\\
292	0\\
293	0\\
294	0\\
295	0\\
296	0\\
297	0\\
298	0\\
299	0\\
300	0\\
};
\end{axis}

\begin{axis}[%
width=4.577in,
height=1.274in,
at={(0.768in,0.65in)},
scale only axis,
xmin=0,
xmax=300,
xlabel style={font=\color{white!15!black}},
xlabel={k},
ymin=-1,
ymax=1,
ylabel style={font=\color{white!15!black}},
ylabel={y3},
axis background/.style={fill=white}
]
\addplot[const plot, color=mycolor1, forget plot] table[row sep=crcr] {%
1	0\\
2	0\\
3	0\\
4	0\\
5	0\\
6	0\\
7	0\\
8	0\\
9	0\\
10	0\\
11	0\\
12	0\\
13	0\\
14	0\\
15	0\\
16	0\\
17	0\\
18	0\\
19	0\\
20	0\\
21	0\\
22	0\\
23	0\\
24	0\\
25	0\\
26	0\\
27	0\\
28	0\\
29	0\\
30	0\\
31	0\\
32	0\\
33	0\\
34	0\\
35	0\\
36	0\\
37	0\\
38	0\\
39	0\\
40	0\\
41	0\\
42	0\\
43	0\\
44	0\\
45	0\\
46	0\\
47	0\\
48	0\\
49	0\\
50	0\\
51	0\\
52	0\\
53	0\\
54	0\\
55	0\\
56	0\\
57	0\\
58	0\\
59	0\\
60	0\\
61	0\\
62	0\\
63	0\\
64	0\\
65	0\\
66	0\\
67	0\\
68	0\\
69	0\\
70	0\\
71	0\\
72	0\\
73	0\\
74	0\\
75	0\\
76	0\\
77	0\\
78	0\\
79	0\\
80	0\\
81	0\\
82	0\\
83	0\\
84	0\\
85	0\\
86	0\\
87	0\\
88	0\\
89	0\\
90	0\\
91	0\\
92	0\\
93	0\\
94	0\\
95	0\\
96	0\\
97	0\\
98	0\\
99	0\\
100	0\\
101	0\\
102	0\\
103	0\\
104	0\\
105	0\\
106	0\\
107	0\\
108	0\\
109	0\\
110	0\\
111	0\\
112	0\\
113	0\\
114	0\\
115	0\\
116	0\\
117	0\\
118	0\\
119	0\\
120	0\\
121	0\\
122	0\\
123	0\\
124	0\\
125	0\\
126	0\\
127	0\\
128	0\\
129	0\\
130	0\\
131	0\\
132	0\\
133	0\\
134	0\\
135	0\\
136	0\\
137	0\\
138	0\\
139	0\\
140	0\\
141	0\\
142	0\\
143	0\\
144	0\\
145	0\\
146	0\\
147	0\\
148	0\\
149	0\\
150	0\\
151	0\\
152	0\\
153	0\\
154	0\\
155	0\\
156	0\\
157	0\\
158	0\\
159	0\\
160	0\\
161	0\\
162	0\\
163	0\\
164	0\\
165	0\\
166	0\\
167	0\\
168	0\\
169	0\\
170	0\\
171	0\\
172	0\\
173	0\\
174	0\\
175	0\\
176	0\\
177	0\\
178	0\\
179	0\\
180	0\\
181	0\\
182	0\\
183	0\\
184	0\\
185	0\\
186	0\\
187	0\\
188	0\\
189	0\\
190	0\\
191	0\\
192	0\\
193	0\\
194	0\\
195	0\\
196	0\\
197	0\\
198	0\\
199	0\\
200	0\\
201	0\\
202	0\\
203	0\\
204	0\\
205	0\\
206	0\\
207	0\\
208	0\\
209	0\\
210	0\\
211	0\\
212	0\\
213	0\\
214	0\\
215	0\\
216	0\\
217	0\\
218	0\\
219	0\\
220	0\\
221	0\\
222	0\\
223	0\\
224	0\\
225	0\\
226	0\\
227	0\\
228	0\\
229	0\\
230	0\\
231	0\\
232	0\\
233	0\\
234	0\\
235	0\\
236	0\\
237	0\\
238	0\\
239	0\\
240	0\\
241	0\\
242	0\\
243	0\\
244	0\\
245	0\\
246	0\\
247	0\\
248	0\\
249	0\\
250	0\\
251	0\\
252	0\\
253	0\\
254	0\\
255	0\\
256	0\\
257	0\\
258	0\\
259	0\\
260	0\\
261	0\\
262	0\\
263	0\\
264	0\\
265	0\\
266	0\\
267	0\\
268	0\\
269	0\\
270	0\\
271	0\\
272	0\\
273	0\\
274	0\\
275	0\\
276	0\\
277	0\\
278	0\\
279	0\\
280	0\\
281	0\\
282	0\\
283	0\\
284	0\\
285	0\\
286	0\\
287	0\\
288	0\\
289	0\\
290	0\\
291	0\\
292	0\\
293	0\\
294	0\\
295	0\\
296	0\\
297	0\\
298	0\\
299	0\\
300	0\\
};
\end{axis}
\end{tikzpicture}%
\caption{Przebieg wyjść obiektu przy stałyćh wejściach: $u_1 = 0, u_2 = 0, u_3 = 0$}
\end{figure}

Każde wyjście ustabilizowało się na wartości 0, więc podany w zadaniu punkt pracy jest zgodny z rzeczywistością.
\chapter{Odpowiedzi skokowe poszczególnych torów}

\begin{figure}[H]
    \centering
    % This file was created by matlab2tikz.
%
%The latest updates can be retrieved from
%  http://www.mathworks.com/matlabcentral/fileexchange/22022-matlab2tikz-matlab2tikz
%where you can also make suggestions and rate matlab2tikz.
%
\definecolor{mycolor1}{rgb}{0.00000,0.44700,0.74100}%
%
\begin{tikzpicture}

\begin{axis}[%
width=1.68in,
height=1.553in,
at={(1.024in,7.552in)},
scale only axis,
xmin=0,
xmax=300,
ymin=0,
ymax=2,
axis background/.style={fill=white},
title style={font=\bfseries},
title={Tor u1 - y1}
]
\addplot[const plot, color=mycolor1, forget plot] table[row sep=crcr] {%
1	0\\
2	0\\
3	0\\
4	0\\
5	0\\
6	0.016391949758997\\
7	0.0322465074841911\\
8	0.0475812909820203\\
9	0.0624133404785264\\
10	0.0767591375546932\\
11	0.0906346234610095\\
12	0.10405521683161\\
13	0.117035830817676\\
14	0.129590889659142\\
15	0.141734344713106\\
16	0.15347968995678\\
17	0.164839976982182\\
18	0.175827829499246\\
19	0.186455457363473\\
20	0.196734670143684\\
21	0.206676890244985\\
22	0.216293165601498\\
23	0.225594181952986\\
24	0.234590274718991\\
25	0.243291440483701\\
26	0.251707348104291\\
27	0.259847349455094\\
28	0.267720489819536\\
29	0.275335517941379\\
30	0.282700895746449\\
31	0.289824807745644\\
32	0.296715170129683\\
33	0.303379639565681\\
34	0.309825621705348\\
35	0.316060279414253\\
36	0.3220905407313\\
37	0.327923106567261\\
38	0.333564458150924\\
39	0.339020864231123\\
40	0.344298388042659\\
41	0.349402894043855\\
42	0.354340054433218\\
43	0.359115355452472\\
44	0.363734103482943\\
45	0.368201430942085\\
46	0.372522301986692\\
47	0.376701518029143\\
48	0.380743723072795\\
49	0.384653408872464\\
50	0.38843491992573\\
51	0.3920924583006\\
52	0.395630088304905\\
53	0.399051741002618\\
54	0.402361218582103\\
55	0.405562198581165\\
56	0.40865823797358\\
57	0.411652777121649\\
58	0.414549143599185\\
59	0.417350555889155\\
60	0.420060126960102\\
61	0.422680867725322\\
62	0.425215690388633\\
63	0.427667411680455\\
64	0.430038755987799\\
65	0.43233235838165\\
66	0.434550767545086\\
67	0.436696448605419\\
68	0.438771785873472\\
69	0.440779085493064\\
70	0.442720578003624\\
71	0.444598420818803\\
72	0.446414700623819\\
73	0.448171435694211\\
74	0.449870578138576\\
75	0.451514016067778\\
76	0.453103575693045\\
77	0.454641023355279\\
78	0.456128067487841\\
79	0.457566360514982\\
80	0.458957500688044\\
81	0.460303033861457\\
82	0.461604455210515\\
83	0.462863210892835\\
84	0.464080699655348\\
85	0.465258274388606\\
86	0.466397243630134\\
87	0.467498873018494\\
88	0.46856438669968\\
89	0.469594968687406\\
90	0.470591764178802\\
91	0.471555880826967\\
92	0.472488389971815\\
93	0.473390327830558\\
94	0.474262696649169\\
95	0.475106465816087\\
96	0.47592257293942\\
97	0.476711924888825\\
98	0.477475398803237\\
99	0.478213843065554\\
100	0.478928078245375\\
101	0.479618898010829\\
102	0.480287070010508\\
103	0.480933336726488\\
104	0.48155841629939\\
105	0.482163003326382\\
106	0.482747769633032\\
107	0.483313365019843\\
108	0.483860417984325\\
109	0.484389536419389\\
110	0.484901308288844\\
111	0.485396302280755\\
112	0.485875068439373\\
113	0.486338138776354\\
114	0.486786027861937\\
115	0.487219233396744\\
116	0.487638236764826\\
117	0.488043503568592\\
118	0.488435484146185\\
119	0.488814614071909\\
120	0.489181314640244\\
121	0.489535993333993\\
122	0.489879044277085\\
123	0.490210848672528\\
124	0.490531775226011\\
125	0.490842180555612\\
126	0.491142409588076\\
127	0.491432795942108\\
128	0.491713662299086\\
129	0.491985320761639\\
130	0.492248073200453\\
131	0.492502211589714\\
132	0.492748018331558\\
133	0.492985766569877\\
134	0.49321572049384\\
135	0.493438135631467\\
136	0.493653259133571\\
137	0.493861330048398\\
138	0.494062579587261\\
139	0.494257231381463\\
140	0.494445501730804\\
141	0.49462759984393\\
142	0.494803728070813\\
143	0.494974082127604\\
144	0.495138851314114\\
145	0.495298218724171\\
146	0.495452361449069\\
147	0.495601450774362\\
148	0.495745652370196\\
149	0.495885126475401\\
150	0.496020028075556\\
151	0.496150507075207\\
152	0.496276708464445\\
153	0.496398772480019\\
154	0.496516834761172\\
155	0.496631026500363\\
156	0.496741474589051\\
157	0.496848301758697\\
158	0.496951626717149\\
159	0.497051564280544\\
160	0.497148225500904\\
161	0.497241717789528\\
162	0.497332145036356\\
163	0.49741960772541\\
164	0.497504203046456\\
165	0.497586025002996\\
166	0.497665164516736\\
167	0.497741709528609\\
168	0.497815745096504\\
169	0.49788735348978\\
170	0.497956614280687\\
171	0.498023604432784\\
172	0.498088398386466\\
173	0.498151068141681\\
174	0.498211683337938\\
175	0.498270311331692\\
176	0.498327017271189\\
177	0.498381864168864\\
178	0.498434912971357\\
179	0.49848622262724\\
180	0.498535850152521\\
181	0.498583850693999\\
182	0.498630277590549\\
183	0.498675182432387\\
184	0.498718615118403\\
185	0.498760623911604\\
186	0.498801255492751\\
187	0.498840555012224\\
188	0.498878566140199\\
189	0.498915331115175\\
190	0.498950890790905\\
191	0.4989852846818\\
192	0.499018551006831\\
193	0.499050726732006\\
194	0.499081847611439\\
195	0.499111948227089\\
196	0.499141062027181\\
197	0.499169221363376\\
198	0.499196457526723\\
199	0.499222800782428\\
200	0.499248280403485\\
201	0.499272924703205\\
202	0.499296761066679\\
203	0.499319815981208\\
204	0.499342115065734\\
205	0.499363683099314\\
206	0.499384544048649\\
207	0.499404721094716\\
208	0.499424236658532\\
209	0.499443112426065\\
210	0.499461369372332\\
211	0.499479027784708\\
212	0.499496107285466\\
213	0.499512626853588\\
214	0.499528604845849\\
215	0.499544059017217\\
216	0.499559006540583\\
217	0.499573464025845\\
218	0.499587447538363\\
219	0.499600972616811\\
220	0.499614054290446\\
221	0.499626707095807\\
222	0.499638945092868\\
223	0.499650781880663\\
224	0.499662230612396\\
225	0.499673304010059\\
226	0.499684014378565\\
227	0.499694373619427\\
228	0.499704393243977\\
229	0.499714084386163\\
230	0.499723457814917\\
231	0.499732523946125\\
232	0.499741292854198\\
233	0.49974977428327\\
234	0.499757977658023\\
235	0.499765912094162\\
236	0.499773586408544\\
237	0.499781009128975\\
238	0.499788188503685\\
239	0.499795132510495\\
240	0.499801848865684\\
241	0.499808345032559\\
242	0.499814628229751\\
243	0.499820705439239\\
244	0.4998265834141\\
245	0.499832268686025\\
246	0.499837767572565\\
247	0.499843086184162\\
248	0.499848230430931\\
249	0.49985320602923\\
250	0.499858018508015\\
251	0.499862673214979\\
252	0.499867175322497\\
253	0.499871529833375\\
254	0.499875741586406\\
255	0.499879815261748\\
256	0.499883755386128\\
257	0.499887566337866\\
258	0.499891252351745\\
259	0.499894817523715\\
260	0.499898265815447\\
261	0.499901601058729\\
262	0.499904826959731\\
263	0.499907947103119\\
264	0.499910964956041\\
265	0.499913883871976\\
266	0.499916707094464\\
267	0.499919437760711\\
268	0.49992207890507\\
269	0.499924633462418\\
270	0.499927104271415\\
271	0.499929494077658\\
272	0.499931805536733\\
273	0.499934041217166\\
274	0.499936203603277\\
275	0.499938295097938\\
276	0.499940318025249\\
277	0.499942274633114\\
278	0.499944167095743\\
279	0.499945997516068\\
280	0.499947767928077\\
281	0.499949480299076\\
282	0.499951136531876\\
283	0.499952738466907\\
284	0.49995428788426\\
285	0.499955786505671\\
286	0.499957235996428\\
287	0.499958637967227\\
288	0.499959993975955\\
289	0.49996130552943\\
290	0.499962574085067\\
291	0.499963801052504\\
292	0.499964987795164\\
293	0.499966135631772\\
294	0.499967245837819\\
295	0.499968319646983\\
296	0.499969358252496\\
297	0.499970362808469\\
298	0.49997133443118\\
299	0.49997227420031\\
300	0.499973183160143\\
};
\end{axis}

\begin{axis}[%
width=1.68in,
height=1.553in,
at={(3.235in,7.552in)},
scale only axis,
xmin=0,
xmax=300,
ymin=0,
ymax=2,
axis background/.style={fill=white},
title style={font=\bfseries},
title={Tor u1 - y2}
]
\addplot[const plot, color=mycolor1, forget plot] table[row sep=crcr] {%
1	0\\
2	0\\
3	0\\
4	0\\
5	0\\
6	0.048770575499286\\
7	0.0951625819640404\\
8	0.139292023574942\\
9	0.181269246922019\\
10	0.221199216928596\\
11	0.259181779318284\\
12	0.29531191028129\\
13	0.329679953964366\\
14	0.362371848378235\\
15	0.393469340287378\\
16	0.423050189619527\\
17	0.451188363905991\\
18	0.477954223239005\\
19	0.503414696208616\\
20	0.527633447259015\\
21	0.550671035882813\\
22	0.572585068051313\\
23	0.593430340259446\\
24	0.613258976545549\\
25	0.632120558828613\\
26	0.650062250888903\\
27	0.667128916301981\\
28	0.683363230621007\\
29	0.698805788087855\\
30	0.713495203139862\\
31	0.727468206966032\\
32	0.740759739354143\\
33	0.753403036058416\\
34	0.765429711906211\\
35	0.776869839851563\\
36	0.787752026173232\\
37	0.798103482005301\\
38	0.807950091379182\\
39	0.81731647594718\\
40	0.826226056549446\\
41	0.834701111778281\\
42	0.842762833686214\\
43	0.850431380777181\\
44	0.857725928413276\\
45	0.864664716763151\\
46	0.871265096411932\\
47	0.877543571746727\\
48	0.883515842226184\\
49	0.889196841637318\\
50	0.894600775437758\\
51	0.899741156276789\\
52	0.904630837784012\\
53	0.909282046710117\\
54	0.913706413500127\\
55	0.917915001375564\\
56	0.921918333998274\\
57	0.925726421785057\\
58	0.929348786938923\\
59	0.932794487259563\\
60	0.936072138792566\\
61	0.939189937374015\\
62	0.942155679124353\\
63	0.944976779942744\\
64	0.947660294050678\\
65	0.950212931631206\\
66	0.952641075607889\\
67	0.954950797605431\\
68	0.957147873131908\\
69	0.95923779602054\\
70	0.961225792167142\\
71	0.963116832597581\\
72	0.964915645897932\\
73	0.966626730038407\\
74	0.968254363620621\\
75	0.969802616576326\\
76	0.971275360344361\\
77	0.972676277551265\\
78	0.97400887121976\\
79	0.975276473528135\\
80	0.976482254142426\\
81	0.977629228142231\\
82	0.978720263559982\\
83	0.979758088552518\\
84	0.9807452982229\\
85	0.981684361109519\\
86	0.982577625358726\\
87	0.983427324596424\\
88	0.984235583513297\\
89	0.98500442317764\\
90	0.985735766089084\\
91	0.986431440985848\\
92	0.987093187417535\\
93	0.987722660094912\\
94	0.988321433027551\\
95	0.988891003459669\\
96	0.989432795614023\\
97	0.989948164253206\\
98	0.99043839806726\\
99	0.990904722896071\\
100	0.991348304794609\\
101	0.991770252948673\\
102	0.99217162244843\\
103	0.992553416926695\\
104	0.992916591068532\\
105	0.993262052998464\\
106	0.99359066655126\\
107	0.993903253431969\\
108	0.994200595270611\\
109	0.994483435576662\\
110	0.994752481598213\\
111	0.995008406090456\\
112	0.99525184899793\\
113	0.995483419054705\\
114	0.995703695306543\\
115	0.995913228558811\\
116	0.996112542753782\\
117	0.99630213628076\\
118	0.996482483222319\\
119	0.99665403453975\\
120	0.996817219200705\\
121	0.996972445251833\\
122	0.997120100839118\\
123	0.997260555178436\\
124	0.997394159478795\\
125	0.997521247820538\\
126	0.997642137990715\\
127	0.997757132277721\\
128	0.99786651822717\\
129	0.997970569360912\\
130	0.998069545860978\\
131	0.998163695220173\\
132	0.998253252860935\\
133	0.998338442724013\\
134	0.99841947782844\\
135	0.998496560804185\\
136	0.998569884398839\\
137	0.998639631959581\\
138	0.998705977891644\\
139	0.998769088094415\\
140	0.998829120376275\\
141	0.998886224849198\\
142	0.99894054430411\\
143	0.998992214567944\\
144	0.999041364843272\\
145	0.999088118031385\\
146	0.999132591039605\\
147	0.999174895073621\\
148	0.99921513591555\\
149	0.999253414188459\\
150	0.999289825607968\\
151	0.999324461221592\\
152	0.999357407636405\\
153	0.999388747235606\\
154	0.999418558384519\\
155	0.999446915626542\\
156	0.999473889869553\\
157	0.999499548563209\\
158	0.99952395586761\\
159	0.999547172813731\\
160	0.999569257456028\\
161	0.999590265017613\\
162	0.999610248028329\\
163	0.999629256456113\\
164	0.999647337831936\\
165	0.999664537368655\\
166	0.999680898074072\\
167	0.999696460858471\\
168	0.99971126463692\\
169	0.999725346426577\\
170	0.999738741439249\\
171	0.999751483169447\\
172	0.999763603478133\\
173	0.999775132672391\\
174	0.999786099581213\\
175	0.999796531627583\\
176	0.999806454897051\\
177	0.999815894202959\\
178	0.999824873148489\\
179	0.999833414185681\\
180	0.999841538671577\\
181	0.999849266921623\\
182	0.99985661826047\\
183	0.999863611070291\\
184	0.999870262836753\\
185	0.999876590192737\\
186	0.999882608959927\\
187	0.999888334188377\\
188	0.99989378019414\\
189	0.999898960595066\\
190	0.999903888344856\\
191	0.999908575765452\\
192	0.999913034577846\\
193	0.999917275931393\\
194	0.999921310431686\\
195	0.999925148167076\\
196	0.999928798733903\\
197	0.999932271260486\\
198	0.99993557442995\\
199	0.99993871650194\\
200	0.99994170533327\\
201	0.999944548397578\\
202	0.999947252804004\\
203	0.999949825314975\\
204	0.999952272363108\\
205	0.999954600067297\\
206	0.999956814248017\\
207	0.999958920441871\\
208	0.999960923915441\\
209	0.999962829678454\\
210	0.99996464249631\\
211	0.999966366901996\\
212	0.999968007207425\\
213	0.999969567514213\\
214	0.99997105172394\\
215	0.999972463547904\\
216	0.9999738065164\\
217	0.999975083987549\\
218	0.999976299155695\\
219	0.999977455059389\\
220	0.999978554588995\\
221	0.999979600493908\\
222	0.999980595389437\\
223	0.999981541763338\\
224	0.999982441982039\\
225	0.999983298296556\\
226	0.999984112848121\\
227	0.999984887673537\\
228	0.999985624710272\\
229	0.999986325801302\\
230	0.999986992699718\\
231	0.999987627073115\\
232	0.999988230507756\\
233	0.999988804512542\\
234	0.999989350522784\\
235	0.999989869903793\\
236	0.999990363954293\\
237	0.999990833909665\\
238	0.999991280945045\\
239	0.999991706178254\\
240	0.999992110672595\\
241	0.999992495439514\\
242	0.99999286144113\\
243	0.999993209592637\\
244	0.999993540764595\\
245	0.999993855785107\\
246	0.999994155441887\\
247	0.999994440484234\\
248	0.999994711624902\\
249	0.999994969541883\\
250	0.999995214880104\\
251	0.999995448253039\\
252	0.999995670244241\\
253	0.999995881408805\\
254	0.999996082274753\\
255	0.999996273344354\\
256	0.999996455095383\\
257	0.999996627982311\\
258	0.999996792437446\\
259	0.999996948872011\\
260	0.999997097677171\\
261	0.999997239225018\\
262	0.999997373869494\\
263	0.999997501947282\\
264	0.999997623778642\\
265	0.999997739668218\\
266	0.999997849905793\\
267	0.999997954767019\\
268	0.999998054514102\\
269	0.999998149396462\\
270	0.999998239651354\\
271	0.999998325504463\\
272	0.999998407170465\\
273	0.999998484853569\\
274	0.999998558748023\\
275	0.999998629038599\\
276	0.999998695901061\\
277	0.9999987595026\\
278	0.999998820002253\\
279	0.999998877551301\\
280	0.999998932293646\\
281	0.999998984366176\\
282	0.999999033899097\\
283	0.999999081016271\\
284	0.999999125835513\\
285	0.999999168468897\\
286	0.999999209023026\\
287	0.999999247599309\\
288	0.999999284294204\\
289	0.999999319199469\\
290	0.999999352402385\\
291	0.999999383985977\\
292	0.999999414029218\\
293	0.999999442607233\\
294	0.99999946979148\\
295	0.999999495649936\\
296	0.999999520247258\\
297	0.999999543644954\\
298	0.99999956590153\\
299	0.999999587072638\\
300	0.999999607211218\\
};
\end{axis}

\begin{axis}[%
width=1.68in,
height=1.553in,
at={(5.446in,7.552in)},
scale only axis,
xmin=0,
xmax=300,
ymin=0,
ymax=2,
axis background/.style={fill=white},
title style={font=\bfseries},
title={Tor u1 - y3}
]
\addplot[const plot, color=mycolor1, forget plot] table[row sep=crcr] {%
1	0\\
2	0\\
3	0\\
4	0\\
5	0\\
6	0.00958843300927808\\
7	0.0189141858826321\\
8	0.0279844548797369\\
9	0.0368062391149706\\
10	0.0453863459583134\\
11	0.0537313962882852\\
12	0.061847829600977\\
13	0.0697419089791175\\
14	0.0774197259250088\\
15	0.0848872050610625\\
16	0.0921501087015615\\
17	0.0992140412991755\\
18	0.106084453769663\\
19	0.112766647698096\\
20	0.119265779429849\\
21	0.125586864049521\\
22	0.131734779250846\\
23	0.137714269100586\\
24	0.143529947699315\\
25	0.149186302741909\\
26	0.154687698980495\\
27	0.160038381592536\\
28	0.165242479456636\\
29	0.170304008338611\\
30	0.175226873990274\\
31	0.180014875163324\\
32	0.184671706540672\\
33	0.189200961587456\\
34	0.19360613532396\\
35	0.19789062702256\\
36	0.202057742830786\\
37	0.206110698322529\\
38	0.210052620979351\\
39	0.213886552603823\\
40	0.217615451666742\\
41	0.221242195590052\\
42	0.224769582967218\\
43	0.228200335722774\\
44	0.231537101212706\\
45	0.234782454267302\\
46	0.237938899178026\\
47	0.241008871629966\\
48	0.243994740581348\\
49	0.246898810091544\\
50	0.249723321099019\\
51	0.252470453150553\\
52	0.255142326083105\\
53	0.257741001659591\\
54	0.260268485159855\\
55	0.262726726928049\\
56	0.265117623877628\\
57	0.267443020955114\\
58	0.269704712563749\\
59	0.271904443948165\\
60	0.274043912541101\\
61	0.27612476927324\\
62	0.278148619847157\\
63	0.280117025976363\\
64	0.282031506590411\\
65	0.283893539006988\\
66	0.285704560071886\\
67	0.287465967267757\\
68	0.289179119792487\\
69	0.290845339608022\\
70	0.292465912460472\\
71	0.294042088872258\\
72	0.295575085107087\\
73	0.297066084108483\\
74	0.298516236412618\\
75	0.29992666103612\\
76	0.301298446339574\\
77	0.302632650867354\\
78	0.303930304164463\\
79	0.305192407570975\\
80	0.306419934994733\\
81	0.307613833662863\\
82	0.308775024852709\\
83	0.309904404602741\\
84	0.311002844403987\\
85	0.312071191872522\\
86	0.313110271403535\\
87	0.31412088480748\\
88	0.315103811928791\\
89	0.316059811247658\\
90	0.316989620465309\\
91	0.317893957073262\\
92	0.318773518906978\\
93	0.319628984684352\\
94	0.320461014529446\\
95	0.32127025048188\\
96	0.322057316992261\\
97	0.322822821404046\\
98	0.323567354422201\\
99	0.32429149056902\\
100	0.324995788627457\\
101	0.325680792072317\\
102	0.326347029489629\\
103	0.326995014984529\\
104	0.327625248577975\\
105	0.32823821659259\\
106	0.328834392027934\\
107	0.329414234925496\\
108	0.329978192723688\\
109	0.330526700603114\\
110	0.331060181822378\\
111	0.33157904804469\\
112	0.332083699655536\\
113	0.332574526071628\\
114	0.333051906041406\\
115	0.333516207937298\\
116	0.333967790039977\\
117	0.334407000814829\\
118	0.334834179180847\\
119	0.335249654772164\\
120	0.335653748192411\\
121	0.336046771262117\\
122	0.336429027259324\\
123	0.336800811153611\\
124	0.337162409833716\\
125	0.337514102328905\\
126	0.337856160024294\\
127	0.338188846870262\\
128	0.338512419586128\\
129	0.338827127858253\\
130	0.339133214532707\\
131	0.339430915802666\\
132	0.339720461390672\\
133	0.340002074725895\\
134	0.340275973116549\\
135	0.340542367917574\\
136	0.340801464693731\\
137	0.341053463378228\\
138	0.341298558426997\\
139	0.341536938968749\\
140	0.341768788950914\\
141	0.341994287281587\\
142	0.342213607967582\\
143	0.342426920248705\\
144	0.342634388728349\\
145	0.342836173500511\\
146	0.34303243027333\\
147	0.343223310489238\\
148	0.343408961441823\\
149	0.343589526389488\\
150	0.343765144665997\\
151	0.343935951787994\\
152	0.344102079559571\\
153	0.344263656173981\\
154	0.344420806312553\\
155	0.344573651240909\\
156	0.344722308902532\\
157	0.344866894009784\\
158	0.345007518132422\\
159	0.34514428978369\\
160	0.345277314504057\\
161	0.345406694942652\\
162	0.345532530936482\\
163	0.345654919587464\\
164	0.345773955337357\\
165	0.345889730040639\\
166	0.346002333035389\\
167	0.346111851212219\\
168	0.346218369081328\\
169	0.346321968837717\\
170	0.346422730424608\\
171	0.34652073159514\\
172	0.346616047972362\\
173	0.34670875310759\\
174	0.346798918537164\\
175	0.346886613837649\\
176	0.346971906679523\\
177	0.347054862879395\\
178	0.347135546450795\\
179	0.347214019653568\\
180	0.347290343041919\\
181	0.347364575511137\\
182	0.347436774343046\\
183	0.347506995250204\\
184	0.347575292418892\\
185	0.347641718550934\\
186	0.347706324904355\\
187	0.347769161332942\\
188	0.347830276324713\\
189	0.347889717039328\\
190	0.347947529344486\\
191	0.348003757851315\\
192	0.348058445948798\\
193	0.348111635837256\\
194	0.348163368560907\\
195	0.348213684039544\\
196	0.348262621099335\\
197	0.348310217502784\\
198	0.348356509977873\\
199	0.348401534246402\\
200	0.348445325051552\\
201	0.348487916184697\\
202	0.348529340511479\\
203	0.348569629997168\\
204	0.34860881573133\\
205	0.348646927951814\\
206	0.348683996068089\\
207	0.348720048683936\\
208	0.348755113619521\\
209	0.348789217932859\\
210	0.348822387940702\\
211	0.348854649238834\\
212	0.348886026721835\\
213	0.348916544602282\\
214	0.348946226429437\\
215	0.348975095107417\\
216	0.349003172912868\\
217	0.349030481512157\\
218	0.34905704197809\\
219	0.349082874806171\\
220	0.349107999930419\\
221	0.349132436738751\\
222	0.349156204087942\\
223	0.349179320318175\\
224	0.349201803267194\\
225	0.349223670284068\\
226	0.349244938242581\\
227	0.34926562355425\\
228	0.349285742180989\\
229	0.349305309647428\\
230	0.349324341052893\\
231	0.349342851083053\\
232	0.349360854021258\\
233	0.349378363759558\\
234	0.349395393809421\\
235	0.349411957312164\\
236	0.349428067049089\\
237	0.349443735451347\\
238	0.349458974609533\\
239	0.349473796283013\\
240	0.349488211908998\\
241	0.349502232611373\\
242	0.349515869209275\\
243	0.349529132225448\\
244	0.349542031894359\\
245	0.349554578170094\\
246	0.349566780734046\\
247	0.349578649002379\\
248	0.349590192133295\\
249	0.349601419034105\\
250	0.349612338368097\\
251	0.349622958561225\\
252	0.349633287808609\\
253	0.34964333408086\\
254	0.34965310513023\\
255	0.349662608496592\\
256	0.349671851513264\\
257	0.34968084131266\\
258	0.3496895848318\\
259	0.349698088817661\\
260	0.349706359832383\\
261	0.349714404258332\\
262	0.349722228303025\\
263	0.349729838003923\\
264	0.349737239233086\\
265	0.349744437701706\\
266	0.349751438964514\\
267	0.349758248424065\\
268	0.349764871334912\\
269	0.349771312807653\\
270	0.34977757781288\\
271	0.349783671185013\\
272	0.34978959762603\\
273	0.349795361709096\\
274	0.349800967882093\\
275	0.34980642047105\\
276	0.349811723683481\\
277	0.349816881611636\\
278	0.349821898235652\\
279	0.349826777426631\\
280	0.349831522949621\\
281	0.349836138466529\\
282	0.349840627538937\\
283	0.349844993630859\\
284	0.349849240111411\\
285	0.349853370257407\\
286	0.349857387255894\\
287	0.349861294206607\\
288	0.349865094124362\\
289	0.349868789941384\\
290	0.349872384509567\\
291	0.349875880602674\\
292	0.349879280918483\\
293	0.349882588080862\\
294	0.349885804641799\\
295	0.349888933083367\\
296	0.349891975819643\\
297	0.349894935198568\\
298	0.34989781350376\\
299	0.349900612956276\\
300	0.349903335716326\\
};
\end{axis}

\begin{axis}[%
width=1.68in,
height=1.553in,
at={(1.024in,5.395in)},
scale only axis,
xmin=0,
xmax=300,
ymin=0,
ymax=2,
axis background/.style={fill=white},
title style={font=\bfseries},
title={Tor u2 - y1}
]
\addplot[const plot, color=mycolor1, forget plot] table[row sep=crcr] {%
1	0\\
2	0\\
3	0\\
4	0\\
5	0\\
6	0.0221199216928595\\
7	0.0393469340287367\\
8	0.0527633447258986\\
9	0.0632120558828559\\
10	0.0713495203139812\\
11	0.0776869839851574\\
12	0.0826226056549562\\
13	0.0864664716763397\\
14	0.0894600775438148\\
15	0.0917915001376117\\
16	0.093607213879331\\
17	0.0950212931632155\\
18	0.0961225792168297\\
19	0.09698026165777\\
20	0.0976482254144007\\
21	0.0981684361111279\\
22	0.0985735766091011\\
23	0.0988891003461764\\
24	0.0991348304796881\\
25	0.0993262053000911\\
26	0.099475248160081\\
27	0.0995913228561522\\
28	0.0996817219203471\\
29	0.0997521247823309\\
30	0.0998069545863742\\
31	0.0998496560806986\\
32	0.0998829120379167\\
33	0.0999088118034399\\
34	0.0999289825611107\\
35	0.0999446915629798\\
36	0.0999569257459367\\
37	0.0999664537372038\\
38	0.0999738741442635\\
39	0.0999796531630923\\
40	0.0999841538674815\\
41	0.0999876590195841\\
42	0.0999903888347864\\
43	0.0999925148170035\\
44	0.0999941705336189\\
45	0.0999954600070155\\
46	0.0999964642499065\\
47	0.0999972463550563\\
48	0.0999978554591594\\
49	0.0999983298299119\\
50	0.0999986992702254\\
51	0.0999989869906308\\
52	0.0999992110675078\\
53	0.0999993855787551\\
54	0.0999995214882511\\
55	0.099999627334673\\
56	0.0999997097679491\\
57	0.099999773967049\\
58	0.0999998239653582\\
59	0.0999998629040804\\
60	0.0999998932295876\\
61	0.0999999168471163\\
62	0.099999935240466\\
63	0.0999999495652211\\
64	0.0999999607213515\\
65	0.0999999694097546\\
66	0.0999999761762898\\
67	0.0999999814460727\\
68	0.0999999855501838\\
69	0.0999999887464686\\
70	0.0999999912357378\\
71	0.0999999931743825\\
72	0.0999999946842005\\
73	0.0999999958600478\\
74	0.0999999967757985\\
75	0.0999999974889857\\
76	0.0999999980444165\\
77	0.0999999984769863\\
78	0.0999999988138718\\
79	0.0999999990762384\\
80	0.0999999992805694\\
81	0.0999999994397024\\
82	0.0999999995636352\\
83	0.0999999996601539\\
84	0.0999999997353225\\
85	0.0999999997938637\\
86	0.0999999998394556\\
87	0.0999999998749624\\
88	0.0999999999026151\\
89	0.0999999999241509\\
90	0.0999999999409229\\
91	0.0999999999539848\\
92	0.0999999999641574\\
93	0.0999999999720795\\
94	0.0999999999782492\\
95	0.0999999999830539\\
96	0.0999999999867957\\
97	0.0999999999897097\\
98	0.0999999999919788\\
99	0.0999999999937458\\
100	0.0999999999951218\\
101	0.0999999999961933\\
102	0.0999999999970277\\
103	0.0999999999976775\\
104	0.0999999999981837\\
105	0.0999999999985779\\
106	0.099999999998885\\
107	0.0999999999991241\\
108	0.0999999999993104\\
109	0.0999999999994554\\
110	0.0999999999995685\\
111	0.0999999999996565\\
112	0.0999999999997252\\
113	0.0999999999997786\\
114	0.0999999999998202\\
115	0.0999999999998526\\
116	0.0999999999998777\\
117	0.0999999999998972\\
118	0.0999999999999125\\
119	0.0999999999999244\\
120	0.0999999999999337\\
121	0.0999999999999409\\
122	0.0999999999999465\\
123	0.0999999999999508\\
124	0.0999999999999541\\
125	0.0999999999999566\\
126	0.0999999999999584\\
127	0.0999999999999597\\
128	0.0999999999999606\\
129	0.0999999999999613\\
130	0.0999999999999618\\
131	0.0999999999999622\\
132	0.0999999999999625\\
133	0.0999999999999626\\
134	0.0999999999999626\\
135	0.0999999999999626\\
136	0.0999999999999624\\
137	0.0999999999999621\\
138	0.0999999999999619\\
139	0.0999999999999615\\
140	0.0999999999999612\\
141	0.099999999999961\\
142	0.0999999999999608\\
143	0.0999999999999606\\
144	0.0999999999999605\\
145	0.0999999999999604\\
146	0.0999999999999602\\
147	0.0999999999999601\\
148	0.09999999999996\\
149	0.0999999999999598\\
150	0.0999999999999596\\
151	0.0999999999999594\\
152	0.0999999999999592\\
153	0.0999999999999589\\
154	0.0999999999999586\\
155	0.0999999999999583\\
156	0.099999999999958\\
157	0.0999999999999578\\
158	0.0999999999999575\\
159	0.0999999999999572\\
160	0.0999999999999569\\
161	0.0999999999999565\\
162	0.0999999999999561\\
163	0.0999999999999557\\
164	0.0999999999999553\\
165	0.099999999999955\\
166	0.0999999999999546\\
167	0.0999999999999543\\
168	0.099999999999954\\
169	0.0999999999999538\\
170	0.0999999999999537\\
171	0.0999999999999536\\
172	0.0999999999999537\\
173	0.0999999999999539\\
174	0.0999999999999542\\
175	0.0999999999999546\\
176	0.0999999999999551\\
177	0.0999999999999557\\
178	0.0999999999999564\\
179	0.099999999999957\\
180	0.0999999999999577\\
181	0.0999999999999583\\
182	0.0999999999999589\\
183	0.0999999999999595\\
184	0.0999999999999601\\
185	0.0999999999999607\\
186	0.0999999999999612\\
187	0.0999999999999618\\
188	0.0999999999999623\\
189	0.0999999999999628\\
190	0.0999999999999634\\
191	0.0999999999999639\\
192	0.0999999999999645\\
193	0.0999999999999651\\
194	0.0999999999999657\\
195	0.0999999999999664\\
196	0.099999999999967\\
197	0.0999999999999677\\
198	0.0999999999999683\\
199	0.099999999999969\\
200	0.0999999999999697\\
201	0.0999999999999704\\
202	0.0999999999999711\\
203	0.0999999999999717\\
204	0.0999999999999723\\
205	0.0999999999999728\\
206	0.0999999999999733\\
207	0.0999999999999737\\
208	0.0999999999999741\\
209	0.0999999999999746\\
210	0.099999999999975\\
211	0.0999999999999754\\
212	0.0999999999999757\\
213	0.0999999999999759\\
214	0.0999999999999761\\
215	0.0999999999999762\\
216	0.0999999999999761\\
217	0.0999999999999759\\
218	0.0999999999999757\\
219	0.0999999999999753\\
220	0.099999999999975\\
221	0.0999999999999746\\
222	0.0999999999999742\\
223	0.0999999999999738\\
224	0.0999999999999735\\
225	0.0999999999999732\\
226	0.0999999999999729\\
227	0.0999999999999726\\
228	0.0999999999999723\\
229	0.0999999999999722\\
230	0.0999999999999721\\
231	0.0999999999999722\\
232	0.0999999999999722\\
233	0.0999999999999723\\
234	0.0999999999999725\\
235	0.0999999999999727\\
236	0.0999999999999728\\
237	0.099999999999973\\
238	0.0999999999999732\\
239	0.0999999999999734\\
240	0.0999999999999737\\
241	0.0999999999999741\\
242	0.0999999999999744\\
243	0.0999999999999748\\
244	0.0999999999999752\\
245	0.0999999999999757\\
246	0.0999999999999761\\
247	0.0999999999999765\\
248	0.0999999999999769\\
249	0.0999999999999772\\
250	0.0999999999999775\\
251	0.0999999999999776\\
252	0.0999999999999777\\
253	0.0999999999999778\\
254	0.0999999999999779\\
255	0.099999999999978\\
256	0.0999999999999781\\
257	0.0999999999999781\\
258	0.099999999999978\\
259	0.0999999999999778\\
260	0.0999999999999776\\
261	0.0999999999999772\\
262	0.0999999999999769\\
263	0.0999999999999764\\
264	0.0999999999999759\\
265	0.0999999999999753\\
266	0.0999999999999746\\
267	0.0999999999999739\\
268	0.0999999999999731\\
269	0.0999999999999723\\
270	0.0999999999999715\\
271	0.0999999999999708\\
272	0.0999999999999702\\
273	0.0999999999999697\\
274	0.0999999999999693\\
275	0.099999999999969\\
276	0.0999999999999687\\
277	0.0999999999999684\\
278	0.0999999999999682\\
279	0.099999999999968\\
280	0.0999999999999679\\
281	0.0999999999999678\\
282	0.0999999999999677\\
283	0.0999999999999677\\
284	0.0999999999999676\\
285	0.0999999999999676\\
286	0.0999999999999676\\
287	0.0999999999999676\\
288	0.0999999999999677\\
289	0.0999999999999678\\
290	0.099999999999968\\
291	0.0999999999999682\\
292	0.0999999999999684\\
293	0.0999999999999687\\
294	0.099999999999969\\
295	0.0999999999999692\\
296	0.0999999999999695\\
297	0.0999999999999698\\
298	0.09999999999997\\
299	0.0999999999999703\\
300	0.0999999999999706\\
};
\end{axis}

\begin{axis}[%
width=1.68in,
height=1.553in,
at={(3.235in,5.395in)},
scale only axis,
xmin=0,
xmax=300,
ymin=0,
ymax=2,
axis background/.style={fill=white},
title style={font=\bfseries},
title={Tor u2 - y2}
]
\addplot[const plot, color=mycolor1, forget plot] table[row sep=crcr] {%
1	0\\
2	0\\
3	0\\
4	0\\
5	0\\
6	0.0204052714454309\\
7	0.0399777926853384\\
8	0.0587515487077024\\
9	0.0767591375546931\\
10	0.0940318269246828\\
11	0.110599608464298\\
12	0.126491249844785\\
13	0.141734344713107\\
14	0.156355360604517\\
15	0.170379684899782\\
16	0.183831668906881\\
17	0.196734670143691\\
18	0.209111092895106\\
19	0.220982427114989\\
20	0.23236928574052\\
21	0.243291440483722\\
22	0.253767856162316\\
23	0.263816723629517\\
24	0.273455491359944\\
25	0.282700895746494\\
26	0.291568990160784\\
27	0.30007517282762\\
28	0.308234213561894\\
29	0.316060279414334\\
30	0.323566959270636\\
31	0.330767287446694\\
32	0.337673766320894\\
33	0.344298388042773\\
34	0.350652655355735\\
35	0.35674760156998\\
36	0.362593809720331\\
37	0.368201430942211\\
38	0.373580202097699\\
39	0.378739462682244\\
40	0.383688171041418\\
41	0.388434919925844\\
42	0.392987951411328\\
43	0.397355171210089\\
44	0.401544162397937\\
45	0.405562198581242\\
46	0.40941625652655\\
47	0.413113028274775\\
48	0.41665893276101\\
49	0.42006012696012\\
50	0.423322516577486\\
51	0.426451766303444\\
52	0.429453309649243\\
53	0.432332358381589\\
54	0.435093911572157\\
55	0.437742764277796\\
56	0.440283515866479\\
57	0.442720578003477\\
58	0.445058182311605\\
59	0.447300387718854\\
60	0.44945108750616\\
61	0.451514016067551\\
62	0.453492755394406\\
63	0.455390741295092\\
64	0.457211269360771\\
65	0.458957500687741\\
66	0.460632467366243\\
67	0.462239077745271\\
68	0.463780121482515\\
69	0.465258274388222\\
70	0.466676103071371\\
71	0.468036069396233\\
72	0.469340534757063\\
73	0.470591764178334\\
74	0.471791930247641\\
75	0.472943116888099\\
76	0.474047322976791\\
77	0.475106465815539\\
78	0.476122384460037\\
79	0.47709684291312\\
80	0.478031533187706\\
81	0.478928078244749\\
82	0.479788034811287\\
83	0.48061289608348\\
84	0.481404094319343\\
85	0.482163003325664\\
86	0.482890940843431\\
87	0.483589170835905\\
88	0.484258905683323\\
89	0.484901308288022\\
90	0.485517494093668\\
91	0.486108533022063\\
92	0.48667545133092\\
93	0.487219233395817\\
94	0.487740823419432\\
95	0.488241127071018\\
96	0.488721013058972\\
97	0.489181314639234\\
98	0.489622831062113\\
99	0.490046328960081\\
100	0.49045254367892\\
101	0.490842180554555\\
102	0.491215916137765\\
103	0.491574399368927\\
104	0.491918252704809\\
105	0.492248073199379\\
106	0.492564433540509\\
107	0.492867883044368\\
108	0.493158948609234\\
109	0.493438135630382\\
110	0.493705928877627\\
111	0.49396279333707\\
112	0.494209175018473\\
113	0.494445501729699\\
114	0.494672183819541\\
115	0.494889614890232\\
116	0.495098172480886\\
117	0.495298218723042\\
118	0.495490100969452\\
119	0.495674152397215\\
120	0.49585069258629\\
121	0.496020028074408\\
122	0.496182452889325\\
123	0.496338249059366\\
124	0.496487687103127\\
125	0.496631026499194\\
126	0.496768516136689\\
127	0.496900394747436\\
128	0.497026891320482\\
129	0.497148225499706\\
130	0.497264607965204\\
131	0.497376240799101\\
132	0.497483317836441\\
133	0.497586025001756\\
134	0.497684540631895\\
135	0.497779035785685\\
136	0.497869674540947\\
137	0.497956614279401\\
138	0.498040005959929\\
139	0.498119994380702\\
140	0.498196718430597\\
141	0.498270311330363\\
142	0.498340900863936\\
143	0.498408609600321\\
144	0.498473555106415\\
145	0.498535850151147\\
146	0.498595602901285\\
147	0.49865291710925\\
148	0.498707892293274\\
149	0.498760623910191\\
150	0.498811203521195\\
151	0.498859718950814\\
152	0.498906254439411\\
153	0.498950890789455\\
154	0.49899370550582\\
155	0.499034772930368\\
156	0.499074164371024\\
157	0.499111948225603\\
158	0.499148190100565\\
159	0.499182952924935\\
160	0.49921629705957\\
161	0.499248280401969\\
162	0.499278958486803\\
163	0.499308384582341\\
164	0.499336609782946\\
165	0.499363683097792\\
166	0.499389651535962\\
167	0.499414560188072\\
168	0.499438452304566\\
169	0.499461369370816\\
170	0.49948335117915\\
171	0.499504435897952\\
172	0.499524660137933\\
173	0.4995440590157\\
174	0.499562666214734\\
175	0.499580514043873\\
176	0.499597633493415\\
177	0.499614054288927\\
178	0.499629804942859\\
179	0.499644912804052\\
180	0.499659404105228\\
181	0.499673304008534\\
182	0.49968663664924\\
183	0.49969942517764\\
184	0.499711691799254\\
185	0.49972345781338\\
186	0.499734743650082\\
187	0.499745568905662\\
188	0.499755952376685\\
189	0.499765912092618\\
190	0.499775465347136\\
191	0.499784628728151\\
192	0.499793418146612\\
193	0.499801848864133\\
194	0.499809935519495\\
195	0.499817692154062\\
196	0.499825132236161\\
197	0.49983226868447\\
198	0.499839113890449\\
199	0.499845679739857\\
200	0.499851977633386\\
201	0.499858018506461\\
202	0.499863812848227\\
203	0.499869370719761\\
204	0.499874701771539\\
205	0.4998798152602\\
206	0.499884720064613\\
207	0.499889424701294\\
208	0.499893937339198\\
209	0.499898265813899\\
210	0.499902417641197\\
211	0.49990640003017\\
212	0.499910219895686\\
213	0.499913883870417\\
214	0.499917398316351\\
215	0.499920769335838\\
216	0.499924002782191\\
217	0.499927104269844\\
218	0.499930079184103\\
219	0.499932932690496\\
220	0.499935669743746\\
221	0.499938295096368\\
222	0.499940813306926\\
223	0.499943228747944\\
224	0.499945545613504\\
225	0.499947767926523\\
226	0.499949899545742\\
227	0.499951944172422\\
228	0.499953905356778\\
229	0.499955786504136\\
230	0.499957590880848\\
231	0.499959321619967\\
232	0.499960981726681\\
233	0.499962574083539\\
234	0.499964101455447\\
235	0.499965566494477\\
236	0.499966971744468\\
237	0.499968319645443\\
238	0.499969612537847\\
239	0.49997085266661\\
240	0.499972042185045\\
241	0.499973183158587\\
242	0.49997427756838\\
243	0.499975327314715\\
244	0.499976334220334\\
245	0.499977300033589\\
246	0.499978226431482\\
247	0.499979115022575\\
248	0.499979967349783\\
249	0.499980784893057\\
250	0.499981569071946\\
251	0.499982321248069\\
252	0.499983042727477\\
253	0.499983734762919\\
254	0.499984398556019\\
255	0.499985035259363\\
256	0.499985645978498\\
257	0.499986231773854\\
258	0.499986793662584\\
259	0.49998733262033\\
260	0.499987849582917\\
261	0.49998834544798\\
262	0.499988821076519\\
263	0.499989277294399\\
264	0.499989714893779\\
265	0.49999013463449\\
266	0.499990537245355\\
267	0.499990923425451\\
268	0.499991293845329\\
269	0.49999164914817\\
270	0.499991989950909\\
271	0.499992316845304\\
272	0.499992630398962\\
273	0.499992931156325\\
274	0.499993219639617\\
275	0.49999349634975\\
276	0.499993761767193\\
277	0.499994016352806\\
278	0.499994260548643\\
279	0.499994494778715\\
280	0.499994719449731\\
281	0.4999949349518\\
282	0.499995141659114\\
283	0.499995339930589\\
284	0.499995530110498\\
285	0.499995712529063\\
286	0.499995887503027\\
287	0.499996055336209\\
288	0.499996216320028\\
289	0.49999637073401\\
290	0.499996518846273\\
291	0.499996660913995\\
292	0.499996797183855\\
293	0.499996927892468\\
294	0.499997053266792\\
295	0.499997173524522\\
296	0.499997288874468\\
297	0.499997399516922\\
298	0.499997505643996\\
299	0.499997607439967\\
300	0.499997705081589\\
};
\end{axis}

\begin{axis}[%
width=1.68in,
height=1.553in,
at={(5.446in,5.395in)},
scale only axis,
xmin=0,
xmax=300,
ymin=0,
ymax=2,
axis background/.style={fill=white},
title style={font=\bfseries},
title={Tor u2 - y3}
]
\addplot[const plot, color=mycolor1, forget plot] table[row sep=crcr] {%
1	0\\
2	0\\
3	0\\
4	0\\
5	0\\
6	0.275428538201157\\
7	0.44248439117999\\
8	0.543808887896099\\
9	0.605265301734372\\
10	0.642540500963272\\
11	0.665149052142497\\
12	0.678861831604379\\
13	0.687179052777888\\
14	0.692223702423233\\
15	0.695283437100644\\
16	0.697139259993079\\
17	0.698264873476337\\
18	0.698947592564918\\
19	0.699361682624113\\
20	0.699612840940896\\
21	0.699765176160465\\
22	0.699857572141687\\
23	0.699913613137132\\
24	0.699947603719068\\
25	0.699968220049154\\
26	0.69998072448544\\
27	0.69998830880943\\
28	0.699992908934462\\
29	0.699995699051332\\
30	0.699997391342756\\
31	0.69999841776939\\
32	0.699999040328613\\
33	0.699999417929869\\
34	0.699999646956609\\
35	0.699999785868349\\
36	0.699999870122579\\
37	0.699999921225355\\
38	0.699999952220756\\
39	0.699999971020418\\
40	0.69999998242299\\
41	0.699999989339001\\
42	0.699999993533773\\
43	0.699999996078031\\
44	0.699999997621201\\
45	0.699999998557179\\
46	0.699999999124878\\
47	0.699999999469204\\
48	0.699999999678047\\
49	0.699999999804718\\
50	0.699999999881547\\
51	0.699999999928146\\
52	0.699999999956409\\
53	0.699999999973552\\
54	0.69999999998395\\
55	0.699999999990257\\
56	0.699999999994082\\
57	0.699999999996403\\
58	0.699999999997811\\
59	0.699999999998664\\
60	0.699999999999182\\
61	0.699999999999496\\
62	0.699999999999688\\
63	0.699999999999804\\
64	0.699999999999875\\
65	0.699999999999919\\
66	0.699999999999947\\
67	0.699999999999966\\
68	0.69999999999998\\
69	0.699999999999991\\
70	0.699999999999999\\
71	0.700000000000007\\
72	0.700000000000013\\
73	0.700000000000019\\
74	0.700000000000024\\
75	0.700000000000029\\
76	0.700000000000033\\
77	0.700000000000036\\
78	0.70000000000004\\
79	0.700000000000043\\
80	0.700000000000045\\
81	0.700000000000047\\
82	0.700000000000049\\
83	0.70000000000005\\
84	0.70000000000005\\
85	0.700000000000049\\
86	0.700000000000046\\
87	0.700000000000043\\
88	0.70000000000004\\
89	0.700000000000036\\
90	0.700000000000032\\
91	0.700000000000028\\
92	0.700000000000024\\
93	0.700000000000019\\
94	0.700000000000015\\
95	0.700000000000012\\
96	0.700000000000008\\
97	0.700000000000005\\
98	0.700000000000001\\
99	0.699999999999997\\
100	0.699999999999993\\
101	0.699999999999989\\
102	0.699999999999986\\
103	0.699999999999982\\
104	0.699999999999977\\
105	0.699999999999973\\
106	0.699999999999968\\
107	0.699999999999963\\
108	0.699999999999958\\
109	0.699999999999953\\
110	0.699999999999948\\
111	0.699999999999943\\
112	0.699999999999938\\
113	0.699999999999933\\
114	0.699999999999927\\
115	0.699999999999921\\
116	0.699999999999914\\
117	0.699999999999908\\
118	0.699999999999901\\
119	0.699999999999894\\
120	0.699999999999888\\
121	0.699999999999881\\
122	0.699999999999875\\
123	0.69999999999987\\
124	0.699999999999866\\
125	0.699999999999864\\
126	0.699999999999863\\
127	0.699999999999863\\
128	0.699999999999864\\
129	0.699999999999867\\
130	0.699999999999872\\
131	0.699999999999877\\
132	0.699999999999884\\
133	0.699999999999892\\
134	0.6999999999999\\
135	0.699999999999909\\
136	0.699999999999917\\
137	0.699999999999925\\
138	0.699999999999933\\
139	0.699999999999939\\
140	0.699999999999946\\
141	0.699999999999952\\
142	0.699999999999957\\
143	0.699999999999962\\
144	0.699999999999967\\
145	0.699999999999971\\
146	0.699999999999974\\
147	0.699999999999976\\
148	0.699999999999977\\
149	0.699999999999978\\
150	0.699999999999978\\
151	0.699999999999978\\
152	0.699999999999978\\
153	0.699999999999978\\
154	0.699999999999978\\
155	0.69999999999998\\
156	0.699999999999981\\
157	0.699999999999982\\
158	0.699999999999983\\
159	0.699999999999984\\
160	0.699999999999985\\
161	0.699999999999986\\
162	0.699999999999986\\
163	0.699999999999986\\
164	0.699999999999986\\
165	0.699999999999986\\
166	0.699999999999985\\
167	0.699999999999984\\
168	0.699999999999982\\
169	0.69999999999998\\
170	0.699999999999977\\
171	0.699999999999975\\
172	0.699999999999972\\
173	0.699999999999968\\
174	0.699999999999964\\
175	0.699999999999959\\
176	0.699999999999954\\
177	0.699999999999949\\
178	0.699999999999944\\
179	0.699999999999939\\
180	0.699999999999936\\
181	0.699999999999933\\
182	0.69999999999993\\
183	0.699999999999928\\
184	0.699999999999926\\
185	0.699999999999924\\
186	0.699999999999923\\
187	0.699999999999922\\
188	0.699999999999921\\
189	0.699999999999921\\
190	0.699999999999921\\
191	0.69999999999992\\
192	0.69999999999992\\
193	0.699999999999918\\
194	0.699999999999917\\
195	0.699999999999916\\
196	0.699999999999915\\
197	0.699999999999914\\
198	0.699999999999913\\
199	0.699999999999912\\
200	0.699999999999912\\
201	0.699999999999912\\
202	0.699999999999911\\
203	0.699999999999911\\
204	0.69999999999991\\
205	0.699999999999907\\
206	0.699999999999904\\
207	0.6999999999999\\
208	0.699999999999896\\
209	0.699999999999891\\
210	0.699999999999887\\
211	0.699999999999883\\
212	0.699999999999881\\
213	0.699999999999879\\
214	0.699999999999878\\
215	0.699999999999877\\
216	0.699999999999878\\
217	0.69999999999988\\
218	0.699999999999883\\
219	0.699999999999887\\
220	0.699999999999892\\
221	0.699999999999897\\
222	0.699999999999903\\
223	0.699999999999908\\
224	0.699999999999912\\
225	0.699999999999916\\
226	0.699999999999919\\
227	0.699999999999922\\
228	0.699999999999924\\
229	0.699999999999927\\
230	0.699999999999929\\
231	0.699999999999932\\
232	0.699999999999934\\
233	0.699999999999935\\
234	0.699999999999936\\
235	0.699999999999935\\
236	0.699999999999934\\
237	0.699999999999932\\
238	0.699999999999931\\
239	0.69999999999993\\
240	0.699999999999929\\
241	0.699999999999928\\
242	0.699999999999927\\
243	0.699999999999925\\
244	0.699999999999923\\
245	0.699999999999921\\
246	0.69999999999992\\
247	0.699999999999918\\
248	0.699999999999916\\
249	0.699999999999914\\
250	0.699999999999911\\
251	0.699999999999908\\
252	0.699999999999905\\
253	0.699999999999901\\
254	0.699999999999896\\
255	0.699999999999892\\
256	0.699999999999887\\
257	0.699999999999882\\
258	0.699999999999876\\
259	0.699999999999871\\
260	0.699999999999866\\
261	0.699999999999862\\
262	0.699999999999859\\
263	0.699999999999857\\
264	0.699999999999856\\
265	0.699999999999856\\
266	0.699999999999857\\
267	0.699999999999859\\
268	0.699999999999861\\
269	0.699999999999863\\
270	0.699999999999866\\
271	0.69999999999987\\
272	0.699999999999875\\
273	0.699999999999879\\
274	0.699999999999884\\
275	0.699999999999889\\
276	0.699999999999895\\
277	0.699999999999902\\
278	0.699999999999909\\
279	0.699999999999916\\
280	0.699999999999923\\
281	0.699999999999928\\
282	0.699999999999933\\
283	0.699999999999937\\
284	0.69999999999994\\
285	0.699999999999942\\
286	0.699999999999943\\
287	0.699999999999943\\
288	0.699999999999943\\
289	0.699999999999941\\
290	0.699999999999938\\
291	0.699999999999933\\
292	0.699999999999928\\
293	0.699999999999921\\
294	0.699999999999914\\
295	0.699999999999907\\
296	0.699999999999898\\
297	0.69999999999989\\
298	0.699999999999881\\
299	0.699999999999873\\
300	0.699999999999865\\
};
\end{axis}

\begin{axis}[%
width=1.68in,
height=1.553in,
at={(1.024in,3.239in)},
scale only axis,
xmin=0,
xmax=300,
ymin=0,
ymax=2,
axis background/.style={fill=white},
title style={font=\bfseries},
title={Tor u3 - y1}
]
\addplot[const plot, color=mycolor1, forget plot] table[row sep=crcr] {%
1	0\\
2	0\\
3	0\\
4	0\\
5	0\\
6	1.20102906177426\\
7	1.64286296185044\\
8	1.80540457010106\\
9	1.86520028611141\\
10	1.88719790070175\\
11	1.89529037086435\\
12	1.89826742426546\\
13	1.899362621007\\
14	1.89976552137225\\
15	1.89991374013347\\
16	1.89996826676851\\
17	1.89998832599654\\
18	1.89999570537413\\
19	1.89999842009542\\
20	1.89999941878557\\
21	1.89999978618313\\
22	1.89999992134113\\
23	1.89999997106297\\
24	1.8999999893546\\
25	1.8999999960837\\
26	1.89999999855918\\
27	1.89999999946985\\
28	1.89999999980485\\
29	1.89999999992807\\
30	1.89999999997338\\
31	1.89999999999004\\
32	1.89999999999615\\
33	1.89999999999838\\
34	1.89999999999918\\
35	1.89999999999946\\
36	1.89999999999954\\
37	1.89999999999956\\
38	1.89999999999955\\
39	1.89999999999953\\
40	1.89999999999951\\
41	1.8999999999995\\
42	1.89999999999948\\
43	1.89999999999946\\
44	1.89999999999944\\
45	1.89999999999943\\
46	1.89999999999941\\
47	1.89999999999939\\
48	1.89999999999936\\
49	1.89999999999934\\
50	1.89999999999932\\
51	1.89999999999931\\
52	1.89999999999929\\
53	1.89999999999927\\
54	1.89999999999925\\
55	1.89999999999924\\
56	1.89999999999922\\
57	1.8999999999992\\
58	1.89999999999919\\
59	1.89999999999917\\
60	1.89999999999916\\
61	1.89999999999915\\
62	1.89999999999914\\
63	1.89999999999913\\
64	1.89999999999912\\
65	1.89999999999911\\
66	1.8999999999991\\
67	1.8999999999991\\
68	1.89999999999909\\
69	1.89999999999909\\
70	1.89999999999909\\
71	1.89999999999909\\
72	1.89999999999909\\
73	1.89999999999909\\
74	1.89999999999909\\
75	1.89999999999909\\
76	1.89999999999909\\
77	1.8999999999991\\
78	1.8999999999991\\
79	1.8999999999991\\
80	1.8999999999991\\
81	1.8999999999991\\
82	1.8999999999991\\
83	1.8999999999991\\
84	1.8999999999991\\
85	1.89999999999911\\
86	1.89999999999911\\
87	1.89999999999911\\
88	1.89999999999911\\
89	1.89999999999911\\
90	1.89999999999912\\
91	1.89999999999912\\
92	1.89999999999912\\
93	1.89999999999912\\
94	1.89999999999912\\
95	1.89999999999911\\
96	1.89999999999911\\
97	1.89999999999911\\
98	1.89999999999911\\
99	1.89999999999911\\
100	1.89999999999911\\
101	1.89999999999911\\
102	1.8999999999991\\
103	1.8999999999991\\
104	1.8999999999991\\
105	1.8999999999991\\
106	1.8999999999991\\
107	1.8999999999991\\
108	1.8999999999991\\
109	1.89999999999909\\
110	1.89999999999909\\
111	1.89999999999909\\
112	1.89999999999908\\
113	1.89999999999908\\
114	1.89999999999908\\
115	1.89999999999908\\
116	1.89999999999907\\
117	1.89999999999907\\
118	1.89999999999907\\
119	1.89999999999907\\
120	1.89999999999907\\
121	1.89999999999906\\
122	1.89999999999906\\
123	1.89999999999906\\
124	1.89999999999905\\
125	1.89999999999904\\
126	1.89999999999903\\
127	1.89999999999903\\
128	1.89999999999902\\
129	1.89999999999902\\
130	1.89999999999901\\
131	1.89999999999901\\
132	1.899999999999\\
133	1.899999999999\\
134	1.899999999999\\
135	1.89999999999899\\
136	1.89999999999899\\
137	1.89999999999899\\
138	1.89999999999899\\
139	1.89999999999899\\
140	1.89999999999899\\
141	1.899999999999\\
142	1.899999999999\\
143	1.899999999999\\
144	1.89999999999901\\
145	1.89999999999901\\
146	1.89999999999901\\
147	1.89999999999901\\
148	1.89999999999902\\
149	1.89999999999902\\
150	1.89999999999903\\
151	1.89999999999903\\
152	1.89999999999904\\
153	1.89999999999905\\
154	1.89999999999906\\
155	1.89999999999908\\
156	1.89999999999909\\
157	1.8999999999991\\
158	1.89999999999911\\
159	1.89999999999913\\
160	1.89999999999914\\
161	1.89999999999915\\
162	1.89999999999917\\
163	1.89999999999918\\
164	1.89999999999919\\
165	1.8999999999992\\
166	1.89999999999922\\
167	1.89999999999923\\
168	1.89999999999924\\
169	1.89999999999925\\
170	1.89999999999926\\
171	1.89999999999927\\
172	1.89999999999928\\
173	1.89999999999928\\
174	1.89999999999928\\
175	1.89999999999929\\
176	1.89999999999929\\
177	1.89999999999929\\
178	1.89999999999929\\
179	1.89999999999928\\
180	1.89999999999928\\
181	1.89999999999928\\
182	1.89999999999928\\
183	1.89999999999927\\
184	1.89999999999927\\
185	1.89999999999926\\
186	1.89999999999925\\
187	1.89999999999924\\
188	1.89999999999923\\
189	1.89999999999922\\
190	1.89999999999921\\
191	1.8999999999992\\
192	1.89999999999919\\
193	1.89999999999919\\
194	1.89999999999918\\
195	1.89999999999918\\
196	1.89999999999917\\
197	1.89999999999917\\
198	1.89999999999916\\
199	1.89999999999916\\
200	1.89999999999916\\
201	1.89999999999916\\
202	1.89999999999916\\
203	1.89999999999916\\
204	1.89999999999916\\
205	1.89999999999916\\
206	1.89999999999916\\
207	1.89999999999916\\
208	1.89999999999916\\
209	1.89999999999916\\
210	1.89999999999916\\
211	1.89999999999917\\
212	1.89999999999917\\
213	1.89999999999917\\
214	1.89999999999916\\
215	1.89999999999916\\
216	1.89999999999916\\
217	1.89999999999916\\
218	1.89999999999915\\
219	1.89999999999914\\
220	1.89999999999914\\
221	1.89999999999913\\
222	1.89999999999913\\
223	1.89999999999912\\
224	1.89999999999911\\
225	1.89999999999911\\
226	1.8999999999991\\
227	1.8999999999991\\
228	1.89999999999909\\
229	1.89999999999909\\
230	1.89999999999909\\
231	1.89999999999909\\
232	1.89999999999909\\
233	1.89999999999909\\
234	1.89999999999909\\
235	1.89999999999909\\
236	1.89999999999909\\
237	1.8999999999991\\
238	1.8999999999991\\
239	1.8999999999991\\
240	1.8999999999991\\
241	1.8999999999991\\
242	1.8999999999991\\
243	1.8999999999991\\
244	1.89999999999911\\
245	1.89999999999911\\
246	1.89999999999911\\
247	1.89999999999911\\
248	1.89999999999911\\
249	1.89999999999911\\
250	1.89999999999911\\
251	1.89999999999911\\
252	1.89999999999911\\
253	1.8999999999991\\
254	1.8999999999991\\
255	1.89999999999909\\
256	1.89999999999909\\
257	1.89999999999908\\
258	1.89999999999908\\
259	1.89999999999907\\
260	1.89999999999906\\
261	1.89999999999906\\
262	1.89999999999905\\
263	1.89999999999905\\
264	1.89999999999904\\
265	1.89999999999904\\
266	1.89999999999904\\
267	1.89999999999903\\
268	1.89999999999903\\
269	1.89999999999902\\
270	1.89999999999902\\
271	1.89999999999902\\
272	1.89999999999901\\
273	1.89999999999901\\
274	1.89999999999901\\
275	1.89999999999902\\
276	1.89999999999902\\
277	1.89999999999903\\
278	1.89999999999904\\
279	1.89999999999904\\
280	1.89999999999905\\
281	1.89999999999906\\
282	1.89999999999907\\
283	1.89999999999908\\
284	1.89999999999908\\
285	1.89999999999909\\
286	1.89999999999909\\
287	1.8999999999991\\
288	1.89999999999911\\
289	1.89999999999911\\
290	1.89999999999912\\
291	1.89999999999912\\
292	1.89999999999912\\
293	1.89999999999913\\
294	1.89999999999913\\
295	1.89999999999913\\
296	1.89999999999913\\
297	1.89999999999913\\
298	1.89999999999913\\
299	1.89999999999912\\
300	1.89999999999912\\
};
\end{axis}

\begin{axis}[%
width=1.68in,
height=1.553in,
at={(3.235in,3.239in)},
scale only axis,
xmin=0,
xmax=300,
ymin=0,
ymax=2,
axis background/.style={fill=white},
title style={font=\bfseries},
title={Tor u3 - y2}
]
\addplot[const plot, color=mycolor1, forget plot] table[row sep=crcr] {%
1	0\\
2	0\\
3	0\\
4	0\\
5	0\\
6	0.0190325163928081\\
7	0.0362538493844037\\
8	0.0518363558636565\\
9	0.0659359907928722\\
10	0.0786938680574735\\
11	0.0902376727811949\\
12	0.100682939241719\\
13	0.110134207176556\\
14	0.118686068051881\\
15	0.126424111765713\\
16	0.133425783260386\\
17	0.139761157617563\\
18	0.145493641393202\\
19	0.150680607211684\\
20	0.15537396797032\\
21	0.159620696401076\\
22	0.163463295189461\\
23	0.166940222355691\\
24	0.170086276155481\\
25	0.172932943352686\\
26	0.175508714349412\\
27	0.177839368327541\\
28	0.179948231255447\\
29	0.181856409342125\\
30	0.183583000275227\\
31	0.185145284357139\\
32	0.186558897452055\\
33	0.187837987474959\\
34	0.188995355988719\\
35	0.190042586326424\\
36	0.190990159521282\\
37	0.191847559204316\\
38	0.192623366519737\\
39	0.193325346007915\\
40	0.193960523315512\\
41	0.194535255510512\\
42	0.195055294705898\\
43	0.195525845628727\\
44	0.195951617710794\\
45	0.196336872222203\\
46	0.196685464919591\\
47	0.197000884635842\\
48	0.197286288197492\\
49	0.197544532019312\\
50	0.197778200692271\\
51	0.197989632850986\\
52	0.198180944579566\\
53	0.198354050590094\\
54	0.198510683385706\\
55	0.198652410600066\\
56	0.198780650686773\\
57	0.198896687115716\\
58	0.199001681218478\\
59	0.19909668381133\\
60	0.199182645712152\\
61	0.19926042725654\\
62	0.199330806908334\\
63	0.199394489050745\\
64	0.199452111036059\\
65	0.199504249564471\\
66	0.199551426455899\\
67	0.199594113872528\\
68	0.199632739044373\\
69	0.199667688545136\\
70	0.199699312161167\\
71	0.199727926392244\\
72	0.199753817619211\\
73	0.199777244970168\\
74	0.199798442913919\\
75	0.19981762360661\\
76	0.199834979015059\\
77	0.199850682838029\\
78	0.199864892244659\\
79	0.199877749447464\\
80	0.199889383125653\\
81	0.199899909712988\\
82	0.199909434563092\\
83	0.199918053003867\\
84	0.199925851291565\\
85	0.19993290747407\\
86	0.199939292172029\\
87	0.199945069285644\\
88	0.199950296634211\\
89	0.199955026534791\\
90	0.199959306325819\\
91	0.199963178840882\\
92	0.199966682837412\\
93	0.199969853384585\\
94	0.199972722214302\\
95	0.199975318038774\\
96	0.199977666837887\\
97	0.199979792119211\\
98	0.199981715153277\\
99	0.199983455186454\\
100	0.199985029633581\\
101	0.199986454252254\\
102	0.199987743300536\\
103	0.199988909679654\\
104	0.199989965063124\\
105	0.199990920013577\\
106	0.199991784088479\\
107	0.199992565935783\\
108	0.199993273380478\\
109	0.199993913502909\\
110	0.199994492709637\\
111	0.199995016797558\\
112	0.199995491011918\\
113	0.199995920098816\\
114	0.199996308352696\\
115	0.199996659659335\\
116	0.199996977534728\\
117	0.199997265160277\\
118	0.199997525414638\\
119	0.199997760902521\\
120	0.19999797398077\\
121	0.199998166781943\\
122	0.199998341235659\\
123	0.199998499087909\\
124	0.199998641918531\\
125	0.199998771157023\\
126	0.199998888096846\\
127	0.199998993908374\\
128	0.199999089650604\\
129	0.199999176281756\\
130	0.199999254668864\\
131	0.199999325596453\\
132	0.199999389774389\\
133	0.199999447844988\\
134	0.199999500389438\\
135	0.199999547933623\\
136	0.199999590953381\\
137	0.199999629879268\\
138	0.199999665100867\\
139	0.199999696970687\\
140	0.199999725807693\\
141	0.199999751900495\\
142	0.199999775510238\\
143	0.199999796873217\\
144	0.199999816203239\\
145	0.199999833693766\\
146	0.199999849519849\\
147	0.199999863839881\\
148	0.199999876797181\\
149	0.199999888521431\\
150	0.19999989912997\\
151	0.199999908728974\\
152	0.199999917414511\\
153	0.199999925273509\\
154	0.199999932384625\\
155	0.199999938819028\\
156	0.199999944641117\\
157	0.19999994990916\\
158	0.199999954675882\\
159	0.19999995898899\\
160	0.199999962891651\\
161	0.199999966422924\\
162	0.199999969618152\\
163	0.199999972509313\\
164	0.199999975125343\\
165	0.199999977492425\\
166	0.199999979634248\\
167	0.19999998157225\\
168	0.199999983325826\\
169	0.199999984912527\\
170	0.199999986348234\\
171	0.199999987647315\\
172	0.199999988822772\\
173	0.19999998988637\\
174	0.199999990848753\\
175	0.199999991719553\\
176	0.199999992507486\\
177	0.199999993220437\\
178	0.199999993865543\\
179	0.199999994449258\\
180	0.199999994977425\\
181	0.199999995455331\\
182	0.199999995887758\\
183	0.199999996279035\\
184	0.199999996633076\\
185	0.199999996953426\\
186	0.199999997243291\\
187	0.199999997505571\\
188	0.199999997742891\\
189	0.199999997957628\\
190	0.19999999815193\\
191	0.199999998327741\\
192	0.199999998486821\\
193	0.199999998630763\\
194	0.199999998761007\\
195	0.199999998878856\\
196	0.19999999898549\\
197	0.199999999081977\\
198	0.199999999169281\\
199	0.199999999248278\\
200	0.199999999319756\\
201	0.199999999384433\\
202	0.199999999442955\\
203	0.199999999495908\\
204	0.199999999543822\\
205	0.199999999587176\\
206	0.199999999626405\\
207	0.199999999661902\\
208	0.19999999969402\\
209	0.199999999723083\\
210	0.19999999974938\\
211	0.199999999773174\\
212	0.199999999794704\\
213	0.199999999814185\\
214	0.199999999831812\\
215	0.199999999847761\\
216	0.199999999862192\\
217	0.199999999875249\\
218	0.199999999887064\\
219	0.199999999897754\\
220	0.199999999907426\\
221	0.199999999916178\\
222	0.199999999924096\\
223	0.199999999931261\\
224	0.199999999937745\\
225	0.199999999943611\\
226	0.199999999948919\\
227	0.199999999953722\\
228	0.199999999958067\\
229	0.199999999961999\\
230	0.199999999965557\\
231	0.199999999968776\\
232	0.199999999971689\\
233	0.199999999974325\\
234	0.199999999976709\\
235	0.199999999978867\\
236	0.199999999980819\\
237	0.199999999982586\\
238	0.199999999984184\\
239	0.199999999985631\\
240	0.199999999986939\\
241	0.199999999988123\\
242	0.199999999989194\\
243	0.199999999990162\\
244	0.199999999991039\\
245	0.199999999991832\\
246	0.19999999999255\\
247	0.1999999999932\\
248	0.199999999993788\\
249	0.19999999999432\\
250	0.199999999994801\\
251	0.199999999995237\\
252	0.199999999995631\\
253	0.199999999995988\\
254	0.199999999996311\\
255	0.199999999996602\\
256	0.199999999996866\\
257	0.199999999997105\\
258	0.199999999997321\\
259	0.199999999997517\\
260	0.199999999997693\\
261	0.199999999997853\\
262	0.199999999997997\\
263	0.199999999998127\\
264	0.199999999998244\\
265	0.199999999998349\\
266	0.199999999998444\\
267	0.19999999999853\\
268	0.199999999998607\\
269	0.199999999998676\\
270	0.199999999998738\\
271	0.199999999998794\\
272	0.199999999998845\\
273	0.199999999998891\\
274	0.199999999998932\\
275	0.19999999999897\\
276	0.199999999999004\\
277	0.199999999999035\\
278	0.199999999999063\\
279	0.199999999999089\\
280	0.199999999999112\\
281	0.199999999999133\\
282	0.199999999999151\\
283	0.199999999999168\\
284	0.199999999999183\\
285	0.199999999999197\\
286	0.199999999999209\\
287	0.19999999999922\\
288	0.19999999999923\\
289	0.199999999999239\\
290	0.199999999999247\\
291	0.199999999999254\\
292	0.19999999999926\\
293	0.199999999999266\\
294	0.199999999999271\\
295	0.199999999999275\\
296	0.19999999999928\\
297	0.199999999999284\\
298	0.199999999999287\\
299	0.199999999999291\\
300	0.199999999999295\\
};
\end{axis}

\begin{axis}[%
width=1.68in,
height=1.553in,
at={(5.446in,3.239in)},
scale only axis,
xmin=0,
xmax=300,
ymin=0,
ymax=2,
axis background/.style={fill=white},
title style={font=\bfseries},
title={Tor u3 - y3}
]
\addplot[const plot, color=mycolor1, forget plot] table[row sep=crcr] {%
1	0\\
2	0\\
3	0\\
4	0\\
5	0\\
6	0.331798825392893\\
7	0.59020401043105\\
8	0.791450170888478\\
9	0.948180838242837\\
10	1.07024280470972\\
11	1.16530475977736\\
12	1.23933908482434\\
13	1.2969970751451\\
14	1.34190116315723\\
15	1.37687250206419\\
16	1.40410820818999\\
17	1.42531939744827\\
18	1.44183868825251\\
19	1.45470392486664\\
20	1.46472338121612\\
21	1.47252654166706\\
22	1.47860364913669\\
23	1.48333650519285\\
24	1.48702245719556\\
25	1.48989307950163\\
26	1.49212872240151\\
27	1.49386984284261\\
28	1.49522582880556\\
29	1.49628187173535\\
30	1.49710431879603\\
31	1.49774484121093\\
32	1.49824368056923\\
33	1.49863217705211\\
34	1.4989347384172\\
35	1.49917037344527\\
36	1.49935388618966\\
37	1.4994968060587\\
38	1.49960811216463\\
39	1.49969479744709\\
40	1.49976230801297\\
41	1.49981488529454\\
42	1.4998558325226\\
43	1.49988772225589\\
44	1.49991255800516\\
45	1.49993190010614\\
46	1.49994696374954\\
47	1.49995869532682\\
48	1.49996783188841\\
49	1.49997494744973\\
50	1.49998048905446\\
51	1.49998480486057\\
52	1.49998816601376\\
53	1.49999078368249\\
54	1.49999282232496\\
55	1.49999441002131\\
56	1.49999564652048\\
57	1.49999660950701\\
58	1.49999735948168\\
59	1.49999794356255\\
60	1.49999839844519\\
61	1.49999875270815\\
62	1.49999902860842\\
63	1.49999924347978\\
64	1.49999941082176\\
65	1.49999954114783\\
66	1.49999964264587\\
67	1.49999972169264\\
68	1.49999978325432\\
69	1.49999983119861\\
70	1.49999986853766\\
71	1.49999989761734\\
72	1.49999992026463\\
73	1.49999993790235\\
74	1.49999995163863\\
75	1.49999996233646\\
76	1.49999997066794\\
77	1.4999999771565\\
78	1.49999998220981\\
79	1.49999998614533\\
80	1.49999998921032\\
81	1.49999999159735\\
82	1.49999999345637\\
83	1.49999999490417\\
84	1.49999999603173\\
85	1.49999999690988\\
86	1.49999999759378\\
87	1.49999999812641\\
88	1.49999999854122\\
89	1.49999999886428\\
90	1.49999999911587\\
91	1.49999999931182\\
92	1.49999999946442\\
93	1.49999999958327\\
94	1.49999999967582\\
95	1.49999999974791\\
96	1.49999999980405\\
97	1.49999999984778\\
98	1.49999999988184\\
99	1.49999999990836\\
100	1.49999999992902\\
101	1.49999999994512\\
102	1.49999999995765\\
103	1.49999999996742\\
104	1.49999999997503\\
105	1.49999999998096\\
106	1.49999999998558\\
107	1.49999999998919\\
108	1.49999999999199\\
109	1.49999999999418\\
110	1.49999999999588\\
111	1.4999999999972\\
112	1.49999999999824\\
113	1.49999999999904\\
114	1.49999999999966\\
115	1.50000000000014\\
116	1.50000000000051\\
117	1.5000000000008\\
118	1.50000000000102\\
119	1.5000000000012\\
120	1.50000000000133\\
121	1.50000000000144\\
122	1.50000000000152\\
123	1.50000000000159\\
124	1.50000000000164\\
125	1.50000000000169\\
126	1.50000000000172\\
127	1.50000000000175\\
128	1.50000000000178\\
129	1.5000000000018\\
130	1.50000000000182\\
131	1.50000000000183\\
132	1.50000000000184\\
133	1.50000000000185\\
134	1.50000000000185\\
135	1.50000000000185\\
136	1.50000000000185\\
137	1.50000000000185\\
138	1.50000000000185\\
139	1.50000000000184\\
140	1.50000000000183\\
141	1.50000000000183\\
142	1.50000000000182\\
143	1.50000000000181\\
144	1.50000000000181\\
145	1.5000000000018\\
146	1.50000000000179\\
147	1.50000000000178\\
148	1.50000000000177\\
149	1.50000000000176\\
150	1.50000000000175\\
151	1.50000000000174\\
152	1.50000000000173\\
153	1.50000000000171\\
154	1.5000000000017\\
155	1.50000000000169\\
156	1.50000000000168\\
157	1.50000000000167\\
158	1.50000000000167\\
159	1.50000000000167\\
160	1.50000000000167\\
161	1.50000000000167\\
162	1.50000000000167\\
163	1.50000000000168\\
164	1.50000000000168\\
165	1.50000000000168\\
166	1.50000000000168\\
167	1.50000000000169\\
168	1.50000000000169\\
169	1.50000000000169\\
170	1.5000000000017\\
171	1.5000000000017\\
172	1.5000000000017\\
173	1.50000000000171\\
174	1.50000000000171\\
175	1.50000000000171\\
176	1.50000000000172\\
177	1.50000000000172\\
178	1.50000000000173\\
179	1.50000000000174\\
180	1.50000000000174\\
181	1.50000000000175\\
182	1.50000000000176\\
183	1.50000000000177\\
184	1.50000000000177\\
185	1.50000000000178\\
186	1.50000000000179\\
187	1.50000000000179\\
188	1.5000000000018\\
189	1.50000000000181\\
190	1.50000000000181\\
191	1.50000000000182\\
192	1.50000000000182\\
193	1.50000000000182\\
194	1.50000000000183\\
195	1.50000000000183\\
196	1.50000000000183\\
197	1.50000000000182\\
198	1.50000000000182\\
199	1.50000000000181\\
200	1.5000000000018\\
201	1.5000000000018\\
202	1.50000000000179\\
203	1.50000000000178\\
204	1.50000000000177\\
205	1.50000000000176\\
206	1.50000000000175\\
207	1.50000000000175\\
208	1.50000000000174\\
209	1.50000000000174\\
210	1.50000000000173\\
211	1.50000000000173\\
212	1.50000000000172\\
213	1.50000000000172\\
214	1.50000000000171\\
215	1.5000000000017\\
216	1.50000000000169\\
217	1.50000000000168\\
218	1.50000000000167\\
219	1.50000000000166\\
220	1.50000000000165\\
221	1.50000000000164\\
222	1.50000000000163\\
223	1.50000000000162\\
224	1.50000000000161\\
225	1.5000000000016\\
226	1.50000000000159\\
227	1.50000000000158\\
228	1.50000000000157\\
229	1.50000000000156\\
230	1.50000000000155\\
231	1.50000000000154\\
232	1.50000000000153\\
233	1.50000000000152\\
234	1.50000000000151\\
235	1.5000000000015\\
236	1.50000000000149\\
237	1.50000000000148\\
238	1.50000000000147\\
239	1.50000000000147\\
240	1.50000000000146\\
241	1.50000000000146\\
242	1.50000000000145\\
243	1.50000000000145\\
244	1.50000000000144\\
245	1.50000000000144\\
246	1.50000000000144\\
247	1.50000000000144\\
248	1.50000000000144\\
249	1.50000000000144\\
250	1.50000000000144\\
251	1.50000000000143\\
252	1.50000000000143\\
253	1.50000000000143\\
254	1.50000000000142\\
255	1.50000000000142\\
256	1.50000000000141\\
257	1.50000000000141\\
258	1.5000000000014\\
259	1.5000000000014\\
260	1.50000000000139\\
261	1.50000000000138\\
262	1.50000000000137\\
263	1.50000000000137\\
264	1.50000000000136\\
265	1.50000000000135\\
266	1.50000000000135\\
267	1.50000000000134\\
268	1.50000000000133\\
269	1.50000000000133\\
270	1.50000000000133\\
271	1.50000000000132\\
272	1.50000000000132\\
273	1.50000000000131\\
274	1.5000000000013\\
275	1.5000000000013\\
276	1.50000000000129\\
277	1.50000000000129\\
278	1.50000000000128\\
279	1.50000000000128\\
280	1.50000000000128\\
281	1.50000000000128\\
282	1.50000000000129\\
283	1.50000000000129\\
284	1.5000000000013\\
285	1.5000000000013\\
286	1.50000000000131\\
287	1.50000000000131\\
288	1.50000000000132\\
289	1.50000000000132\\
290	1.50000000000133\\
291	1.50000000000134\\
292	1.50000000000135\\
293	1.50000000000135\\
294	1.50000000000136\\
295	1.50000000000137\\
296	1.50000000000138\\
297	1.50000000000139\\
298	1.5000000000014\\
299	1.50000000000141\\
300	1.50000000000142\\
};
\end{axis}

\begin{axis}[%
width=1.68in,
height=1.553in,
at={(1.024in,1.083in)},
scale only axis,
xmin=0,
xmax=300,
ymin=0,
ymax=2,
axis background/.style={fill=white},
title style={font=\bfseries},
title={Tor u4 - y1}
]
\addplot[const plot, color=mycolor1, forget plot] table[row sep=crcr] {%
1	0\\
2	0\\
3	0\\
4	0\\
5	0\\
6	0.0787630183424815\\
7	0.152754026640026\\
8	0.22226214636548\\
9	0.287558982007174\\
10	0.348899682369366\\
11	0.406523937571737\\
12	0.460656915643741\\
13	0.511510142373577\\
14	0.5592823278498\\
15	0.604160142925312\\
16	0.646318948637774\\
17	0.685923481436677\\
18	0.72312849689459\\
19	0.758079374417931\\
20	0.790912685320151\\
21	0.821756726477113\\
22	0.85073202164992\\
23	0.877951792434131\\
24	0.903522400675608\\
25	0.927543764081738\\
26	0.950109746652047\\
27	0.971308525453817\\
28	0.991222935175894\\
29	1.00993079180703\\
30	1.02750519670357\\
31	1.04401482223454\\
32	1.0595241801205\\
33	1.07409387351442\\
34	1.08778083380985\\
35	1.10063854310159\\
36	1.1127172431681\\
37	1.12406413179239\\
38	1.13472354718834\\
39	1.14473714125325\\
40	1.15414404232353\\
41	1.16298100806955\\
42	1.171282569127\\
43	1.17908116402611\\
44	1.18640726594592\\
45	1.19328950178889\\
46	1.19975476404115\\
47	1.20582831585543\\
48	1.21153388976744\\
49	1.21689378043123\\
50	1.22192893173605\\
51	1.22665901864503\\
52	1.23110252407558\\
53	1.23527681112172\\
54	1.23919819090079\\
55	1.24288198628951\\
56	1.2463425917984\\
57	1.24959352981869\\
58	1.25264750346136\\
59	1.25551644619476\\
60	1.25821156847499\\
61	1.26074340155091\\
62	1.2631218386153\\
63	1.26535617346266\\
64	1.26745513680498\\
65	1.26942693038711\\
66	1.27127925903534\\
67	1.27301936076412\\
68	1.27465403505875\\
69	1.27618966944457\\
70	1.27763226444632\\
71	1.27898745703531\\
72	1.28026054265603\\
73	1.28145649591822\\
74	1.28257999003524\\
75	1.28363541508476\\
76	1.2846268951631\\
77	1.28555830450022\\
78	1.28643328259833\\
79	1.28725524845337\\
80	1.28802741391479\\
81	1.28875279623589\\
82	1.28943422986388\\
83	1.29007437751544\\
84	1.29067574058145\\
85	1.29124066890116\\
86	1.29177136994424\\
87	1.29226991743656\\
88	1.29273825946328\\
89	1.29317822608105\\
90	1.29359153646898\\
91	1.2939798056464\\
92	1.29434455078355\\
93	1.29468719712999\\
94	1.29500908358376\\
95	1.29531146792318\\
96	1.29559553172161\\
97	1.29586238496454\\
98	1.29611307038679\\
99	1.29634856754712\\
100	1.29656979665578\\
101	1.29677762217033\\
102	1.29697285617349\\
103	1.29715626154636\\
104	1.29732855494943\\
105	1.29749040962291\\
106	1.29764245801745\\
107	1.29778529426547\\
108	1.29791947650269\\
109	1.29804552904914\\
110	1.29816394445787\\
111	1.29827518543968\\
112	1.2983796866711\\
113	1.29847785649298\\
114	1.29857007850603\\
115	1.29865671306977\\
116	1.29873809871063\\
117	1.29881455344479\\
118	1.29888637602076\\
119	1.29895384708684\\
120	1.29901723028768\\
121	1.2990767732945\\
122	1.29913270877291\\
123	1.29918525529201\\
124	1.29923461817845\\
125	1.29928099031879\\
126	1.29932455291317\\
127	1.29936547618338\\
128	1.29940392003799\\
129	1.2994400346972\\
130	1.29947396127981\\
131	1.2995058323547\\
132	1.29953577245877\\
133	1.29956389858364\\
134	1.29959032063274\\
135	1.29961514185081\\
136	1.2996384592273\\
137	1.29966036387537\\
138	1.2996809413879\\
139	1.29970027217197\\
140	1.29971843176304\\
141	1.29973549112011\\
142	1.29975151690298\\
143	1.29976657173275\\
144	1.29978071443649\\
145	1.29979400027714\\
146	1.29980648116939\\
147	1.29981820588261\\
148	1.29982922023136\\
149	1.29983956725445\\
150	1.29984928738311\\
151	1.29985841859894\\
152	1.29986699658237\\
153	1.29987505485206\\
154	1.29988262489587\\
155	1.2998897362939\\
156	1.29989641683411\\
157	1.29990269262085\\
158	1.29990858817689\\
159	1.29991412653925\\
160	1.2999193293492\\
161	1.29992421693682\\
162	1.29992880840048\\
163	1.29993312168142\\
164	1.29993717363388\\
165	1.29994098009095\\
166	1.29994455592644\\
167	1.29994791511301\\
168	1.29995107077676\\
169	1.29995403524851\\
170	1.29995682011199\\
171	1.29995943624912\\
172	1.29996189388252\\
173	1.29996420261544\\
174	1.2999663714693\\
175	1.29996840891894\\
176	1.29997032292576\\
177	1.29997212096876\\
178	1.29997381007385\\
179	1.29997539684123\\
180	1.29997688747123\\
181	1.29997828778853\\
182	1.29997960326489\\
183	1.29998083904057\\
184	1.29998199994438\\
185	1.29998309051259\\
186	1.29998411500661\\
187	1.29998507742967\\
188	1.29998598154247\\
189	1.29998683087784\\
190	1.29998762875458\\
191	1.29998837829041\\
192	1.29998908241416\\
193	1.29998974387721\\
194	1.29999036526424\\
195	1.29999094900334\\
196	1.29999149737547\\
197	1.29999201252341\\
198	1.29999249646012\\
199	1.29999295107658\\
200	1.29999337814923\\
201	1.29999377934685\\
202	1.29999415623713\\
203	1.29999451029279\\
204	1.2999948428973\\
205	1.29999515535032\\
206	1.29999544887277\\
207	1.2999957246116\\
208	1.29999598364425\\
209	1.2999962269829\\
210	1.29999645557841\\
211	1.29999667032402\\
212	1.29999687205885\\
213	1.29999706157119\\
214	1.29999723960155\\
215	1.2999974068456\\
216	1.29999756395684\\
217	1.29999771154919\\
218	1.29999785019938\\
219	1.29999798044917\\
220	1.29999810280754\\
221	1.29999821775258\\
222	1.29999832573346\\
223	1.29999842717211\\
224	1.2999985224649\\
225	1.29999861198419\\
226	1.29999869607979\\
227	1.29999877508029\\
228	1.29999884929439\\
229	1.29999891901209\\
230	1.29999898450581\\
231	1.29999904603146\\
232	1.29999910382947\\
233	1.29999915812567\\
234	1.29999920913222\\
235	1.29999925704845\\
236	1.29999930206158\\
237	1.29999934434751\\
238	1.29999938407146\\
239	1.29999942138866\\
240	1.29999945644492\\
241	1.29999948937724\\
242	1.29999952031428\\
243	1.29999954937695\\
244	1.2999995766788\\
245	1.29999960232652\\
246	1.29999962642032\\
247	1.29999964905434\\
248	1.29999967031704\\
249	1.2999996902915\\
250	1.29999970905577\\
251	1.29999972668317\\
252	1.29999974324258\\
253	1.2999997587987\\
254	1.29999977341233\\
255	1.29999978714056\\
256	1.29999980003704\\
257	1.29999981215217\\
258	1.29999982353327\\
259	1.29999983422483\\
260	1.29999984426863\\
261	1.29999985370389\\
262	1.29999986256751\\
263	1.2999998708941\\
264	1.29999987871621\\
265	1.29999988606441\\
266	1.2999998929674\\
267	1.29999989945216\\
268	1.29999990554403\\
269	1.29999991126681\\
270	1.29999991664286\\
271	1.2999999216932\\
272	1.29999992643755\\
273	1.29999993089446\\
274	1.29999993508133\\
275	1.29999993901454\\
276	1.29999994270944\\
277	1.29999994618048\\
278	1.29999994944122\\
279	1.2999999525044\\
280	1.29999995538199\\
281	1.29999995808523\\
282	1.29999996062469\\
283	1.2999999630103\\
284	1.29999996525136\\
285	1.29999996735665\\
286	1.29999996933438\\
287	1.29999997119229\\
288	1.29999997293763\\
289	1.29999997457723\\
290	1.29999997611749\\
291	1.29999997756444\\
292	1.29999997892372\\
293	1.29999998020065\\
294	1.29999998140021\\
295	1.2999999825271\\
296	1.29999998358571\\
297	1.29999998458019\\
298	1.29999998551441\\
299	1.29999998639203\\
300	1.29999998721648\\
};
\end{axis}

\begin{axis}[%
width=1.68in,
height=1.553in,
at={(3.235in,1.083in)},
scale only axis,
xmin=0,
xmax=300,
ymin=0,
ymax=2,
axis background/.style={fill=white},
title style={font=\bfseries},
title={Tor u4 - y2}
]
\addplot[const plot, color=mycolor1, forget plot] table[row sep=crcr] {%
1	0\\
2	0\\
3	0\\
4	0\\
5	0\\
6	0.373795873272998\\
7	0.600514530887129\\
8	0.73802634785899\\
9	0.821431480925214\\
10	0.872019251307288\\
11	0.902702285050514\\
12	0.921312485748772\\
13	0.932600143055663\\
14	0.939446453288612\\
15	0.943598950350788\\
16	0.946117567133352\\
17	0.947645185432029\\
18	0.948571732766497\\
19	0.949133712132509\\
20	0.9494745698481\\
21	0.949681310503184\\
22	0.949806705049079\\
23	0.949882760685702\\
24	0.949928890761133\\
25	0.949956870066191\\
26	0.949973840372521\\
27	0.949984133383588\\
28	0.949990376410353\\
29	0.949994162997468\\
30	0.949996459678621\\
31	0.949997852686127\\
32	0.94999869758786\\
33	0.949999210046637\\
34	0.949999520868569\\
35	0.94999970939157\\
36	0.94999982373652\\
37	0.949999893090207\\
38	0.949999935155312\\
39	0.949999960669055\\
40	0.949999976143889\\
41	0.949999985529817\\
42	0.949999991222637\\
43	0.949999994675475\\
44	0.949999996769694\\
45	0.949999998039871\\
46	0.949999998810242\\
47	0.949999999277465\\
48	0.94999999956082\\
49	0.949999999732654\\
50	0.949999999836848\\
51	0.949999999900017\\
52	0.949999999938303\\
53	0.949999999961499\\
54	0.949999999975541\\
55	0.949999999984032\\
56	0.949999999989156\\
57	0.949999999992237\\
58	0.94999999999408\\
59	0.949999999995171\\
60	0.949999999995807\\
61	0.949999999996167\\
62	0.94999999999636\\
63	0.949999999996453\\
64	0.949999999996484\\
65	0.94999999999648\\
66	0.949999999996453\\
67	0.949999999996413\\
68	0.949999999996366\\
69	0.949999999996315\\
70	0.949999999996262\\
71	0.949999999996208\\
72	0.949999999996154\\
73	0.9499999999961\\
74	0.949999999996048\\
75	0.949999999995997\\
76	0.949999999995948\\
77	0.949999999995901\\
78	0.949999999995857\\
79	0.949999999995815\\
80	0.949999999995777\\
81	0.949999999995742\\
82	0.94999999999571\\
83	0.949999999995682\\
84	0.949999999995656\\
85	0.949999999995631\\
86	0.949999999995607\\
87	0.949999999995584\\
88	0.949999999995561\\
89	0.949999999995538\\
90	0.949999999995515\\
91	0.949999999995493\\
92	0.949999999995472\\
93	0.949999999995453\\
94	0.949999999995435\\
95	0.949999999995419\\
96	0.949999999995404\\
97	0.949999999995392\\
98	0.949999999995382\\
99	0.949999999995375\\
100	0.949999999995368\\
101	0.949999999995364\\
102	0.94999999999536\\
103	0.949999999995358\\
104	0.949999999995357\\
105	0.949999999995357\\
106	0.949999999995358\\
107	0.949999999995359\\
108	0.949999999995361\\
109	0.949999999995365\\
110	0.949999999995369\\
111	0.949999999995374\\
112	0.94999999999538\\
113	0.949999999995387\\
114	0.949999999995396\\
115	0.949999999995405\\
116	0.949999999995415\\
117	0.949999999995426\\
118	0.949999999995437\\
119	0.949999999995447\\
120	0.949999999995457\\
121	0.949999999995466\\
122	0.949999999995475\\
123	0.949999999995484\\
124	0.949999999995492\\
125	0.949999999995499\\
126	0.949999999995506\\
127	0.949999999995512\\
128	0.949999999995517\\
129	0.949999999995522\\
130	0.949999999995526\\
131	0.94999999999553\\
132	0.949999999995534\\
133	0.949999999995538\\
134	0.949999999995543\\
135	0.949999999995548\\
136	0.949999999995554\\
137	0.949999999995561\\
138	0.949999999995568\\
139	0.949999999995577\\
140	0.949999999995586\\
141	0.949999999995593\\
142	0.9499999999956\\
143	0.949999999995604\\
144	0.949999999995606\\
145	0.949999999995607\\
146	0.949999999995606\\
147	0.949999999995603\\
148	0.949999999995598\\
149	0.94999999999559\\
150	0.94999999999558\\
151	0.949999999995567\\
152	0.949999999995552\\
153	0.949999999995533\\
154	0.949999999995512\\
155	0.949999999995487\\
156	0.949999999995461\\
157	0.949999999995434\\
158	0.949999999995406\\
159	0.949999999995377\\
160	0.949999999995346\\
161	0.949999999995314\\
162	0.949999999995281\\
163	0.949999999995246\\
164	0.949999999995211\\
165	0.949999999995175\\
166	0.949999999995138\\
167	0.949999999995102\\
168	0.949999999995065\\
169	0.949999999995028\\
170	0.949999999994994\\
171	0.949999999994962\\
172	0.949999999994933\\
173	0.949999999994906\\
174	0.949999999994883\\
175	0.949999999994862\\
176	0.949999999994842\\
177	0.949999999994825\\
178	0.949999999994809\\
179	0.949999999994795\\
180	0.949999999994784\\
181	0.949999999994773\\
182	0.949999999994764\\
183	0.949999999994757\\
184	0.94999999999475\\
185	0.949999999994743\\
186	0.949999999994737\\
187	0.949999999994731\\
188	0.949999999994727\\
189	0.949999999994723\\
190	0.949999999994719\\
191	0.949999999994716\\
192	0.949999999994715\\
193	0.949999999994713\\
194	0.949999999994711\\
195	0.949999999994709\\
196	0.949999999994705\\
197	0.949999999994699\\
198	0.949999999994691\\
199	0.949999999994682\\
200	0.949999999994672\\
201	0.949999999994661\\
202	0.949999999994648\\
203	0.949999999994633\\
204	0.949999999994617\\
205	0.949999999994601\\
206	0.949999999994584\\
207	0.949999999994565\\
208	0.949999999994545\\
209	0.949999999994524\\
210	0.949999999994503\\
211	0.949999999994481\\
212	0.94999999999446\\
213	0.949999999994438\\
214	0.949999999994417\\
215	0.949999999994396\\
216	0.949999999994374\\
217	0.949999999994352\\
218	0.949999999994331\\
219	0.949999999994309\\
220	0.949999999994287\\
221	0.949999999994266\\
222	0.949999999994245\\
223	0.949999999994224\\
224	0.949999999994204\\
225	0.949999999994184\\
226	0.949999999994164\\
227	0.949999999994144\\
228	0.949999999994123\\
229	0.949999999994104\\
230	0.949999999994084\\
231	0.949999999994065\\
232	0.949999999994048\\
233	0.949999999994032\\
234	0.949999999994019\\
235	0.949999999994009\\
236	0.949999999994002\\
237	0.949999999993997\\
238	0.949999999993996\\
239	0.949999999993998\\
240	0.949999999994002\\
241	0.949999999994007\\
242	0.949999999994013\\
243	0.94999999999402\\
244	0.949999999994029\\
245	0.949999999994037\\
246	0.949999999994046\\
247	0.949999999994054\\
248	0.949999999994061\\
249	0.949999999994067\\
250	0.949999999994071\\
251	0.949999999994072\\
252	0.949999999994069\\
253	0.949999999994064\\
254	0.949999999994055\\
255	0.949999999994045\\
256	0.949999999994033\\
257	0.949999999994021\\
258	0.949999999994009\\
259	0.949999999993998\\
260	0.949999999993987\\
261	0.949999999993978\\
262	0.949999999993969\\
263	0.949999999993961\\
264	0.949999999993955\\
265	0.949999999993948\\
266	0.949999999993943\\
267	0.949999999993938\\
268	0.949999999993935\\
269	0.949999999993934\\
270	0.949999999993936\\
271	0.94999999999394\\
272	0.949999999993947\\
273	0.949999999993957\\
274	0.949999999993971\\
275	0.949999999993988\\
276	0.949999999994008\\
277	0.949999999994031\\
278	0.949999999994057\\
279	0.949999999994083\\
280	0.949999999994111\\
281	0.949999999994139\\
282	0.949999999994168\\
283	0.949999999994198\\
284	0.949999999994229\\
285	0.949999999994261\\
286	0.949999999994291\\
287	0.94999999999432\\
288	0.949999999994347\\
289	0.949999999994373\\
290	0.949999999994399\\
291	0.949999999994424\\
292	0.949999999994449\\
293	0.949999999994475\\
294	0.9499999999945\\
295	0.949999999994526\\
296	0.949999999994551\\
297	0.949999999994575\\
298	0.949999999994599\\
299	0.949999999994621\\
300	0.949999999994642\\
};
\end{axis}

\begin{axis}[%
width=1.68in,
height=1.553in,
at={(5.446in,1.083in)},
scale only axis,
xmin=0,
xmax=300,
ymin=0,
ymax=2,
axis background/.style={fill=white},
title style={font=\bfseries},
title={Tor u4 - y3}
]
\addplot[const plot, color=mycolor1, forget plot] table[row sep=crcr] {%
1	0\\
2	0\\
3	0\\
4	0\\
5	0\\
6	0.0383795687773465\\
7	0.0708671723565527\\
8	0.0983673350718417\\
9	0.121645720241852\\
10	0.141350447873231\\
11	0.15803013970714\\
12	0.172149194021352\\
13	0.18410071547107\\
14	0.194217459962896\\
15	0.202781099290614\\
16	0.210030063480084\\
17	0.216166179190857\\
18	0.221360289001841\\
19	0.225757008033916\\
20	0.229478750344047\\
21	0.232629137194326\\
22	0.235295882089425\\
23	0.237553232908072\\
24	0.239464039122725\\
25	0.241081501663237\\
26	0.242450654144477\\
27	0.243609616698436\\
28	0.244590657320196\\
29	0.245421090277892\\
30	0.24612403660033\\
31	0.246719067815853\\
32	0.247222750865535\\
33	0.247649109362228\\
34	0.248010014037931\\
35	0.248315513250343\\
36	0.248574112750619\\
37	0.24879301250167\\
38	0.248978307140519\\
39	0.249135155666025\\
40	0.249267925076444\\
41	0.249380311955988\\
42	0.249475445395641\\
43	0.249555974113734\\
44	0.249624140201929\\
45	0.249681841549845\\
46	0.249730684686358\\
47	0.249772029508803\\
48	0.249807027145423\\
49	0.249836652005236\\
50	0.249861728907672\\
51	0.249882956047302\\
52	0.24990092443307\\
53	0.24991613434325\\
54	0.249929009254256\\
55	0.249939907631132\\
56	0.249949132907989\\
57	0.249956941936256\\
58	0.249963552135973\\
59	0.249969147549233\\
60	0.249973883964301\\
61	0.249977893253098\\
62	0.249981287042795\\
63	0.249984159823753\\
64	0.249986591580334\\
65	0.249988650017839\\
66	0.24999039244757\\
67	0.249991867382495\\
68	0.249993115887954\\
69	0.24999417272501\\
70	0.249995067318264\\
71	0.249995824575105\\
72	0.249996465579182\\
73	0.24999700817742\\
74	0.249997467476913\\
75	0.249997856265541\\
76	0.249998185368009\\
77	0.249998463947235\\
78	0.24999869975946\\
79	0.249998899370199\\
80	0.249999068337042\\
81	0.249999211364387\\
82	0.249999332434421\\
83	0.249999434917992\\
84	0.249999521668463\\
85	0.249999595101151\\
86	0.249999657260579\\
87	0.249999709877399\\
88	0.249999754416576\\
89	0.249999792118175\\
90	0.24999982403189\\
91	0.249999851046266\\
92	0.249999873913443\\
93	0.24999989327009\\
94	0.249999909655138\\
95	0.249999923524782\\
96	0.249999935265183\\
97	0.249999945203217\\
98	0.249999953615582\\
99	0.249999960736495\\
100	0.249999966764218\\
101	0.249999971866576\\
102	0.249999976185628\\
103	0.249999979841627\\
104	0.249999982936364\\
105	0.249999985556002\\
106	0.249999987773477\\
107	0.24999998965053\\
108	0.24999999123942\\
109	0.249999992584387\\
110	0.249999993722877\\
111	0.249999994686588\\
112	0.249999995502352\\
113	0.249999996192882\\
114	0.249999996777403\\
115	0.249999997272189\\
116	0.249999997691017\\
117	0.249999998045547\\
118	0.249999998345651\\
119	0.249999998599683\\
120	0.249999998814716\\
121	0.249999998996738\\
122	0.249999999150817\\
123	0.249999999281242\\
124	0.249999999391644\\
125	0.249999999485097\\
126	0.249999999564204\\
127	0.249999999631167\\
128	0.249999999687849\\
129	0.24999999973583\\
130	0.249999999776445\\
131	0.249999999810825\\
132	0.249999999839927\\
133	0.249999999864562\\
134	0.249999999885415\\
135	0.249999999903066\\
136	0.249999999918008\\
137	0.249999999930655\\
138	0.249999999941361\\
139	0.249999999950423\\
140	0.249999999958095\\
141	0.249999999964588\\
142	0.249999999970085\\
143	0.249999999974737\\
144	0.249999999978676\\
145	0.24999999998201\\
146	0.249999999984832\\
147	0.249999999987221\\
148	0.249999999989244\\
149	0.249999999990956\\
150	0.249999999992406\\
151	0.249999999993634\\
152	0.249999999994673\\
153	0.249999999995553\\
154	0.249999999996298\\
155	0.249999999996929\\
156	0.249999999997464\\
157	0.249999999997916\\
158	0.249999999998299\\
159	0.249999999998623\\
160	0.249999999998897\\
161	0.249999999999129\\
162	0.249999999999325\\
163	0.249999999999492\\
164	0.249999999999633\\
165	0.249999999999752\\
166	0.249999999999853\\
167	0.249999999999939\\
168	0.250000000000011\\
169	0.250000000000072\\
170	0.250000000000123\\
171	0.250000000000166\\
172	0.250000000000203\\
173	0.250000000000233\\
174	0.250000000000259\\
175	0.25000000000028\\
176	0.250000000000298\\
177	0.250000000000313\\
178	0.250000000000326\\
179	0.250000000000336\\
180	0.250000000000345\\
181	0.250000000000352\\
182	0.250000000000357\\
183	0.250000000000362\\
184	0.250000000000366\\
185	0.250000000000369\\
186	0.250000000000371\\
187	0.250000000000374\\
188	0.250000000000376\\
189	0.250000000000378\\
190	0.25000000000038\\
191	0.250000000000381\\
192	0.250000000000383\\
193	0.250000000000384\\
194	0.250000000000386\\
195	0.250000000000387\\
196	0.250000000000389\\
197	0.250000000000391\\
198	0.250000000000393\\
199	0.250000000000394\\
200	0.250000000000396\\
201	0.250000000000397\\
202	0.250000000000398\\
203	0.2500000000004\\
204	0.250000000000401\\
205	0.250000000000403\\
206	0.250000000000404\\
207	0.250000000000405\\
208	0.250000000000407\\
209	0.250000000000408\\
210	0.250000000000409\\
211	0.250000000000411\\
212	0.250000000000412\\
213	0.250000000000414\\
214	0.250000000000416\\
215	0.250000000000417\\
216	0.250000000000419\\
217	0.250000000000421\\
218	0.250000000000423\\
219	0.250000000000425\\
220	0.250000000000427\\
221	0.250000000000429\\
222	0.250000000000431\\
223	0.250000000000434\\
224	0.250000000000436\\
225	0.250000000000438\\
226	0.250000000000441\\
227	0.250000000000443\\
228	0.250000000000445\\
229	0.250000000000447\\
230	0.250000000000449\\
231	0.25000000000045\\
232	0.25000000000045\\
233	0.25000000000045\\
234	0.25000000000045\\
235	0.250000000000449\\
236	0.250000000000448\\
237	0.250000000000447\\
238	0.250000000000446\\
239	0.250000000000443\\
240	0.250000000000441\\
241	0.250000000000438\\
242	0.250000000000435\\
243	0.250000000000432\\
244	0.250000000000429\\
245	0.250000000000425\\
246	0.250000000000422\\
247	0.250000000000419\\
248	0.250000000000416\\
249	0.250000000000413\\
250	0.250000000000411\\
251	0.250000000000409\\
252	0.250000000000407\\
253	0.250000000000406\\
254	0.250000000000405\\
255	0.250000000000404\\
256	0.250000000000404\\
257	0.250000000000404\\
258	0.250000000000405\\
259	0.250000000000406\\
260	0.250000000000407\\
261	0.250000000000408\\
262	0.250000000000409\\
263	0.250000000000411\\
264	0.250000000000412\\
265	0.250000000000414\\
266	0.250000000000415\\
267	0.250000000000417\\
268	0.250000000000418\\
269	0.250000000000419\\
270	0.25000000000042\\
271	0.250000000000421\\
272	0.250000000000423\\
273	0.250000000000424\\
274	0.250000000000426\\
275	0.250000000000428\\
276	0.25000000000043\\
277	0.250000000000432\\
278	0.250000000000434\\
279	0.250000000000436\\
280	0.250000000000438\\
281	0.25000000000044\\
282	0.250000000000443\\
283	0.250000000000444\\
284	0.250000000000446\\
285	0.250000000000448\\
286	0.25000000000045\\
287	0.250000000000451\\
288	0.250000000000452\\
289	0.250000000000453\\
290	0.250000000000453\\
291	0.250000000000453\\
292	0.250000000000453\\
293	0.250000000000452\\
294	0.25000000000045\\
295	0.250000000000449\\
296	0.250000000000447\\
297	0.250000000000445\\
298	0.250000000000443\\
299	0.250000000000441\\
300	0.250000000000439\\
};
\end{axis}
\end{tikzpicture}%
    \caption{Odpwiedzi poszczególnych torów dla skoku 0 - 1}
\end{figure}
    


\end{document}

