\chapter{Sprawdzenie poprawno?ci podanych warto?ci}
?eby sprawdzi? poprawno?? podanych warto?ci podajemy na wej?cie warto?? $U_{\mathrm{PP}}$ i patrzymy na jakiej warto?ci si? ustali $Y_{\mathrm{PP}}$.

Je�eli dost�pne s� rysunki w~formacie \verb+pdf+, najwygodniej do przetworzenia dokumentu u�y� polecenia \verb+pdflatex+, kt�re bezpo�rednio generuje dokument w formacie \verb+pdf+. Polecenie \verb+latex+ wymaga rysunk�w w formacie \verb+ps+ lub \verb+eps+ i~generuje dokument w~formacie \verb+dvi+, kt�ry nast�pnie mo�na przekszta�ci� do formatu \verb+eps+ lub \verb+pdf+. Nie u�ywamy rysunk�w zapisanych w~plikach bitmapowych (\verb+bmp+, \verb+jpg+, \verb+png+). Jedynym wyj�tkiem s� zdj�cia.

Istnieje wiele podr�cznik�w do nauki zasad sk�adania dokument�w w~\LaTeX u, np. doskona�a praca zbiorowa \cite{litOetiker2007} lub ew. podr�cznik Wikibooks \cite{litlatexwiki2017}. Do edycji dokument�w mo�na wykorzysta� np. program \TeX nicCenter, dost�pny pod adresem \url{http://www.texniccenter.org}. W~przypadku problem�w warto poszuka� rozwi�zania na forum \url{http://tex.stackexchange.com}.

W~dalszej cz�ci dokumentu podano najwa�niejsze wymagania dotycz�ce wzor�w matematycznych, tabeli i~rysunk�w. Najszybsz� metod� prowadz�c� do otrzymania dokumentu jest modyfikacja niniejszego szablonu.