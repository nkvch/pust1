%! TEX encoding = utf8
\chapter{Laboratorium}

\section{Określenie wartości pomiaru temperatury w punkcie pracy}

W celu określenia wartości pomiaru temperatury w punkcie pracy ustawiono moc wentylatora  $W1 = 50\%$,a moc grzałki $G1 = 25\%$.
Po czasie około 5 minut temperatura odczytywana przez czujnik temperatury zaczeła się stabilizować  na poziomie  $T1 = 28,4^{\circ} C$. 
Niestety z powodu ciągłego ruchu powietrza związanego z przemieszczaniem się osób w sali i dużej ilości tych osób wpływających na temperaturę sali oraz czułość stanowiska pomiarowego temperatura odczytywana przez czujnik zaczeła odbiegać i lekko oscylować od tej temperatury.

\begin{figure}[H]
\centering
% This file was created by matlab2tikz.
%
%The latest updates can be retrieved from
%  http://www.mathworks.com/matlabcentral/fileexchange/22022-matlab2tikz-matlab2tikz
%where you can also make suggestions and rate matlab2tikz.
%
\definecolor{mycolor1}{rgb}{0.00000,0.44700,0.74100}%
%
\begin{tikzpicture}

\begin{axis}[%
width=4.521in,
height=3.566in,
at={(0.758in,0.481in)},
scale only axis,
xmin=0,
xmax=400,
xlabel style={font=\color{white!15!black}},
xlabel={k},
ymin=21,
ymax=29,
ylabel style={font=\color{white!15!black}},
ylabel={$\text{T[}^\circ\text{C]}$},
axis background/.style={fill=white}
]
\addplot[const plot, color=mycolor1, forget plot] table[row sep=crcr] {%
1	21.12\\
2	21.12\\
3	21.12\\
4	21.06\\
5	21.06\\
6	21.06\\
7	21.06\\
8	21.12\\
9	21.12\\
10	21.12\\
11	21.12\\
12	21.18\\
13	21.18\\
14	21.18\\
15	21.25\\
16	21.31\\
17	21.31\\
18	21.37\\
19	21.43\\
20	21.56\\
21	21.56\\
22	21.62\\
23	21.68\\
24	21.75\\
25	21.87\\
26	21.93\\
27	22\\
28	22.06\\
29	22.12\\
30	22.18\\
31	22.31\\
32	22.37\\
33	22.43\\
34	22.5\\
35	22.62\\
36	22.68\\
37	22.75\\
38	22.81\\
39	22.93\\
40	23\\
41	23.06\\
42	23.12\\
43	23.18\\
44	23.25\\
45	23.31\\
46	23.43\\
47	23.5\\
48	23.56\\
49	23.62\\
50	23.68\\
51	23.75\\
52	23.81\\
53	23.87\\
54	23.93\\
55	24\\
56	24.06\\
57	24.12\\
58	24.18\\
59	24.25\\
60	24.31\\
61	24.37\\
62	24.43\\
63	24.43\\
64	24.5\\
65	24.56\\
66	24.62\\
67	24.68\\
68	24.75\\
69	24.75\\
70	24.81\\
71	24.87\\
72	24.93\\
73	25\\
74	25\\
75	25.06\\
76	25.06\\
77	25.12\\
78	25.18\\
79	25.25\\
80	25.25\\
81	25.31\\
82	25.31\\
83	25.37\\
84	25.43\\
85	25.43\\
86	25.5\\
87	25.56\\
88	25.56\\
89	25.62\\
90	25.68\\
91	25.68\\
92	25.75\\
93	25.75\\
94	25.81\\
95	25.81\\
96	25.81\\
97	25.87\\
98	25.87\\
99	25.93\\
100	25.93\\
101	26\\
102	26\\
103	26.06\\
104	26.06\\
105	26.12\\
106	26.12\\
107	26.18\\
108	26.18\\
109	26.18\\
110	26.25\\
111	26.25\\
112	26.31\\
113	26.31\\
114	26.31\\
115	26.37\\
116	26.37\\
117	26.37\\
118	26.37\\
119	26.43\\
120	26.43\\
121	26.43\\
122	26.5\\
123	26.5\\
124	26.5\\
125	26.5\\
126	26.5\\
127	26.56\\
128	26.56\\
129	26.62\\
130	26.62\\
131	26.68\\
132	26.68\\
133	26.68\\
134	26.75\\
135	26.75\\
136	26.81\\
137	26.81\\
138	26.81\\
139	26.81\\
140	26.87\\
141	26.93\\
142	26.93\\
143	26.93\\
144	26.93\\
145	26.93\\
146	27\\
147	27\\
148	27.06\\
149	27.06\\
150	27.06\\
151	27.12\\
152	27.12\\
153	27.18\\
154	27.18\\
155	27.18\\
156	27.18\\
157	27.25\\
158	27.25\\
159	27.25\\
160	27.31\\
161	27.31\\
162	27.31\\
163	27.37\\
164	27.37\\
165	27.37\\
166	27.37\\
167	27.37\\
168	27.43\\
169	27.43\\
170	27.43\\
171	27.5\\
172	27.5\\
173	27.5\\
174	27.5\\
175	27.5\\
176	27.56\\
177	27.56\\
178	27.56\\
179	27.56\\
180	27.56\\
181	27.56\\
182	27.62\\
183	27.62\\
184	27.62\\
185	27.62\\
186	27.62\\
187	27.62\\
188	27.62\\
189	27.62\\
190	27.62\\
191	27.68\\
192	27.68\\
193	27.68\\
194	27.68\\
195	27.68\\
196	27.68\\
197	27.75\\
198	27.68\\
199	27.68\\
200	27.68\\
201	27.75\\
202	27.75\\
203	27.75\\
204	27.75\\
205	27.75\\
206	27.75\\
207	27.75\\
208	27.75\\
209	27.81\\
210	27.81\\
211	27.81\\
212	27.81\\
213	27.81\\
214	27.81\\
215	27.87\\
216	27.87\\
217	27.87\\
218	27.87\\
219	27.87\\
220	27.87\\
221	27.87\\
222	27.87\\
223	27.93\\
224	27.93\\
225	27.93\\
226	28\\
227	28\\
228	28\\
229	28\\
230	28.06\\
231	28.06\\
232	28.06\\
233	28.12\\
234	28.12\\
235	28.12\\
236	28.12\\
237	28.12\\
238	28.12\\
239	28.12\\
240	28.12\\
241	28.06\\
242	28.12\\
243	28.12\\
244	28.06\\
245	28.12\\
246	28.06\\
247	28.06\\
248	28.06\\
249	28.06\\
250	28.06\\
251	28.06\\
252	28.06\\
253	28.06\\
254	28.06\\
255	28.06\\
256	28.06\\
257	28.06\\
258	28.06\\
259	28.12\\
260	28.12\\
261	28.12\\
262	28.18\\
263	28.18\\
264	28.18\\
265	28.18\\
266	28.25\\
267	28.25\\
268	28.25\\
269	28.25\\
270	28.25\\
271	28.25\\
272	28.25\\
273	28.25\\
274	28.25\\
275	28.25\\
276	28.25\\
277	28.25\\
278	28.25\\
279	28.25\\
280	28.25\\
281	28.25\\
282	28.25\\
283	28.25\\
284	28.25\\
285	28.25\\
286	28.25\\
287	28.25\\
288	28.25\\
289	28.25\\
290	28.25\\
291	28.25\\
292	28.31\\
293	28.31\\
294	28.31\\
295	28.31\\
296	28.31\\
297	28.31\\
298	28.37\\
299	28.37\\
300	28.37\\
301	28.31\\
302	28.37\\
303	28.31\\
304	28.31\\
305	28.31\\
306	28.31\\
307	28.37\\
308	28.37\\
309	28.37\\
310	28.37\\
311	28.43\\
312	28.43\\
313	28.43\\
314	28.43\\
315	28.43\\
316	28.43\\
317	28.43\\
318	28.43\\
319	28.43\\
320	28.43\\
321	28.43\\
322	28.43\\
323	28.43\\
324	28.43\\
325	28.43\\
326	28.43\\
327	28.43\\
328	28.43\\
329	28.43\\
330	28.43\\
331	28.43\\
332	28.43\\
333	28.37\\
334	28.37\\
335	28.43\\
336	28.43\\
337	28.37\\
338	28.43\\
339	28.43\\
340	28.43\\
341	28.43\\
342	28.43\\
343	28.43\\
344	28.5\\
345	28.43\\
346	28.43\\
347	28.43\\
348	28.5\\
349	28.5\\
350	28.5\\
351	28.5\\
352	28.5\\
353	28.5\\
354	28.5\\
355	28.5\\
356	28.5\\
357	28.5\\
358	28.5\\
359	28.5\\
360	28.5\\
361	28.5\\
362	28.5\\
363	28.56\\
364	28.5\\
365	28.56\\
366	28.56\\
367	28.56\\
368	28.5\\
369	28.5\\
370	28.5\\
371	28.5\\
372	28.5\\
373	28.5\\
374	28.5\\
375	28.5\\
376	28.5\\
377	28.5\\
378	28.56\\
379	28.56\\
380	28.56\\
381	28.62\\
382	28.62\\
383	28.62\\
384	28.68\\
385	28.68\\
386	28.62\\
387	28.68\\
388	28.68\\
389	28.68\\
390	28.68\\
391	28.68\\
392	28.68\\
393	28.68\\
394	28.68\\
395	28.68\\
396	28.68\\
397	28.68\\
398	28.68\\
399	28.68\\
400	28.68\\
};
\end{axis}
\end{tikzpicture}%
\caption{Pomiar temperatury w punkcie pracy}
\end{figure}

\section{Wyznaczenie odpowiedzi skokowych}

Rozpoczynając z punktu pracy wyznaczono odpowiedzi skokowe dla trzech różnych wartości sygnału sterującego  $G1 = 35\%$  $G1 = 45\%$ i $G1 = 55\%$.

\begin{figure}[H]
\centering
% This file was created by matlab2tikz.
%
%The latest updates can be retrieved from
%  http://www.mathworks.com/matlabcentral/fileexchange/22022-matlab2tikz-matlab2tikz
%where you can also make suggestions and rate matlab2tikz.
%
\definecolor{mycolor1}{rgb}{0.00000,0.44700,0.74100}%
\definecolor{mycolor2}{rgb}{0.85000,0.32500,0.09800}%
\definecolor{mycolor3}{rgb}{0.92900,0.69400,0.12500}%
%
\begin{tikzpicture}

\begin{axis}[%
width=4.521in,
height=3.566in,
at={(0.758in,0.481in)},
scale only axis,
xmin=0,
xmax=350,
xlabel style={font=\color{white!15!black}},
xlabel={k},
ymin=28,
ymax=40,
ylabel style={font=\color{white!15!black}},
ylabel={$\text{T[}^\circ\text{C]}$},
axis background/.style={fill=white}
]
\addplot[const plot, color=mycolor1, forget plot] table[row sep=crcr] {%
1	28.5\\
2	28.5\\
3	28.56\\
4	28.56\\
5	28.56\\
6	28.5\\
7	28.5\\
8	28.5\\
9	28.5\\
10	28.5\\
11	28.5\\
12	28.5\\
13	28.5\\
14	28.5\\
15	28.5\\
16	28.5\\
17	28.43\\
18	28.43\\
19	28.43\\
20	28.43\\
21	28.43\\
22	28.43\\
23	28.43\\
24	28.43\\
25	28.43\\
26	28.43\\
27	28.43\\
28	28.43\\
29	28.5\\
30	28.5\\
31	28.56\\
32	28.56\\
33	28.56\\
34	28.62\\
35	28.62\\
36	28.62\\
37	28.68\\
38	28.68\\
39	28.68\\
40	28.68\\
41	28.75\\
42	28.75\\
43	28.75\\
44	28.81\\
45	28.87\\
46	28.87\\
47	28.87\\
48	28.93\\
49	28.93\\
50	29\\
51	29\\
52	29.06\\
53	29.06\\
54	29.12\\
55	29.18\\
56	29.25\\
57	29.25\\
58	29.25\\
59	29.31\\
60	29.37\\
61	29.37\\
62	29.37\\
63	29.43\\
64	29.5\\
65	29.5\\
66	29.56\\
67	29.62\\
68	29.62\\
69	29.68\\
70	29.68\\
71	29.68\\
72	29.75\\
73	29.75\\
74	29.81\\
75	29.81\\
76	29.81\\
77	29.87\\
78	29.87\\
79	29.93\\
80	29.93\\
81	29.93\\
82	29.93\\
83	30\\
84	30\\
85	30\\
86	30.06\\
87	30.06\\
88	30.06\\
89	30.12\\
90	30.12\\
91	30.12\\
92	30.18\\
93	30.18\\
94	30.18\\
95	30.18\\
96	30.18\\
97	30.25\\
98	30.25\\
99	30.25\\
100	30.25\\
101	30.31\\
102	30.31\\
103	30.31\\
104	30.37\\
105	30.37\\
106	30.37\\
107	30.37\\
108	30.37\\
109	30.43\\
110	30.43\\
111	30.43\\
112	30.5\\
113	30.5\\
114	30.5\\
115	30.5\\
116	30.56\\
117	30.56\\
118	30.62\\
119	30.62\\
120	30.62\\
121	30.68\\
122	30.68\\
123	30.68\\
124	30.68\\
125	30.68\\
126	30.68\\
127	30.68\\
128	30.75\\
129	30.75\\
130	30.75\\
131	30.75\\
132	30.75\\
133	30.75\\
134	30.75\\
135	30.81\\
136	30.81\\
137	30.81\\
138	30.87\\
139	30.87\\
140	30.87\\
141	30.87\\
142	30.93\\
143	30.93\\
144	30.93\\
145	31\\
146	31\\
147	31\\
148	31\\
149	31.06\\
150	31.06\\
151	31.06\\
152	31.06\\
153	31.06\\
154	31.06\\
155	31.12\\
156	31.12\\
157	31.12\\
158	31.12\\
159	31.12\\
160	31.12\\
161	31.12\\
162	31.12\\
163	31.12\\
164	31.12\\
165	31.18\\
166	31.18\\
167	31.18\\
168	31.18\\
169	31.18\\
170	31.18\\
171	31.18\\
172	31.18\\
173	31.18\\
174	31.18\\
175	31.18\\
176	31.18\\
177	31.25\\
178	31.25\\
179	31.25\\
180	31.25\\
181	31.25\\
182	31.25\\
183	31.25\\
184	31.25\\
185	31.25\\
186	31.25\\
187	31.25\\
188	31.25\\
189	31.18\\
190	31.18\\
191	31.18\\
192	31.25\\
193	31.18\\
194	31.25\\
195	31.25\\
196	31.25\\
197	31.25\\
198	31.25\\
199	31.25\\
200	31.25\\
201	31.25\\
202	31.25\\
203	31.25\\
204	31.25\\
205	31.25\\
206	31.25\\
207	31.25\\
208	31.25\\
209	31.25\\
210	31.25\\
211	31.25\\
212	31.31\\
213	31.31\\
214	31.31\\
215	31.31\\
216	31.31\\
217	31.37\\
218	31.37\\
219	31.37\\
220	31.37\\
221	31.37\\
222	31.43\\
223	31.43\\
224	31.43\\
225	31.43\\
226	31.37\\
227	31.37\\
228	31.37\\
229	31.37\\
230	31.37\\
231	31.37\\
232	31.37\\
233	31.37\\
234	31.37\\
235	31.37\\
236	31.37\\
237	31.37\\
238	31.37\\
239	31.43\\
240	31.43\\
241	31.43\\
242	31.43\\
243	31.43\\
244	31.43\\
245	31.43\\
246	31.43\\
247	31.43\\
248	31.43\\
249	31.43\\
250	31.43\\
251	31.43\\
252	31.43\\
253	31.43\\
254	31.43\\
255	31.5\\
256	31.43\\
257	31.43\\
258	31.5\\
259	31.5\\
260	31.5\\
261	31.5\\
262	31.5\\
263	31.5\\
264	31.5\\
265	31.5\\
266	31.43\\
267	31.5\\
268	31.5\\
269	31.5\\
270	31.43\\
271	31.5\\
272	31.43\\
273	31.43\\
274	31.43\\
275	31.43\\
276	31.43\\
277	31.43\\
278	31.43\\
279	31.43\\
280	31.43\\
281	31.43\\
282	31.43\\
283	31.43\\
284	31.43\\
285	31.43\\
286	31.43\\
287	31.43\\
288	31.43\\
289	31.37\\
290	31.37\\
291	31.43\\
292	31.43\\
293	31.43\\
294	31.43\\
295	31.43\\
296	31.43\\
297	31.43\\
298	31.43\\
299	31.43\\
300	31.43\\
301	31.43\\
302	31.43\\
303	31.43\\
304	31.43\\
305	31.43\\
306	31.43\\
307	31.43\\
308	31.5\\
309	31.5\\
310	31.5\\
311	31.5\\
312	31.56\\
313	31.56\\
314	31.56\\
315	31.56\\
316	31.56\\
317	31.56\\
318	31.56\\
319	31.56\\
320	31.56\\
321	31.56\\
322	31.62\\
323	31.56\\
324	31.56\\
325	31.56\\
326	31.56\\
327	31.56\\
328	31.56\\
329	31.56\\
330	31.56\\
331	31.56\\
332	31.56\\
333	31.56\\
334	31.56\\
335	31.56\\
336	31.56\\
337	31.56\\
338	31.56\\
339	31.56\\
340	31.56\\
341	31.62\\
342	31.62\\
343	31.62\\
344	31.62\\
345	31.62\\
346	31.62\\
347	31.62\\
348	31.62\\
349	31.62\\
350	31.56\\
};
\addplot[const plot, color=mycolor2, forget plot] table[row sep=crcr] {%
1	28.56\\
2	28.56\\
3	28.59\\
4	28.59\\
5	28.59\\
6	28.56\\
7	28.59\\
8	28.59\\
9	28.59\\
10	28.59\\
11	28.59\\
12	28.59\\
13	28.59\\
14	28.59\\
15	28.59\\
16	28.59\\
17	28.555\\
18	28.555\\
19	28.555\\
20	28.555\\
21	28.555\\
22	28.59\\
23	28.59\\
24	28.59\\
25	28.59\\
26	28.62\\
27	28.62\\
28	28.62\\
29	28.655\\
30	28.685\\
31	28.715\\
32	28.745\\
33	28.78\\
34	28.81\\
35	28.84\\
36	28.87\\
37	28.93\\
38	28.965\\
39	28.995\\
40	29.055\\
41	29.125\\
42	29.185\\
43	29.215\\
44	29.31\\
45	29.37\\
46	29.435\\
47	29.465\\
48	29.555\\
49	29.59\\
50	29.685\\
51	29.715\\
52	29.81\\
53	29.84\\
54	29.935\\
55	29.995\\
56	30.09\\
57	30.125\\
58	30.185\\
59	30.245\\
60	30.34\\
61	30.37\\
62	30.435\\
63	30.525\\
64	30.59\\
65	30.625\\
66	30.715\\
67	30.775\\
68	30.84\\
69	30.9\\
70	30.965\\
71	30.995\\
72	31.06\\
73	31.125\\
74	31.185\\
75	31.215\\
76	31.245\\
77	31.31\\
78	31.34\\
79	31.4\\
80	31.43\\
81	31.465\\
82	31.495\\
83	31.59\\
84	31.625\\
85	31.655\\
86	31.715\\
87	31.745\\
88	31.78\\
89	31.84\\
90	31.87\\
91	31.9\\
92	31.965\\
93	31.995\\
94	32.025\\
95	32.055\\
96	32.09\\
97	32.155\\
98	32.185\\
99	32.215\\
100	32.25\\
101	32.28\\
102	32.34\\
103	32.34\\
104	32.4\\
105	32.435\\
106	32.465\\
107	32.525\\
108	32.525\\
109	32.59\\
110	32.62\\
111	32.62\\
112	32.715\\
113	32.715\\
114	32.75\\
115	32.75\\
116	32.81\\
117	32.84\\
118	32.87\\
119	32.87\\
120	32.9\\
121	32.965\\
122	32.965\\
123	32.965\\
124	32.965\\
125	32.995\\
126	32.995\\
127	33.025\\
128	33.06\\
129	33.09\\
130	33.125\\
131	33.125\\
132	33.155\\
133	33.185\\
134	33.185\\
135	33.245\\
136	33.28\\
137	33.28\\
138	33.34\\
139	33.34\\
140	33.37\\
141	33.37\\
142	33.4\\
143	33.43\\
144	33.43\\
145	33.465\\
146	33.465\\
147	33.5\\
148	33.5\\
149	33.53\\
150	33.53\\
151	33.56\\
152	33.56\\
153	33.56\\
154	33.59\\
155	33.65\\
156	33.65\\
157	33.65\\
158	33.65\\
159	33.685\\
160	33.685\\
161	33.685\\
162	33.715\\
163	33.715\\
164	33.715\\
165	33.775\\
166	33.775\\
167	33.805\\
168	33.805\\
169	33.84\\
170	33.87\\
171	33.87\\
172	33.9\\
173	33.93\\
174	33.93\\
175	33.93\\
176	33.965\\
177	34\\
178	34\\
179	34\\
180	34\\
181	34\\
182	34\\
183	34\\
184	34\\
185	34\\
186	34.03\\
187	34.06\\
188	34.06\\
189	34.055\\
190	34.09\\
191	34.09\\
192	34.125\\
193	34.12\\
194	34.155\\
195	34.125\\
196	34.125\\
197	34.125\\
198	34.125\\
199	34.125\\
200	34.125\\
201	34.125\\
202	34.155\\
203	34.155\\
204	34.155\\
205	34.185\\
206	34.215\\
207	34.215\\
208	34.215\\
209	34.25\\
210	34.25\\
211	34.28\\
212	34.31\\
213	34.31\\
214	34.34\\
215	34.34\\
216	34.37\\
217	34.4\\
218	34.435\\
219	34.435\\
220	34.465\\
221	34.465\\
222	34.495\\
223	34.495\\
224	34.495\\
225	34.495\\
226	34.465\\
227	34.465\\
228	34.495\\
229	34.495\\
230	34.495\\
231	34.525\\
232	34.525\\
233	34.525\\
234	34.525\\
235	34.525\\
236	34.525\\
237	34.525\\
238	34.525\\
239	34.555\\
240	34.555\\
241	34.555\\
242	34.555\\
243	34.555\\
244	34.555\\
245	34.555\\
246	34.525\\
247	34.555\\
248	34.555\\
249	34.525\\
250	34.555\\
251	34.555\\
252	34.555\\
253	34.59\\
254	34.62\\
255	34.655\\
256	34.65\\
257	34.68\\
258	34.715\\
259	34.75\\
260	34.75\\
261	34.78\\
262	34.78\\
263	34.78\\
264	34.78\\
265	34.78\\
266	34.745\\
267	34.78\\
268	34.78\\
269	34.78\\
270	34.775\\
271	34.81\\
272	34.775\\
273	34.775\\
274	34.775\\
275	34.775\\
276	34.775\\
277	34.775\\
278	34.775\\
279	34.805\\
280	34.805\\
281	34.84\\
282	34.84\\
283	34.84\\
284	34.84\\
285	34.84\\
286	34.84\\
287	34.84\\
288	34.805\\
289	34.775\\
290	34.775\\
291	34.775\\
292	34.775\\
293	34.775\\
294	34.745\\
295	34.745\\
296	34.745\\
297	34.745\\
298	34.745\\
299	34.745\\
300	34.775\\
301	34.745\\
302	34.745\\
303	34.745\\
304	34.745\\
305	34.775\\
306	34.775\\
307	34.805\\
308	34.84\\
309	34.84\\
310	34.84\\
311	34.84\\
312	34.87\\
313	34.87\\
314	34.87\\
315	34.905\\
316	34.905\\
317	34.905\\
318	34.935\\
319	34.905\\
320	34.935\\
321	34.935\\
322	34.965\\
323	34.935\\
324	34.965\\
325	34.965\\
326	34.995\\
327	34.995\\
328	34.995\\
329	35.03\\
330	35.03\\
331	35.03\\
332	35.06\\
333	35.06\\
334	35.06\\
335	35.03\\
336	35.03\\
337	35.06\\
338	35.03\\
339	35.06\\
340	35.06\\
341	35.09\\
342	35.06\\
343	35.06\\
344	35.06\\
345	35.025\\
346	35.025\\
347	34.995\\
348	34.995\\
349	34.995\\
350	34.935\\
};
\addplot[const plot, color=mycolor3, forget plot] table[row sep=crcr] {%
1	28.62\\
2	28.62\\
3	28.62\\
4	28.62\\
5	28.62\\
6	28.62\\
7	28.68\\
8	28.68\\
9	28.68\\
10	28.68\\
11	28.68\\
12	28.68\\
13	28.68\\
14	28.68\\
15	28.68\\
16	28.68\\
17	28.68\\
18	28.68\\
19	28.68\\
20	28.68\\
21	28.68\\
22	28.75\\
23	28.75\\
24	28.75\\
25	28.75\\
26	28.81\\
27	28.81\\
28	28.81\\
29	28.81\\
30	28.87\\
31	28.87\\
32	28.93\\
33	29\\
34	29\\
35	29.06\\
36	29.12\\
37	29.18\\
38	29.25\\
39	29.31\\
40	29.43\\
41	29.5\\
42	29.62\\
43	29.68\\
44	29.81\\
45	29.87\\
46	30\\
47	30.06\\
48	30.18\\
49	30.25\\
50	30.37\\
51	30.43\\
52	30.56\\
53	30.62\\
54	30.75\\
55	30.81\\
56	30.93\\
57	31\\
58	31.12\\
59	31.18\\
60	31.31\\
61	31.37\\
62	31.5\\
63	31.62\\
64	31.68\\
65	31.75\\
66	31.87\\
67	31.93\\
68	32.06\\
69	32.12\\
70	32.25\\
71	32.31\\
72	32.37\\
73	32.5\\
74	32.56\\
75	32.62\\
76	32.68\\
77	32.75\\
78	32.81\\
79	32.87\\
80	32.93\\
81	33\\
82	33.06\\
83	33.18\\
84	33.25\\
85	33.31\\
86	33.37\\
87	33.43\\
88	33.5\\
89	33.56\\
90	33.62\\
91	33.68\\
92	33.75\\
93	33.81\\
94	33.87\\
95	33.93\\
96	34\\
97	34.06\\
98	34.12\\
99	34.18\\
100	34.25\\
101	34.25\\
102	34.37\\
103	34.37\\
104	34.43\\
105	34.5\\
106	34.56\\
107	34.68\\
108	34.68\\
109	34.75\\
110	34.81\\
111	34.81\\
112	34.93\\
113	34.93\\
114	35\\
115	35\\
116	35.06\\
117	35.12\\
118	35.12\\
119	35.12\\
120	35.18\\
121	35.25\\
122	35.25\\
123	35.25\\
124	35.25\\
125	35.31\\
126	35.31\\
127	35.37\\
128	35.37\\
129	35.43\\
130	35.5\\
131	35.5\\
132	35.56\\
133	35.62\\
134	35.62\\
135	35.68\\
136	35.75\\
137	35.75\\
138	35.81\\
139	35.81\\
140	35.87\\
141	35.87\\
142	35.87\\
143	35.93\\
144	35.93\\
145	35.93\\
146	35.93\\
147	36\\
148	36\\
149	36\\
150	36\\
151	36.06\\
152	36.06\\
153	36.06\\
154	36.12\\
155	36.18\\
156	36.18\\
157	36.18\\
158	36.18\\
159	36.25\\
160	36.25\\
161	36.25\\
162	36.31\\
163	36.31\\
164	36.31\\
165	36.37\\
166	36.37\\
167	36.43\\
168	36.43\\
169	36.5\\
170	36.56\\
171	36.56\\
172	36.62\\
173	36.68\\
174	36.68\\
175	36.68\\
176	36.75\\
177	36.75\\
178	36.75\\
179	36.75\\
180	36.75\\
181	36.75\\
182	36.75\\
183	36.75\\
184	36.75\\
185	36.75\\
186	36.81\\
187	36.87\\
188	36.87\\
189	36.93\\
190	37\\
191	37\\
192	37\\
193	37.06\\
194	37.06\\
195	37\\
196	37\\
197	37\\
198	37\\
199	37\\
200	37\\
201	37\\
202	37.06\\
203	37.06\\
204	37.06\\
205	37.12\\
206	37.18\\
207	37.18\\
208	37.18\\
209	37.25\\
210	37.25\\
211	37.31\\
212	37.31\\
213	37.31\\
214	37.37\\
215	37.37\\
216	37.43\\
217	37.43\\
218	37.5\\
219	37.5\\
220	37.56\\
221	37.56\\
222	37.56\\
223	37.56\\
224	37.56\\
225	37.56\\
226	37.56\\
227	37.56\\
228	37.62\\
229	37.62\\
230	37.62\\
231	37.68\\
232	37.68\\
233	37.68\\
234	37.68\\
235	37.68\\
236	37.68\\
237	37.68\\
238	37.68\\
239	37.68\\
240	37.68\\
241	37.68\\
242	37.68\\
243	37.68\\
244	37.68\\
245	37.68\\
246	37.62\\
247	37.68\\
248	37.68\\
249	37.62\\
250	37.68\\
251	37.68\\
252	37.68\\
253	37.75\\
254	37.81\\
255	37.81\\
256	37.87\\
257	37.93\\
258	37.93\\
259	38\\
260	38\\
261	38.06\\
262	38.06\\
263	38.06\\
264	38.06\\
265	38.06\\
266	38.06\\
267	38.06\\
268	38.06\\
269	38.06\\
270	38.12\\
271	38.12\\
272	38.12\\
273	38.12\\
274	38.12\\
275	38.12\\
276	38.12\\
277	38.12\\
278	38.12\\
279	38.18\\
280	38.18\\
281	38.25\\
282	38.25\\
283	38.25\\
284	38.25\\
285	38.25\\
286	38.25\\
287	38.25\\
288	38.18\\
289	38.18\\
290	38.18\\
291	38.12\\
292	38.12\\
293	38.12\\
294	38.06\\
295	38.06\\
296	38.06\\
297	38.06\\
298	38.06\\
299	38.06\\
300	38.12\\
301	38.06\\
302	38.06\\
303	38.06\\
304	38.06\\
305	38.12\\
306	38.12\\
307	38.18\\
308	38.18\\
309	38.18\\
310	38.18\\
311	38.18\\
312	38.18\\
313	38.18\\
314	38.18\\
315	38.25\\
316	38.25\\
317	38.25\\
318	38.31\\
319	38.25\\
320	38.31\\
321	38.31\\
322	38.31\\
323	38.31\\
324	38.37\\
325	38.37\\
326	38.43\\
327	38.43\\
328	38.43\\
329	38.5\\
330	38.5\\
331	38.5\\
332	38.56\\
333	38.56\\
334	38.56\\
335	38.5\\
336	38.5\\
337	38.56\\
338	38.5\\
339	38.56\\
340	38.56\\
341	38.56\\
342	38.5\\
343	38.5\\
344	38.5\\
345	38.43\\
346	38.43\\
347	38.37\\
348	38.37\\
349	38.37\\
350	38.31\\
};
\end{axis}
\end{tikzpicture}%
\caption{Odpowiedzi skokowe dla trzech różnych wartości sygnału sterującego}
\end{figure}

Analizując otrzymane wykresy można wywnioskować że właściwości statyczne procesu są w przybliżeniu liniowe, zmiany wartości odpowiedzi skokowej dla tych samych chwil są w przybliżeniu proporcjonalne jak również sam kształt wykresów jest w przybliżeniu podobny. W konsekwencji postanowiono wyznaczyć wzmocnienie statyczne procesu.

\begin{equation}
K_s_t_a_t = 0.3303
\end{equation}


\begin{figure}[H]
\centering
% This file was created by matlab2tikz.
%
%The latest updates can be retrieved from
%  http://www.mathworks.com/matlabcentral/fileexchange/22022-matlab2tikz-matlab2tikz
%where you can also make suggestions and rate matlab2tikz.
%
\definecolor{mycolor1}{rgb}{0.00000,0.44700,0.74100}%
%
\begin{tikzpicture}

\begin{axis}[%
width=4.521in,
height=3.566in,
at={(0.758in,0.481in)},
scale only axis,
xmin=25,
xmax=55,
xlabel style={font=\color{white!15!black}},
xlabel={u},
ymin=28,
ymax=40,
ylabel style={font=\color{white!15!black}},
ylabel={y},
axis background/.style={fill=white}
]
\addplot [color=mycolor1, forget plot]
  table[row sep=crcr]{%
25	28.4\\
35	31.56\\
45	34.935\\
55	38.31\\
};
\end{axis}

\begin{axis}[%
width=5.833in,
height=4.375in,
at={(0in,0in)},
scale only axis,
xmin=0,
xmax=1,
ymin=0,
ymax=1,
axis line style={draw=none},
ticks=none,
axis x line*=bottom,
axis y line*=left
]
\end{axis}
\end{tikzpicture}%
\caption{Charakterystyka statyczna procesu}
\end{figure}

\section{Aproksymacja odpowiedzi skokowej}

Dokonano aproksymacji odpowiedzi skokowej dla wartości sygnału starującego $G1 = 35\%$.

\begin{figure}[H]
\centering
% This file was created by matlab2tikz.
%
%The latest updates can be retrieved from
%  http://www.mathworks.com/matlabcentral/fileexchange/22022-matlab2tikz-matlab2tikz
%where you can also make suggestions and rate matlab2tikz.
%
\definecolor{mycolor1}{rgb}{0.00000,0.44700,0.74100}%
\definecolor{mycolor2}{rgb}{0.85000,0.32500,0.09800}%
%
\begin{tikzpicture}

\begin{axis}[%
width=4.521in,
height=3.566in,
at={(0.758in,0.481in)},
scale only axis,
xmin=0,
xmax=350,
xlabel style={font=\color{white!15!black}},
xlabel={k},
ymin=-0.05,
ymax=0.35,
ylabel style={font=\color{white!15!black}},
ylabel={s},
axis background/.style={fill=white},
legend style={at={(0.03,0.97)}, anchor=north west, legend cell align=left, align=left, draw=white!15!black}
]
\addplot[const plot, color=mycolor1] table[row sep=crcr] {%
1	0\\
2	0\\
3	0.00599999999999987\\
4	0.00599999999999987\\
5	0.00599999999999987\\
6	0\\
7	0\\
8	0\\
9	0\\
10	0\\
11	0\\
12	0\\
13	0\\
14	0\\
15	0\\
16	0\\
17	-0.00700000000000003\\
18	-0.00700000000000003\\
19	-0.00700000000000003\\
20	-0.00700000000000003\\
21	-0.00700000000000003\\
22	-0.00700000000000003\\
23	-0.00700000000000003\\
24	-0.00700000000000003\\
25	-0.00700000000000003\\
26	-0.00700000000000003\\
27	-0.00700000000000003\\
28	-0.00700000000000003\\
29	0\\
30	0\\
31	0.00599999999999987\\
32	0.00599999999999987\\
33	0.00599999999999987\\
34	0.0120000000000001\\
35	0.0120000000000001\\
36	0.0120000000000001\\
37	0.018\\
38	0.018\\
39	0.018\\
40	0.018\\
41	0.025\\
42	0.025\\
43	0.025\\
44	0.0309999999999999\\
45	0.0370000000000001\\
46	0.0370000000000001\\
47	0.0370000000000001\\
48	0.043\\
49	0.043\\
50	0.05\\
51	0.05\\
52	0.0559999999999999\\
53	0.0559999999999999\\
54	0.0620000000000001\\
55	0.068\\
56	0.075\\
57	0.075\\
58	0.075\\
59	0.0809999999999999\\
60	0.0870000000000001\\
61	0.0870000000000001\\
62	0.0870000000000001\\
63	0.093\\
64	0.1\\
65	0.1\\
66	0.106\\
67	0.112\\
68	0.112\\
69	0.118\\
70	0.118\\
71	0.118\\
72	0.125\\
73	0.125\\
74	0.131\\
75	0.131\\
76	0.131\\
77	0.137\\
78	0.137\\
79	0.143\\
80	0.143\\
81	0.143\\
82	0.143\\
83	0.15\\
84	0.15\\
85	0.15\\
86	0.156\\
87	0.156\\
88	0.156\\
89	0.162\\
90	0.162\\
91	0.162\\
92	0.168\\
93	0.168\\
94	0.168\\
95	0.168\\
96	0.168\\
97	0.175\\
98	0.175\\
99	0.175\\
100	0.175\\
101	0.181\\
102	0.181\\
103	0.181\\
104	0.187\\
105	0.187\\
106	0.187\\
107	0.187\\
108	0.187\\
109	0.193\\
110	0.193\\
111	0.193\\
112	0.2\\
113	0.2\\
114	0.2\\
115	0.2\\
116	0.206\\
117	0.206\\
118	0.212\\
119	0.212\\
120	0.212\\
121	0.218\\
122	0.218\\
123	0.218\\
124	0.218\\
125	0.218\\
126	0.218\\
127	0.218\\
128	0.225\\
129	0.225\\
130	0.225\\
131	0.225\\
132	0.225\\
133	0.225\\
134	0.225\\
135	0.231\\
136	0.231\\
137	0.231\\
138	0.237\\
139	0.237\\
140	0.237\\
141	0.237\\
142	0.243\\
143	0.243\\
144	0.243\\
145	0.25\\
146	0.25\\
147	0.25\\
148	0.25\\
149	0.256\\
150	0.256\\
151	0.256\\
152	0.256\\
153	0.256\\
154	0.256\\
155	0.262\\
156	0.262\\
157	0.262\\
158	0.262\\
159	0.262\\
160	0.262\\
161	0.262\\
162	0.262\\
163	0.262\\
164	0.262\\
165	0.268\\
166	0.268\\
167	0.268\\
168	0.268\\
169	0.268\\
170	0.268\\
171	0.268\\
172	0.268\\
173	0.268\\
174	0.268\\
175	0.268\\
176	0.268\\
177	0.275\\
178	0.275\\
179	0.275\\
180	0.275\\
181	0.275\\
182	0.275\\
183	0.275\\
184	0.275\\
185	0.275\\
186	0.275\\
187	0.275\\
188	0.275\\
189	0.268\\
190	0.268\\
191	0.268\\
192	0.275\\
193	0.268\\
194	0.275\\
195	0.275\\
196	0.275\\
197	0.275\\
198	0.275\\
199	0.275\\
200	0.275\\
201	0.275\\
202	0.275\\
203	0.275\\
204	0.275\\
205	0.275\\
206	0.275\\
207	0.275\\
208	0.275\\
209	0.275\\
210	0.275\\
211	0.275\\
212	0.281\\
213	0.281\\
214	0.281\\
215	0.281\\
216	0.281\\
217	0.287\\
218	0.287\\
219	0.287\\
220	0.287\\
221	0.287\\
222	0.293\\
223	0.293\\
224	0.293\\
225	0.293\\
226	0.287\\
227	0.287\\
228	0.287\\
229	0.287\\
230	0.287\\
231	0.287\\
232	0.287\\
233	0.287\\
234	0.287\\
235	0.287\\
236	0.287\\
237	0.287\\
238	0.287\\
239	0.293\\
240	0.293\\
241	0.293\\
242	0.293\\
243	0.293\\
244	0.293\\
245	0.293\\
246	0.293\\
247	0.293\\
248	0.293\\
249	0.293\\
250	0.293\\
251	0.293\\
252	0.293\\
253	0.293\\
254	0.293\\
255	0.3\\
256	0.293\\
257	0.293\\
258	0.3\\
259	0.3\\
260	0.3\\
261	0.3\\
262	0.3\\
263	0.3\\
264	0.3\\
265	0.3\\
266	0.293\\
267	0.3\\
268	0.3\\
269	0.3\\
270	0.293\\
271	0.3\\
272	0.293\\
273	0.293\\
274	0.293\\
275	0.293\\
276	0.293\\
277	0.293\\
278	0.293\\
279	0.293\\
280	0.293\\
281	0.293\\
282	0.293\\
283	0.293\\
284	0.293\\
285	0.293\\
286	0.293\\
287	0.293\\
288	0.293\\
289	0.287\\
290	0.287\\
291	0.293\\
292	0.293\\
293	0.293\\
294	0.293\\
295	0.293\\
296	0.293\\
297	0.293\\
298	0.293\\
299	0.293\\
300	0.293\\
301	0.293\\
302	0.293\\
303	0.293\\
304	0.293\\
305	0.293\\
306	0.293\\
307	0.293\\
308	0.3\\
309	0.3\\
310	0.3\\
311	0.3\\
312	0.306\\
313	0.306\\
314	0.306\\
315	0.306\\
316	0.306\\
317	0.306\\
318	0.306\\
319	0.306\\
320	0.306\\
321	0.306\\
322	0.312\\
323	0.306\\
324	0.306\\
325	0.306\\
326	0.306\\
327	0.306\\
328	0.306\\
329	0.306\\
330	0.306\\
331	0.306\\
332	0.306\\
333	0.306\\
334	0.306\\
335	0.306\\
336	0.306\\
337	0.306\\
338	0.306\\
339	0.306\\
340	0.306\\
341	0.312\\
342	0.312\\
343	0.312\\
344	0.312\\
345	0.312\\
346	0.312\\
347	0.312\\
348	0.312\\
349	0.312\\
350	0.306\\
};
\addlegendentry{odpowiedź układu}

\addplot[const plot, color=mycolor2] table[row sep=crcr] {%
1	0\\
2	0\\
3	0\\
4	0\\
5	0\\
6	0\\
7	0\\
8	0\\
9	0\\
10	0\\
11	0\\
12	0\\
13	0.0001448414725457\\
14	0.000428209042231097\\
15	0.000843993922684382\\
16	0.0013862879075529\\
17	0.00204937752540018\\
18	0.00282773835422124\\
19	0.00371602949139211\\
20	0.00470908817497558\\
21	0.00580192455240992\\
22	0.00698971659270791\\
23	0.00826780513839316\\
24	0.00963168909349638\\
25	0.0110770207440287\\
26	0.0125996012074409\\
27	0.0141953760076655\\
28	0.0158604307724282\\
29	0.0175909870495974\\
30	0.0193833982394246\\
31	0.0212341456396099\\
32	0.0231398346002038\\
33	0.0250971907854343\\
34	0.0271030565396237\\
35	0.0291543873544299\\
36	0.0312482484347216\\
37	0.0333818113604628\\
38	0.0355523508420518\\
39	0.0377572415666248\\
40	0.0399939551328979\\
41	0.0422600570721855\\
42	0.0445532039532921\\
43	0.0468711405690354\\
44	0.0492116972022163\\
45	0.051572786968907\\
46	0.0539524032369846\\
47	0.0563486171178911\\
48	0.0587595750296514\\
49	0.0611834963292357\\
50	0.0636186710123969\\
51	0.0660634574791676\\
52	0.0685162803632449\\
53	0.0709756284235381\\
54	0.0734400524962003\\
55	0.0759081635055075\\
56	0.0783786305319914\\
57	0.0808501789362746\\
58	0.083321588537096\\
59	0.0857916918420557\\
60	0.0882593723296441\\
61	0.090723562781161\\
62	0.0931832436611634\\
63	0.0956374415451199\\
64	0.0980852275929806\\
65	0.100525716067408\\
66	0.102958062895448\\
67	0.105381464272446\\
68	0.107795155307051\\
69	0.11019840870619\\
70	0.112590533498884\\
71	0.114970873797879\\
72	0.117338807598004\\
73	0.11969374561028\\
74	0.122035130130772\\
75	0.124362433943218\\
76	0.126675159254525\\
77	0.128972836662187\\
78	0.131255024152766\\
79	0.133521306130557\\
80	0.135771292475596\\
81	0.138004617630206\\
82	0.140220939713266\\
83	0.142419939661443\\
84	0.144601320396624\\
85	0.146764806018813\\
86	0.148910141023783\\
87	0.151037089544785\\
88	0.153145434617632\\
89	0.155234977468506\\
90	0.157305536823838\\
91	0.159356948241649\\
92	0.161389063463727\\
93	0.163401749788069\\
94	0.165394889460991\\
95	0.167368379088361\\
96	0.169322129065412\\
97	0.171256063024585\\
98	0.173170117300912\\
99	0.175064240414413\\
100	0.176938392569044\\
101	0.178792545167692\\
102	0.180626680342775\\
103	0.182440790501996\\
104	0.184234877888801\\
105	0.186008954157126\\
106	0.18776303996002\\
107	0.189497164551724\\
108	0.191211365402843\\
109	0.192905687828204\\
110	0.194580184627038\\
111	0.196234915735131\\
112	0.197869947888584\\
113	0.199485354298849\\
114	0.201081214338703\\
115	0.202657613238836\\
116	0.204214641794754\\
117	0.205752396083662\\
118	0.207270977191071\\
119	0.208770490946799\\
120	0.210251047670113\\
121	0.211712761923732\\
122	0.213155752276424\\
123	0.21458014107393\\
124	0.21598605421799\\
125	0.217373620953196\\
126	0.218742973661461\\
127	0.220094247663851\\
128	0.221427581029581\\
129	0.222743114391934\\
130	0.224040990770902\\
131	0.225321355402351\\
132	0.226584355573489\\
133	0.227830140464467\\
134	0.229058860995904\\
135	0.230270669682169\\
136	0.231465720490227\\
137	0.232644168703889\\
138	0.233806170793283\\
139	0.234951884289402\\
140	0.236081467663548\\
141	0.237195080211537\\
142	0.2382928819425\\
143	0.239375033472146\\
144	0.240441695920339\\
145	0.241493030812849\\
146	0.242529199987153\\
147	0.243550365502146\\
148	0.244556689551646\\
149	0.245548334381558\\
150	0.246525462210594\\
151	0.247488235154418\\
152	0.248436815153113\\
153	0.249371363901865\\
154	0.250292042784744\\
155	0.251199012811495\\
156	0.252092434557233\\
157	0.252972468104936\\
158	0.253839272990665\\
159	0.254693008151399\\
160	0.255533831875402\\
161	0.256361901755043\\
162	0.257177374641981\\
163	0.257980406604631\\
164	0.25877115288784\\
165	0.259549767874693\\
166	0.260316405050369\\
167	0.261071216967991\\
168	0.261814355216388\\
169	0.262545970389702\\
170	0.263266212058781\\
171	0.263975228744293\\
172	0.264673167891497\\
173	0.265360175846608\\
174	0.266036397834712\\
175	0.266701977939158\\
176	0.267357059082385\\
177	0.268001783008128\\
178	0.268636290264946\\
179	0.269260720191031\\
180	0.269875210900252\\
181	0.270479899269371\\
182	0.271074920926406\\
183	0.271660410240089\\
184	0.27223650031037\\
185	0.272803322959942\\
186	0.273361008726729\\
187	0.273909686857315\\
188	0.274449485301264\\
189	0.274980530706308\\
190	0.275502948414351\\
191	0.276016862458274\\
192	0.276522395559497\\
193	0.277019669126263\\
194	0.277508803252629\\
195	0.277989916718118\\
196	0.278463126988012\\
197	0.278928550214255\\
198	0.279386301236948\\
199	0.279836493586387\\
200	0.280279239485653\\
201	0.280714649853699\\
202	0.281142834308926\\
203	0.281563901173225\\
204	0.281977957476455\\
205	0.282385108961349\\
206	0.282785460088811\\
207	0.2831791140436\\
208	0.283566172740372\\
209	0.283946736830064\\
210	0.284320905706601\\
211	0.284688777513917\\
212	0.285050449153262\\
213	0.285406016290783\\
214	0.285755573365366\\
215	0.286099213596724\\
216	0.286437028993713\\
217	0.286769110362866\\
218	0.287095547317127\\
219	0.287416428284779\\
220	0.287731840518547\\
221	0.288041870104865\\
222	0.288346601973302\\
223	0.288646119906125\\
224	0.288940506547999\\
225	0.289229843415809\\
226	0.289514210908587\\
227	0.289793688317552\\
228	0.290068353836234\\
229	0.290338284570685\\
230	0.290603556549775\\
231	0.290864244735536\\
232	0.291120423033589\\
233	0.291372164303605\\
234	0.291619540369822\\
235	0.291862622031595\\
236	0.292101479073979\\
237	0.29233618027834\\
238	0.292566793432977\\
239	0.292793385343771\\
240	0.293016021844824\\
241	0.293234767809117\\
242	0.293449687159153\\
243	0.293660842877598\\
244	0.293868297017907\\
245	0.294072110714936\\
246	0.29427234419553\\
247	0.294469056789082\\
248	0.294662306938074\\
249	0.29485215220857\\
250	0.295038649300684\\
251	0.295221854059002\\
252	0.295401821482962\\
253	0.295578605737191\\
254	0.295752260161784\\
255	0.295922837282541\\
256	0.296090388821145\\
257	0.29625496570528\\
258	0.296416618078693\\
259	0.296575395311195\\
260	0.296731346008596\\
261	0.296884518022577\\
262	0.297034958460488\\
263	0.297182713695085\\
264	0.297327829374191\\
265	0.297470350430287\\
266	0.297610321090026\\
267	0.297747784883673\\
268	0.297882784654471\\
269	0.298015362567928\\
270	0.298145560121022\\
271	0.298273418151331\\
272	0.298398976846081\\
273	0.29852227575111\\
274	0.298643353779753\\
275	0.298762249221644\\
276	0.298878999751433\\
277	0.29899364243742\\
278	0.299106213750104\\
279	0.299216749570647\\
280	0.299325285199255\\
281	0.29943185536347\\
282	0.299536494226378\\
283	0.299639235394734\\
284	0.299740111926992\\
285	0.29983915634126\\
286	0.299936400623162\\
287	0.300031876233613\\
288	0.300125614116514\\
289	0.30021764470635\\
290	0.300307997935716\\
291	0.300396703242744\\
292	0.300483789578454\\
293	0.300569285414011\\
294	0.300653218747906\\
295	0.30073561711304\\
296	0.300816507583739\\
297	0.300895916782667\\
298	0.300973870887671\\
299	0.301050395638531\\
300	0.301125516343629\\
301	0.301199257886543\\
302	0.301271644732547\\
303	0.301342700935036\\
304	0.301412450141872\\
305	0.301480915601637\\
306	0.30154812016982\\
307	0.301614086314919\\
308	0.301678836124456\\
309	0.301742391310924\\
310	0.301804773217652\\
311	0.301866002824588\\
312	0.301926100754014\\
313	0.301985087276175\\
314	0.302042982314837\\
315	0.302099805452772\\
316	0.302155575937164\\
317	0.302210312684943\\
318	0.302264034288045\\
319	0.302316759018602\\
320	0.302368504834059\\
321	0.302419289382214\\
322	0.302469130006199\\
323	0.30251804374938\\
324	0.302566047360194\\
325	0.302613157296917\\
326	0.30265938973236\\
327	0.302704760558506\\
328	0.302749285391071\\
329	0.302792979574004\\
330	0.302835858183927\\
331	0.302877936034497\\
332	0.302919227680723\\
333	0.3029597474232\\
334	0.3029995093123\\
335	0.303038527152282\\
336	0.303076814505363\\
337	0.303114384695706\\
338	0.303151250813368\\
339	0.303187425718177\\
340	0.303222922043556\\
341	0.303257752200289\\
342	0.303291928380227\\
343	0.303325462559947\\
344	0.303358366504342\\
345	0.303390651770172\\
346	0.303422329709545\\
347	0.303453411473358\\
348	0.303483908014679\\
349	0.303513830092078\\
350	0.303543188272903\\
};
\addlegendentry{odpowiedź aproksymowana}

\end{axis}

\begin{axis}[%
width=5.833in,
height=4.375in,
at={(0in,0in)},
scale only axis,
xmin=0,
xmax=1,
ymin=0,
ymax=1,
axis line style={draw=none},
ticks=none,
axis x line*=bottom,
axis y line*=left
]
\end{axis}
\end{tikzpicture}%
\caption{Aproksymacja odpowiedzi skokowej}
\end{figure}

W celu wyznaczenia optymalnych parametrów optymalizacji posłużono się algorytmem genetycznym o losowej populacji początkowej

\section{Dobranie nastaw regulatora PID i parametrów algorytmu DMC}
 
Wartości nastaw regulatora PID

\begin{equation}
K = 30  T_i = 35  T_d = 4.5  T_p = 1
\end{equation}

Wskaźnik jakości regulacji:

\begin{equation}
E = 20,5988
\end{equation}

