%! TEX encoding = utf8
\chapter{Algorytmy regulacji}

\section{Algorytm cyfrowy regulacji PID}

Prawo regulacji cyfrowego PID:

\begin{equation}
u(k)=r_2e(k-2) + r_1e(k-1) + r_0e(k) + u(k-1) \textrm{, dla } k=12, 13  \ldots N
\label{control_rule}
\end{equation}

Współczynniki regulacji $r_i$ są wyliczane ze standardowych wzorów i zależą od wzmocnienia członu proporcjonalnego $K_p$, parametru całkowania $T_i$, parametru różniczkowania $T_d$, które są parametrami dostrojalnymi regulatora, oraz od wartości okresu próbkowania $T = 0,5s$.

Wartości wejścia i wyjścia będą wystawiane na $U_{PP}$ i $Y_{PP}$ dla $k = 1,2 \ldots 11 $, a dla następnych chwil czasu wejście będzie liczone z prawa regulacji, uwzględniając ograniczenia $u_{min}$, $u_{max}$, $\triangle u^{max}$.

\section{Algorytm DMC w wersji analitycznej}

Prawo regulacji DMC:

\begin{equation}
\triangle U(k)=K(Y^{zad}(k)-Y^0(k))
\end{equation}

Gdzie $\triangle U(k)$ to wektor $N_u$ (horyzont sterowania) przyszłych wartości sterowania, $Y^0(k)$ to przewidywana odpowiedź z modelu procesu, $K$ - macierz policzona raz na początku ze współczynników odpowiedzi skokowej, uwzględniając wybrany współczynnik $\lambda$ oraz horyzonty predykcji i sterowania.

